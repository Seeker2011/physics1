
\chapter{机械振动和机械波}
\minitoc[n]
\section{教学要求}


这一章在前面学过的知识基础上讲解机械振动和机械
波。为了清楚起见,本章教材分为三部分:第一部分讲机械振
动,第二部分讲机械波,第三部分讲声学初步知识。

这一章的教学要求是:
\begin{enumerate}
\item 了解振动产生的条件,理解回复力的概念.
\item 理解振幅、周期和频率等概念的意义。了解相和相差
的概念,知道什么是同相和反相。
\item 从受力情况、速度和加速度、能量几个方面明确简谐
振动的特点;掌握简谐振动的周期公式。
\item 了解振动图象的物理意义.
\item 了解自由振动和受迫振动的意义,明确产生共振的
条件。
\item 理解机械波是怎样产生的,知道什么是横波和纵波。
\item 了解波动图象的物理意义,知道振动图象和波动图象
的区别。
\item 了解波的干涉和波的衍射.
\item 了解声波的产生,了解声波的反射、干涉、衍射以及声
音的共鸣。
\item 了解音调、响度、音品的意义,知道它们各是由什么
决定的。
\item 知道什么是乐音和噪声,了解噪声的危害和控制,知
道什么是超声波和次声波,了解超声波的应用。
\end{enumerate}

下面对这一章的教学内容作些具体说明。

讲解产生振动的条件时,要使学生很好地理解回复力的
概念,知道它是根据力的效果命名的。介绍表征振动的物理
量即振幅、周期和频率时,应注意说明振动有它的特点,需要
:引入新的物理量来描述这种特点,讲解周期的概念时,要着
重说明什么是一次全振动,这是正确理解周期这个概念的
基础。

讲解简谐振动时,应该让学生理解为什么要先研究简谐
振动,再一次说明理想化的方法,要求学生对这种研究方法进
一步有所领会。

简谐振动的周期公式,虽然是就弹簧振子给出的,但对任
何简谐振动都适用;只是对不同的简谐振动,由于受力的性质
不同,$k$的含义也不同。对于单摆,要使学生明确:只有摆角
很小时回复力才跟位移成正比,单摆才做简谐振动。单摆的
周期公式要求学生能够根据简谐振动的一般周期公式自己推
导出来。

相这个概念比较抽象,学生不容易体会它的意义。因此
教材没有给出相的定义,只要求了解:两个简谐振动的振动步
调不一致,就表示它们的相不同,或者说存在着相差。这里,
不讲初相的概念,只要求学生知道什么是同相什么是反相。

由于没有讲振动方程,更需要学生理解振动图象的物理
意义;它表示了振子对平衡位置的位移怎样随时间而变化。还
要求学生明确知道:在振动图象上可以表示出周期和振幅;利
用振动图象还可以比较振动的相。

单摆中能量的转化,在机械能一章中已经讨论过,这里着
重说明单摆的能量跟振幅有关,振幅越大,能量越大,但振动
中能量的转化不要求定量讨论,对于阻尼振动,只要求学生知
道:什么是阻尼振动;在什么条件下可以把阻尼振动作为简谐
振动来处理。

波的概念初学者较难理解,要做好演示,使学生清楚地看
出波是振动的传播,媒质本身并不随波迁移。要求学生对波
的形成有明确的认识,知道振动为什么不会局限在媒质的一
个地方,而要传播出去;知道在振动的传播中,后一个质点总
比前一个质点迟一些开始振动,相邻质点振动的相不同,因而
在整体上看才形成波向前传播。对于横波和纵波,只要求学
生知道,什么是横波,什么是纵波,不要求对它们的传播过程
作过细的分析。

波速的公式$v=\lambda f$是对各种波普遍适用的公式,要求学
生掌握。这里,首先是对波长的概念要有清楚的理解,其次是
知道在一个周期的时间内振动传播的距离等于一个波长。

关于波的图象,要求学生了解它的物理意义,它表示的是
某一时刻各个质点的位移,它是数学图象,只是对横波来说才
直观地表示波形,对纵波的图象,要求学生理解图象的意义
即纵坐标所表示的是各个质点离开平衡位置向左或向右的位
移,不要求仔细地讲述怎样得出这个图象,要求学生能够区
分振动图象和波动图象二者的不同的意义,不要求综合利用
两种图象来分析问题。

关于波的干涉,要求学生知道干涉现象是怎样产生的,即
波峰和波峰(波谷和波谷)相遇处,叠加的结果,振动最强,波
峰和波谷相遇处,叠加的结果,振动最弱。要求学生知道什么
是相干波源。这里提到相差恒定,可以要求学生从波源总保
持同相这种特殊情形来理解。关于波的衍射,只作简单介绍。

第三部分介绍声学的初步知识,重点是介绍乐音的三种
特性,说明它们各是由什么决定的,目前,噪声已成为污染城
市环境的公害之一,教材单列一节讲述噪声的危害和控制,
希望引起学生注意,并知道这方面的简单常识。

“超声波”一节是选讲内容,即使不讲授这节内容,也应该
使学生知道什么是超声波和次声波。

\section{教学建议}
全章分为三个单元。第一单元从第一节至第七节,介绍
了机械振动的基础知识,主要讨论了简谐振动的规律,介绍了
受迫振动和共振的知识,本单元是全章的基础和重点。

第二单元从第八节至第十二节,介绍了机械波的产生、传
播形式,描述手段和波的两个主要特性-干涉和衍射。
第三单元从第十三节到第十七节,介绍了声波的产生、基
本声现象、乐音的三要素和噪声的危害。

\subsection{第一单元}
\subsubsection{机械振动的定义和产生的条件}

为了培养学生科学
的观察和分析能力,可以先学生举一些他们在生活中观察
到的振动的例子,接着教师可展示一组不同物体的振动,例如
可选取弹簧振子、单摆、钟摆的摆轮、一端夹紧的锯条、内燃机
汽缸模型中的活塞、水中的浮子、不倒翁以及天平指针等。启
发他们归纳出这些运动的共同特点即物体或物体的一部分在
某位置附近沿着直线或弧线作往复的周期性运动。接着再让
学生仔细观察一下竖直放置的弹簧振子的振动,分析一下振
子在某位置附近作往复运动的这个“某位置”有什么特点,从
而帮助学生认识课本上指的平衡位置就是振动停止时物体所
在的位置。这样引导学生通过观察,掌握机械振动这种运动
形式的特点。同时也为后面引入产生振动的两个条件作了
准备。

在引入回复力概念时,可先提出前面讲过振动是一种作
用力大小、方向都变的运动。那末振动的物体所受的作用力
有什么共同的特点呢?要求学生分别观察弹簧振子以及水中
浮子的运动,思考物体为什么会作往复运动。在分析物体受
力基础上得出振动物体离开平衡位置后就受到一个指向平衡
位置的力的作用,因此称这个力为回复力,而这个力可以是不
同性质的力或者它们的合力。这样通过典型振动实例的受力
分析来引入回复力的概念,有助于学生认识回复力同向心力
一样,是根据力的效果命名的。

在引入振动的第二个条件时,可先提出为什么前面演示
中的振动物体最终都要停下来?在学生回答有阻力存在后,
可进一步提出如果阻力越来越大会怎样?接着便让学生观察
摆在水中和油中的运动,由此说明阻力过大单摆根本无法实
现往复运动,只有阻力足够小时,才能多次往复运动。

\subsubsection{表征振动的物理量}
在引入振幅概念时,要让学生通
过观察明确振子或摆在振动时,以平衡位置为原点,有一个最
大位移。这个最大位移的绝对值叫做振幅,所以课本中说振
幅是振动物体离开平衡位置最大的距离,而不说最大位移。

学生对周期这一概念并不陌生,首先应该指出振动最基
本的特点是它的周期性,在此基础上,着重帮助学生理解全
振动这一概念,教学中特别要交代清课本中的“位移和速度回
到原来的数值”,所指的“数值”不仅表示它们的大小而且包括
正负。为了使学生掌握振动物体的位移和速度这两个矢量经
过一次往复运动均返回到原先值,就完成一次全振动,有必要
让学生做一次观察练习。用一硬纸板做一红色箭头标志,将
此标志先后放在振子或单摆的不同位移处,让学生反复观察、
明确一次全振动的意义。这样,周期和频率这两个概念和其
相互关系就不难掌握了。为了使学生明确周期和频率是两个
表征振动快慢的物理量,还可让学生观察比较两个摆长相差
较大的单摆的振动,要求学生用脉搏或秒表计时比较一下这
两个摆的周期和频率,从而认识频率越大,周期越小,它们
之间的关系是互为倒数。

教学中还要注意防止学生将“振动的快慢”和“振动物体
运动的快慢”这两种表述混淆起来。因此要点明前者用周期
或频率来描述,对一个确定的振动物体讲是恒定的,而后者用
速度来描写,它是随时间而变的,由此使学生认识振幅、周期、
频率都是从整体上描述振动特点的物理量。

\subsubsection{简谐振动的回复力}

在第二节和第三节中应使学生
对简谐振动回复力的特点、来源以及分析方法有一个逐步深
入的理解。第二节通过弹簧振子的实例引入简谐振动回复力
的表达式$F=-kx$后,应该指出式中回复力与位移的比例常
数,是由振动系统本身结构决定的物理量,应该指出,如果
物体除受回复力作用外,在振动方向上还受其它不平衡力的
作用(如阻力),物体的振动就不是简谐振动了。

\subsubsection{简谐振动的运动规律}

分析简谐振动的规律是教学
中的难点,学生对加速度最大时速度为零,加速度最小时速
度最大往往不易接受,错误地认为,在平衡位置处加速度最
大,在位移最大处加速度最小,这主要在于学生仍没有真正
理解加速度的物理意义以及速度和加速度之间的关系。在分
析简谐振动的规律时,要帮助学生澄清以上错误。

为了培养学生的观察、分析能力,建议在分析振子运动规
律时将课本练习二3作为课堂练习,让学生一边认真观察弹
簧振子(最好选一个$k$较小、$m$较大的振子)的速度随位移变
化的情况,一边将观察结果先填人表中第一项和第四项(回复
力和速度随位移变化的规律)。其中振子速度换向时速度为
零,可提醒学生回忆一下竖直上抛物体达到最高点时的速度
等于什么,然后要求学生分别根据已填好的第一和第四项来
判断第三项(加速度随位移变化规律)和第二项(加速度的方
向和大小变化规律)应怎样填写,在以上观察、填表、分析的
基础上,最后再让学生阅读课本283页对简谐振动的分析,作
为小结。

关于$k$和$m$对振动周期的影响可以进行定性演示。演
示时可将振子放在气垫导轨上,让学生用秒表测出多次振动
的平均$T$值,通过比较用同一弹簧不同质量振子的$T$值和
同一振子不同$k$值弹簧的$T$值,使学生具体认识周期$T$随
的增大而减小,随$m$的增大而增大,为学生理解和接受周期
公式做好准备。


还要指出,$T=2\pi\sqrt{\dfrac{m}{k}}$
是一个适用于一切简谐振动系统
的表达式,只是对不同的振动系统,因回复力的性质不同,式
中$k$的形式也不同。对于固有周期与振幅无关,也要通过演
示使学生信服。

\subsubsection{单摆的周期公式}

单摆周期公式可通过实验观察、设
疑、释疑的方式引入,以培养学生探求和分析新问题的能力并
加深对公式中$k$的物理意义的理解。为此,可以首先介绍一
下伽利略发现教堂吊灯振动规律的故事,并用演示说明摆的
振动周期$T$与振幅、摆锤质量无关,而仅与摆长$\ell$有关。这
样,一方面能进一步加深学生对前面讲过的固有振动周期与
振幅无关的认识,另一方面由于$T$与$m$无关的实验结论和上
节学过的简谐振动周期公式中$T\propto\sqrt{m}$形式上的不一致,可以
提出一引导学生进一步探索的新问题。然后指出解决这一
表面上“矛盾”的关键,是找出单摆振动系统的$k$取决于什么
因素。接着通过对单摆回复力的分析,得出单摆的$k=mg/\ell$,
推导出单摆周期公式$T=2\pi\sqrt{\dfrac{\ell}{g}}$,
解决了前面提出的“矛盾”,
并说明理论上的分析推导与演示实验得出的结论是一致的。


\subsubsection{相和相差}

“相”对学生来说是一个抽象和陌生的新概
念,教学时主要应通过演示实验引导学生观察振动的步调是
否一致来认识相和相差的物理意义,而不必引入相的定义。可
先让学生观察两个相同的单摆作振幅相同但步调不一致的振
动。要求学生指出这两个单摆的振动有什么相同和不同的地
方。从分析这两个摆振动的不同之处,重点启发学生认识振
动步调是否一致就是指是否能保持“同时、同向”(同时向一
个方向运动)。在此基础上指出为了能对这两个摆的振动情
况分别加以描述,就必须引入一个表示振动步调的物理
量——相。接着可分别演示频率相同的摆同相和反相的振
动。演示反相时可先将两单摆从平衡位置左右两侧同时放
手,然后再让学生考虑一下如果两个单摆从同一侧放手,怎
样实现反相,并试着做一下。

\subsubsection{简谐振动图象}

做好绘制振动图象的演示,使学生理
解振动图象的物理意义,是本节教学的关键。为了增加演示
的可见度,便于边演示边分析,建议对课本图9.6的实验作
适当改进,用投影仪来演示。

将振动曲线视为振动质点的运动轨迹,认为振动物体的
速度方向沿着曲线的切线方向是学生中常见的错误。为了帮
助他们理解振动图线的物理意义,关键是使学生搞清沿着拉
动玻璃板方向的横轴所表示的是时间而不表示振动物体的位
移。演示时,可先使摆振动但不拉动下面的玻璃板,让砂或笔
头在它上面来回划出一条直线。说明振动物体仅仅只在平衡
位置两侧来回运动,但由于各个不同时刻的位移在玻璃板上
留下的痕迹相互重叠而呈现为一条直线。在此可让学生思考
一下如何将不同时刻的位移分别显示出来,接着匀速拉动玻
璃板,结果原先成一条直线的痕迹展开成一条曲线,这样便
清楚显示了不同时刻振动物体的位移,从而说明横轴表示的
是时间。教师还可指出匀速拉动(或转动)记录纸来记录参量
随时间变化的技术,被广泛应用于各种仪器中,例如脑电图、
心电图、温度、压力、地震波记录仪所记录的曲线的横坐标都
表示时间,条件允许还可让学生看一下这些仪器的实物和记
录下的曲线图。

\subsubsection{简谐振动的能量}

在从能量角度对简谐振动进行描
述前,可要求学生复习一下机械能守恒定律及其守恒条件,接
着可让学生阅读课文和观察摆或振子的振动。将练习五作为
课堂练习,让学生当堂巩固、并加深对振动过程中能量转化规
律的理解。

教学中应该说明振动系统的总能量取决于外界提供给振
动系统的能量大小,而与振动系统本身的结构无关。还应该指
出,只有与外界没有能量交换的系统作简谐振动时机械能守
恒,才遵循上述能量转化规律,与外界有能量交换的系统,情
况则不一定如此。

\subsubsection{受迫振动和共振}

在受迫振动演示实验中,要指出
只有当受迫振动达到稳定状态后,其频率才等于策动力的频
率。此时策动力对振动系统做功所传递给系统的能量恰好补
偿系统因阻力而损耗的能量,系统的机械能保持不变,成为等
幅振动。

共振的演示实验,除了可用课本图9.12和图9.13两个
装置来演示外,还可增加几个简单的演示,结合练习六2
布置几个课外实验习作,以扩展学生对共振现象的认识。课
本上关于对共振产生原因的定性解释,也可用单摆代替“秋
千”,分别施加跟单摆振动“合拍”和“不合拍”的推力,让学生
观察振幅的变化,启发学生从能量的角度,根据策动力做正功
和负功去认识共振的成因。

至此本单元已介绍了简谐振动、阻尼振动、自由振动、受
迫振动的概念,学生往往搞不清它们的区别和关系,教师可作
一简单的归纳。

\subsection{第二单元}
\subsubsection{机械波的基本特征}

在引入媒质概念时,可让学生
观察下面的演示:在钟罩中置一发声的电铃,将罩中空气抽空
便无法听到铃声,接着将钟罩中的电铃换成一个大功率的灯
泡,抽空罩中空气接通电源,在罩外仍可看到灯泡发光和感到
灯泡发出的热量,由此模拟太阳光波把光和热送到地球上是
不需要任何媒质的,通过以上演示的比较,再举水波和地震波
的例子说明机械波的基本特征是必须依靠某种媒质来传播。

\subsubsection{机械波为什么会在媒质中传播}

可首先用发波水槽
演示一下槽中水面上的浮子不随水波向前运动,使学生对机
械波是振动的传播而不是媒质的迁移获得初步的感性认识。
接着提出振动为什么不会局限在媒质内一个地方的问题让学
生思考,再慢慢转动波动演示器的手柄,让学生观察沿着波的
传播方向相邻质点依次振动的过程,并从媒质各部分之间存
在相互作用力来分析机械波的成因。

\subsubsection{机械波是怎样在媒质中传播的}

左右甩动放在光滑
地面上的长弹簧和推动水平悬挂的长弹簧,让学生从整体上
初步观察一下弹簧中凹凸相间波和疏密相间波的传播,并提
出这两种波怎么形成的问题加以研究。然后在弹簧上的某处
作出醒目的记号,重复以上演示,要求学生注意观察在波的传
播过程中该处质点作什么运动,看它是否一直向前迁移。接
着在弹簧上不同处分别做不同颜色的记号,要求学生从整体
上仔细观察这几处质点振动时步调是否一致,频率是否相同。
然后再用波动演示器或活动投影幻灯片模拟波的传播,放慢
披的传播速度,再现课本上图9.15和图9.16的动态过程,
让学生进一步证实他们在实际观察中得到的初步结论,通过
以上步步深入的观察,引导学生认识弹簧上每个质点只在它
自己的平衡位置附近振动,不同质点的振动频率相同,但相位
不同,它们在传播方向上依次落后,就形成了在整体上所观察
到的凹凸相间和疏密相间的波。

在以上分析的基础上,最后可引导学生归纳一下振动和
波的区别与联系。明确它们的研究对象不同,振动是波的起
因,波是振动在媒质中的传播,并由此得出形成机械波的两个
条件——波源和媒质。

\subsubsection{机械波也是能量在媒质中的传播}

在阐明机械波的
传播实质上就是能量在媒质中传播时可先提出以下问题让学
生思考:波源质点的振动是否可能是一种无阻尼的自由振动?
通过对这一问题的讨论、分析,帮助学生认识媒质质点间的
相互作用力对波源讲是阻力,对波传播方向上的质点讲它既
是动力又是阻力,因此只有不断供给波源能量,它的振动才能
保持下去,并不断地向外输出能量。而沿着振动传播方向的
媒质质元则起了能量传递者的作用,它不断从前面的质元
获取能量,又把这能量传给后面的质元,(它的能量是随时间
作周期性变化的,这与前面第六节中介绍的孤立质元作无阻
尼振动时机械能守恒的情况不同。)于是波在传递振动形式
的同时,也将波源的能量传递开去。

\subsubsection{波长、波速与频率的关系}
掌握波长的定义的关键,是让学生弄清在波的传播方向
上哪两个点是相邻的同相质点。可以结合课本上图9.15和
用手摇波动器 做演示让学生找一下哪些点是相邻的同相质
点。引入波长定义后,还可要求学生进一步集中注意力观察
一下某时刻波动演示器上横波相邻波峰(或波谷)上的两个质
点或者纵波相邻密部(或疏部)中央的两个质点在振动过程中
是否同相,并可将练习七4作为课堂练习,让学生当堂巩固
对波长定义的理解。

掌握波长、波速和频率关系的关键是要把波长等于一个
周期内振动在媒质中传播的距离交代清楚。为此除了用图
9.15进行分析外,仍可在波动演示器上边演示边说明以达到
加深学生印象的效果。此外还可将波长、周期、波速与步行时
的步长,走一步的时间和步行的速度类比(将频率理解为单位
时间内走几步),以帮助学生理解和掌握这个公式。

\subsubsection{横波图象和纵波图象}

应使学生明确,无论是横波还
是纵波图象,都是表示某一时刻媒质各质点离开平衡位置位
移的函数图象。横波图象直观地表示了波的形状,犹如在某
时刻给波传播方向上全体质元拍的一张“照片”,故称横波图
象为波形曲线。学生较难理解的是纵波图象,因为它不太直
观.教学时,可将练习八3作为课堂练习,此外还可补充一
个问题:纵波密部中央质点和疏部中央质点的位移有何特征?
是否与横波中的波峰和波谷相对应。

\subsubsection{波的图象与振动图象} 

比较波动和振动图象的教学,
可在引入简谐波的概念基础上,引导学生分析讨论以下问题:
这两种图象描述的对象是否相同?纵、横坐标的意义是否相
同?相邻两个位移最大值之间距离所表示的意义是否相同?判
断质点在某时刻运动趋势的方法是否相同?由此将它们在物
理意义上的本质区别作一简要归纳。

\subsubsection{波的独立传播和叠加} 

课本图9.21的演示可用长弹
簧(或灌满铅粒的细橡皮管)代替绳子,将它们放在光滑地面
上同时甩动一下两端,要求学生观察两个脉冲波在相遇时振
幅怎样变化,相遇后是否会消失或改变方向和形状,为了使学
生对波的这一基本性质获得更为具体生动的认识,可在波纹
槽或盛水面盆中用两个手指同时轻点水面,就能看到两列水
波互相穿过,这说明了水波的独立传播。在以上演示基础上
再用运动和位移的合成来解释波的叠加现象,学生便较易接
受和理解了。

\subsubsection{波的干涉} 

波的干涉的教学可采用两种方式处理,第
一种方式可按课本的顺序,在前面介绍波的独立传播和叠加
原理的基础上,先从理论上分析两列频率和相都相同、振动方
向一致的水波叠加后会出现什么现象,然后再用演示实验验
证预期的现象,并得出结论。在演示水波的干涉图样时,用音
叉作为两个相干波源,也可取得较好效果。敲击音叉,将它的
两臂接触水面,用投影仪或直接利用阳光的反射都可将水面
上的干涉图样清晰地显示在屏幕或墙壁上。第二种方式可先
用波纹槽演示两列频率和相都相同的水波叠加后产生的干涉
现象,以激起学生探索兴趣,然后再根据波的传播和叠加原
理从理论上分析形成所观察到的现象的原因。

\begin{figure}[htp]
    \centering
\includegraphics[scale=.6]{fig/9-1.png}
    \caption{}
\end{figure}

在用波的叠加原理分析干涉现象时,最好先让学生用两
条截下的瓦楞纸表示两列横波,如图9.1所示,用两个大头针
将它们的端部分别固定在木板上两点表示两个波源,使两条
瓦楞纸在波源前方交叉,交叉点沿着平行于连接两个波源的
一条直线移动。分别用两种不同记号表示交叉处波峰与波峰
(或波谷与波谷)以及波峰与波谷的叠加。这样,可使学生形
象地认识两列同频率的波迭加所产生的振动加强和削弱互相
间隔的效果,还可启发学生通过仔细的观察思考一下,在振
动最强处和最弱处质点所参与的两个振动的相位关系。讲述
时还可用两张印有许多同心圆的投影片的重叠所产生的视觉
形象来清晰显示课本图9.22所示的两个相干波源产生的干
涉条纹。

\subsubsection{产生波的干涉以及衍射的条件}

首先要求学生明确
产生干涉现象的两列波的振动方向必须一致,关于相干波源
的条件的教学,可用演示实验来说明,两个波源的频率必须相
同。关于要求两个波源相差恒定这一点只需从演示干涉现象
时两个波源保持同相这一特殊情况推广即可,不必再引伸解
释。如果时间允许还可用上面介绍的瓦楞纸教具结合章末习
题9做一下,两个波源频率相同但振动反相是否也会出现干
涉现象的演示(用两枚大头针分别将一条瓦楞纸端部的波峰
和另一条端部的波谷固定在木板上两点,代表两个反相的
波源)。

有关波的衍射条件,只需用发波水槽通过演示课本图9.24的情况给出,而不必作理论上的分析解释。应该向学生指
出课本上讲的障碍物或孔的大小尺寸与波长差不多是指产生
明显衍射现象的起码条件,若障碍物或孔的尺寸远小于波长,
无疑是可以产生衍射的,这一点可以通过演示水波通过的孔
缩小至比波长还小时衍射现象越来越明显来说明。

在指出波的干涉、衍射是一切波所特有的现象时可举无
线电波衍射的例子和让学生观察光通过指缝时的衍射现象,
从而使学生获得初步的感性认识。有条件的学校还可用厘米
波发生接收器来演示无线电波通过双缝时的干涉和通过单缝
时的衍射现象。

\subsection{第三单元}
\subsubsection{声源发声时的振动}

在介绍声源是振动物体时,要做
好几个不同的演示,除了课本图9.25的实验外,还可以选用
以下几个演示实验。
\begin{enumerate}
\item 用敲响的音叉接触蒸发皿中的水面,可看到水向外
飞溅。
\item 在鼓面或向上放置的喇叭纸盒上撒一些碎的硬质泡
沫塑料屑,敲响鼓面和使喇叭发声,可看到碎屑的跳动。
\item 弹拨绷紧的橡皮筋或弦可看到橡皮筋和弦的轮廓变
模糊,发音时用手摸咽喉可感到声带的振动。
\end{enumerate}

\subsubsection{声波}

说明声波是声源的振动在媒质中的传递时,可
在一发出低频信号声的喇叭纸盆前置一烛焰,观察烛焰随着
信号声而抖动的现象,从而形象地显示出图9.26所示的声
波是纵波。教学中还可选用以下几个随堂小实验来显示声波
也可在固体、液体中传播以及在不同的媒质中传播的速率
不同。
\begin{enumerate}
\item 将耳朵贴在桌面上,在离耳朵不同距离处用指甲轻敲
桌子。
\item 如图9.2所示,将匙子系在绳子的中间,把绳子的两
端分别用两具手的拇指按在两个耳孔上,敲响匙子,接着松开
按住耳朵的手指,比较前后听到的两种声音有何不同。
\begin{figure}[htp]
    \centering
\includegraphics[scale=.6]{fig/9-2.png}
    \caption{}
\end{figure}

\item 把耳朵紧贴在一个盛满水的塑料袋上,能听到贴在水
袋另一边表的滴嗒声。把塑料水袋取走,比较前后两次听到
的声音有何不同?
\item 将一敲响的音叉放在鱼缸水面的上方,观察缸内鱼的
反应。
\end{enumerate}

\subsubsection{声音的现象} 

讲声波的反射时,要注意讲清人耳能分
清回声的间隔时间与建筑物内交混回响时间的区别,免得学
生把两者混淆起来。

讲解声波的干涉和共鸣现象时,可把音叉发给学生,让他
们自己做课本图9.29的实验,此外还可将面对学生的两个
扬声器(相距约1米)接到同一音频信号发生器上,让学生在
座位上左右晃动身体,便可听到有的地方声音增强,有的地
方声音减弱。在做声音共鸣实验时可用扩音机把共鸣声放大,
使全班同学都能听到。


\subsubsection{乐音和乐音的波形曲线}

学生往往容易把一些声学
名词和概念混淆起来,因此教学中应尽可能使学生通过自身
的听觉和观察来认识和描述人耳对乐音感觉的三个特征以及
它们取决于什么因素。

在介绍课本图9.30乐音的周期性波形曲线时,应说明是
由单一乐器(钢琴)所发出的,并非表示许多乐器或同一乐器
演奏一首乐曲时的波形曲线。此外可以点明这里的波形曲线
实质上也是波源或媒质质点的振动曲线,它的横轴是时间轴,
纵轴是位移轴.后面课本图9.33(乙)和图9.34(乙)所表示的
波形曲线也是如此。有条件的话,最好用示波器演示乐音和
噪声的波形曲线。

\subsubsection{音调和响度}

课本图9.32的演示实验可见度大,效
果较明显,但发声较轻,可用扩音机放大声音。在说明声源振
幅对响度的影响时,应设法增加声源振幅大小的可见度,例如
可轻拨和重拨拉紧的弦或一端夹紧的锯条,使学生看到声音
强弱不同时振幅不同。用课本介绍的音叉或鼓做实验时,可将
敲响的音叉接触通草球或碟子中的水面,在鼓面上撒一些爆
米花来间接显示振幅大小。

在引入声强概念时应使学生明确响度是人们主观上感觉
到的声音强弱,它跟客观上的声强有密切关系。但是,人耳能
听到的最低声强随频率而异。所以,不同频率的声音,即使声
强相同,响度也是不一致的。

\subsubsection{音品}

学生对音品这一概念较难理解,而且容易将它
与音调混淆起来,为此在引入音品这一概念时可播放一段事
先准备好的用不同乐器(如钢琴、小提琴、笛子等)以同一音调
演奏的同一乐曲的录音,让学生辨别 这样能使学生对人耳
之所以能辨别出不同乐器的声音,是由于音品不同,留下一个
鲜明生动的印象。为了使学生了解音品取决于什么因素,关
键是要讲清基音和泛音这两个概念。讲解时最好用示波器来
显示音叉发出的声波是简谐波,而其他乐器发出的声波是类
似图9.33、9.34所示的周期性波.不同乐器发出声音的音调
相同即是指其基音频率相同,而人耳之所以能分辨出不同乐
器发出的声音,是因为它们发出的泛音的多少、频率、振幅不
同所致。

\subsubsection{噪声的危害和控制、超声波}

这两节课可以学生阅读
讨论为主,提高他们的阅读和表达能力。例如让学生列出一
张生活环境中各种噪声源的表,考虑一下有什么控制的方法,
采用讨论的方式,互相交流、补充。此外还可组织看有关超声
波和噪声的教学电影并选择一些杂志上有关的科普文章推荐
给学生看,以开拓他们的知识面。对有兴趣的同学还可组织
他们对周围环境中的噪声(噪声的强弱及噪声源)作些调查,
对噪声治理提出一些建议和办法。

\section{实验指导}
\subsection{演示实验}
\subsubsection{机械振动}
(1)课本图9.1的实验可用倔强系数较小的弹簧来做,
重物的质量可以稍大些,这样可使振动较慢。如果没有$k$值
较小的弹簧,也可以把弹簧取得长一些,这样就相当于$k$值变
小。在演示时可在铁架台上用标记指示重物的平衡位置,以
便于观察重物以这一位置为中心上下往复运动。

(2)也可以用图9.3的装置来演示机械振动,在一乒乓
球的两侧,分别用橡皮胶粘贴
一段细橡皮绳(航模材料),将两
端的橡皮绳固定在两个铁架台
上,调节两个铁架台间的距离,
使得两段橡皮绳都被绷紧,将
乒乓球向上(或向下)拉开一段
距离,释放后,乒乓球就以平衡位置$O$点为中心做上下往复运
动,如果把两个铁架台前后放置,也可以观察乒乓球以平衡位
置为中心的左右往复运动。
\begin{figure}[htp]\centering
    \begin{minipage}[t]{0.48\textwidth}
    \centering
\includegraphics[scale=.7]{fig/9-3.png}
    \caption{}
    \end{minipage}
    \begin{minipage}[t]{0.48\textwidth}
    \centering
\includegraphics[scale=.7]{fig/9-4.png}
    \caption{}
    \end{minipage}
    \end{figure}

    (3)如图9.4所示,在大量筒(800—1000毫升)里盛满
酒精(粘滞系数比水小),将一密度计(比重计——轻表)浮在
酒精中,注意不要让密度计与筒壁相接触。当它静止后,把它
向下压(或向上提起)一小段距离,释放后,可观察到密度计在
酒精中以原来的平衡位置为中心上下振动。

\subsubsection{产生振动的条件}

(1)当物体离开平衡位置后必须受到回复力的作用.如
图9.5所示,弹簧$A$和由普通铁丝绕成的形状和$A$相仿的“弹
簧”$B$, 在它们的下方各挂一个50克的钩码.并使两个钩码
都在同一高度,这一位置就是两个钩码的平衡位置。将弹簧$A$
下的钩码再往下拉一小段距离,释放后,可观察到钩码做上下
振动;而将“弹簧”$B$下的钩码也往下拉一小段相同的距离,释
放后钩码根本不运动。以此来说明物体离开平衡位置后必须
受到回复力的作用,才能产生振动。

(2)将一单摆放在空气中观察其振动,然后再把一盛
有粘性很大的油(如10号或15号机油)的玻璃缸放在桌上,
缸内油面的深度要能使摆球经过平衡位置时全部被浸没(图
9.6)。将摆球拉出油面,释放后,摆球只能比较缓慢地运动到
平衡位置附近,而不能继续做往复运动。这说明产生振动的
第二个必要条件是:阻力要足够小。

\begin{figure}[htp]\centering
    \begin{minipage}[t]{0.48\textwidth}
    \centering
\includegraphics[scale=.7]{fig/9-5.png}
    \caption{}
    \end{minipage}
    \begin{minipage}[t]{0.48\textwidth}
    \centering
\includegraphics[scale=.7]{fig/9-6.png}
    \caption{}
    \end{minipage}
    \end{figure}

    (3)将一钢球放在离心轨道(可用两条粗铁丝自制,如图
9.7所示)的底部$O$点,指出这是钢球的平衡位置,然后将钢
球移到位置$B$, 释放后,可观察到钢球将以$O$点为中心沿着圆
弧做往复运动。这说明钢球离开平衡位置后,受到重力和弹
力的作用(阻力可以不计),这两个力的合力,是使钢球回到平
衡位置的回复力。

\begin{figure}[htp]\centering
    \begin{minipage}[t]{0.48\textwidth}
    \centering
\includegraphics[scale=.7]{fig/9-7.png}
    \caption{}
    \end{minipage}
    \begin{minipage}[t]{0.48\textwidth}
    \centering
\includegraphics[scale=.7]{fig/9-8.png}
    \caption{}
    \end{minipage}
    \end{figure}

\subsubsection{简谐振动}
(1)可利用气垫导轨,在
滑块上固定一根倔强系数较小
的细弹簧(弹簧的直径约1.5—2.0厘米),弹簧的另一端固定在
气垫导轨的一侧,按课本图9.2进行演示和分析.演示时可在
导轨旁放一大的刻度板,用来指示滑块原来的平衡位置,并可
观察滑块在平衡位置两侧的位移变化。

(2)也可利用图9.8所示的装置.在小车的一端固定一
根值较小的弹簧,小车下面垫放一块玻璃板,放在水平桌面
上进行演示。

(3)利用课本图9.1的演示,说明挂在弹簧下端的重物
的振动是简谐振动时,应指出如重物在平衡位置弹簧的形变
为
$x_0$, 则弹簧的弹力$kx_0$和重力$mg$相平衡的位置即是振动
中的平衡位置(图9.9甲)。当重物向下距平衡位置的距离为
$x$ ($x<x_0$) 时,它受的合力$F=k(x_0+x)-mg=kx$, 方向向上
(指向平衡位置),如图9.9乙所示.在重物向上运动的过程
中,这一合力将逐渐变小,当经过平衡位置时,合力$F=0$.
而当重物继续向上运动离开平衡位置的距离为$x$时,重物所
受合力$F=mg-k(x_0-x)=kx$, 方向向下(指向平衡位置,
图9.9丙),这一合力将随着上升距离$x$的增大而增大,因此
从整个振动过程来分析,重物所受的合力跟它离开平衡位置
的位移$x$成正比,而方向始终跟位移相反,所以挂在弹簧下端
的重物在竖直方向上的振动是简谐振动。

\begin{figure}[htp]
    \centering
    \includegraphics[scale=.7]{fig/9-9.png}
    \caption{}
\end{figure}

\subsubsection{简谐振动的周期和频率}
做课本图9.1的实验时,如果把重物的质量减小为原来
的1/4, 振动周期约减小一半,也就是频率约增加一倍.重物
质量不变的情况下,还可以改用$k$值较大或较小(可采用不同
长度的同种弹簧,弹簧原长增加一倍,相当于$k$值减小一半,
弹簧原长减少一半,相当于$k$值增加一倍)的弹簧,重新测定
振动周期,以获得弹簧振子的周期跟它的质量和弹簧$k$值大
小有关的认识。


\subsubsection{简谐振动的图象}
(1)在做课本图9.6的演示实验时,要使用烘干的并用
细端筛过的砂粒(若用金刚砂则效果更好),平板可以用玻璃
板(或透明的有机玻璃板),整个装置放在投影仪上进行演示,
让学生看到振动图象的描绘过程。为了使摆锤的摆动稳定,
可以采用双线摆的结构(图9.10)。
\begin{figure}[htp]
    \centering
    \includegraphics[scale=.7]{fig/9-10.png}
    \caption{}
\end{figure}

(2)或可将一毛笔头缚在锯条的一端,锯条的另一端用
夹具夹住(图9.11),把毛笔尖蘸上少许墨水,将锯条拨动使
之发生振动,同时将与笔尖刚接触的纸片,沿着垂直于笔尖的
振动方向匀速拉动,在纸片上就描绘出笔尖做简谱振动的
图象。
\begin{figure}[htp]
    \centering
    \includegraphics[scale=.7]{fig/9-11.png}
    \caption{}
\end{figure}

(3)也可以像图9.12那样,在两根轻质弹簧之间拴一具
有较大质量的金属螺母,在螺母上固定一个蘸有墨水的毛笔
头,将两根弹簧的另一端$A$和$B$固定起来,使两根弹簧适当绷
紧,弹簧基本上在水平方向,将金属螺母向右(或向左)拉过一
小段距离,释放后带动笔尖在水平方向做简谐振动。调节两
根弹簧固定端$A$、$B$的高度,使得笔尖刚能和放在下面的一张
纸片接触,将纸片沿着垂直于笔尖的振动方向匀速拉动,在纸
片上就描绘出笔尖做简谐振动的图象。

(4)在上述(2)、(3)两个演示实验中,如果适当降低锯条
和弹簧的高度,使得笔尖和纸片间的压力增大,从而增大了振
动时的摩擦力,则可描绘出明显的阻尼振动的图象。
\begin{figure}[htp]\centering
    \begin{minipage}[t]{0.48\textwidth}
    \centering
\includegraphics[scale=.7]{fig/9-12.png}
    \caption{}
    \end{minipage}
    \begin{minipage}[t]{0.48\textwidth}
    \centering
\includegraphics[scale=.7]{fig/9-13.png}
    \caption{}
    \end{minipage}
    \end{figure}

\subsubsection{受迫振动和共振}
(1)课本所示的受迫振动实验,可以在振子下方放一个
盛着水的烧杯(图9.13),适当调节烧杯的高度,使得振子在水
中振动,以增大阻尼,使它的固有振动尽快得到衰减,呈现受
迫振动,但要控制摇手柄的转速,使策动力的频率不要远大于
振子的固有频率,因为由于水的阻力,振子会来不及振动,如
果摇手柄的转速较小,策动力的频率小于振子的固有频率,演
示效果较好。为了使转速均匀,可以把手柄拆下,在曲轴上安
装一个直径约为10—12厘米的皮带轮(可用厚有机玻璃片车
制),利用玩具电动机(2—6伏)通过皮带传动装置来带动曲
轴转动(图9-13),而且通过调节串联在电动机电路中的滑动
变阻器,可以方便地改变电动机转速(相当于改变策动力的
频率)。

(2)摆的共振:用课本图9.13的装置观察摆的共振时,$A$摆的摆球质量
要大些,以仅贮存较多的能量,在张紧的绳上不一定要同时放
上七个单摆,例如除$A$外保留$B$、$D$、$E$三个摆,能观察到跟$A$
摆摆长相等的$B$摆振动的振幅最大也就足以说明问题了。

\begin{figure}[htp]
    \centering
    \includegraphics[scale=.7]{fig/9-14.png}
    \caption{}
\end{figure}

(3)如图9.14所示,将一长木片(航模材料,$50\x1\x300{\rm mm^3}$
)用铁块压在桌子边,长木片露出桌边的长度约为12—14厘米。木片的端部用螺丝固定一个小的玩具电动机(2—6伏),在机轴上缠绕一段焊锡丝并留出一段(约2厘米)不绕上
去。这样,玩具电机运转时就对长木片产生一个周期性的
策动力。在电动机电路中串联一个滑动变阻器,当变阻器的
阻值较小时,电动机转速较大,形成的策动力频率也较大,露
出桌边部分的木片就按相同频率做受迫振动,可观察到这时
木片的振幅较小,调节变阻器,使电阻逐渐增大,电动机转速
逐渐变小,当形成的策动力频率等于露出桌边部分木片的固
有频率时,便可观察到木片做受迫振动的振幅达到最大——
即产生共振的现象。

\subsubsection{横波的形成和传播}
(1)课本图9.15的演示,如果把绳子放在地上做,由于
地面支持力抵消了重力作用,效果较好。但最好选择象磨光
水泥地那种摩擦较小的地面来做。

如果用弹簧来代替绳子,则效果更好。可采用直径为
12—15毫米左右钢丝直径约为0.30—0.35毫米的密绕弹簧。
如果买来的弹簧比较短,则可以用几段接到一起使总长度约
为4—5米.联接的方法:可以用铜丝把两段弹簧一端的一圈
钢丝绞合在一起。
\begin{figure}[htp]
    \centering
    \includegraphics[scale=.7]{fig/9-15.png}
    \caption{}
\end{figure}

(2)如图9.15所示的简单装置,可用来演示横波的形
成。一根拉长到发生均匀范性形变的弹簧,根据需要可截取
数圈装置在有机玻璃支架上。当缓慢转动弹簧时,通过投影
仪,在屏幕上可形象地观察横波的传播过程,改变弹簧的转
向,波的传播方向也随之改变。

这类装置制作简便,但要注意每圈弹簧间的间隔应相等,
并且转动轴在弹簧的轴线上,为了观察横波中质点的振动方
向,可在弹簧的任一圈上滴上一滴熔融的火漆成一小球。当
转动弹簧时,可观察到凹、凸相间的横波波形向前推进的同
时,质点以平衡位置为中心作振动(图9.16)。

\begin{figure}[htp]
    \centering
    \includegraphics[scale=.7]{fig/9-16.png}
    \caption{}
\end{figure}

为比较同相点或反相点的振动,可在弹簧的相应位置上
滴制二个或四个火漆小球(图9.17)。

\begin{figure}[htp]
    \centering
    \includegraphics[scale=.7]{fig/9-17.png}
    \caption{}
\end{figure}

为了不同的观察目的,这类装置可以制作一批,例如弹簧
的不同圈距可以表示不同波长,弹簧的不同直径则可表示不
同的波幅。

\subsubsection{波长、频率和波速的关系}

\begin{figure}[htp]\centering
    \begin{minipage}[t]{0.48\textwidth}
    \centering
\includegraphics[scale=.6]{fig/9-18.png}
    \caption{}
    \end{minipage}
    \begin{minipage}[t]{0.48\textwidth}
    \centering
\includegraphics[scale=.7]{fig/9-19.png}
    \caption{}
    \end{minipage}
    \end{figure}

(1)可利用发波水槽和投影仪进行演示.如图9.18所
示。在玩具电动机的机轴上用焊锡丝缠绕成一个偏心装置,
将玩具电动机固定在具有弹性的金属片的一端,金属片固定
在发波水槽的边上,它的伸出部分的长度可以调节,在金属
片的伸出端下方装-个用来跟水面接触的平板型振子(图9.19)。振子可用1.5—2毫米厚的木板制成,用螺丝固定时要注
意调节安装位置,使得电动机转动时它只做上下振动而没有
横向摆动(还可以通过改变金属片的伸长部分的长度来调
节)。这样,通过投影就可以在屏幕上观察到平板型振子所产
生的平面波。从波纹明暗相间的间隔宽度可以大致后出波长。
改变电动机电路中串联的滑动变阻器的阻值,使电动机转速
变快,相当于使波源的频率变大,这时可以观察到,波纹间的
间隔变窄,即波长变小;使变阻器电阻增大,电动机转速变慢,
波源频率变小,则波纹间距变宽,即波长变大。这就定性地说
明了在波速一定时(机械波的波速是由媒质性质所决定的),
波长跟频率成反比关系。

(2)用圆盘频闪观察器观察当频率一定时,波长随波速
的增大而增大。
\begin{figure}[htp]
    \centering
    \includegraphics[scale=.7]{fig/9-20.png}
    \caption{}
\end{figure}

如图9.20所示,在发波水槽的一边装一个框架,把带有
偏心装置的玩具电动机安装在一块木条上,木条用两根橡皮
筋(或弹簧)吊在框架上,使木条下部稍许浸入水中,当电动机
转动时,使木条振子发生上下振动,在水面上形成一系列平面
水波。在水槽的另一边叠放几块矩形玻璃板,调节玻璃板的
厚度,使水层变得很浅。这样,当平面波进入到这一薄层的区
域时,相当于传播振动的媒质性质发生了改变,于是波速有了
改变(变小),由于波源的频率不变,因此波长也有了相应改变
(变短)。上述现象可以投影在屏幕上直接观察(为了消除槽
边反射波的影响,可在水槽四周垫放一些塑料回丝)。

\begin{figure}[htp]
    \centering
    \includegraphics[scale=.7]{fig/9-21.png}
    \caption{}
\end{figure}

屏幕上的波纹可以利用手动式圆盘频闪观察器来进行
观察.频闪观察器可以自制,在一块直径约25厘米的圆盘状
硬纸板或薄木板上开6条或12条等距对称、成辐射状的条形
观察孔,如图9.21所示,在圆盘中心开孔,用螺栓做轴,使
圆盘能绕轴转动,在轴上用螺母固定一个手柄,并在盘上靠
近中心处开一个直径约1.5—1.8厘米的指孔,用来拨动圆盘,
这便制成了圆盘频闪观察器。使用时一手持柄,另一手用食指
插入指孔转动圆盘,眼睛通过盘上的辐射状观察孔注视屏幕
上的波纹,逐渐增大圆盘转速,当圆盘上的观察孔经过眼前的
频率与水波频率同步时,所观察到的波纹就会“固定”下兼静
止不动(否则波纹不是前进便是后退)。这时可看到玻璃板上
方的波纹和前面部分的波纹同时被“固定”住。但这两个区域
内相邻两条波纹的间距大小是不相等的,即波长不同。这就
直观地显示了波进入不同媒质时频率不变,波速和波长发生
了变化的情况。

\subsubsection{波的叠加}
(1)课本图9.21所示的在绳子上两个相遇的波互相穿
过的实验效果不是很明显。可以改用前述的4—5米长的弹
簧进行演示。演示时让学生围在四周观察,请一位学生当助
手。教师与学生各拿着弹簧的一端,把弹簧平放在地板上先
让学生把弹簧的$A$端按在地上,教师把弹簧的$B$端迅速向上
抖动一下,这时可观察到一个凸起的(振动平面垂直于地面)
半波,从$B$向$A$传播。分别从弹簧的$A$、$B$端发出一个波,可
观察到这两列波相遇,互相穿过后,仍然各自保持原有的状态
继续向前传播的现象,这个实验也可以贴着地面抖动弹簧,
使波动在水平方向上发生。

(2)利用发波水槽也可以观察两列水波互相穿过的现
象。演示时可用两支口径粗细不同的滴管,在水槽中同时滴
下两颗水滴,大水滴激起的水面波的能量较大,从屏幕上看,
圆面的波纹较粗,小水滴激起的水而波的波纹较细。可以看
到粗、细两个圆形波互相穿过后,仍保持各自原有的粗细程度
向前传播。

(3)还可利用课本图9.16的装置,在细长弹簧的左端推
动摆球振动一次,发出一个波的同时,在右端用手掌推动一
下弹簧形成一个频率较小的波,可以观察到这两列纵波相互
穿过的现象。

\subsubsection{水波的干涉}
利用图9.18的发波水槽和投影装置,可以观察课本图
9.23的水面波的干涉图样,演示时可以在弹性金属片下安装
两个相隔一定距离的金属小球(直径约4—5毫米),作为两个
频率和相都相同的波源(相干波源)。观察时要调节聚光镜的
位置,使水面的像最为清晰。

发生干涉时只要求两个波源频率相同、相差恒定,两个波
源的振幅不一定要相等,这一点可以通过演示证明。把安装
双球振子的固定片稍稍倾斜,使一个球接触水面的深度深些,
另一个浅些,这样,发出的两列波的振幅就不相等,但还是能
看到稳定的干涉图样。

\subsubsection{声波的干涉}
将正在发声的音叉放在耳旁徐徐转动,就能辨别出声音
忽强忽弱的现象。也可以将正在发声的音叉放在一个话筒前
转动,把信号放大后接在扬声器上,可以听出声音忽强忽弱。
还可以把示波器并联在扬声器两端进行观察,可以看到随着
音叉的转动,所形成的声波波幅的变化。

以上几个现象都说明了声波的干涉。但对如何形成声波
干涉的具体过程,建议不必行分析,因为这是比较复杂的,
不象发波水槽中水面波产生干涉那样的单纯。

\subsubsection{声音的共鸣}
(1)利用课本所述的共振音叉演示声音的共鸣时,要使
两个共鸣箱的开口端互相对着,比较靠近些,并且使两个音
叉的振动方向在同一平面上。当用橡皮锤敲击一个音叉时,要
稍待一会儿,使得通过空气的振动把能量较多地传给另一个
音叉,然后用手按住被敲出的音叉,去听另一音叉发出的
声音。

(2)也可利用弦的共振来进行演示(沈括在《梦溪笔谈》当
中所介绍的方法),把两根弦固定在弦音计上,调整到相同的
频率,拨动一根弦时,可以看到骑放在另一根弦上的小纸片会
发生弹跳飞落的现象。

(3)课本图9.29所示的空气柱的共鸣实验中,所用的玻
璃管的直径约为2厘米左右.如果选用频率为520赫兹的音
叉,则玻璃管的长度应不小于20厘米,因为声波在这一频率
时它的四分之一波长约为16厘米,这样才能上下移动玻璃管
调节气柱的长度。

\subsubsection{音调跟频率有关}
(1)按课本图9.32的装置进行演示时,所用的纸片应选
用薄而硬的材料(譬如可用一小块新的牛皮纸效果很好)。

(2)敲击不同频率的音叉,由话筒通过放大器用示波器
观察它们的波形.若以384赫的波形为标准(譬如调到出现
三个全波),再换上256赫兹或520赫兹的音叉,可以明显地
看到波数变少或变多,说明频率越大,音调越高。

(3)利用音频信号发生器,当音调连续(或不连续)改变
时,可观察到示波器上的波数出现相应的改变。

\subsubsection{响度跟振幅有关}
(1)音叉插在共鸣箱上,用橡皮锤轻敲,音叉发出比较轻
的声音,同时用悬挂在支架上的小木球靠近音叉的一个叉股,
观察小球被弹开的角度。然后再用橡皮锤较重地敲击音叉,
音叉发生较响的声音,用小木球靠近时,可观察到小木球被弹
开的角度要大得多。

(2)利用示波器进行观察,轻敲音叉时波形的波幅较小,
较重地敲击音叉时,可观察到波幅明显增大。

\subsubsection{音品}
用示波器观察,对纯音(可用频率为256赫兹的音叉)和
其他乐器(或人声)所发出的中央C音的波形进行对比。











































































































































































\section{参考资料}
\subsection{周期运动、振荡和振动}
周期运动是任何一种在相等的间隔中完全重复的运动。
设$X(t)$代表系统在时刻$t$在某坐标系中的位移,则对于时间
变量的每一个$t$值周期运动都具有方程$X(t+T)=X(t)$所
定义的性质。一个循环持续的时间$T$称为运动的周期。钟
表的擒纵机构的运动,地球绕太阳的公转以及发动机在匀速
运转时曲柄、连杆和活塞更复杂的运动都是周期运动的例子。
任何周期运动都可以表示为傅里叶级数,即一些正弦项或余
弦项之和,各项的频率是整个周期运动频率$f$的整数倍。例
如通过傅里叶分析任何一个复杂的周期波都可以表示为一系
列正弦波分量的级数之和,分量的周期是这个复合波的基本
周期的$1/n$, $n$为正整数。

振荡是任何一种往复变化的现象,振荡的例子包括声波
中压力的变化,以及数学函数的起伏,即某数值在某一平均值
上下的重复交替变化。振荡这个词在很多场合与振动同义,
虽然后者主要指机械振动。例如为了检测产品抗震性能的机
器称为振动台,而使电流方向周期性反复交变的电子器件通
常称为振荡器。电磁场的交替变换称为电磁振荡。如给系统
以某一初扰动,然后让它自己进行振荡,这种现象称为\textbf{自由振
荡}。系统对恒定作用的周期激扰的响应称为\textbf{受迫振荡}。振幅不
断减小的任何振荡称为\textbf{衰减振荡}。通常是由于系统有能量输
出。振幅保持不变的振荡称为非衰减振荡,这通常是由于外
部能源补充能量。

振动这个术语描述相对于某一规定的中心基准(平衡位
置)位移的持续周期变化,这种周期运动可以包括从摆的简
单来回摆动、钢板受锤击后的较为复杂的振动,直到大型结构
的极其复杂的振动。例如汽车在粗糙路面上行驶时发生的振
动。所有的原子、分子与核子也都在不断振动,一个机械系
统要能自身维持其自由振动必须具有质量与弹性的特性或者
与此相当的特性,具有弹性是指系统从正常形态的任何偏离
将引起回复力促使系统返回正常形态。具有质量或惯性是指
系统在回到正常形态时所获得的速度又可使系统超越这一形
态。正是由于质量和弹性的相互作用,机械系统才有可能发
生周期振动。

\subsection{谐运动和谐振子}
谐运动是以时间的正弦(或余弦)函数表示的一种周期运
动,通常称为简谐运动,它是最简单形式的振动,这种运动对
它的平衡位置是对称的,在平衡位置处速度最大,加速度为零;
在最大位移(或转向点)处速度为零,加速度最大。这种运动的
特征是由单一的频率(无泛音)来表示的。谐运动可出现于非
常简单的机构中,例如匀速圆周运动物体上任一点在固定直
线上的投影。谐运动也可以是振动系统对某一周期正弦力的
响应。谐运动是许多简单系统的典型运动,只要使系统偏离
其稳定平衡位置然后释放并略去阻力得到的即是简谐运动。

谐振子是被回复力或回复力矩束缚在稳定平衡位置的任
何物理系统,其中回复力或回复力矩与离开平衡位置的线位
移或角位移成正比。如果这样一个物体从它的平衡位置被扰
动后释放并且阻尼可以忽略,则由此引起的振动是简谐振
动而没有谐波。振动频率是振子的固有频率,取决于它的惯
性(质量)和回复力的劲度(倔强系数)。谐振子并不限于力学
系统也可以是电学系统。(然而典型的电子振荡器只是近似
的谐振子)

\subsection{声压、声强}
声压:没有声波传播时媒质中的压强为$p_0$, 当声波在媒
质中传播时,某点的压强为$p_1$, 定义$p=p_1-p_0$为该点的声
压。随着质点位移的周期性变化,声压也作周期性的变化,声
压与媒质中质点的速度成正比,而且两者的位相相同。频率越
高,越容易获得较大的声压。正是由于声压的变化,引起耳膜
的振动才产生了听觉。在室内大声说话的声压大约为$10^5$帕。









