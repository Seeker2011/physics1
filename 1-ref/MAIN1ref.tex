\documentclass[a4paper, openany]{ctexbook}

%\usepackage{anysize}
%\papersize{26cm}{18.5cm}
%\marginsize{2.25cm}{2.25cm}{2cm}{2cm}
\usepackage[margin=3cm]{geometry}


% 修改脚注的编号为加圈样式,并且各页单独编号

\usepackage{pifont}
\usepackage[perpage,symbol*]{footmisc}
\DefineFNsymbols{circled}{{\ding{192}}{\ding{193}}{\ding{194}}
{\ding{195}}{\ding{196}}{\ding{197}}{\ding{198}}{\ding{199}}{\ding{200}}{\ding{201}}}
\setfnsymbol{circled}



\usepackage{amsmath,amsfonts,mathrsfs,amssymb}
\usepackage{graphicx}

\usepackage[font=bf,labelfont=bf,labelsep=quad]{caption}

\usepackage{tikz}



\usepackage{physics}
%\usepackage{amsthm}
\usepackage{ntheorem}
\theoremseparator{\;}



\usepackage{blkarray}
\usepackage{bm}
\usepackage[colorlinks=true, linkcolor=black]{hyperref}

%\usepackage{enumerate}


\theoremstyle{plain}
\theoremheaderfont{\normalfont\bfseries} 
\theorembodyfont{\normalfont}


\newtheorem{example}{\bf 例题}[chapter]
\newenvironment{solution}{\noindent {\bf 解答:}}{}%{\hfill $\blacksquare$\par}


\renewcommand{\proofname}{\bf 证明:}
\newenvironment{proof}{{\noindent \bf 证明:}}{}%{\hfill $\square$\par}

\newcommand{\E}{\mathbb{E}}
\renewcommand{\Pr}{\mathbb{P}}
\newcommand{\EP}{\mathbb{E}^{\mathbb{P}}}
\newcommand{\EQ}{\mathbb{E}^{\mathbb{Q}}}
\newcommand{\dif}{\,{\rm d}}
\newcommand{\Var}{{\rm Var}}
\newcommand{\Cov}{{\rm Cov}}

\newcommand{\x}{\times}

 \usepackage{tcolorbox}
 \tcbuselibrary{breakable}
 \tcbuselibrary{most}

\setcounter{tocdepth}{1}

\setcounter{secnumdepth}{3}



\ctexset {
section = {
	name = {第,节},
	number = \chinese{section}},
subsection = {
	name = {\hspace{2em},、\hspace{-1em}},
	number = \chinese{subsection}
},
subsubsection = {
	name = {\hspace{2em}(,)\hspace{-1em}},
	number = \chinese{subsubsection},
}
}


\renewcommand{\contentsname}{目~~录}

\usepackage{paralist}
\let\itemize\compactitem

\let\enumerate\compactenum
\let\description\compactdesc


\usepackage{titlesec}
\titlespacing{\chapter}{0pt}{*1}{*1}
\titlespacing{\section}{0pt}{*1}{*1}
\titlespacing{\subsection}{0pt}{*1}{*1}

\titlespacing{\subsubsection}{0pt}{*1}{*1}

\renewcommand{\le}{\leqslant}
\renewcommand{\ge}{\geqslant}
\usepackage{mathtools}

\setlength{\abovecaptionskip}{0.cm}
\setlength{\belowcaptionskip}{-0.cm}

\usetikzlibrary{decorations.pathmorphing, patterns}
\usetikzlibrary{calc, patterns, decorations.markings}
\usetikzlibrary{positioning, snakes}



\usepackage{yhmath}
\newcommand{\ms}{\text{m}/\text{s}}
\newcommand{\cms}{\text{cm}/\text{s}}
\newcommand{\msq}{\text{m}/\text{s}^2}
\newcommand{\cmsq}{\text{cm}/\text{s}^2}
\newcommand{\kmh}{\text{km}/\text{h}}

\usepackage{tkz-euclide}

\usepackage{minitoc}
\setcounter{minitocdepth}{3}  






\begin{document}
\fontsize{11}{14}\selectfont














\title{\Huge\bfseries 高级中学\\物理(甲种本)第一册\\教学参考书\vspace*{2cm} }



\author{\Large 人民教育出版社物理室~~编}
\date{\Large 1986年2月}

\maketitle

\dominitoc
\tableofcontents


\frontmatter

\chapter{前~~言}

为了帮助教师使用好高中物理(甲种本)第一册教材,我
们编写了这本教学参考书,内容包括全册书的说明,引言和
各章的教学说明和资料.

全册书的说明对这册教材的内容安排、全书的一些重要
问题以及课时安排,作了简要的说明.

各章的教学说明和资料,包括教学要求、教学建议、实验
指导、习题解答、参考资料五项内容.在“教学要求”中对各章
的教学内容提出了具体的要求和说明,在“教学建议”中对怎
样进行教学提出了参考性意见.在“实验指导”中提出了演示
实验、学生实验及课外实验活动中应当注意的问题,还提供了
自制简单仪器的制作方法和不同的实验方法,补充了一些实
验内容,供教师选用.在“习题解答”中给出了课本中全部练习
和习题的解答,供教师参考.在“参考资料”中提供了一些教
学中可供参考的材料,这些参考资料只供教师参考,个别的也
可在教学中引用.

本书第一章和第二章“教学建议”、“习题解答”由吴孟
明编写,第三章和第四章的“教学建议”、“习题解答”、“参考资
料”及全书的“实验指导”由袁哲诚编写,第五章和第九章的
“教学建议”、“习题解答”及部分“参考资料”由曹磊编写,第六
章、第七章和第八章的“教学建议”、“习题解答”及第七章的
“参考资料”由唐锦顺编写,其余部分由郭连璧编写.全书由
郭连璧统稿,统稿中刘克桓、董振邦给予了很多帮助,最后经
雷树人审阅修改.

欢迎教师对本书提出宝贵意见.


\chapter{高中物理甲种本第一册的说明}

(1)高中物理课本(甲种本)第一册全部讲力学知识.第
一章讲解力的基本知识,第二章讲解怎样描述运动.在这两
章的基础上第三章讲解运动和力的关系.前两章是必要的预
备知识,而牛顿定律是力学的核心,第四章讲解曲线运动,是
运动定律知识的具体应用和扩展,第五章讲解万有引力定
律,把已学知识应用到天体运动中去,第六章讲解静力学的
基本知识,可视为动力学的特殊情形,第七章和第八章分别
讲解机械能和动量,是牛顿力学的进一步展开,特别是得出守
恒定律,为解决力学问题开辟了新的途径.最后一章是利用
以前学过的知识来分析更为复杂的运动——振动和波.

上述安排,第一章不涉及力的平衡,第二章不讲述平抛和
斜抛物体的运动,这样,使得教学难点后移,头一、二章的内
容相对集中,只包括了学习牛顿定律所必需的预备知识,有助
于降低初高中的台阶,另外,万有引力定律单独设章,使每一
章的中心突出,便于教学.

(2)高中物理难学主要在力学部分,原因之一是某些内
容要求偏高.为了减轻学生的学习困难,使他们对基础知识
掌握得更好些,教学要从实际出发,要求要适当.这里,对大
家一直很关心的几个问题作些说明.

关于矢量.中学物理中需要用矢量解决的问题,一般比
较简单,只要有一维矢量运算以及矢量的合成和分解的知识,
也就够用了,不需要系统地讲解矢量运算的知识,因此,物理
量和公式可以不用矢量符号和矢量式来表示,关于矢量的教
学要求可以归结为三点:第一,懂得什么是矢量,什么是标量,
知道它们之间有什么不同;第二,知道矢量加法服从平行四边
形法则,并且会用它对矢量进行合成和分解;第三,知道一维
矢量的运算可以化成代数运算,并且学会这种运算.

关于连接体.一般的连接体问题本来不是很难,对于培
养学生综合运用知识和分析问题的能力也有好处.但是讲了
连接体,会派生出大量的难题,大大增加了学生负担.因此,
从整体来看,目前,在中学以不讲连接体为宜.对于个别学
校,如果学生理解力较强,教师可以自己补充少量不太难的连
接体问题.这样师生都比较主动,也符合因材施教的原则.

不讲连接体问题,并不是不讲隔离法.把所研究的对象
从周围物体中孤立出来或者隔离出来,单独对它进行受力分
析,并运用动力学或静力学的规律来解决问题,这种分析问题
和解决问题的方法,学生应当学会.

(3)教学要符合学生的认识水平和认识规律,努力做到
循序渐进.

物理的概念和规律的特点之一是有严格的含义,但是在
中学阶段,对物理概念的讲解,不宜过分追求严谨.例如即时
速度这个概念,严格的讲,用极限和微分的知识才能讲解清楚.
由于学生没有学过极限和微分,初学者要透彻理解这个概念
比较困难.因此,讲解这个概念,不宜采用数学意义较浓的讨
论,而要着重从物理意义上予以说明.又如加速度的概念,-
开始就区分平均加速度和即时加速度,这固然严谨,学生接受
却有困难.就匀变速运动来讲解加速度,不再区分平均加速
度和即时加速度,可以克服上述困难,便于教学.

学生掌握知识要有个过程.一个概念,只知道它的定义,
远不能说掌握了它.要在具体问题中不断运用它,逐步体会
它的含义,要在逐步揭示与其他概念的联系中,逐步加深对它
的理解.掌握物理规律和某种分析问题的方法,情况也类
似.例如力这个概念,学生对它的认识是逐步丰富起来的.一
开始只明确力是物体间的相互作用;讲过力的合成和分解之
后,又明确了力是矢量,服从矢量运算规则;讲到动力学揭示
出力和加速度的关系,学生才认识到力是产生加速度的原因,
懂得了牛顿第二定律,在这个基础上,再提出力的独立作用
原理加以讨论,对于物体受力分析的教学,也要逐步深入.一
开始就分析很多复杂的事例,企图一下子就掌握好,一劳永逸
地解决问题,实际上是办不到的.

(4)教学中,在讲解知识的同时,应该注意渗透物理学的
研究方法,以提高学生分析和处理物理问题的能力.

研究和处理问题,首先要明确研究对象,这一点看起来简
单,在具体运用中学生却往往不能把握,造成分析上的混乱.
这一点要在教学中通过具体事例强调说明,使学生能够掌握.

研究问题要从简单情况入手,这不仅是为了方便,而且是
一种科学的研究方法.在简单情况下考虑的因素少,容易把
问题搞清楚.然在此基础上逐步把以前未考虑的因素考虑
进去,情况逐渐复杂,研究逐步展开.研究直线运动从匀速运
动开始,研究振动从简谐振动开始,都属于这种情形,希望学
生对此有所体会.

从简单情况入手研究问题,需要理想化的方法,需要科学
的抽象.教材在讲述质点这一概念时,第一次介绍了这种方
法,此后,凡遇到这种情况,也都予以说明,理想实验也属于
理想化方法.伽利略的斜面对接实验是一个理想实验,介绍
这个实验,可以使学生了解用理想实验这种推理形式能够深
入地把握现象的本质.

从简单情况入手以及理想化的方法,需要分清主次,即抓
住主要因素,暂时舍去次要因素,把问题予以简化,这在研究
和处理问题时十分重要,例如在分析物体受力情况时,如果
物体在光滑平面上运动,可以忽略滑动摩擦力;如果物体的截
面积较小而且运动速度不大,可以不考虑空气阻力,都是采用
了这种简化方法.

还有一些研究处理问题的方法,不再一一列举,希望在教
学中予以注意.

(5)高中物理中抽象思维的作用虽然有所加强,但实验
的重要性不能削弱.高中物理仍然是以实验为基础,要重视
实验教学.

对于学生的实验要求有这样的设想.懂得实验原理,能
根据实验课文的叙述自己确定实验步骤,能正确地使用仪器,
会读取数据并设计表格记录数据,知道怎样分析数据得出结
论,会写简明的实验报告-这些是对所有实验的共同要求.
至于分析误差以及对实验结果的进一步讨论等,则只对部分
实验作要求.

习题中的实验题,要求学生课外自己做,课外实验,不作
要求,但应该鼓励学生去做.有的实验活动,需要教师给予必
要的指导,有的器材学生没有,学校可以借给学生使用.

(6)课本中的练习题分为两种.一是练习,设在每节或
每单元之后,是基本练习题,一部分可随堂做,一部分可留为
作业,一是习题,设在每章之后,其中有综合题和较难的题,
一部分可在习题课中解决,一部分可留为作业.习题中个别
有代表性的题或较难的题,给出了解.这类带解的习题可让
学生自己看,也可作为例题讲,由教师酌定.

习题的安排,要求是逐步提高的.头两章内容是基础性
的,学生又是刚到高中学习,因此安排的都是基本题目.从第
三章开始增加了综合题.第七、八章的要求更高些,注意培养
学生灵活运用知识的能力.经验证明,对学生解题能力的培养,
必须循序渐近.一开始就布置过多过难的练习题,往往是师
生负担重,教学效果不够好的重要原因.这一经验教训,应充
分吸取.

每一章后都有复习题.复习题大都很简单,目的是让学
生通过解答复习题自己写出全章的复习提纲.有的复习题要
求高些,让学生理一下全章的基本思路,总结一下学习经验.

(7)高中物理课本(甲种本)第一册的教学内容可按每周
4课时,全学年共128课时讲授完.各章所用的课时数是:引
言6(3)课时(括号内的数字是学生实验的课时数,下同),第
一章力11(2)课时,第二章直线运动14(3)课时,第三章运动
定律12(2)课时,第四章曲线运动12(2)课时,第五章万有引
力定律6课时,第六章物体的平衡7(1)课时,第七章机械能
15(2)课时,第八章动量13(3)课时,第九章机械振动和机械
波18(1)课时,平时复习和机动时间14课时.

\chapter{引言——怎样学好物理知识}
\minitoc[n]
\section{教学要求}
学生在开始学习高中物理时,往往会不适应,感到困难.
这是因为跟初中物理比较起来,高中物理在广度上有所扩大,
在程度上也有明显的提高.抽象思维和推理论证的作用增大
了,数学的应用增多了,学生实验的要求也提高了.课本安排
了《引言》这一部分内容,先集中讲一讲怎样学好物理知识,就
是要在学习方法上给学生一些引导,使他们能较快地适应高
中物理的学习.

学生实验要测量读取数据和分析处理数据.在做第一次
学生实验前,有必要让学生了解误差和有效数字的初步知识.
因此,在教过《引言》之后,要进行学生实验部分的《误差和有
效数字》的教学.

《引言》及《误差和有效数字》的教学要求是:
\begin{enumerate}
    \item 了解进一步学习物理知识的重要性以及怎样才能学
好高中物理.
\item 了解误差的概念和有效数字的意义,知道在实验测量
中要按有效数字规则读数.
\end{enumerate}

在《引言》的教学中,对于教材中提出的学好高中物理应
该注意的几个问题,学生只要有个初步的了解就可以了,不能
要求他们一下子领会得很好,以后还可以结合各章的教学,使
他们逐步加深认识,不断改进学习方法,适应高中阶段的学习
要求.

关于误差的教学,要求学生了解误差的概念,即知道什么
是误差,什么是系统误差和偶然误差,知道误差是不可避免
的,初步知道怎样分析产生误差的原因,但不要求定量地讨论
和计算误差.

由于有效数字的运算规则比较复杂,教材未作介绍,也不
要求教师讲解.只要求学生懂得有效数字的意义,在实验测
量中能按有效数字规则读数,在处理实验数据和解题时,运算
结果一般取两位或三位数字就可以了.

\section{教学建议}
《引言》从回顾初中学过的物理知识讲起,进一步明确物
理知识的重要,指出了高中物理的特点.重点是对怎样学好
高中物理提出了几个应当注意的问题,即如何做好物理实验、
如何学好物理概念和规律以及如何做好练习.

《引言》的教学不必拘泥于课文的叙述,教师可以根据《引
言》的教学要求、自己的教学经验和学生的实际情况,灵活地
组织教学.

\begin{enumerate}
    \item \textbf{消除物理难学的惧怕心理}\quad  在说明高中物理的特点,
指出跟初中物理相比较有明显提高时,还要注意消除一般学
生思想上存在的物理难学的惧怕心理,树立学好高中物理的
信心,这可以从两个方面来说明:一是已有两年初中物理的
学习基础;二是只要注意不断改进学习方法,就会逐步适应高
中物理的学习.
\item \textbf{激发学生学习物理的兴趣}\quad  学生只有对学习物理发
生了兴趣,才能积极主动、心情愉快地学习物理,收到良好的
学习效果,因而在引言的教学中,如何激发学生学习物理的
兴趣,显得十分重要.学生的学习兴趣在很大程度上产生于
求知的渴望,教学中应该多举一些学生熟悉和感兴趣的事例,
来说明物理知识的重要.例如,人造卫星为什么能够围绕地
球运行?电冰箱为什么能够致冷?收音机和电视机为什么能接
收声音和图像信号,把它们再现出来?照相机和电影放映机的
原理又是什么?什么是原子能?等等.要了解这些问题,都需
要学习物理知识.还可以通过能为学生接受的,生动具体的
事例,说明物理学对促进科学技术的发展和人类社会的进步
所起的巨大作用.让学生了解,目前世界已进入了原子能、电
子计算机、自动化、半导体、激光、空间科学等新技术的时代,
现代科学技术正经历一场伟大的革命,使学生把学习物
面向现代化、面向世界、面向未来联系起来,增强他们的学习
责任感,激发他们的求知欲望.
\item \textbf{密切联系实际,指导学习方法}\quad  讲述如何做好物理实
验、如何学好物理概念和规律以及如何做好练习时,切忌照本
宣科,空洞地说教.要通过具体的事例来说明问题.例如,讲
述做实验的要求,可结合学生在初中实验中存在的一般问题,
来说明怎样做是对的,怎样做是不对的;讲述要注意观察演示
实验时,可实际做一、两个简单的实验,具体说明这个实验研
究的是什么问题,应当观察什么,怎样观察,应当记录什么现
象和数据,怎样记录,怎样分析看到的现象和处理实验数据,
最后得到怎样的结论,等等;讲述如何解题时,也应结合具体
例题说明解题的思考方法和一般步骤,给学生做一个良好的
示范.
\item \textbf{误差和有效数字的教学}\quad  这部分内容按照课本的要
求进行教学就可以了,不宜作更多的补充.参考资料中介绍
的有关内容,只供教师参考,不宜向学生讲述.但可以告诉学
生,误差和有效数字之间的关系;有效数字的最后一位就是误
差所在的一位;由误差决定有效数字,这是处理一切有效数字
问题的依据.因此在中学物理实验中,-一般来说,要求估读到
测量仪器最小分度的十分之几.由于误差和有效数字问题比
较复杂,在处理实验数据和解题时,运算结果一般取两位或三
位数字就可以了.还应告诉学生,对于原来有效数字位数不
多的数字,运算结果再取更多的位数,也是毫无意义的.
\end{enumerate}


\section{实验指导}
\subsection{学生实验}
\subsubsection{练习分析实验数据}
这个实验并不要求学生进行实际测量,而是根据课
本给出的某一次实验中所得的数据,来练习分析和处理实验
数据的技能,了解分析实验数据对于得出物理量之间的关系
和研究物理量变化规律所具有的重要意义.

教材要求通过给出的实验数据,用图象方法找出一
定量的水,从容器底部不同直径的排水孔流出时,排尽水的时
间$t$跟排水孔直径$d$的定量关系.在讲解这种方法时,要使学
生了解“自变量”和“因变量”是根据所要研究的两个物理量的
变化关系来确定的.对于容器中一定量的水来说,排尽水的
时间$t$由水孔直径的大小$d$所决定,所以排水孔直径$d$是
自变量,排尽水的时间$t$是因变量.但是,应该让学生注意,
由于研究的问题不同,同一个物理量,有时是自变量,有时又
是因变量.譬如,在研究做匀速直线运动的物体经过的路程
对时间的关系时,时间就不再是因变量,而是自变量了.

在得出$t$-$d$图象是一条曲线之后,可以看出$t$随$d$
的增大而减小,但这只是一个定性的关系,从生活经验或单
从数据表也就足以判断了.所以必须作进一步的分析与猜想,
以便得出定量的关系.由于水是从整个圆孔排出的,排尽水
的时间$t$是否跟圆孔的横截面积$S$存在简单的反比关系呢?
因为圆面积$S$是跟直径$d$的平方成正比的,如果这一猜想成
立,那么,排尽水的时间$t$应该和$d$成反比,即$t$与$1/d^2$成正
比.因此要计算出$1/d^2$的数值,描绘出$t$-$1/d^2$图象,看看是
不是一条直线.要使学生明白建立在实验基础上的合理的猜
想,是研究物理问题的一种重要的思想方法.

在分别画出水深为30厘米和水深为10厘米的
$t$-$1/d^2$图象后,可以让学生自己得出排尽水的时间$t$跟排水
孔横截面积$S$存在着怎样的关系,还可以让他们考虑,这一关
系是否会因排水时的水深不同而有所改变?

\subsubsection{游标卡尺的使用}

这个实验要求学生通过实际操作,了解游标卡尺的
构造和它的读数原理,正确掌握读数方法,学会正确使用游标
卡尺.

实验前可利用游标卡尺的放大模型(也可以按课本
图10.2所示的游标尺部分自制)先使学生理解主尺最小分度
是1毫米、游标尺上有10个小的等分刻度、总长度等于9毫
米、准确度为0.1毫米的游标卡尺的读数原理.使用这种游
标卡尺读数时在0.1毫米后还可估读一位,但不要求估读(详
见参考资料).

根据学生实际使用的游标卡尺制成游标尺部分的放
大模型,让学生结合所用的游标卡尺,观察了解这种卡尺的主
尺最小分度是多少毫米?游标尺上有多少个小的等分刻度t总
长度等于多少毫米?读数时的准确度为多少毫米?
如果使用准确度为0.05毫米或0.02毫米的游标卡尺,由
于读数误差发生在毫米读数的百分位,因此不应再估读一位
(详见参考资料).

在使用游标卡尺练习测量金属管的长度、内径和外
径时,课本要求在测长度时每次测量后让金属管绕轴转过
$45^{\circ}$再测量下一次;测内径和外径时,要在管子的两端分别量
出两个互相垂直的内,外径,然后分别求出它们的平均值.应
该使学生了解为什么要这样测量的原因.采用多次测量求平
均值的方法可以减小测量时的偶然误差,但是在这个实验中
多测量是为了弥补金属管可能不是一个理想圆柱体而造成的
误差,这也是一种常规的测量方法.

得出金属管的内径和外径的平均值求得金属管的体
积后,还可以启发学生思考,利用这些数据如何来计算金属管
的管壁厚度?

\subsubsection{螺旋测微器的使用}
这个实验要求学生通过操作了解螺旋测微器的构造
和它的读数原理,正确掌握读数方法,学会正确使用螺旋测
微器.

实验前可利用螺旋测微器的放大模型,让学生结合
所用的螺旋测微器进行观察,了解其结构,并让学生将旋钮$K$
向逆时针方向慢慢旋动,使测微螺杆$P$慢慢后退,同时观察当
$P$后退0.5毫米时,可动刻度$H$恰好转过50格,即转过一周.
因此,可动刻度每转过1格,就相当于沿着螺旋的轴线方向移
动$0.5/50$毫米$=0.01$毫米(不要误认为可动刻度上每1格的
长度-两条刻度线间的距离等于0.01毫米),所以用螺旋
测微器测量长度时可以准确到0.01毫米.

关于读数问题.螺旋测微器的零误差在使用前应先
调整好,使得基本上没有零误差,这样,在读数时只需先读出
固定刻度尺上的毫米读数,再加上可动刻度的读数,而不必考
虑零误差的存在.读数时要特别注意观察固定刻度尺上表示
半毫米的刻线是否已经露出,由于固定刻度尺上的刻线较粗,
有时很难判断究竟半毫米的刻线已经露出还是即将露出,这
时就应看可动刻度$H$上零刻线的位置是在固定刻度尺的准线
之上,还是在准线之下,如果在准线之下,就表示半毫米的刻
线已经露出(图1),则在读数时应加上0.5毫米.此外,在
读数时还应估计一位读数(不可靠的),如果认为可动刻度$H$
上的基条刻线正好跟固定刻度尺$S$上的准线重合,则读数时
最后一位的估计读数应为0, 在记录读数时,这个0也应写
出,虽然这个0并不是可靠的.
如图1所示的读数应为6.540
毫米.

\begin{figure}[htp]
    \centering
\includegraphics[scale=.8]{fig/1.png}
    \caption{}
\end{figure}


关于正确使用螺旋测
微器的方法.应指导学生阅读
课本340页上有关的叙述,并通过演示示范,要求学生严格
遵守.

在完成课本要求的测量后,还可让学生用螺旋测微
器测量一下自己的头发的直径,如果时间允许,还可以事先准
备一些包装香烟用的薄铝箔(去除后面的衬纸,铝箔厚度约为
0.01毫米左右),让学生分别用游标卡尺和螺旋测微器测量它
的厚度,从而体会螺旋测微器是比游标卡尺更精密的湖量长
度的工具.

\section{参考资料}
\subsection{测量误差的估计}
测量的结果不可能绝对精确,总会产生误差,对于测量
结果,可以信任到何种程度,需要知道测量误差,由于误差是
测量值与真实值之差,而真实值本身是不能确切知道的,因此
对于测量误差只能是估计.

\subsubsection{多次测量结果偶然误差的估计}
为了减少偶然误差,在可能的情况下,总是采用多次测
量,以多次测量的算术平均值作为测量结果,即
\[N=\overline{N}=\frac{1}{k}(N_1+N_2+\cdots+N_k)\]
根据误差的统计理论,算术平均值$N$最接近于真实值.

在这种情况下,通常简单的估计方法,是用算术平均偏差
$\delta$来表示多次测量结果的偶然误差.

设第$i$次测量值$N_i$, 与平均值$N$的偏差为$\delta_i$, $i=1,2,\ldots,k$,即
\[\delta_1=N_1-N,\; \delta_2=N_2-N,\ldots,\delta_k=N_k-N\]
则算术平均偏差
\[\delta=\frac{1}{k}(|\delta_1|+|\delta_2|+\cdots+|\delta_k|)\]

\subsubsection{一次测量结果误差估计}
有些情况下,测量不能重复或者不需要精确测量,则需要
估计一次测量结果的误差,这时可以根据测量仪表所注明的
误差来估计.例如2.5级电表,它的一次测量误差可估计为
满刻度的2.5\%, 如果没有注明,可取仪表最小分度值的一半
作为测量误差,例如最小分度为毫米的刻度尺,它的一次测
量误差可估计为0.5毫米.

需要说明的是,测量误差应当包括系统误差和偶然误差
两个方面,有些情况下,主要是系统误差;有些情况下,主要
是偶然误差.估计误差时,要作具体的分析,实际上,往往情
况比较复杂,要对测量结果的误差作出估计,不是一件容易的
事,在中学,并不要求学生估计误差.

\subsection{绝对误差和相对误差}
实际上,常把测量结果写成$N\pm\Delta N$的形式.其中$N$是
测量值,它可以是一次测量值,也可以是多次测量的平均值
$\overline N$; $\Delta N$是测量误差值,叫做绝对误差.绝对误差给出了测量值
的误差范围,但这并不排除多次测量中有的测量值在$N\pm\Delta N$
以外.

用绝对误差不能对测量结果的好坏给出一个十分清楚的
概念.于是引入了相对误差的概念,相对误差用$\Delta N/N$表示,
也叫做百分误差.

相对误差与绝对误差的关系是
\[\Delta N=N\x \frac{\Delta N}{N}\]

误差与有效数字密切相关,由于误差本身是一个估计数,
所以,一般情况下误差的有效数字只取一位,在特殊情况下也
不超过两位,多了是没有意义的.

根据有效数字的含义,有效数字的最后一位是有误差的.
因此,有效数字的最后一位一定要同误差所在的一位取齐.这
就是说,有效数字的位数取决于绝对误差.

相对误差与有效数字之间的关系,大体上讲,有效数字的
位数越多,相对误差就越小;有效数字的位数越少,相对误差
就越大,例如,$1.320\pm 0.001$厘米,有效数字是4位,相对误
差$\frac{0.001}{1.320}\approx 0.08\%$; $1.3\pm 0.1$厘米,有效数字是2位,相对误差$\frac{0.1}{1.3}\approx 8\%$, 一般说来,两位有效数字的相对误差为十分之
几至百分之几,三位有效数字的相对误差为百分之几至千分
之几,依此类推.

\subsection{测量仪器的读数规则}
原教育部颁布的《高中物理教学纲要(草案)》,要求学生
实验测量中能按有效数字规则读数.那么,有效数字的读数
规则是什么,如何要求学生按有效数字规则读数呢?

有效数字的最后一位一定要同误差所在的一位取齐,这
是考虑有效数字的依据.因此,测量仪器的读数规则应当是:
测量误差出现在哪一位,读数时就应读到哪一位,这样,就要
首先估计测量误差,然后再确定读到哪一位.
这里讨论一下电表的读数问题,中学生使用的电表的准
确度是2.5级,即在规定的使用条件下,最大误差不超过满刻
度的2.5\%.

安培表有3安培和0.6安培两个量程,使用3安培量程
时,误差是3安$\x2.5\%=0.075$安,即误差出现在安培的百分
位.这时表的最小分度是0.1安培,即可以准确读到安培的
十分位,因此,使用3安培量程时,应估读一位到安培的百分
位.使用0.6安培量程时,误差0.6安$\x2.5\%=0.015$安,即
误差也是出现在安培的百分位,这时表的最小分度是0.02安
培,即可以准确读到0.02安培,因此,使用0.6安培量程时,
只能读到安培的百分位,可以估读半小格,如果估读到安培
的千分位,则是无意义的了.

伏特表有15伏特和3伏特两个量程使用15伏特量程
时,误差是0.37伏特,即误差出现在伏特的十分位.这时表
的最小分度是0.5伏特,即可以唯确读到0.5伏特.因此,使
用15伏特量程时,应读到伏特的十分位,即可以估该五分之
一小格,使用3伏特量程时,跟安培表3安培量程的情况一
样,应估读一位到伏特的百分位.

可以看出,实验测量中,究竟读取几位有效数字,要作具
体分桥、然而,中学关于误差和有效数字的教学要求不高,并
不要求估计测量误差,在处理实验数据和解题时,运算结果一
般取两位或三位数字就可以了.因此,在中学的实验测量中,
一般来说,可以要求学生估读到测量仪器最小分量的十分
之几.

\subsection{游标卡尺的读数问题}
课本里讲的十分度游标卡尺,游标有10个等分刻度,总长
等于9毫米(图2).这种卡尺的游标读数值(即主尺与游标每
个分度的差值)是$1.0{\rm mm}-\frac{9.0}{10}{\rm mm}=0.1{\rm mm}$,还有一种十
分度游标卡尺的游标总长等于19毫米(图3).这种情况下,
主尺上的两个分度(2毫米)与游标上的一个分度相当,因此它
的游标读数值是$2.0{\rm mm}-\frac{19}{10}{\rm mm}=0.1{\rm mm}$,即也是0.1毫米.

\begin{figure}[htp]\centering
    \begin{minipage}[t]{0.48\textwidth}
    \centering
\begin{tikzpicture}[>=latex, scale=.8]
\foreach \x in {1,2,...,14}
{
    \draw(\x/3, 0)--(\x/3,.5);
}
\foreach \x in {0,5,10}
{
    \draw(\x/3, 0)--(\x/3,.8)node[above]{$\x$};
    \draw(\x/3*.9, 0)--(\x/3*.9,-.8)node[below]{$\x$};
}
\foreach \x in {1,2,...,9}
{
    \draw(\x/3*.9, 0)--(\x/3*.9,-.4);
}
\draw(-.5,0)--(5.25,0);
\node at (2.5,1.8){主尺};
\node at (2.25,-.8){游标};
\node at (5,.8){mm};
    \end{tikzpicture}
    \caption{}
    \end{minipage}
    \begin{minipage}[t]{0.48\textwidth}
    \centering
    \begin{tikzpicture}[>=latex, scale=.8]
\foreach \x in {1,2,...,9}
{
    \draw(\x/2*.95, 0)--(\x/2*.95,-.4);
}
\foreach \x in {0,5,10}
{
    \draw(\x/2*.95, 0)--(\x/2*.95,-.8)node[below]{$\x$};
}
\foreach \x in {1,2,...,19,21,22,23}
{
    \draw(\x/4, 0)--(\x/4,.4);
}
\foreach \x in {0,5,...,20}
{
    \draw(\x/4, 0)--(\x/4,.8)node[above]{$\x$};
}
\draw(-.5,0)--(6.25,0);
\node at (2.8,1.8){主尺};
\node at (3,-.8){游标};
\node at (6,.8){mm};
    \end{tikzpicture}
    \caption{}
    \end{minipage}
    \end{figure}



不难看出,游标卡尺的测量误差为游标读数值的一半.因
为当我们采用游标上的某一刻度读数时,这一刻度与主尺上
相当的刻度的距离就不会超过游标读数值的一半.否则,跟
游标上这一刻度左右相邻的两个刻度中,必有一个跟主尺上
的刻度更为接近.

因此,游标读数值是0.1毫米的游标卡尺,误差是0.05毫
米,即误差出现在毫米的百分位.于是,这种游标卡尺,毫米
的百分位可以估读为“0”,表示误差出现在毫米的百分位,如
果无法判断游标上相邻的两条刻度哪一条跟主尺上的刻度重
合或更接近,则毫米的百分位可估读为“5”.如图4所示,读
作0.55毫米.

\begin{figure}[htp]\centering
    \begin{minipage}[t]{0.48\textwidth}
    \centering
\begin{tikzpicture}[>=latex, scale=.8]
\foreach \x in {1,2,...,14}
{
    \draw(\x/3, 0)--(\x/3,.5);
}
\foreach \x in {0,5,10}
{
    \draw(\x/3, 0)--(\x/3,.8)node[above]{$\x$};
    \draw(\x/3*.9+.1667, 0)--(\x/3*.9+.1667,-.8)node[below]{$\x$};
}
\foreach \x in {1,2,...,9}
{
    \draw(\x/3*.9+.1667, 0)--(\x/3*.9+.1667,-.4);
}
\draw(-.5,0)--(5.25,0);
\node at (2.5,1.8){主尺};
\node at (2.35,-1.8){游标};
\node at (5,.8){mm};
    \end{tikzpicture}
    \caption{}
    \end{minipage}
    \begin{minipage}[t]{0.48\textwidth}
    \centering
    \begin{tikzpicture}[>=latex, scale=.8]
\foreach \x in {1,2,...,25}
{
    \draw(\x/5, 0)--(\x/5,.4);
}     
\foreach \x in {0,1,2}
{
    \draw(\x*2, 0)--(\x*2,.6)node[above]{\x};
}   
\draw(-.5,0)--(5.5,0);
\foreach \x in {1,2,...,19}
{
    \draw(\x*.19,0)--(\x*.19,-.4);
}
\foreach \x in {0,25,...,100}
{
    \draw(\x*.19/5,0)--(\x*.19/5,-.6)node[below]{\x};
}
\node at (5.6,.8){cm};
\node at (2.7,1.8){主尺};
\node at (2.7,-1.8){游标};
    \end{tikzpicture}
    \caption{}
    \end{minipage}
    \end{figure}

二十分度(即游标有20个等分刻度)游标卡尺的游标总
长等于19毫米(图5)或等于39毫米(图6).它们游标读
数值是$1.0{\rm mm}-\frac{19}{20}{\rm mm}=0.05{\rm mm}$或$2.0{\rm mm}-\frac{39}{20}{\rm mm}=
0.05{\rm mm}$,即都是0.05毫米,还有一种十分度游标卡尺,主尺
的最小分度是0.5毫米,游标总长等于4.5毫米(图7).它
的游标读数值也是0.05毫米(注意:这种情况下,主尺上的一
个分度是0.5毫米,跟游标上的一个分度0.45毫米相当).

\begin{figure}[htp]\centering
    \begin{minipage}[t]{0.48\textwidth}
    \centering
\begin{tikzpicture}[>=latex, scale=1]
\draw(-.5,0)--(5,0);
\node at (5,.5){cm};
\foreach \x in {1,2,...,44}
{
    \draw(\x*.1,0)--(\x*.1,.4);
}
\foreach \x in {0,1,...,4}
{
    \draw(\x,0)--(\x,.7)node[above]{\x};
}
\foreach \x in {5,15,...,35}
{
    \draw(\x*.1,0)--(\x*.1,.6);
}
\foreach \x in {1,2,...,19}
{
    \draw(\x*.195,0)--(\x*.195,-.4);
}
\foreach \x in {0,25,...,100}
{
    \draw(\x*.195/5,0)--(\x*.195/5,-.6)node[below]{\x};
}
\node at (2.3,1.5){主尺};
\node at (2.3,-1.5){游标};

    \end{tikzpicture}
    \caption{}
    \end{minipage}
    \begin{minipage}[t]{0.48\textwidth}
    \centering
    \begin{tikzpicture}[>=latex, scale=1]
        \draw(-.5,0)--(3.5,0)node[above]{mm};
\foreach \x in {1,2,...,6}
{
    \draw(\x*.5,0)--(\x*.5,.6)node[above]{\x};
    \draw(\x*.5-.25,0)--(\x*.5-.25,.4);
}
\foreach \x in {10,20,...,50}
{
    \draw(\x*.045,0)--(\x*.045,-.6)node[below]{\x};
    \draw(\x*.045-.225,0)--(\x*.045-.225,-.4);
}
\draw(0,.6)node[above]{0}--(0,-.6)node[below]{0};
\node at (1.3,1.5){主尺};
\node at (1.3,-1.5){游标};
    \end{tikzpicture}
    \caption{}
    \end{minipage}
    \end{figure}

对于游标读数值是0.05毫米的游标卡尺,误差是0.025
毫米,即误差出现在毫米的百分位,而用这种游标卡尺测量
长度时,可以直接读到毫米的百分位,因此无需再估读了.

还有两种游标读数值是0.02毫米的游标卡尺,如图8
和图9所示,它们的误差是0.01毫米,即误差也是出现在
毫米的百分位,用这种游标卡尺测量长度时,也是可以直接
读到毫米的百分位,因此也无需再估读了.

\begin{figure}[htp]\centering
    \begin{minipage}[t]{0.48\textwidth}
    \centering
\begin{tikzpicture}[>=latex, scale=.8]
\draw (-.5,0)--(6,0);
\foreach \x in {.1,.2,...,5.4}
{
    \draw(\x,0)--(\x,.4);
    
}
\foreach \x in {.1,.2,...,5}
{
    \draw(\x*4.9/5,0 )--(\x*4.9/5,-0.4);
}
\foreach \x in {0,1,...,5}
{
    \draw(\x*1,0)--(\x*1,.7)node[above]{$\x$};
}
\foreach \x in {0,1,...,10}
{
    \draw(\x*4.9/10,0)--(\x*4.9/10,-.7)node[below]{$\x$};
}
\node at (2.5,1.6){主尺};
\node at (2.5,-1.6){游标};
\node at (5.8,.6){cm};
    \end{tikzpicture}
    \caption{}
    \end{minipage}
    \begin{minipage}[t]{0.48\textwidth}
    \centering
    \begin{tikzpicture}[>=latex, scale=.8]
\draw (-.5,0)--(6.5,0);
\foreach \x in {1,3,...,29}
{
    \draw(\x*.2,0)--(\x*.2,.3);
    \draw(\x*.2+.2,0)--(\x*.2+.2,.4);
}
\foreach \x in {0,5,...,15}
{
    \draw(\x*.4,0)--(\x*.4,.6)node[above]{$\x$};
}
\foreach \x in {1,2,...,24}
{
    \draw (\x*4.8/25,0)--(\x*4.8/25,-0.3);
}
\foreach \x in {0,10,...,50}
{
    \draw (\x*4.8/50,0)--(\x*4.8/50,-0.5)node[below]{$\x$};
}



\node at (3.5,1.6){主尺};
\node at (3.5,-1.6){游标};
\node at (6.8,.6){mm};
    \end{tikzpicture}
    \caption{}
    \end{minipage}
    \end{figure}


综上所述,用各种常用的游标卡尺测量长度时,误差都出
现在毫米的百分位.由于游标卡尺的误差问题比较复杂,游
标读数值为0.02毫米和0.05毫米的游标卡尺也不需要估读,
因此中学课本里没有讲解游标卡尺的误差问题,也不要求中
学生使用游标卡尺时进行估读.
















\mainmatter

 \chapter{力}\minitoc[n]
\section{教学要求}
这一章讲述力的初步知识,为了减少学生开始学习高中
物理时遇到的困难,降低与初中物理的台阶,本章不讲静力学
的知识,只讲学习动力学所必需的预备知识。


这一章的教学要求是:

\begin{enumerate}
\item 正确理解力的概念,认识力是物体对物体的作用;知
道重力、弹力、摩擦力的产生条件以及它们的大小和方向;
掌握倔强系数,会计算弹簧的弹力;掌握滑动摩擦系数,会计
滑动摩擦力。
\item 进一步认识力是物体间的相互作用,掌提牛顿第三
定律。
\item 
初步学会分析物体的受力情况,会画物体受力图。
\item 理解合力和分力的概念,掌握平行四边形法则,知道
三角形法,会用作图法和公式法求合力和分力。
\end{enumerate}

在重力的教学中讲到了物体静止时拉紧悬绳的力或压在
水平支持物上的力,其大小等于物体所受的重力。在这节很
难从道理上把这一点讲清楚,先要求学生作为事实接受下来,
讲过牛顿第三定律后再来解决,这里所以要提到这一点,是
使学生对重力的大小有个具体认识,但主要还是因为在讲牛
顿第三定律之前就要用到它,如在计算摩擦力的习题中,在
测定滑动摩擦系数的学生实验中,都要用到.

弹力的方向问题比较复杂,在中学阶段经常遇到的弹力
大都是支持力和拉力,因此,弹力的方向是就支持力和拉力这
两种情形来讲解的,并要求学生掌握,以便以后进行力的分析。

在讲述胡克定律之前,为了扩展学生的眼界,先就几种基
本的形变定性说明:形变越大,弹力也越大,对金属丝的扭转
形变,说明扭转角度越大,弹力也越大,是考虑到以后讲卡文
迪许扭秤和库仑扭秤时要用到,这里并不要求详细地讨论。胡
克定律是就弹簧的弹力来讲的,给出了倔强系数的概念,但不
要求全面分析它跟弹簧的哪些因素有关,胡克定律的公式写
成$f=kx$, 没有写成$f=-kx$, 是考虑到这里还没有讲一维矢
量的运算,写成前者初学者容易接受;讲到简谐振动时再考虑
弹力的方向,写成后者。

摩擦力的教学,重点是讲述滑动摩擦力,静摩擦的教学,
只要求学生了解静摩擦和最大静擦力的概念,不讲静摩擦系
数。在讲解滑动摩擦时提到相对滑动,在讲解静摩擦时提到
相对运动趋势,只是为了讲解得确切一些,不要求涉及较为复
杂的情形(例如一个物体在传送带上相对滑动或相对静止)来
展开讲解。

讲述物体受力分析,既是教给学生一种分析方法,也是前
面学过的知识的综合运用。考虑到学会物体受力分析要贯穿
在整个力学教学中,本章只限于分析最基本的事例。讲述分
析方法,要强调明确研究的对象,分析时应强调力是物体对物
体的作用。至于按照什么顺序(如按重力、弹力、摩擦力的顺
序)来分析力,不宜过分强调,强调得过分,甚至要求学生死
记住一个分析的顺序,对于学生灵活运用知识,学会分析方
法,都是没有好处的。物体的受力情况实际上往往是很复杂
的,根据具体的问题,可以略去某些次要因素。这种研究问题
的方法,应该作为一项要求向学生提出,在以后作力的分析时
也要注意这一点。

力的合成和分解的教学,主要是使学生掌握力的平行四
边形法则。三角形法在实际中常常用到,学生应当知道。但
应该使学生明确,三角形法并不是另外一种新方法,只是平行
四边形法的简化。


\section{教学建议}
这一章内容,建议在教学中分为三个单元:

第一单元(第1节——第7节)讲述力的概念和力学中常见
的三种力,并在此基础上讲述牛顿第三定律和物体受力情况
分析。

第二单元(第8节——第10节)讲述力的合成和分解.

第三单元(第11节——第12节)在前面讲过的力的矢量性
的基础上,讲述矢量的初步知识。

\subsection{第一单元}
这一单元,首先通过重力、弹力、摩擦力这三种常见的力
和牛顿第三定律的教学,使学生比较具体地认识力的概念和
性质,包括力是物体之间的相互作用,力有大小、方向和作用
点等,然后引导学生初步掌握物体受力情况分析的方法,为进
一步学习力学知识打下基础。

\subsubsection{力的概念和图示}

这部分内容大部分在初中已经学
过,是复习性的。但是有一些学生,虽然能记住学过的知识,
并不真正理解、会用,所以对这部分内容的教学仍然要给予
足够的重视。

力的概念的教学,最主要的是通过演示和说明使学生真
正理解力是物体对物体的作用,只要有力就一定有施力物体
和受力物体,力不能离开物体而存在。这是力学中最基本的
事实。可以通过一些实际的例子让学生指出施力物体和受力
物体,学生牢牢地掌握这一点,就不会离开物体的作用凭空设
想出多余的力来,有助于学好后面的物体受力分析。

关于力的图示,要强调表示力的每条有向线段,都要根据
选择好的标度,按照一定的比例来画。有的同学,往往用不同
的标度来画同一物体所受的不同的力,这种错误应该及时给
予纠正,一般情况下,力的作用点都可画在物体的质心上。由
于教材没有介绍质心这个概念,在实际作图时,只要在表示物
体的图形中间选择一个适当的点来表示力的作用点就可以
了,但是也有个别情况,例如第一次介绍摩擦力的方向时(课
本图1.12),为了清楚起见,把摩擦力$f$画在接触面上,而且
还把相互接触的两个物体画得离开一些。以后还应逐步使学
生知道,在分析力学问题时,有时只须画出力的示意图。力的
示意图常常是为了使物体的受力分析更清楚而作的,它在力
的大小、标度上的要求,不象力的图示中要求的那么严格,但
是对力的方向、力的相对大小也不能画错。在另一些情况下,
例如,在物体的受力分析之后,需作它的受力图并用图示法来
解题,这时的受力图又必须按力的图示的要求来作图了。但
是在本节的教学中,只须按课本上的要求进行就可以了。

\subsubsection{重力} 关于重力的概念,按教材上所讲的“由于地球
的吸引而使物体受到的力叫做重力”来进行教学是适宜的,它
既浅显易懂,也没有把引力与重力等同起来。这里也不宜过
早的把地球对物体的引力与重力相区别,以后在教学中会逐
步讲清楚的。

应该注意的是,教材上把重量定义为物体所受的重力,按
照国务院颁布的法定计量单位,重量是质量的同义词。这种
差别的产生是因为教材是在法定计量单位颁布前编写的。这
一点,在教学中应该向同学们讲清楚。

物体所受重力的大小,可以用静力学的方法来确定,这就
是教材上讲的物体静止时拉紧悬绳的力或压在水平支持物上
的力,它们的大小都跟物体所受的重力相等。应该让学生注
意的是,重力是地球作用在物体上的力,受力者是物体,而物
体对悬绳的拉力或对支持物的压力,受力者是悬绳或支持物,
它们跟重力是作用在不同物体上的力,不能把它们跟重力相
混淆。

重力这一节的教学,也是具体地通过这种常见的力来表
明:力是物体对物体的作用,力有大小、方向和作用点,从而使
学生逐步加深对力这个抽象概念的理解。重心的概念,也是
从重力的概念引伸出来的,因为课本不讲同向平行力的合成,
不要追求概念的严谨,补充重心的定义,通过课本中图1.2
的演示,很容易说明“一个物体的各部分都要受到地球对它的
作用力,我们可以认为重力的作用集中于一点,这一点叫做物
体的重心。”

\subsubsection{形变与弹力}
这一内容的教学,应着重于有关形变与
弹力的实验演示(包括显示微小形变的实验演示),使学生从
直观上来理解和接受,而不宜增加关于形变产生弹力的微观
解释。课本中说明弹力与重力不同,弹力只有在物体直接接
触并产生形变时才能产生。而实际上,有时微小形变又不易
察觉,这样从表观上就不易直接判别出相互接触的物体之间
究竞是否有形变与弹力产生。原则上这里只须提醒同学注意
这个问题,至于进一步具体判断的方法,应在以后学习物体的
平衡等节内容时再作讨论。有关弹力方向的问题,宜按课本
中的提法,即第13页和第14页两段有波纹线的文字,说明支
持力和拉力的方向,由于还存在扭转、切变等种种形变,教学
中不宜笼统地表述为“弹力的方向总是指向……”的形式。

\subsubsection{摩擦力} 摩擦力一节的教学重点是滑动摩擦力,教学
时可在初中学过的知识基础上,通过演示得出关系式$f=\mu N$,
引出滑动摩擦系数的概念,对于这个公式,有的同学往往误
认为压力$N$的大小总是跟滑动物体所受的重力相等,教师
应该让他们知道,压力$N$是跟两个物体的接触面垂直的。只
有物体在水平拉力作用下沿水平面滑动时,压力$N$的大小才
跟物体所受的重力相等,在其他情况下,例如物体沿斜面下滑
时,压力$N$并不等于物体所受的重力。

摩擦力不是教学的重点,在判断静摩擦力的方向时,
同学们对相对运动趋势常常感到比较抽象。在不讨论静摩擦
力作为动力的情况下(如传送带上的物体等),可引导同学这样
来认识:按照已经给定的力来看,物体本是要运动的,但实际
上物体却处于静止状态,那阻碍物体运动的力便是静摩擦力。
这样静摩擦力的方向也就随之而明确了。

\subsubsection{作用力和反作用力}
牛顿第三定律是一个基本定律,
是本章的重点,讲好重点知识,应该引导学生抓住定律的主
要之点。对初学者来说,牛顿第三定律的主要之点就是作用
力和反作用力分别作用在相互作用的两个物体上,教材正是
抓住了这一点,通过实例和演示,反复加以说明,而没有侧重
于作用力和反作用力是同种性质的力,它们同时产生、同时消
失。在教学中应该注意这个问题,不能把主要之点当作自明
之理一带而过,把力量耗费在讲述一些次要问题上。讲解牛
顿第三定律,也是在进一步扩展学生对力的概念的认识,明确
力是物体对物体的相互作用。关于作用力和反作用力跟平衡
力之间的区别,学生常常理论上知道,实际上还会混淆。教学
中要通过一些实例,引导学生搞清两者的区别。例如,可以分
析放在桌面上的静止物体,找出它所受的一对平衡力,以及物
体所受的重力和它对桌面的压力的反作用力,还可以分析用
悬绳挂在天花板下的物体,找出作用在物体和悬线上的平衡
力,以及地球和物体,物体和悬绳、悬绳和天花板间的作用力
和反作用力。当然这个问题也不是一堂课所能解决的,下一
节物体受力分析,还要讨论这类问题。

\subsubsection{物体受力情况分析}  这一节内容,是在以前各节预备
知识的基础上提出的,是前面各节有关知识的应用,这一节内
容的安排也是循序渐近的:从静止物体到运动物体,从平面上
的物体到斜面上的物体,从具体实例上升到分析物体受力情
况的一般方法——隔离法。

在教学上要注意:
\begin{enumerate}
    \item 引导学生正确地搞清楚研究对象,
施力物体与受力物体;
\item 要找到分析对象受到的所有的力,
不能遗漏,但也不能“无中生有”,不能“张冠李戴”;
\item 不能只
讲一般原则和注意事项,不能仅靠课堂上受力分析的示范,还
要行适量的实例练习,画出受力图,及时发现问题,及时引
导同学自觉纠正错误,逐步掌握正确的受力分析方法。
\end{enumerate}

物体受力情况是各种各样的,因此不可能在这一节教学
中要求学生完全掌握,要有一个过程。循序渐进在这里特别
重要,切不可一次就补充很多复杂的题目让同学分析。这样
反而会使同学无所适从,甚至产生畏难情绪。对隔离法的教
学要求尤宜如此,所以课本中没有提出连接体之类的繁难问
题,而把重点放在引导学生理解和掌握受力分析的方法和思
路上。如果学生能掌握正确的方法和思路,则他们自己也会
逐渐独立地解析各种力学题目。

在受力分析时,常常忽略某些次要因素,这也是使所研究
的问题理想化,应该在教学中引导同学重视并逐渐熟悉这种
方法。例如物体下落时,相对于重力来说,可以忽略空气阻力
这一次要因素,通过理想化才构成了自由落体运动这一模型。
以后的教学中还要不止一次地运用这种方法,使本来很复杂
的问题,能够较容易地入手研究。当然这些话并不是都要在
这节课中向学生一一说明的。

\subsection{第二单元}
这一单元讲力的合成和分解,主要使学生掌握力的平行
四边形法则。这个法则是进行力学计算的基本规律,是本章
的叉一重点,由于学生初次接触这种运算,很不习惯,因此也
是难点。教学中应该通过演示,实验和力的图示,使学生理解
和掌握这一规律。

\subsubsection{力的合成}
在讲合力和力的合成的概念时,首先要从
生活中的事例出发,让学生理解儿个力共同作用的效果可以
跟另外一个力单独作用的效果相同,例如,一件行李,可以由
两个人共同提,也可以由一个人提;吊起一个重物,可以用两
根悬绳,也可以只用一根悬绳;一辆车子,可以由几个人推着
它匀速前进,也可以由一个人推着它匀速前进,等等,然后再
引入合力和力的合成的概念,这可以使抽象的概念具体化,便
于学生理解.力的平行四边形法则的教学,做好课本中图1.
22的演示实验是个关键。可以把实验装置装在竖立的小黑
板上,边讲边画出力的图示,还要事先设计好几组不同的数
据,不要只由一次实验的结果就总结出规律来。在处理力的
三角形法时,可以只把它当作代替力的平行四边形法则的简
单的作图法,介绍三角形法时,要学生搞清楚“首尾相接”的
意思,以免把合力的方向搞错,有的同学对于合力跟分力之
间的关系认识不清,学了力的合成以后,往往认为物体在受
到几个力作用的同时,还要受到它们的合力的作用。这种错
误认识应该纠正,要使学生认识,合力与分力之间是等效“代
替”的关系,而不是合力跟分力同时作用在物体上。

\subsubsection{力的分解}   在讲力的分解时,也要通过实例和演示,
使学生体会:一个力往往可以产生几个效果。例如,在同一悬
点上,用两根悬绳吊起一个物体,物体对悬点的竖直向下的
拉力,产生了同时拉紧两根悬绳的效果,由此可以引出分力
和力的分解的概念。

在讲述怎样分解一个力时,教材是通过两个实际的例子,
得出一个力可以根据它产生的效果进行分解的结论,在这里,
教材上的提法是谨慎的、留有余地的。因为在许多情况下,力
的分解不是根据它产生的实际效果,而是要按照研究问题的
方便来进行的,大家熟知的力的正交分解法就是一个很好的
例子。

\subsection{第三单元}
这个单元是在前两个单元的知识基础上,提出矢量的概
念,并学习同一直线上的矢量的运算方法,为以后直线运动中
位移和速度的合成提供了依据。

\subsubsection{矢量与标量} 
这节教材的教学,主要在于讲清楚矢量
与标量的不同含义和不同运算法则。这里从力的合成要按照
平行四边形法则来进行,外推到平行四边形法则也是矢量合
成(矢量加法运算)的普遍法则就够了,无须再举例展开,矢
量与标量在运算规则上的不同,教学中也只举加法运算为例,
至于矢量的减法和乘、除运算就更不须提出来进行比较了。

\subsubsection{一维矢量运算}  在同一直线上的矢量的运算,是矢量
运算中最简单的情况,在讲过一般的矢量加法运算之后,在这
里作较为详细讲述,是因为以后讲直线运动时要用到这一知
识,要着重讲清:先要沿着矢量所在的直线选定一个正方向,
规定凡是方向跟正方向相同的矢量都取正值,凡是方向跟正
方向相反的矢量都取负值。这样,就可以用一个带有正负号的
数值把矢量的大小和方向都表示出来,从而把同一直线上的
矢量运算简化为代数运算,为下一章直线运动中位移和速度
的运算提供了很大的方便。教学中应该引导学生重视这一节
内容并仔细阅读认真领会课文,要弄清楚课本上说的“可以用
一个带有正负号的数值把矢量的大小和方向都表示出来”的
含义,是把矢量的大小和方向分开来表示的.例如$F=-6$牛,
力的大小用数值和单位(6牛)表示,而力的方向则按跟预先
规定的正方向相同或相反(取正号或负号)来表示,掌提了
这个基本点,一维矢量的运算就容易弄清楚了。

\section{实验指导}
\subsection{演示实验}
\subsubsection{重心的实验测定方法}
在演示课本图1.4所介绍的用悬挂法测物体的重心位置
时,应使学生明确:
\begin{enumerate}
\item 利用这种方法测重心只能用于薄板状(即厚度极小
可以忽略)的物体,任何有厚度的物体的重心不会在物体的某
一个表面上。
\item 重心可以不在物体上,如用悬挂法可以测出一个薄
板状的塑料衣架的重心$C$并不在衣架上(图1.1)。一个薄板
状的圆环的重心一定在环心上。
\end{enumerate}

\begin{figure}[htp]
    \centering
\includegraphics[scale=.8]{fig/1-1.png}
    \caption{}
\end{figure}


\subsubsection{物体形变时产生的弹力}
在做课本图1.5的实验时,为了突出所观察的是一端固
定的弹簧被拉长和被压缩时所产生的弹力对小车的作用,小
车上不必加放砝码,以免分散学生的注意力。如果为了使效
果明显可用倔强系数较小的弹簧。

课本图1.6所示的现象,要用较大的、一侧透明的水槽,
圆木的质量又必须足够大,一般不容易演示,如果用投影仪,
则不易看清细木棍的弯曲形变的恢复过程,因此建议改用细
木条(用制作模型飞机的木条$1\x2\x300 {\rm mm}^3$)把一辆原来静
止的小车推开的现象来演示。

演示课本图1.7的现象时,可用洗澡用的或做沙发床垫
用的塑料,观察发生形变和形变的恢复,效果较好。

\subsubsection{显示微小形变的实验装置}
可按课本图1.10的装置进行演示,为了使效果明显,应
使两块平面镜$M$和$N$间的距离相隔得尽可能远些,屏幕可
利用教室的墙壁,以便离开平面镜$N$的距离更大些。如果用
激光作为光源,在教室内就可演示。如果用白炽灯制成的平
行光源,则需在作为屏幕的墙的一边,用黑窗帘(或黑纸)把窗
户局部遮光。

这一装置是利用光在均匀媒质中的直线传播和平面镜组
对于光线的二次反射原理制成的,在入射光线的方向不改变
的情况下,如果平面镜转过$\theta$角,则反射光线将偏转$2\theta$角(图
1.2),当在桌面上施加一个压力,桌面发生微小的弯曲形变
时,两块原来平行放置的平面镜就不再平行了,$M$将向左侧
偏斜一个微小角度$\theta$, $N$将向右侧偏斜一个微小角度.由
于$M$的偏斜,使得反射角变小了$2\theta$. 对平面镜$N$来说,入射
光线的方向已发生了变化,又由于$N$的偏斜,使得入射角比原
来减小了$2\theta- \theta'$. 于是从$N$反射的光线的偏离程度就更大。
即反射角将变小$2(2\theta+\theta')$. 再加上由于屏幕离$N$的距离$r$
较远,即使$\theta$和$\theta'$都十分微小,光点在屏幕上的移动距离$\Delta s\approx r\x 2(2\theta+\theta')$仍是十分明显的,这种利用光线的二次反射
把微小效应放大的作用,常称为“光杠杆的放大作用”。
\begin{figure}[htp]\centering
    \begin{minipage}[t]{0.48\textwidth}
    \centering
\includegraphics[scale=.7]{fig/1-2.png}
    \caption{}
    \end{minipage}
    \begin{minipage}[t]{0.48\textwidth}
    \centering
    \includegraphics[scale=.7]{fig/1-3.png}
    \caption{}
    \end{minipage}
    \end{figure}

\subsubsection{扭转形变}
课本图1.11所示的扭转形变,可用直径为20—25毫米
的橡皮管进行演示,沿着管轴方向平行地用颜色漆划几条线,
在橡皮管的两端各塞上一小段圆木,把管子的一端用试管夹
固定起来,在另一端用一根粗铁丝沿塞子的直径方向穿过(图
1.3)。这样,当在铁丝上加一力偶使它转过一个角度时,就可
明显地显示出扭转形变。


\subsubsection{静摩擦和滑动摩擦}
如图1.4所示,先用水平仪把长木板调节水平,并调节滑
轮$P$或调节固定在木块$B$上的钩子$C$的位置,使得细绳被拉
直时与长木板表面平行。在砝码盘$A$中加一小砝码,木块$B$
仍能保持静止。通过对木块$B$进行受力情况的分析,说明静
摩擦力的存在,然后在砝码盘$A$中逐渐增加砝码,直到$B$开始
运动。这说明在木块开始运动之前所受到的静摩擦力是随着
细绳拉力的增大而增大的,但静摩擦力的增大有一限度,即存
在一个最大静摩擦力。
\begin{figure}[htp]
    \centering
    \includegraphics[scale=.7]{fig/1-4.png}
    \caption{}
\end{figure}

适当调整盘中砝码的数量,使得木块$B$能沿着长木板做
匀速直线运动(不要求很精确,只要大致是做匀速运动就可以
了),通过对木块$B$进行受力情况的分析,可知木块所受到的
滑动摩擦力的大小等于细绳的拉力。通过对砝码盘$A$(连同
其中的砝码)的受力分析,可知细绳的拉力等于盘和盘中所加
砝码所受重的和,从而可知这时木块所受到的滑动摩擦力
的大小就等于盘$A$和盘中砝码所受重力的和。

在木块$B$上加一大砝码(100克或200克),必须重新调
整盘中砝码的数量,才能使$B$做匀速直线运动,测出增大了
压力后滑动摩擦力的数值,由此可说明滑动摩擦力的大小是
跟接触面间的压力大小有关。

在木块$B$的另一个面上事先贴好一张比较粗糙的纸,再
用这个粗糙面跟木板接触,重做实验,发现必须重新调整盘中
砝码的数量,才能使$B$做匀速直线运动。这说明在压力不变
的情况下,滑动摩擦力的大小还跟接触面的材料性质和粗糙
程度有关。

\subsubsection{作用力和反作用力}
课本图1.13的实验在课堂上不能做,可以要求学生
在划船活动时有意识地进行。在课堂教学中可以用两辆小平
板车,请两位学生蹲在小车上,面对面地用手互相推动,来观
察两辆小车互相被推开,向相反方向运动的情况,演示时可
让甲同学将手掌对着乙同学,由乙同学推甲同学的手;再让乙
同学将手掌伸向甲同学,由甲同学施力于乙同学;最后再由
甲、乙两同学同时通过手掌施力于对方。可以观察到这三种
情况下,小车的运动情况是相同的。

课本图1.14的现象要用投影仪来观察,所用的小磁
铁和铁块的质量要相近,但不要太大,软木塞要用大些的,不
然由于重心较高,在软木塞运动过程中磁铁(或铁块)可能会
翻落下去,也可以将磁铁和铁块分别用细线缚在软木塞上,并
且都把它们翻过来放置,让磁铁和铁块在水中相互吸引,这在
投影时的效果是相同的。

为了便于直接观察,也可以用一块较大的条形磁铁和一
块质量相近的铁块,分别在它们的下面垫放几支粉笔(如果粉
笔是圆台形的,刚应将粉笔粗的一端和细的一端交错放置)以
减少摩擦,直接在讲台上演示,效果很明显。演示时可先将磁
铁用手按住,使铁块接近到某一适当距离,可观察到铁块被吸
引过来;再将铁块按住,使磁铁接近到某一适当距离,可观察
到磁铁被吸引过来;然后再将两手控制住磁铁和铁块,调节它
们之间的距离,同时将手放开,可观察到磁铁和铁块相互吸
引,彼此靠近直到吸在一起,这一演示说明了力是物体与物
体间的相互作用,任何施力物体也必然是受力物体。

课本图1.15的实验定量地说明了作用力和反作用
力的关系,是讲解牛顿第三定律的实验基础。在进行这个实验
之前,事先应做好准备,试一下这两个弹簧秤的钩子勾在一起
沿水平方向拉动时,两个弹簧秤的示数是否总是相等。'为了
增加可见度,可在弹簧秤的指示器上贴两块小纸。

将一玩具汽车(电动的或弹簧发条的)放在一块较大
的泡沫塑料平板上,平板下面垫放若干玻璃试管以减小摩擦。
当汽车开动时,可观察到塑料
平板向相反方向运动(图1.5),
说明物体间的摩擦力也是相互作用的。
\begin{figure}[htp]
    \centering
    \includegraphics[scale=.8]{fig/1-5.png}
    \caption{}
\end{figure}

\subsubsection{力的合成的平行四边形法则}
课本图1.22的实验,为了便于计算,可使$F_1$等于三个等
重钩码的重量,使$F_2$等于四个等重钩码的重量,并使当拉绳
的结点到达位置$O$时,两根细绳的夹角恰为$90^{\circ}$, 于是可认为
这两个共点力$F_1$和$F_2$的方向夹角为$90^{\circ}$, 在这两个力的共同
作用下,使橡皮条发生了长度为$EO$的伸长形变。

当演示用一个定滑轮和一组钩码,通过细绳拉橡皮条使
之发生相同的伸长形变时,可以逐次增加钩码,一直加到五个
钩码,可以看到结点恰好到达位置$O$, 加到第六个钩码时结点
已超过位置$O$, 或者可以先加七个钩码,观察结点已远远超过
位置$O$, 然后逐步减少钩码数,直到还剩五个钩码时,观察结
点恰到达位置$O$. 于是这五个钩码所引起的绳子拉力就等于
合力的大小。

\subsubsection{合力的大小跟分力夹角的关系}
课本36页最后有一段文字叙述:“当$F_1$和$F_2$的夹角$\theta$在
$0^{\circ}$到$180^{\circ}$之间时,$\theta$越大,$\cos\theta$的值就越小,合力就越小,而且
合力的方向也随着夹角$\theta$的变化而变化.”关于这一段叙述,
除了从合力的计算式$F=\sqrt{F^2_1+F^2_2+2F_1F_2\cos\theta}$来加以理解
外,还可以用如下的模拟演示来获得感性认识。
\begin{figure}[htp]
    \centering
    \includegraphics{fig/1-6.png}
    \caption{}
\end{figure}


先让学生参阅课本图1.23丙用三角形法来表示共点力
$F_1$和$F_2$的合力$F$, 然后将模拟演示仪介绍给学生,这是用五
合板(或2—2.5毫米厚的有机玻璃)制成的两根长度不等的
箭头,分别用来代表分力$F_1$和$F_2$, 并设$F_1>F_2$. 根据三角形
法,将$F_1$的始端和$F_2$的末端用铆钉铆
合在一起(不要铆死,使它能转动),在$F_1$
的末端和$F_2$的始端间固定一根橡皮条(或宽紧带)用来表示合力(图1.6)。橡皮条的原长应为$F_1-F_2$. 演示时可从$F_1$和$F_2$的夹角$\theta=0^{\circ}$开始,逐渐增大,直到
$\theta=180^{\circ}$. 在这过程中可以看到橡皮条的长度(表示合力$F$的
大小)从$F=F_1+F_2$逐渐减小到$F=F_1-F_2$, 如图1.7所示.
\begin{figure}[htp]
    \centering
    \includegraphics[scale=.8]{fig/1-7.png}
    \caption{}
\end{figure}
因此在两个共点力大小不变的情况下,它们的合力大小是随
着分力夹角的增大而变小的。

\subsubsection{力的分解}
力的分解的演示实际上都是利用二力平衡的原理来显示
的。

(1)课本图1.29,放在斜面上的物体的重力分解。
在一个倾角约为$15^{\circ}$的光滑斜面上,放一辆质量约为200
克的金属小车,小车的一端用细绳跟固定在斜面顶端的弹簧
秤$A$相连(图1.8)。用手托住弹簧秤外壳使它跟斜面平行,观
察这时弹簧秤的示数。将斜面倾角逐渐增大,可以观察到弹簧
秤示数也逐渐增大,当斜面倾角到达$30^{\circ}$时,可读得弹簧秤示
数约为100克力。这说明放在斜面上的物体的重力所产生的一
个效果是使物体沿着斜面下滑,而且斜面倾角越大,使物体沿
斜面下滑的重力分力$F_1$也越大.
\begin{figure}[htp]
    \centering
    \includegraphics[scale=.8]{fig/1-8.png}
    \caption{}
\end{figure}

如图1.9所示,再用一个
铁架台,在复夹上固定另一个
弹簧秤$B$, 通过细绳钩住小车
的中部,在不改变斜面倾角、使
弹簧秤$A$的示数不变的条件
下,调节铁架台的位置和复夹
的高度,使弹簧秤$B$的拉绳方向跟斜面垂直,且使拉绳逐渐绷
紧,直到使小车不对斜面施加压力(即小车刚脱离斜面)。这时
从弹簧秤$B$的示数就可以知道放在斜面上的物体的又一个效
果是使物体紧压在斜面上,可读得斜面倾角为$30^{\circ}$时弹簧秤
$B$的示数约为170克力.
\begin{figure}[htp]
    \centering
    \includegraphics[scale=.8]{fig/1-9.png}
    \caption{}
\end{figure}

如果在这基础上,使斜面倾角继续增大,则可观察到弹簧
秤$A$的示数逐渐增大,弹簧秤$B$的示数逐渐减小。在这过程
中,需要不断调节铁架台的位置和复夹的高度,使得弹簧秤$B$
的拉绳方向始终保持跟斜面垂直。当使斜面的倾角继续增大
到$90^{\circ}$时,可观察到弹簧秤$A$的示数增大到等于小车重量,而
弹簧秤$B$的示数逐渐减小到等于零。当使斜面的倾角逐渐减
小到$0^{\circ}$时,弹簧秤$A$的示数也减小到零,而弹簧秤$B$的示数
将增大到等于小车的重量。

从这个实验可以说明,重力$G$的平行于斜面方向的分力
$F_1=G\sin\theta$, 它的作用效果是使物体沿斜面下滑;重力$G$的
垂直于斜面方向的分力$F_2=G\cos\theta$,它的作用效果是使物体
紧压在斜面上。

(2)课本图1.30, 直角支架的端点所受向下作用力的
分解。

如图1.10所示,直角支架的横梁$OM$是一根装有压缩
弹簧测力计的金属管,$M$端有一套环,可以套在通过复夹安装
在铁架台上的固定转动轴上;斜梁$ON$可用细绳连接弹簧秤
组成,弹簧秤的圆环钩在铁架台上侧的复夹$N$处,细绳的下
端固定在横梁端点$O$处.当在$O$点悬挂重物$G$时,可观察到
细绳被拉紧、横梁被压缩,从弹簧秤和测力计分别可以读出
拉力和压力的数值。这个实验说明了支架$O$点受到一个向下
的作用力$F$时,它对支架的两个梁产生的效果是不同的。
\begin{figure}[htp]
    \centering
    \includegraphics[scale=.8]{fig/1-10.png}
    \caption{}
\end{figure}

演示时,应先调节细绳的长度使得横梁$OM$稍向上倾
斜.这样,当在$O$点挂上重物时,恰好使横梁保持水平,才能
应用直角三角形关系来表示$F$的两个分力$F_1$和$F_2$的大小,
即$F_1=\frac{F}{\cos\theta}$
(它的作用效果使斜梁$ON$受到拉力),$F_2=F\tan\theta$
(它的作用效果使横梁$OM$受到压力)。
\begin{figure}[htp]
    \centering
    \includegraphics[scale=.8]{fig/1-11.png}
    \caption{}
\end{figure}

如果没有压缩弹簧测方计,则可以用另一根细绳在相反
方何通过弹簧秤$C$来拉$O$端,使得$OM$恰巧不变压力,用这
样的方法来进行间接测量(图1.11)。具体做法如下:

用一小方木块$D$, 在它的上部、下部和右侧各装一小钩并
都系上一段细线。上部的细线和弹簧秤$A$相连,下部的细线
悬挂重物,右侧的细线可以系在弹簧秤$C$上。横梁$OM$可用
一根木棒来做,横梁左端装一套环,右端锯成齿状,小木块$D$
的左侧也锯成齿状以增大摩擦,演示时可用手捏住小木块$D$
和横梁的右端不使松动,挂上重物后,由干摩擦,小木块不会
滑动,这时可观察斜梁$ON$受拉力的情况(细线被拉紧),从弹
簧秤$A$可以读出拉力的大小;至于横梁$OM$受压力的情况,
则可间接地用弹簧秤$C$, 通过小木块$D$右侧钩上的细绳,保
持水平地慢慢地拉动,直到横梁不受压力由于自身重力作用,
将会离开小木块$D$而转动,读得这时弹簧秤$C$的示数即为横
梁所受的压力.这个演示同样可以证明当没有用弹簧秤$O$拉
小木块$D$时,横梁$OM$确是受到压力作用的。

\subsection{学生实验}
\subsubsection{测量滑动摩擦系数}

这个实验的目的是定量地测定两种木料之间的滑动
摩擦系数。如何处理实验数据,不宜作过细的规定,可以让学
生复习第一个实验(练习分析处理实验数据),由学生自行设
计表格,进行数据分析和处理。

实验时必须首先利用水平仪把长木板调成水平,而
且要调节滑轮的高度(或调节小木块上钩往细绳的钩子的
位置),使得拉小木块的细绳和长木板保持平行。在这样的情
况下,当小木块沿长木板作匀速运动时,它所受到的滑动摩擦
力$f$的大小,才可以用挂在绳端的砝码盘和盘中砝码受到的
重力之和来量度。

用弹簧秤称出小木块的重量,就求出了压力$N$. 要改变
压力的大小,只要往小木块上加放砝码就可以了。

测得在一系列不同压力时的滑动摩擦力的大小后,
如果利用画图象的方法(不一定用这个方法)来找出它们之间
的定量关系,则可启发学生思考:
\begin{enumerate}
\item 这个实验中的压力和滑动摩擦力这两个物理量,哪
个是自变量,哪一个是因变量?
\item 画图象时平面直角坐标的横坐标应表示哪一个物理
量,纵坐标应表示哪一个物理量?
\item 画出图象后,应怎样得出滑动摩擦系数?
\item 滑动摩擦系数是否有单位?
\end{enumerate}

实验后还可思考以下问题:
\begin{enumerate}
    \item 滑动摩擦力的大小跟压力有怎样的关系?
    \item 滑动摩擦系数的大小是由什么因素决定的?是否可以
    说滑动摩擦系数跟摩擦力成正比,跟压力成反比?压力发生
    了变化,滑动摩擦系数是否会有相应的变化?
    \item 我国东北农村的一些地区,在冬季里人们乘坐的爬
    犁,用几匹狗就可以拉着在雪地里奔驰,如果到了夏季把爬犁
    放在泥地上拉动就很困难,这是为什么?
\end{enumerate}

\subsubsection{互成角度的两个力的合成}

这是一个验证性的实验,在已知两个互成角度的共
点力的情况下,先用力的平行四边形法则求出它们的合力,然
后再用弹簧秤来进行验证。

在实验前可以先把所用的两个弹簧秤的钩子互相钩
住,平放在桌上向相反方向拉动,看看它们的示数在拉力的大
小改变时是否总是相同的。如果是相同的,在实验时就可以
同时使用这两个弹簧秤进行读数。如果不相同,则可以只用
其中一个弹簧秤进行读数(每一个分力或两个分力的合力,都
用这同一个弹簧秤来读数)。

用两个弹簧秤同时拉动一端固定的橡皮条时(要注
意不使伸长量太大,以免用一个弹簧秤测合力时由于超出它
的测量范围而无法读数),要使两个弹簧秤贴着方木板平拉,
弹簧秤互成的角度可以是任意的。但为了便于作图,可以使
两个弹簧秤拉着的细绳沿着一块三角板的任一特殊角(如
$30^{\circ}$、$45^{\circ}$或$60^{\circ}$等)拉动,使橡皮条的另一端伸长到某一位置
$O$. 用细铅笔描下两条细绳的方向,记下两条细绳端点的位
置$O$以及两个弹簧秤的读数$F_1$和$F_2$, 按相同比例(所选定
的标度要在记录纸上明确画出)作出这两个互成角度的共点
力$F_1$和$F_2$的图示,它们的共同作用点也就是绳子的结点位
置$O$.

作平行四边形求出$F_1$和$F_2$的合力$F$时,要求学生
认真地用三角板来作平行线,然后用同一标度量出合力$F$的
大小。

只用一个弹簧秤来拉橡皮条时,可启发学生思考为
什么要使细绳结点拉到同一位置$O$, 这时弹簧秤的读数$F'$和
用两个弹簧秤拉橡皮条时的合力$F$应该有怎样的关系?

用细铅笔描下这时细绳的方向,记下$F'$, 并按相同
的标度在同一力图上画出$F'$的图示。

对有兴趣的学生可以让他们按课本
342页最后一段的叙述进行实验,此外还可
指导他们对刚才实验中的误差进行分析。

比较$F'$和$F$就可以知道误差的大小,如图1.12所示,线段$CC'$就是合力$F'$和$F$之间的误差,线段$CC'$越长,表示误差越大。

\begin{figure}[htp]
    \centering
    \includegraphics[scale=.8]{fig/1-12.png}
    \caption{}
\end{figure}

\subsection{课外实验活动}
\subsubsection{测量尼龙丝的抗断拉力}
这个课外实验是利用最简单的仪器,根据学生所学过的
力的合成和分解的知识,来测量尼龙丝的抗断拉力。

要使学生知道什么是抗断拉力.一段材料发生拉伸
形变对物体产生的弹力是随形变的增大而增大,但是任何材
料对物体产生的弹力都有一个最大值,当材料将被拉断时对
物体产生的最大弹力就是材料的抗断拉力。各种不同材料的
抗断拉力是不同的。

{实验的原理} 

从课本362页图 10.22可知,在尼龙丝的中点悬挂了质
量为1千克的重物,使中点$O$受到向下的力$F=C$, 根据力的
平行四边形法则可知,力$F$跟$F$沿着被拉紧的尼龙丝方向的
分力$F_1$间存在这样的关系:$F=2F_1\cos\frac{\alpha}{2}$(图1.13), 因此将
绳子的另一端沿箭头$\alpha$所示的方向,保持水平地缓慢拉动,随
着。角的逐渐增大,分力$F_1$也逐渐增大,用量角器量出尼龙
丝刚被拉断时的$\alpha$角,就可计算出分力$F$, 而这时的大小
也就等于尼龙丝对$O$点的弹力,于是便可测得尼龙丝的抗断
拉力。

\begin{figure}[htp]
    \centering
    \includegraphics[scale=.8]{fig/1-13.png}
    \caption{}
\end{figure}

实验时的注意事项
\begin{enumerate}
\item 尼龙丝要用很细的一种,不然不易拉断.
\item 将重物悬挂在尼龙丝的中点时,可以通过 一个拉窗
帘用的小环来悬挂,即不要把悬挂重物的细绳在尼龙丝的中
点$O$处打结(打结后就变成两段尼龙丝了)。这样,即使将尼
龙丝的另一端拉动时不在水平方向,由于小环可以滑动,使得
力的合成的平行四边形始终是一个菱形,尼龙丝中的拉力和
向下力的关系仍可满足上述关系式
\item 在拉尼龙丝时可在手指上缠一些布条以免拉紧尼龙丝时会把手指捋破。另一只手可拿着量角器,并使量角器的
圆心始终位于尼龙丝的中点,而且量角器的零刻度始终跟右
侧的尼龙丝重合,如图1.14所示.这样,当尼龙丝被拉断时
就可以方便地读出$\alpha$角的大小。


\begin{figure}[htp]
    \centering
    \includegraphics[scale=.8]{fig/1-14.png}
    \caption{}
\end{figure}


\item 参考数据:如果尼龙丝的直径为0.125毫米,在它的
中点所挂重物的重量为1千克力,当$\alpha=130^{\circ}$时,尼龙丝将被
拉断,则测得抗断拉力
\[F_1=\frac{F}{2\cos\frac{\alpha}{2}}=\frac{1}{2\x \cos65^{\circ}}=1.18\text{千克力}\]

如果找不到尼龙丝,也可以用缝纫机上使用的细线进行
测量。
\end{enumerate}

在实验室中测量尼龙丝的抗断拉力,所用的最方便而
且直接的方法是:把尼龙丝的一端固定,另一端套在一个大量
程的测力计的钩环上,然后将尼龙丝水平地拉紧,并逐渐增大
所施加的拉力,同时观察测力计示数的变化,直到尼龙丝被拉
断,记下这时测力计的读数,就等于尼龙丝的抗断拉力。

也可采用这样的方法:将尼龙丝的上端固定,在其下端悬
挂砝码,并且逐渐增大所加砝码的重量,直到尼龙丝被拉断,
记下这时所加砝码的总重量,就等于尼龙丝的抗断拉力。

\section{习题解答}

\subsection{练习一}
\begin{enumerate}
\item  举出几个实例来说明力是物体对物体的作用.

\begin{solution}
    这样的例子可以举出很多。用铁锤钉钉子,铁锤对
锤子施了力;用手推墙壁,人手对墙壁施了力;跳水运动员用
力蹬跳板,运动员的脚对跳板施了力;汽车停在路面上,车轮
对路面施了力,等等。
\end{solution}
\item 放在水平面上的物体(课本图1.8甲)受到几个力的作用?
  各是什么物体对它的作用,是哪种力?画出物体受力的示意
  图.

  \begin{solution}
    放在水平面上的物体受到两个力的作用,如图1.15
    所示,一个是地球对它作用的重力$G$, 方向竖直向下;一个是
    水平面对它作用的支持力$N$, 方向竖直向上,支持力是弹力。
  \end{solution}
\item 用一根绳子把小球挂在天花板上(课本图1.9甲),小球受
  到几个力的作用?各是什么物体对它的作用,是哪种力?画出
  小球受力的示意图.

  \begin{solution}
    用一根绳子把小球挂在天花板上,小球受到两个力
    的作用,如图1.16所示,一个是地球对它作用的重力$G$, 方
    向竖直向下;一个是绳子对它的拉力$F$, 方向竖直向上,拉力
    是弹力。
  \end{solution}

  \begin{figure}[htp]\centering
    \begin{minipage}[t]{0.32\textwidth}
    \centering
\begin{tikzpicture}[>=latex, scale=1]
\fill [pattern=north east lines] (0,0) rectangle (2,.2);
\draw(0,0.2)--(2,.2);
\draw (.5,.2) rectangle (1.5,.8);
\draw[<->, thick](1,1.5)node[above]{$N$}--(1,.5)--(1,-.5)node[below]{$G$};
\tkzDrawPoint(1,.5)
    \end{tikzpicture}
    \caption{}
    \end{minipage}
        \begin{minipage}[t]{0.32\textwidth}
    \centering
\begin{tikzpicture}[>=latex, scale=1]
    \fill [pattern=north east lines] (0.5,2) rectangle (1.5,2.2);
    \draw(0.5,2)--(1.5,2); 
    \draw(1,2)--(1,.5);\draw[->, thick](1,.5)--(1,1.5)node[right]{$F$};
    \draw[fill=white] (1,.5) circle (.2);
    \tkzDrawPoint(1,.5)
\draw[->,thick](1,.5)--(1,-.5)node[below]{$G$};
    \end{tikzpicture}
    \caption{}
    \end{minipage}
    \begin{minipage}[t]{0.32\textwidth}
    \centering
    \begin{tikzpicture}[>=latex, scale=1]
\fill [pattern=north east lines] (-1,2) rectangle (-1.5,2.2);
    \draw(-1,2)--(-1.5,2); 
\fill [pattern=north east lines] (1,2) rectangle (1.5,2.2);
    \draw(1.5,2)--(1,2); 
\draw(-1.2,2)--(0,.2)--(1.2,2);
\draw(-.3,-.2) rectangle (.3,.2);
\tkzDrawPoint(0,0)
\draw[->,thick](0,0)--(0,-1)node[below]{$G$};
\draw[->,thick](0,.2)--(-.6,1.1)node[left]{$F_1$};
\draw[->,thick](0,.2)--(.6,1.1)node[right]{$F_2$};
    \end{tikzpicture}
    \caption{}
    \end{minipage}
    \end{figure}

\item 用两根绳子把物体挂在天花板上(课本图1.9乙),这个物
  体受到几个力的作用?各是什么物体对它的作用,是哪种力?
  画出物体受力的示意图.

  \begin{solution}
    用两根绳子把物体挂在天花板上,这个物体受到三
    个力的作用,如图1.17所示.
    一个是地球对它作用的重力
    $G$, 方向竖直向下;一个是左边
    绳子对它的拉力$F_1$, 方向沿着
    左边的绳子斜向上;另一个是右边绳子对它的拉力$F_2$, 方向
    沿着右边的绳子斜向上。两根绳子的拉力都是弹力。
  \end{solution}
 \item 找一个薄板状的物体,用书中所讲的悬挂方法求出
  这个物体的重心.

  \begin{solution}
    (略)
    说明:本题是实验题,应该要求学生动手去做。
  \end{solution}
 \item 放在水平桌面上的两个小球,它们靠在一起但不互
相挤压,它们之间有弹力作用吗?为什么?

\begin{solution}
    它们之间没有弹力作用,因为它们没有互相挤压就
不发生形变,也就不会产生使它们恢复原状的弹力。
\end{solution}
\item 用下面的简单装置也可以显示微小形变.找一个大
玻璃瓶,装满水,塞上中间插有细管的瓶塞.用手按压玻璃瓶,
细管中的水面就上升;松开手,水面又降回原处.这说明玻璃瓶
遇到按压时发生弹性形变.实际做一下这个实验.

\begin{solution}
    提示:应先将玻璃瓶盛满水,再塞上中间插有一根开口
细玻璃管的瓶塞,玻璃瓶内不应留有气泡。
\end{solution}
\end{enumerate}

\subsection{练习二}
\begin{enumerate}
\item 把一个重量为2牛的物体挂在弹簧上,物体静止时受
到的弹簧的弹力有多大?为什么?

\begin{solution}
    弹簧的弹力也是2牛.因为物体静止时,它受到的重
力和弹簧的弹力是一对平衡力,大小相等。
\end{solution}
\item 把重量相同的两个物体分利挂在两根不同的弹簧
上,一根弹簧伸长的长度小,另一根伸长的长度大,哪根弹簧
的倔强系数大?

\begin{solution}
    由公式$f=kx$可以看出,当$f$一定时,$x$小的$k$大,$x$
大的$k$小,所以,伸长长度小的弹簧倔强系数大。
\end{solution}
\item 一根弹簧的倔强系数是100牛/米,伸长的长度为2厘
米时,弹簧的弹力有多大?另一根弹簧的倔强系数是2000
牛/米,缩短的长度为3厘米时,弹簧的弹力有多大?

\begin{solution}
   \[\begin{split}
       f_1&=k_1x_1=100{\rm N/m}\x0.02{\rm m}=2{\rm N}\\
f_2&=k_2x_2=2000{\rm N/m}\x0.03{\rm m}=60{\rm N}
   \end{split} \]
两根弹簧的弹力分别为2牛和60牛.
\end{solution}
\item 一根弹簧,不挂物体时长15厘米,挂上0.5千克的物
体时长18厘米.这根弹簧的倔强系数有多大?

\begin{solution}
    物体所受的重力$G=0.5{\rm kg}\x9.8{\rm N/kg}=4.9
{\rm N}$,伸长$x=0.18-0.15=0.03{\rm m}$,弹力$f=G=4.9{\rm N}$。
由$f=kx$, 得弹簧倔强系数
\[k=\frac{f}{x}=\frac{4.9{\rm N}}{0.03{\rm m}}=163{\rm N/m}\]
\end{solution}
\end{enumerate}

\subsection{练习三}
\begin{enumerate}
\item  在东北的冬李伐木工作中,许多伐下的木料被装在
雪橇上,用马拉着在冰道上运出去.一个有钢制滑板的雪橇,
上面装着木料,共重$4.9\times 10^4$牛.在水平的冰道上,马要在水平
方向用多大的力才能够拉着雪橇匀速前进?

\begin{solution}
    匀速前进时,拉力$F$等于摩擦力$f$, 所以拉力
    \[f=\mu N=0.02\x4.9\x10^4{\rm N}=9.8\x 10^2{\rm N}\]
\end{solution}
\item  用20牛的水平的力拉着一块重量是40牛的砖,可以
使砖在水平地面上匀速滑动.求砖和地面之间的滑动摩擦系
数.

\begin{solution}
    砖与地面之间的滑动摩擦系数
    \[\mu=\frac{f}{N}=\frac{20{\rm N}}{40{\rm N}}=0.5\]
\end{solution}
\item  要使重量是400牛的桌子从原地移动,必须最小用
200牛的水平推力.桌子从原地移动以后,为了使它继续做匀
速运动,只要160牛的水平推力就行了.求最大静摩擦力和滑
动摩擦系数.如果用100牛的水平推力推桌子,这时静摩擦力
有多大?


\begin{solution}
    使桌子移动必须用的最小水平推力就是最大静摩擦
    力$f_n=200$牛.

    由于匀速前进时,推力$F=f$, 而$f=\mu N$. 所以滑动摩擦
    系数
    \[\mu=\frac{f}{N}=\frac{160{\rm N}}{400{\rm N}}=0.4\]
    静止时,静摩擦力等于推力,
    $\therefore\quad f=F=100{\rm N}$
\end{solution}
\item  做下面的实验:用一根橡皮绳把书吊起来,当书静
止不动的时候,测出橡皮绳伸长的长度.把书放在桌子上,水
平拉橡皮绳,使书做匀速运动,再测出橡皮绳伸长的长度.设
橡皮绳伸长的长度跟外力成正比,根据测出的数据粗略地算
出书和桌面之间的滑动摩擦系数.

\begin{solution}
    在实验基础上解题步骤:设被吊起的书静止不动时,橡皮
    绳伸长的长度为$x_1$, 此时,橡皮绳的弹力$F_1$等于书所受的重力
    $G$, 即$$F_1=kx_1,\qquad F_1=G$$ 因此:
\begin{equation}
     G=kx_1
\end{equation}   
    拉着书在桌面上作匀速运动时,橡皮绳伸长的长度为$x_2$
    此时,橡皮绳的弹力$F_2$等于书受到的摩擦力$\mu N$, 即
  \[  F_2=kx_2,\qquad F_2=\mu N=\mu G\]
因此:
\begin{equation}
    G=\frac{kx_2}{\mu}
\end{equation}
    由(1.1)(1.2)式,得滑动摩擦系数
    \[\mu=\frac{x_2}{x_1}\]
\end{solution}
\end{enumerate}


\subsection{练习四}
\begin{enumerate}
\item  有人说“施力物体同时也一定是受力物体”,这句话
正确吗?用两三个实例来说明.

\begin{solution}
    正确。

    例如,吊绳吊灯,绳对灯来说是施力物体,对灯施一个拉
    力;但同时它又是受力物体,受到灯拉它的力。

    又如,把物休放在桌面上,物体是施力物体,给桌面一
    个压力;同时它又是受力物体,受到桌面给予它的支持力(弹力)。
\end{solution}
\item  地球的质量大约是$6\times 10^{24}$千克.地球对地面上质
贵是1千克的石块的引力跟这个石块时地球的引力相比较,哪
个力大?根据牛顿第三定律,正确的答案是什么?

\begin{solution}
    两个引力一样大。因为根据牛顿第三定律,地球对
    石块的引力和石块对地球的引力,是一对作用力和反作用力,
    所以一样大。
\end{solution}

\item  用牛顿第三定律判断下列说法是否正确.
\begin{enumerate}
\item  只有你站在地上完全不动,你和地球之间的相互作用
力才是一对大小相等方向相反的力.
\item  物体$A$静止在物体$B$上,$A$的质量是$B$的质量的100
倍,因此,$A$作用于$B$的力大于$B$作用于$A$的力.
\end{enumerate}


\begin{solution}
\begin{enumerate}
    \item 不正确,你和地球之间的相互作用力是作用力
    和反作用力,因此大小相等方向相反,与物体之间有无相对运
    动无关。
    \item 不正确。因为$A$作用于$B$的力和$B$作用于$A$的力是
    一对作用力和反作用力。它们的大小总是相等的。
\end{enumerate}
\end{solution}
\item  放在水平面上的物体(课本图1.8甲)受到两个力的作用.
这两个力的反作用力各作用在什么物体上?在这四个力中,哪
两对力是作用力和反作用力?哪两个力是相互平衡的力?

\begin{solution}
    在课本图1.8甲中,物体所受重力的反作用力(引
    力)作用于地球;支持力的反作用力(压力)作用于水平面,重
    力和物体对地球的引力,支持力和压力是两对作用力和反作
    用力。重力和支持力是相互平衡的力。
\end{solution}
\item  挂在绳子(或弹簧)上的物体(课本图1.9甲)受到两个力
的作用.这两个力的反作用力各作用在什么物体上?在这四
个力中,哪两对力是作用力和反作用力?哪两个力是相互平衡
的力?

\begin{solution}
    在课本图1.9甲中,物体所受重力的反作用力(引
    力)作用于地球,物体所受拉力的反作用力(拉力)作用于绳
    子。物体所受的重力和物体对地球的引力,物体所受拉力和绳
    子所受拉力,是两对作用力和反作用力。物体所受的拉力和
    物体所受的重力是相互平衡的力。
\end{solution}
\item 从上述两题的解答中,试找出一对作用力和反作用
力跟两个相互平衡的力之间的区别.

\begin{solution}
    作用力和反作用力分别作用于相互作用的两个物体
    上,相互平衡的力作用于同一物体上。
\end{solution}
\end{enumerate}


\subsection{练习五}
    在下面各题中,在画受力图的时候,如果已知力的大小和
方向,要按照一定的标度做力的图示;如果未给出力的大小,
可以只画出力的方向.
\begin{enumerate}
\item 竖直向上抛出的石块受到几个力的作用?水平抛出
的石块受到几个力的作用?竖直向下抛出的石块受到几个力
的作用?放开手,让石块自由下落,石块受到几个力的作用?分
别画出石块的受力图.不考虑空气阻力.

\begin{solution}
    所述的四种情况,石块都只受
到竖直向下的重力作用,如图1.18所示
是它们的受力图。
\end{solution}

\begin{figure}[htp]\centering
    \begin{minipage}[t]{0.15\textwidth}
    \centering
\begin{tikzpicture}[>=latex, scale=1]
       \draw(0,0) rectangle (.5,.5);
       \draw[->,thick](.25,.25)--(.25,-1)node[below]{$G$};
       \tkzDrawPoint(.25,.25)
    \end{tikzpicture}
    \caption{}
    \end{minipage}
    \begin{minipage}[t]{0.25\textwidth}
    \centering
\begin{tikzpicture}[>=latex]
\draw [rotate=30] (0,0) rectangle (.5,.5);
\draw[rotate=30] (-1,0)--(2.3,0);
\fill[rotate=30] (.25,.25) circle[radius=1.5pt];
\draw[rotate=30, ->, thick](.25,.25)--(.25,1.5)node[right]{$N$};
\draw[rotate=30, ->, thick](.25,.25)--(-.5,.25-1.3)node[below]{$G$};
\draw(-.85,-.5)--(2,-.5)--(2,1.15);    
\end{tikzpicture}
    \caption{}
    \end{minipage}
    \begin{minipage}[t]{0.26\textwidth}
        \centering
\begin{tikzpicture}[>=latex]
\draw [rotate=30] (0,0) rectangle (.5,.5);
\draw[rotate=30] (-1,0)--(2.3,0);
\fill[rotate=30] (.25,.25) circle[radius=1.5pt];
\draw[rotate=30, <->, thick](1,.25)node[above]{$f$}--(.25,.25)--(.25,1.5)node[right]{$N$};
\draw[rotate=30, ->, thick](.25,.25)--(-.5,.25-1.3)node[below]{$G$};
\draw(-.85,-.5)--(2,-.5)--(2,1.15);    
\end{tikzpicture}
        \caption{}
        \end{minipage}
            \begin{minipage}[t]{0.23\textwidth}
        \centering
\begin{tikzpicture}[>=latex]
\draw [rotate=30] (0,0) rectangle (.5,.5);
\draw[rotate=30] (-1,0)--(2.3,0);
\fill[rotate=30] (.25,.25) circle[radius=1.5pt];
\draw[rotate=30, <->, thick](-.5,.25)node[left]{$f$}--(.25,.25)--(.25,1.5)node[right]{$N$};
\draw[rotate=30, ->, thick](.25,.25)--(-.5,.25-1.3)node[below]{$G$};
\draw(-.85,-.5)--(2,-.5)--(2,1.15);    
\end{tikzpicture}
        \caption{}
        \end{minipage}
    \end{figure}

\item 一个物体沿着光滑的斜面滑下来,物体受到几个力
的作用?物体原来具有某一速度,它沿着光滑的斜面滑上去的
时候受到几个力的作用?分别画出物体的受力图.

    如果物体和斜面之间有滑动摩擦,受力情况又怎样?再分
别画出物体的受力图.

\begin{solution}
    沿光滑斜面上滑或下滑都只受到重力和支持力这两
个力的作用(图1.19)。如果物体和斜面之间有滑动摩擦,物
体和斜面间就会有滑动摩擦力,斜面上的物体将受到重力、支
持力和滑动摩擦力这三个力的作用。下滑时,滑动摩擦力沿
斜面向上(图1.20),上滑时,滑
动摩擦力沿斜面向下(图1.21)。
\end{solution}
\item 雨滴下落的速度较大,空气阻力不能忽略不计.无
风的时候雨滴匀速竖直下落,雨滴受到几个力的作用?设雨滴
的重量是0.001牛,画出雨滴的受力图.

\begin{solution}
    雨滴受到重力和空气阻力的作用,由于雨滴匀速竖
    直下落,这两个力平衡,大小相等,方向相反。雨滴的受力图
    如图1.22所示.
\end{solution}
\item 用水平绳拉着木块在水平面上运动,木块的重量是
5牛,绳的拉力是10牛,滑动摩擦系数是0.3.画出木块的受力图.


\begin{solution}
    木块共受四个力:重力$G=5$牛,竖直向下;水平面
的支持力$N=5$牛,竖直向上;拉力$F=10$牛,方向水平;摩
擦力$f=0.3\x5=1.5$牛,方向与$F$相反,受力图如图1.23
所示。


\end{solution}

\begin{figure}[htp]\centering
    \begin{minipage}[t]{0.48\textwidth}
    \centering
\begin{tikzpicture}[>=latex]
\draw(0,0) circle(0.25);
\draw[<->](0,2)node[above]{$f$}--(0,-2)node[below]{$G$};
\tkzDrawPoint(0,0)
\foreach \x in {-3,-2,-1,1,2,3}
{
    \draw(-.1,\x/2)--(0,\x/2);
}
\draw[|-|](1.5,-1)--node[rotate=90, above]{$2.5\x 10^{-4}$N}(1.5,-.5);
\end{tikzpicture}
    \caption{}
    \end{minipage}
    \begin{minipage}[t]{0.48\textwidth}
    \centering
\begin{tikzpicture}[>=latex]
    \draw[thick](-.2,-.2) rectangle (.2,.2);
    \draw[<->](-1,0)node[left]{$f$}--(5,0)node[right]{$F$};
\draw[<->](0,-2.5)node[right]{$G$}--(0,2.5)node[right]{$N$};
\foreach \x in {-1,1,2,...,9}
{
    \draw(\x/2,0)--(\x/2,-.1);
}
\foreach \x in {-4,-3,-2,-1,1,2,3,4}
{
    \draw(-.1,\x/2)--(0,\x/2);
}
\draw[|-|](2,2)--node[above]{1N}(2.5,2);
\end{tikzpicture}
    \caption{}
    \end{minipage}
    \end{figure}

\item 如课本图1.21那样用一根绳子$a$把物体挂起来,再用另一根
水平的绳子$b$把物体拉向一旁固定起来.这个物体受到几个力的
作用?画出物体的受力图.

\begin{solution}
    物体受到重力,绳子$a$和$b$对它的拉力三个力的
作用.受力图如图1.24所示.
\end{solution}

\item  在课本图1.16中没有画出书对桌面的压力$N'$,把这个力
画出来,并回答下面的问题:
\begin{enumerate}
\item 压力$N'$是什么性质的力?
\item 压力$N'$跟哪个力是一对作用力和反作用力?
\item 压力$N'$和重力$G$是不是作用在同一个物体上的力?
\item 在书静止地压在桌面上的情况下,压力$N'$和重力$G$的大小有什么关系?
\end{enumerate}

\begin{figure}[htp]\centering
    \begin{minipage}[t]{0.48\textwidth}
    \centering
\begin{tikzpicture}[>=latex]
\tkzDefPoints{0/0/O, -1.5/0/B, 0/-2/C, 1.5/2/A}
\draw(-.25,-.25) rectangle (.25,.25);
\draw[->, thick](O)--(A)node[above]{$F_a$};
\draw[->, thick](O)--(B)node[above]{$F_b$};
\draw[->, thick](O)--(C)node[right]{$F_c$};
\end{tikzpicture}
    \caption{}
    \end{minipage}
    \begin{minipage}[t]{0.48\textwidth}
    \centering
    \includegraphics[scale=.7]{fig/1-25.png}
    \caption{}
    \end{minipage}
    \end{figure}

\begin{solution}
    书对桌面的压力$N'$如图1.25所示.
\begin{enumerate}
\item 压力$N'$是弹力;    
\item 压力$N'$跟桌面对
书的支持力$N$是作用力和反作
用力;    
\item 压力$N'$和重力$G$不
是作用在同一物体上的力,$N'$作用在桌面上,$G$作用在书上;    
\item 书静止在桌面上时,压力
$N'$和重力$G$大小相等。
\end{enumerate}

\end{solution}
\end{enumerate}

\subsection{练习六}
\begin{enumerate}
\item 两个力的合力总大于原来的每一个力,这话对吗?为
什么?

\begin{solution}
    根据力的平行四边形法则,合力的大小由平行四边
形的对角线表示,原来两个力的大小由平行四边形的两个邻
边表示;而平行四边形的对角线并不总是大于其邻边的。所
以,“两个力的合力总大于原来的每一个力”的说法是不对的。
\end{solution}
\item 有两个力$F_1$和$F_2$,用作图法求出当它们之间的夹角
$\theta =0^\circ$, $30^\circ$, $60^\circ$, $90^\circ$, $120^\circ$, $150^\circ$, $180^\circ$时的合力.研究你所作
的图,能不能得到结论:夹角$\theta$在$0^\circ$到$180^\circ$之间时,$\theta $越大,合
力就越小.

\begin{solution}
    按照题中给出的各个夹角,用同一标度分别做出两
个力$F_1$和$F_2$的合力$R$, 如下列各图所示,测量$R$的大小,可
以得到结论:在$0^{\circ}$到$180^{\circ}$之间,夹角越大,合力就越小.

\begin{figure}[htp]\centering
    \begin{minipage}[t]{0.48\textwidth}
    \centering
\begin{tikzpicture}[>=latex, scale=1]
\draw[->](0,0)node[below]{$O$}--(2,0)node[below]{$F_1$};
\draw[->](0,0)--(3,0)node[below]{$F_2$};
\draw[->](0,0)--(5,0)node[below]{$R=F_1+F_2$};
\node at (2.5,.5){$\theta=0^{\circ}$};
\foreach \x in {1,2,3,4}
{
    \draw(\x,0)--(\x,.1);
}
    \end{tikzpicture}
    \caption{}
    \end{minipage}
    \begin{minipage}[t]{0.48\textwidth}
    \centering
    \begin{tikzpicture}[>=latex, scale=1]
\draw[->](0,0)--(2,0)node[below]{$F_1$};
\draw[->](0,0)--(30:3)node[above]{$F_2$};
\draw[dashed](30:3)--+(2,0)--(2,0);
\draw[thick,->](0,0)node[left]{$O$}--(4.6,1.5)node[right]{$R$};
\node at (0.5,1){$\theta=30^{\circ}$};

    \end{tikzpicture}
    \caption{}
    \end{minipage}
    \end{figure}

\begin{figure}[htp]\centering
    \begin{minipage}[t]{0.3\textwidth}
    \centering
\begin{tikzpicture}[>=latex, scale=1]
    \draw[->](0,0)--(2,0)node[below]{$F_1$};
    \draw[->](0,0)--(60:3)node[above]{$F_2$};
    \draw[dashed](60:3)--+(2,0)--(2,0);
    \draw[thick,->](0,0)node[left]{$O$}--(1.5+2,1.5*1.732)node[right]{$R$};
    \node at (0.5,1)[left]{$\theta=60^{\circ}$}; 
    \end{tikzpicture}
    \caption{}
    \end{minipage}
    \begin{minipage}[t]{0.27\textwidth}
    \centering
    \begin{tikzpicture}[>=latex, scale=1]
    \draw[->](0,0)--(2,0)node[below]{$F_1$};
    \draw[->](0,0)--(90:3)node[above]{$F_2$};
    \draw[dashed](90:3)--+(2,0)--(2,0);
    \draw[thick,->](0,0)node[left]{$O$}--(2,3)node[right]{$R$};
    \node at (0,2)[left]{$\theta=90^{\circ}$};      
    \end{tikzpicture}
    \caption{}
    \end{minipage}
\begin{minipage}[t]{0.4\textwidth}
    \centering
\begin{tikzpicture}[>=latex, scale=1]
    \draw[->](0,0)--(2,0)node[below]{$F_1$};
    \draw[->](0,0)--(120:3)node[above]{$F_2$};
    \draw[dashed](120:3)--+(2,0)--(2,0);
    \draw[thick,->](0,0)node[left]{$O$}--(-1.5+2,1.5*1.732)node[right]{$R$};
    \node at (-0.5,1)[left]{$\theta=120^{\circ}$};      
    \end{tikzpicture}
    \caption{}
    \end{minipage}
    \end{figure}

\begin{figure}[htp]\centering
    \begin{minipage}[t]{0.48\textwidth}
    \centering
\begin{tikzpicture}[>=latex, scale=1]
    \draw[->](0,0)--(2,0)node[below]{$F_1$};
    \draw[->](0,0)--(150:3)node[above]{$F_2$};
    \draw[dashed](150:3)--+(2,0)--(2,0);
    \draw[thick,->](0,0)node[below]{$O$}--(-4.6+4,1.5)node[right]{$R$};
    \node at (-0.5,.2)[left]{$\theta=150^{\circ}$};      
    \end{tikzpicture}
    \caption{}
    \end{minipage}
    \begin{minipage}[t]{0.48\textwidth}
    \centering
    \begin{tikzpicture}[>=latex, scale=1]
      \draw[<->](-2,0)node[above]{$F_1$}--(0,0)node[below]{$O$}--(3,0)node[above]{$F_2$};
\draw[->, thick](0,0)--(1,0)node[below]{$R$};
\node at (1,.5){$\theta=180^{\circ}$};
\foreach \x in {-1,0,1,2}
{
    \draw(\x,0)--(\x,.1);
}
    \end{tikzpicture}
    \caption{}
    \end{minipage}
    \end{figure}

\end{solution}
\item 两个力的合力什么情况下最大,什么情况下最小?设
有两个力,一个是20牛,一个走5牛.合力的最大值是多大,最
小位是多大?

\begin{solution}
    当两个力间夹角$\theta=0^{\circ}$时合力最大,$\theta=180^{\circ}$时,合
力最小.合力的最大值$R_{\text{最大}}=20+5=25$牛;合力的最
小值$R_{\text{最小}}=20-5=15$牛.
\end{solution}
\item 2牛和10牛的两个力,它们的合力能够等于5牛、10牛、
15牛吗?

\begin{solution}
    2牛和10牛的两力,合力的最小值为$10-2=8$牛,合力的最大值为$10+2=12$牛.所以这两个力
的合力不能等于5牛和15牛,可以等于10牛.
\end{solution}
\item   两个力互成$30^\circ$角, 大小分别是90牛和120牛.用作图法求出合力的
大小和方向,然后再用公式来求.

\begin{figure}[htp]
    \centering
\begin{tikzpicture}[>=latex, scale=1]
    \draw[->](0,0)--(3,0)node[below]{90N};
    \draw[->](0,0)--(30:4)node[above]{120N};
    \draw[dashed](30:4)--+(3,0)--(3,0);
    \draw[thick,->](0,0)node[left]{$O$}--(2*1.732+3, 2)node[right]{$R$};
\draw[|-|](5,.5)--node[above]{30N}(6,.5);
\foreach \x in {1,2}
{
    \draw(\x,0)--(\x,.1);
}
\draw[|-|](30:2)--(30:1);
\draw[-|](30:2)--(30:3);
\end{tikzpicture}
    \caption{}
\end{figure}


\begin{solution}
    利用平行四边形法则作图得合力$R$如图1.33所示。
按照标度测量$R$的长度,得合力的大小为204牛.用量角器
测得合力与90牛的力的夹角为$17^{\circ}$.

利用公式,合力的大小
\[R=\sqrt{90^2+120^2+2\x90\x120\cos30^{\circ}}=203{\rm N}\]
合力与90牛的力的夹角的正切
\[\tan\phi = \frac{120\sin 30^{\circ}}{90+120\cos30^{\circ}}=
0.31\]
查表得$\phi=17^{\circ}$.
\end{solution}
\end{enumerate}


\subsection{练习七}
\begin{enumerate}

\item 一个物体的重量是20牛,把它放在一个斜面上,斜面
长$AB$与斜面高$BC$之比是$5:3$.把重力分解,求出平行于斜面
使物体下滑的力和垂直于斜面使物体压紧斜面的力.
 \begin{figure}[htp]\centering
    \begin{minipage}[t]{0.48\textwidth}
    \centering
\includegraphics[scale=.6]{fig/1-34.png}
    \caption{}
    \end{minipage}
    \begin{minipage}[t]{0.48\textwidth}
    \centering
    \includegraphics[scale=.6]{fig/1-35.png}
    \caption{}
    \end{minipage}
    \end{figure}

\begin{solution}
根据题意作图1.34. 平行于斜面使物体下滑的力
\[F_1=\frac{BC}{AB}G=\frac{3}{5}\x 20{\rm N}=12{\rm N}\]
垂直于斜面使物体压紧斜面的力
\[F_2=\frac{AC}{AB}G=\frac{4}{5}\x20{\rm N}=16{\rm N}\]    
\end{solution}


\item 图1.35是塔式起重机,钢索$NO$与水平悬臂$MO$成
$30^\circ$角,当起重机吊着$4.0\times 10^4$牛的货物时,钢索和悬臂分别受
多大的力?
 
\begin{solution}
    根据题意得知$F=4.0\x10^4$牛.从图得钢索受的力
\[F_1=\frac{F}{\sin 30^{\circ}}=\frac{4.0\x10^4}{0.50}=8.0\x10^4{\rm N}\]
悬臂受的力
\[F_2=\frac{F}{\tan 30^{\circ}}=\frac{4.0\x10^4}{0.577}=6.9\x10^4{\rm N}\]
\end{solution}
\item 如图1.36所示,垂直作用在帆上的风力$F=1.0\times 10^4$
牛.沿着船身方向的分力$F_1$使帆船前进,垂直于船身方向的
分力$F_2$使船身侧倾.设$F$与船身方向成$45^\circ$角,求力$F_1$是多大.
\begin{figure}[htp]\centering
    \begin{minipage}[t]{0.48\textwidth}
    \centering
       \includegraphics[scale=.7]{fig/1-36.png}
    \caption{}
    \end{minipage}
    \begin{minipage}[t]{0.48\textwidth}
    \centering
    \begin{tikzpicture}[>=latex, scale=.8]
\tkzDefPoints{0/0/F2, 4/0/G, 4/3/O, 8/3/F1}
\draw[->, thick](O)node[above]{$O$}--(F2)node[below]{$F_2$};
\draw[->, thick](O)--(G)node[right]{$G=180$N};
\draw[->, thick](O)--(F1)node[right]{$F_1$};
\draw[dashed](F1)--(G)--(F2);
\draw[|-|](0,2)--node[above]{60N}(1,2);
\foreach \x in {1,2}
{
    \draw(4+\x,3+0)--(4+\x,3+.1);
    \draw(4+0,\x)--(4+.1,\x);
}
\draw(7,3)--(7,3.1);

\foreach \x in {1,3}
{
    \draw[|-|](36.87:\x)--(36.87:\x+1);
}
\tkzMarkAngle[mark=none, size=.6](F2,O,G)
\tkzMarkRightAngle[mark=none, size=.3](G,O,F1)
\tkzLabelAngle(F2,O,G){$\theta$}
    \end{tikzpicture}
    \caption{}
    \end{minipage}
    \end{figure}

\begin{solution}
    由图可以看出沿着船身方
向的分力
\[F_1=F\cos45^{\circ}=1.0\times 10^4\x \frac{\sqrt{2}}{2}=7.1\times 10^3{\rm N}\]
\end{solution}

\item 把竖直向下的180牛的力分解为两个分力,一个分力
在水平方向上并等于240牛,求另一个分力的大小和方向.

 
\begin{solution}
    根据题意作图,如图1.37所示,$G=180$牛,$F_1=240$牛,则另一个分力的大小
 \[   F_2=\sqrt{G^2+F_1^2}=\sqrt{180^2+240^2}=300{\rm N}\]

 $F_2$和$G$夹角的余弦
 \[\cos\theta=\frac{G}{F_2}=\frac{180{\rm N}}{300{\rm N}}=0.6\]

 所以夹角$\theta=53^{\circ}8'$, 即$F_2$的方向和竖直向下的方向成
$53^{\circ}8'$的角.
\end{solution}
\item 一个小同学跟一个大同学拔河,小同学拉不动大同
学,可是用下述办法,小同学就可以拉动大同学.在树干上
拴一条绳子,大同学拿着绳子的另一端,沿水平方向把绳子拉
紧.小同学用力推绳子的中点,就可以拉动大同学了.实际
做一做,并解释所发生的现象.

\begin{figure}[htp]
    \centering
\includegraphics[scale=.8]{fig/1-38.png}
    \caption{}
\end{figure}

 \begin{solution}
    图1.38是根据题意画的示意图。小同学推绳子
    的力为$F$,大同学受到的拉力是$F$的一个分力$F_1$
\[F_1=\frac{F}{2\cos\frac{\theta}{2}}\]
 $\theta$   是两段绳子间的夹角。当角$\theta$接近$180^{\circ}$时,$\cos\frac{\theta}{2}$
    接近于零,分力$F_1$比小同学用力$F$大得多,所以能拉动大同学。
\end{solution}
\end{enumerate}

\subsection{习题}

\begin{enumerate}
\item   如图1.39所示,为了防止电线杆倾倒,常在两侧对称
地拉上钢绳.如果两条钢绳间的夹角是$60^\circ$,每条钢绳的拉力
都是300牛,求两条钢绳作用在电线杆上的合力.

\begin{figure}[htp]
\centering
\begin{minipage}[t]{0.48\textwidth}
\centering
\begin{tikzpicture}[>=stealth,  thick, scale=.7]

\fill [pattern = north east lines] (-3,-.25) rectangle (3,0);
\draw(-3,0)--(3,0);


\draw (0,4.5)--(-2.5,0);
\draw (0,4.5)--(2.5,0);
\foreach \x in{1,2,3}
{
    \draw [fill=white] (-.5,4.8+\x*0.2) rectangle (.5,4.8+\x*0.2+.1);
}
\draw [fill=white] (-.1,0) rectangle (.1,6);


\draw [->](-1.4,3)--(-2,2);
\draw [->](1.4,3)--(2,2);
\end{tikzpicture}
\caption{}
\end{minipage}
\begin{minipage}[t]{0.48\textwidth}
\centering
\begin{tikzpicture}[>=stealth,  thick, scale=.8]

\draw [->] (0,0)node [below]{$O$}--node [below]{$F$}(6,0);
\draw [->] (0,0)--node [left]{$F_1$}(-.05, 2);
\draw [->] (0,0)--node [above]{$F_2$}(3, .5);
\draw [dashed] (-.05, 2)--(-.05+3, 2.5)--(3, .5);
\draw [dashed] (-.05+3, 2.5)--(6,0)--(6-3+.05, -2.5);
\draw [->,dashed] (0,0)--(-.05+3, 2.5);
\draw [->] (0,0)--node [above]{$F_3$} (6-3+.05, -2.5);
\end{tikzpicture}
\caption{}
\end{minipage}
\end{figure}

\begin{solution}
    解法一:根据力的平行四边形法则
\[\begin{split}
    F&=\sqrt{F^2_1+F^2_2+2F_1F_2\cos\theta}\\
    &=\sqrt{300^2+300^2+2\x300\x300\cos60^{\circ}}\\
    &=300\sqrt{3}=519{\rm N}
\end{split}\]
方向竖直向下.

    解法二:由于$F_1=F_2$, 这一平行四边形是一个菱形
    \[F=2F_1\cos\frac{\theta}{2}=2\x300\x\cos\frac{60^{\circ}}{2}=300\sqrt{3}=519{\rm N}\]
    方
    向竖直向下。
\end{solution}
\item  图1.40表示用平行四边形法则求三个共点力$F_1$、$F_2$、$F_3$
的合力$F$.先求出$F_1$和$F_2$的合力,再求出这个合力与$F_3$的
合力$F$.改用三角形法求出这三个力的合力.改变求和的顺
序,再分别用平行四边形法则和三角形法求出这三个力的合
力.

\begin{solution}
用三角形法求$F_1$、$F_2$、$F_3$的合力$F$如图1.41所示。
图中$F'$是$F_1$、$F_2$的合力.
\begin{figure}[htp]\centering
    \begin{minipage}[t]{0.48\textwidth}
    \centering
\begin{tikzpicture}[>=latex, scale=.8]
\draw [->, thick] (0,0)node [below]{$O$}--(6,0)node [below]{$F$};
\draw [->, thick] (0,0)--node [left]{$F_1$}(-.05, 2);
\draw [->, thick] (-.05, 2)--+(3, .5)node [above]{$F_2$};
\draw [->, thick] (-.05+3, 2.5)--+(6-3+.05, -2.5)node [above]{$F_3$} ; 
\draw [->,dashed] (0,0)--(-.05+3, 2.5)node [below]{$F'$};
    \end{tikzpicture}
    \caption{}
    \end{minipage}
    \begin{minipage}[t]{0.48\textwidth}
    \centering
    \begin{tikzpicture}[>=latex, scale=.8]
\draw [->, thick] (0,0)node [below]{$O$}--(6,0)node [right]{$F$};
\draw [->, thick] (0,0)--(-.05, 2)node [left]{$F_1$};
\draw [->, thick] (0,0)--(3, .5)node [above]{$F_2$};
\draw [->, thick] (0,0)-- (6-3+.05, -2.5)node [below]{$F_3$};
\draw [dashed] (-.05, 2)--(6,0);
\draw [dashed] (3, .5)--+(6-3+.05, -2.5)--(6-3+.05, -2.5);
\draw [->,dashed, thick] (0,0)--node[above]{$F'$}(5.95,-2);
\draw [->,dashed, thick] (5.95,-2)--(6,0);
    \end{tikzpicture}
    \caption{}
    \end{minipage}
    \end{figure}

改变求和的顺序,用平行四边形法则求这三个力的合力,
如图1.42所示,即先求$F_2$、$F_3$的合力$F'$, 再求$F'$、$F_1$的合
力,即三个力的合力$F$.

改变求和的顺序,用三角形法求这三个力的合力,如图1.
43所示,即先求$F_2$、$F_3$的合力$F'$, 再求$F'$, $F_1$的合力,即三个力的合力$F$.
\end{solution}

\begin{figure}[htp]\centering
    \begin{minipage}[t]{0.48\textwidth}
    \centering
\begin{tikzpicture}[>=latex, scale=.8]
\draw [->, thick] (0,0)node [below]{$O$}--(6,0)node [right]{$F$};
\draw [->, thick] (0,0)--(-.05, 2)node [left]{$F_1$};
\draw [->, thick] (0,0)--(3, .5)node [above]{$F_2$};
\draw [->, thick] (0,0)-- (6-3+.05, -2.5)node [below]{$F_3$};

\draw [->,dashed, thick] (0,0)--(5.95,-2)node[below]{$F'$};
\draw [dashed] (5.95,-2)--(3, .5);
\draw [->,dashed, thick] (5.95,-2)--(6,0);
    \end{tikzpicture}
    \caption{}
    \end{minipage}
    \begin{minipage}[t]{0.48\textwidth}
    \centering
    \begin{tikzpicture}[>=latex, scale=.8]
\foreach \x/\y in {90/2, -30/3, -150/4}
{
    \draw[->,thick] (0,0)--(\x:\y);
}   
\draw[dashed](0,2)--+(-30:3)node(R1)[right]{$R_1$}--(-30:3);
\draw[->,thick, dashed](0,0)--(R1);
\draw[dashed](R1)--+(-150:4)node(R2)[below]{$R_2$}--(-150:4);
\draw[->,thick](0,0)--(R2);

\node at (90:2)[above]{20N};
\node(A) at (-30:3)[right]{30N};
\node(B) at (-150:4)[left]{40N};
\tkzDefPoints{0/0/O}
\tkzMarkAngles[mark=none, size=.6](A,O,R1 B,O,R2)
\tkzLabelAngle(A,O,R1){$\theta_1$}
\tkzLabelAngle(B,O,R2){$\theta_2$}
    \end{tikzpicture}
    \caption{}
    \end{minipage}
    \end{figure}



\item   20牛、30牛和40牛的三个力作用于物体的一点,它们
之间的夹角都是$120^\circ$.求合力的大小和方向.

\begin{solution}
根据题意画出受力图,如图1.44所示.先合成20
牛和30牛二力得$R_1$, 再将$R_1$和40牛合成得$R_2$
\[R_1=\sqrt{20^2+30^2+2\x20\x30\cos120^{\circ}}=10\sqrt{7}=26.5{\rm N}\]
\[\tan\theta_1=\frac{20\sin 120^{\circ}}{30+20\cos 120^{\circ}
}=0.866,\qquad \theta_1=41^{\circ}\]
\[R_2=\sqrt{40^2+(10\sqrt{7})^2+2\x40\x10\sqrt{7}\cos(120^{\circ}+\theta_1)}=17.3{\rm N}\]
\[\tan\theta_2=\frac{26.5\sin 161^{\circ}}{40+26.5\cos161^{\circ}}=0.577,\qquad \theta_2=30^{\circ}\]


\end{solution}
\item 如图1.45所示,把一个重量为10牛的物体挂在绳子
上,已知$AC=BC=3$米,$CD=1$米.求绳$AC$和$BC$所受的拉
力.
\begin{figure}[htp]
\centering
\begin{tikzpicture}[>=stealth,  thick, scale=.8]

\fill [pattern = north east lines] (-4.1,-.5) rectangle (-3.8,.5);
\draw(-3.8,.5)--(-3.8,-.5);
\fill [pattern = north east lines] (4.1,-.5) rectangle (3.8,.5);
\draw(3.8,.5)--(3.8,-.5);

\draw [dashed] (-3.8,0)  --node[above]{$D$} (3.8,0);
\node at (-3.5,0) [below]{$A$};
\node at (3.5,0) [below]{$B$};
\draw [dashed] (0,0)  -- (0,-1.5)node[above]{$C$};
\draw  (0,-1.5)  -- (0,-2.5);
\draw (-3.8,0)--(0,-1.5)--(3.8,0);
\draw [fill=black!30] (-.5,-2.5) rectangle (.5,-2.75-.6);


\end{tikzpicture}
\caption{}
\end{figure}


\begin{solution}
    根据题意画受力图,如图1.46所示,由图得
\[\frac{AC}{DC}=\frac{F_{AC}}{\frac{1}{2}G}\]
所以绳$AC$所受的拉力
\[F_{AC}=\frac{G}{2}\cdot \frac{AC}{DC}=\frac{10}{2}\x 3=15{\rm N}\]
绳$BC$所受的拉力
\[F_{BC}=F_{AC}=15{\rm N}\]
\end{solution}

\begin{figure}[htp]\centering
    \begin{minipage}[t]{0.48\textwidth}
    \centering
\begin{tikzpicture}[>=latex, scale=1.4]
    \fill [pattern = north east lines] (-2,-.5) rectangle (-1.8,.5);
    \draw(-1.8,.5)--(-1.8,-.5);
    \fill [pattern = north east lines] (2,-.5) rectangle (1.8,.5);
    \draw(1.8,.5)--(1.8,-.5);
\draw[dashed](-1.8,0)--(1.8,0);
\draw[dashed](1.5,-1.5)--(-1.5,-1.5);
\draw[dashed](0,0)--(0,-1);
\draw[->, thick](0,-.8)--(0,-2.2)node[below]{$G$};
\draw[dashed](-1.5,-1.5)--(0,-2.2)--(1.5,-1.5);
\draw[->, thick](-1.8,0)--(1.5,-1.5)node[right]{$F_{AC}$};
\draw[->, thick](1.8,0)--(-1.5,-1.5)node[left]{$F_{BC}$};
\node at (.2,-1.5)[above]{$E$};
\node at (.2,-.7)[above]{$C$};\node at (0,0)[above]{$D$};
\node at (1.6,0)[below]{$B$};\node at (-1.6,0)[below]{$A$};
\end{tikzpicture}
    \caption{}
    \end{minipage}
    \begin{minipage}[t]{0.48\textwidth}
    \centering
    \begin{tikzpicture}[>=latex, scale=1.4]
\draw[->, thick](-1.8,0)--(1.5,-1.5)node(B)[right]{$F$};
\draw[->, thick](1.8,0)--(-1.5,-1.5)node(C)[left]{$F$};
\draw[->, thick](0,-.8)node(A){}--(0,-2.2)node[below]{$G$};
\draw[dashed](-1.5,-1.5)--(0,-2.2)--(1.5,-1.5);
\tkzMarkAngle[mark=none, size=.3](C,A,B)
\node at (.2,-1.5)[above]{$\theta$};
    \end{tikzpicture}
    \caption{}
    \end{minipage}
    \end{figure}

\item   用手握着橡皮绳的两端,在橡皮绳的中间挂一个重
物,当两手之间的距离增大或减小的时候,物体对橡皮绳的
拉力是否改变?怎样改变?实际做一下,并说明道理.


\begin{solution}
    要改变,道理如下:如图1.47所示,重物$G$挂在绳中
    间,可看成在绳中间加一拉力,拉力的大小等于重物所受的重
    力,这个拉力可按平行四边形法则分解成两个分力$F$, 当两
    手拉开时,两个分力的夹角$\theta$增大,两个分力也随之增大.

    说明:这个问题还可以做一些定量分析。由于
    \[G^2=F^2+F^2+2F\cdot F\cos\theta\]
    因此:
    \[F=\sqrt{\frac{1}{2(1+\cos\theta)}}\cdot G\]
    当$90^{\circ}<\theta<180^{\circ}$时,$\cos\theta$为负值,$\theta$越大,$|\cos\theta|$越大,$F$也
    随之增大。
\end{solution}
\item  刀、斧、凿、刨等切削工具的刃部叫做劈,劈的纵截面
是一个三角形,如图1.48所示.使用劈的时候,在劈背上加力
$F$,这个力产生两个效果,这就是使劈的两个侧面推压物体,
把物体劈开.设劈的纵截面是一个等腰三角形,劈背的宽度
是$d$,劈的侧面的长度是$\ell$,可以证明:
\[f_1=f_2=\frac{\ell}{d}F \]

从上式可知,当$F$一定的时侯,劈的两个侧面之间的夹角
越小,$\ell/d$就越大,$f_1$和$f_2$就越大.这说明了为什么越锋利的
切削工具越容易劈开物体.试证明上式.
\begin{figure}[htp]\centering
    \begin{minipage}[t]{0.48\textwidth}
    \centering
\includegraphics[scale=.6]{fig/1-48.png}
    \caption{}
    \end{minipage}
    \begin{minipage}[t]{0.48\textwidth}
    \centering
    \includegraphics[scale=.6]{fig/1-49.png}
    \caption{}
    \end{minipage}
    \end{figure}


\begin{solution}
    如图1.48所示,$f_1$、$f_2$是$F$的两个分力,由于三角
    形$ABC$与三角形$OPQ$相似,则有
\[\frac{f_1}{F}=\frac{\ell}{d}\]
因此:\[f_1=f_2=\frac{\ell}{d}F\]
\end{solution}


\item 一个物体放在倾角为$\theta$的光滑斜面上,求物体受到
的合力.

\begin{solution}
    略(课本已作解答)。
\end{solution}
   

\item 一个滑雪人沿着山坡滑下.滑雪人的重量是700牛,
山坡的倾角是$30^\circ$,滑雪板和雪地的滑动摩擦系数是0.04.求
滑雪人所受的合力.

\begin{solution}
如图1.49所示,滑雪人所受的重力$G$分解为平行于
山坡的分力$F_1$和垂直于山坡的分力$F_2$。

在垂直于山坡方向上,滑
雪人受到的重力的分力$F_2$和
山坡的支持力,两者大小相等,
方向相反,合力为零。

在平行于山坡方向上,滑
雪人受到重力的分力$F_1$和相
反方向的摩擦力$f$。
\[\begin{split}
   F_1&=G \sin 30^{\circ} =700\x\frac{1}{2}=350{\rm N}\\
   f&=\mu N=\mu F_2=\mu G\cos 30^{\circ}=0.04\x700\x\frac{\sqrt{3}}{2}=24{\rm N}
\end{split}\]
所以滑雪人受到的
合力$F=F_1-f=350-24=326$牛.方向平行于山坡、
向下。
\end{solution}
\end{enumerate}






















































































\section{参考资料}
\subsection{四种基本的力}
按照现代物理学的观点,自然界存在
四种基本的相互作用:万有引力(简称引力)、电磁力、强相互
作用力和弱相互作用力,在宏观世界里能显示其作用
的只有两种,即引力和电磁力。这两种力是长程力,它们的
作用范围从理论上说是无限的。强相互作用和弱相互作用则
是短程力,强作用力只在$10^{-15}$米范围内才有显著作用;弱
作用的力程更短,不超过$10^{-16}$米,这两种力只有在原子核内
部和基本粒子的相互作用中才显示出来,在宏观世界里不能
察觉它们的存在。

四种相互作用,按其作用的强弱次序排列,依次为:强相
互作用、电磁相互作用、弱相互作用、引力相互作用。一对质
子在相距$10^{-15}$米时,各种相互作用的相对强度为
\begin{center}
    \begin{tabular}{p{.25\textwidth}l}
  强相互作用&1\\
电磁相互作用&$10^{-2}$\\
弱相互作用&$10^{-14}$\\
引力相互作用&$10^{-40}$ 
    \end{tabular}
\end{center}

尽管四种相互作用的差别如此巨大,仍然有一些物理学
家致力于寻求各种相互作用的统一理论,近年来,在弱作用和
电磁作用的统一方面,已取得成功,实验已经证明,正如电和
磁是电磁作用的两种不同表现一样,弱作用和电磁作用也只
不过是统一的弱电相互作用的两种不同表现而已。

值得注意的是自然界中是否还有新的未被发现的基本作
用力呢?这个问题也有人在研究。最近发表的研究报告表明,
可能有第五种力存在。这种力在影响落体的加速度方面起着
跟引力相反的作用。它的相对强度可能只有引力作用的百分
之一,作用范围只有几百米,是一种中程力。

\subsection{力的等效移动}

作用在物体上的力,可以用有向线段
来表示,有向线段应当从力的作用点画起。但实际上,在对物
体进行受力分析时,常常把力的作用点沿着力的作用线移动
或者把力在物体上平移,而不改变力的作用效果。

\begin{figure}[htp]
    \centering
\includegraphics[scale=.45]{fig/1-50.png}
    \caption{}
\end{figure}

在外力作用下,形状和体积都不起变化的物体叫刚体。
作用在刚体上的可以沿着力的作用线的方向移到任意一点
而不改变力的效果。力的这种性质,叫做刚体内的力的可传
性.这一性质可以用图1.50来说明,甲图表示刚体的$A$点
受一外力$F_A$. 乙图表示再在$F$的作用线上任意一点$B$附加
一对平衡力$F_B$和$F_{B'}$, 这样并不会改变刚体的运动状态。如:
果使附加的平衡力$F_B$、$F_{B'}$的大小和$F_A$的大小相等,这时
$F_{B'}$和$F_{A}$也是一对平衡力;于是,去掉这对平衡力对刚体的
运动状态也没有影响,如图丙所示。可见,作用在刚体$A$点的
力$F_A$被作用在$B$点的力$F_B$等效代替,这样的效果就相当于
$F_A$沿其作用线从$A$点移到了$B$点。

一切实际物体都不是刚体,但是一般固体在外力作用下
发生的形变并不显著,可以近似看成刚体。中学研究物体
受力时,实际上是把物体看成是质点或近似看成刚体来做受
力图的。

这里再说明一下力的平移。研究平面上的物体在重力
$G$、支持力$N$、拉力$F$和摩擦力$f$作用下保持平衡时,受力图
常使$G$、$N$、$F$、$f$交于一点,如图1.51所示.
\begin{figure}[htp]\centering
    \begin{minipage}[t]{0.48\textwidth}
    \centering
\begin{tikzpicture}[>=latex, scale=1]
\fill[pattern=north east lines](-1.5,-.2) rectangle (1.5,0);
\draw(-1.5,0)--(1.5,0);
\draw (-.7,0) rectangle(.7,1);
\draw[<->, thick](-1.2,.5)node[left]{$f$}--(1.2,.5)node[right]{$F$};
\draw[<->, thick](0,-1)node[below]{$G$}--(0,2)node[above]{$N$};
\tkzDrawPoint(0,.5)
    \end{tikzpicture}
    \caption{}
    \end{minipage}
    \begin{minipage}[t]{0.48\textwidth}
    \centering
    \begin{tikzpicture}[>=latex, scale=1]
\fill[pattern=north east lines](-2.5,-.2) rectangle (2.5,0);
\draw(-2.5,0)--(2.5,0);
\draw (-.7,0) rectangle(.7,1);
\draw[->, thick](-.7,0)node[below]{$A$}--(-1.5,0)node[above]{$f$};
\draw[->, thick](0,.5) --(1.2,.5)node[right]{$F$};
\draw[->, thick](0,.5) -- (0,-1)node[below]{$G$};
\node at (.7,0) [below]{$B$};
\draw[dashed](-2,.5)--(0,.5)--(0,2.3);
\draw[<->](-1.8,0)--node[left]{$d_1$}(-1.8,.5);
\draw[->, thick](.4,0)--(.4,1.5)node[right]{$N$};
\draw[dashed](.4,1.5)--(.4,2.3);
\draw[<->](0,1.8)--node[above]{$d_2$}(.4,1.8);
\tkzDrawPoint(0,.5)      
    \end{tikzpicture}
    \caption{}
    \end{minipage}
    \end{figure}

这样做在许多情况下是把物体看成质点来研究的。而实
际的物体受力图应如图1.52所示.这时,拉力$F$和摩擦力$f$
是一对力偶,力偶矩$M_1=fd_1$, 这个力偶矩使物体沿顺时针
方向转动;重力$G$和支持力$N$是另一对力偶,其力偶矩$M_2
=Nd_2$, 这个力偶矩使物体沿逆时针方向转动.物体之所以
没有转动,是因为这两对力偶矩相互平衡,即$fd_1=Nd_2$. 这
是物体在$G$、$N$、$F$、$f$四个力作用下保持平衡的一个条件。
另一个条件是:$F=f$, $G=N$. 从$fd_1=Nd_2$这一平衡条件可以
看出,当静摩擦力$f$随外力增大时,$d_2$也相应地增大才能保
持平衡.若$d_2$已增大到无法再增大时,如图中$d_2=AB/2$
时,就能
使$fd_1>Nd_2$, 这时物体将以$B$为轴沿顺时针方向转动。当然
静摩擦力$f$的增大也有个最大限度$f_m=\mu_0 N$. 当$f$增大到
$f_m$ 仍然不能使$f_md_1>Nd_2$时,物体就不会转动.

在图1.52的分析中考虑到物体受力时可能产生的转动。
如果实际问题表明,物体受力后不会有转动,则把图1.52简
化为图1.51并不影响力的作用效果。这就是我们在研究物体
受力平衡时,可以用图1.51来代替图1.52的原因.

上面是就物体受到几个力保持平衡状态的情况而说的。
事实上,只要物体受力后会有转动,都可以将受力图简化,
把几个力绘成交于一点,而不改变原来力的作用效果。

更一般的情况而言,力也可以平移,但要满足力的平移原
理,即把作用在物体上的力平行于它的作用线移到物体上任
意一点而不改变原来力对物体的作用效果,则必须附加一个
力偶,这个力偶产生的力偶矩等于原来的力对这一点的力矩。

\subsection{关于塔式起重机的简单说明}

课本图1.31是塔式起重机。图中钢索$NO$是用来改变
起重臂(悬臂)仰角的,它属于起重机的变辐机构,由装在平衡
臂上的电动卷扬机牵引,图中悬挂重物的钢索是用来起重的,
它属于起重机的起升机构,由装在塔身中部的电动卷扬机牵
引。钢索$NO$的拉力和起重钢索的拉力大小是不同的。起重
机悬臂(起重臂)$MO$与塔身是铰链连接,悬臂仰角可以改变。
而不是固接在塔身上不能动的。




 \chapter{直线运动}\minitoc[n]
\section{教学要求}
这一章讲授的运动学知识,跟第一章一样,都是基础性
的,是后面学习动力学所必需的预备知识.

为了减少学生学习的困难,适应学生的知识水平和接受
能力,本章只讲直线运动,而把运动的合成和分解以及平抛和
斜抛的知识移到第四章的曲线运动中去讲.

通过这一章的教学,应该使学生了解一些描述物体运动
的基本概念和方法,掌握匀速直线运动和匀变速直线运动的
规律,会用这些规律来分析解决一些比较简单的实际问题.

这一章的教学要求是:
\begin{enumerate}
\item 了解参照物的概念,知道研究物体的运动要选择合适
的参照物,了解质点的概念,知道在什么情况可以把物体看
成质点.了解位移的概念,知道位移和路程的区别.
\item 明确什么是匀速直线运动,掌握匀速直线运动的公
,理解匀速直线运动的图象的物理意义.
\item 明确什么是匀变速直线运动,理解平均速度、即时速
,加速度等概念,明确知道速度和加速度的区别,掌握匀变
直线运动的公式,理解匀变速直线运动的图象的物理意义.
\item 认识自由落体运动和竖直上抛运动的特点和规律.
\end{enumerate}

下面对这一章的教学内容作些具体说明.

第一节开始先复习初中学过的机械运动和参照物的
概念,以加强与初中知识的联系,同时强调参照物的重要性,
使学生初步了解怎样选择参照物.第一节还简单介绍了平动
和转动,目的是使学生对物体运动的这两种基本形式有所认
识,后面用到时方便.教材没有给平动和转动下严格的定义,
也不要求补充讲解,只要求学生知道平动和转动的特点和
区别.

质点是力学中的一个重要概念,它是通过科学抽象得出
的理想化模型,运用理想化模型来研究问题,是物理学经常
用的方法.学生在这里初次接触这个问题,需要引起他们注
意,因此把质点单独作为一节来讲述.关于质点的定义,教材
采用了有质量的点这种说法,目的在于强调质点是物理学上
的点,不同于几何学中所说的点,在“质点”这节的最后,给出
了研究质点运动的基本线索——确定质点在任一时刻的位置
和速度,是为了使学生明确讲解本章知识的思路.

描述物体的运动,首先要懂得如何描述物体的位置和位
置变化,为要讲解位置的坐标表示和位移的概念.对位移
的坐标表示,只要求学生知道位移的数值可以用初末位置的
坐标来表示;在后面计算位移的公式中,除了平抛和斜抛外,
都不要求写出位移的坐标表示,而只写出位移本身.

匀速直线运动的知识,学生在初中已学过.这里要在复
习的基础上予以扩展和提高,为讲授匀变速直线运动作好准
备.用比值来定义物理量是物理学中常用的方法,这里用位
移和时间的比值重新定义了速度.在用比值给出速度的定义
之后,说明速度在数值上等于单位时间内位移的大小,使学生
既懂得可以用比值来定义速度,又能跟初中学过的知识联系
起来,理解其意义.

从速度的定义式$v=s/t$,
可以直接写出匀速运动的位移
公式$s=vt$. 根据这个公式就可以确定做匀速运动的物体在
任意时间内的位移,从而确定物体在任意时刻的位置.指出
这一点,可以使学生和前面提出的研究物体运动的总线索联
系起来,认识这一公式表示出了匀速运动的规律.

关于匀速直线运动的定义,教材没有明确“相等的时间”
是“任意”的.这样做是为了把问题叙述得简明一些,减少一
些过细的分析和冗长的论述.

第五节讲解匀速直线运动的图象,主要要求学生会认识
图象,知道图象的物理意义,会画简单的图象.学生刚开始学
习图象,不要求他们用图象去解决比较复杂的问题,以免增加
教学上的难度和学生的负担.教学中,应注意引导学生把数
学中学过的函数及其图象的知识运用到物理中来.这一节讲
述了用速度图象求位移,为后面用面积法推导匀变速运动的
位移公式做准备.

关于即时速度的概念,着重讲述它的物理意义.教材中
写了用数学语言可以精确地表达即时速度,只是让学生知道
个意思,不要求细讲.

为了减少学生学习加速度的概念时的困难,本章只讲匀
变速运动的加速度,不讨论一般变加速运动的情况,不引入
平均加速度和即时加速度的概念,以后各章中再逐步扩大对
加速度的认识,例如讲牛顿第二定律时,提到加速度可以改
变,即外力随时间改变,加速度就随时间而改变,讲圆周运动
时进一步认识加速度的方向的改变.

关于匀变速直线运动位移公式,教材采用了求面积的方
法推导,比较直观形象.这种通过计算面积来求物理量的方
法在物理学中经常用到,需要学生熟悉.但是,对于什么是无
限分割,限于学生的数学知识水平,不能细讲,只要求他们能
体会其意思就行了.

教材没有把匀加速运动和匀减速运动分开来讲,而是把
它统一看作匀变速运动,用统一的位移公式和速度公式来处
理.这样便于后面用统一的方法来处理竖直上抛等问题,有
助于培养学生的概括能力.

匀变速直线运动的两个基本公式皮映了匀变速直线运动
的规律并包括了前面讲过的匀速直线运动.要使学生清楚这
一点,提高他们对两个基本公式的认识.这里提到初速不为
零的匀变速直线运动,可以看作是速度为0.的匀速直线运动
和初速为零的匀变速直线运动的合运动,为了使学生理解这
几种运动之间的关系,加深对匀变速直线运动公式的认识.关
于运动合成的知识这里不宜多讲,到第四章再具体讲解.
教材通过对自由落体闪光照片的分析,得出自由落体运
动是匀变速直线运动,利用闪光照片来研究物体的运动情况,
这种方法以后还要用到.

对竖直上抛运动,教材是把它作为统一的匀变速运动来
处理的,以提高学生的理解能力和运用知识的能力.如果这
样处理有困难,也可以先把上升过程和下降过程分开来计算,
再用统一的运动来处理.讨论竖直上抛物体的上升时间、上
升的最大高度、下落时间、落地速度等,是为了使学生练习运
用匀变速运动的一般规律来分析具体问题,而不是单纯地记
住几个计算公式.

\section{教学建议}
这一章的内容可以分为四个单元:

第一单元(第1节——第3节)讲授描述物体运动的一些预
备知识.

第二单元(第4节——第5节)讲述匀速直线运动.

第三单元(第6节——第10节)讲述匀变速直线运动.

第四单元(第11节——第12节)讲述自由落体运动和竖直
上抛运动.

这一章的重点是第三单元,并通过第四单元两种常见的
实例来巩固和加深对于匀变速直线运动的认识.

\subsection{第一单元}
这一单元是在复习初中学过的机械运动有关知识的基础
上讲述参照物、质点、平动与转动、位置和位移等基本概念,内
容比较简单,但对以后的学习是重要的.

\subsubsection{运动的相对性}

使学生从理性上接受“在自然界中没
有不运动的初体”并且跟“总有许多物体停在原地不动”这种
日常经验统一起来,逻辑上自然要求提出参照物和运动的相
对性问题.为了避免引伸出去讲得很多,可以不提出“运动
的相对性”这个概念,要通过实例使学生清楚地知道:
\begin{enumerate}
    \item 一个物体对某个参照物来说是运动还是静止,要看这个物体
    对参照物来说位置是否变化;
    \item 对于相同的运动,由于选取
    的参照物不同,观测得出的结果可以是不同的;
    \item 虽然参照
    物的选取是人为的,但是在实际选参照物时,总是要使观测方
    便和使运动的描述尽可能简单,例如,在研究地面上物体的运
    动时,一般总是选取地面作参照物.
\end{enumerate}

质点是力学中的一个重要概念.在讲这个概念时,首先
要抓住“物体都具有大小和形状,在运动中物体中各点的位置
变化一般说来是各不相同的,所以要详细描述物体的运动,并
不是一件简单的事情”这一段叙述作为出发点,结合实例(例
如,改变桌子的方位)先讨论在运动中物体中各点的位置变化
各不相同的问题,然后再结合另外一些实例(远途行驶的汽
车、公转轨道上的地球,等等)讨论物体的大小和形状可以忽
略的问题,使学生知道在什么情况下可以只考虑问题的主要
方面、忽略次要因素的影响,使本来并不简单的事情得到简
化,从而便于找出它的规律来.这种使事物或问题简化、理想
化、抽象化、模型化的思想方法和研究方法,在“自由落体运
动”中以及后面还有很多内容都要接触到.

在讲质点的概念时,另一种情况,即平动的物体可以看成
质点,也要讲到,不可忽略.它也是科学的抽象.

\subsubsection{位置和位移}
 这一节一开始就提出,研究物体(质点)
的运动,首先要确定物体(质点)的位置,并可梁取建立坐标系
的方法来确定.在教学中要向学生指出,这里可以把所选定
的参照物,作为坐标系的原点以确定运动物体(质点)的位置,
并按上一章一维矢量的运算方法,来确定质点的位移.但是,
为了计算的方便,更常见的是取质点的初位置作为坐标的
原点.

这一节,要重点讲清楚“位移”这个新概念.主要是要讲
清位移是表示质点在运动过程中位置变化的物理量,它只与
运动的起点(初位置)和终点(末位置)有关,而与物体运动的
路径无关.要通过实际例子讲清楚位移不但有大小,而且有
方向,是个矢量,并把它跟初中学过的路程的概念进行对比和
区别.在讲位移和路程的区别时,要着重提醒学生,只有质点
始终向着同一个方向作直线运动时,位移的大小才等于路程.
在直线运动中,对位移的坐标表示,学生只要知道用初、末位
置的坐标来得出位移的数值,其方向则可按规定好的坐标轴
的正方向依一维矢量的运算方法得知.

\subsection{第二单元}
这一单元就它的内容-匀速直线运动来说,是学生所
熟知的,但是其中速度的概念和匀速直线运动的图象,比初中
的要求高得多,而且在方法上又要为以后学习其他内容作准
备,教学时要予以重视.

\subsubsection{匀速运动的速度}

这一节的教学在讲了什么是匀速
直线运动之后,重点应放在速度的定义、意义和矢量性上.对
中学生来说,用比值的形式来定义一个物理量,也是他们所不
熟悉的.关键在于使同学们认识位移跟时间的比值表示的就
是他们在日常生活中很熟悉的物体运动的快慢,应该从匀速
直线运动的定义出发,即物体在相等时间内的位移相等,推出
匀速运动的位移和时间的比值是一个不随时间而改变的恒
量,这个比值越大,表示物体在相同时间里的位移越大,即运
动得越快.

用位移跟时间的比值来定义速度,突出了速度概念的矢
量性.速度矢量的方向,就是物体位移的方向,根据一维矢量
的运算法则,由于匀速直线运动是沿着同一方向运动的,只要
取位移的方向为正方向,它的位移和速度就都是正值.

了解时间和时刻的区别,对于初学者来说也是很必要的.
这里,不需要在形式上给时间和时刻下什么定义,可以通过具
体的例子说明两者的区别.如通常所说的“几秒内”、“第几秒
内”都是表示一段时间,而“第几秒末”表示的就是一个时刻.
物体在运动过程中,每一时刻都有一定的对应位置,而一定的
时间,则对应于一段位移.

\subsubsection{匀速运动的图象}

物体运动的规律不仅可以用数学
公式去描述它,还可用图象的方法去描述它,而且用图象的方
法,有时更形象直观一些.这一节教学就是要引导学生把函
数图象的知识运用到物理中来.

讲解匀速直线运动的图象,主要是让学生学会认识图象,
理解图象的物理意义,会画简单的图象,学习怎样用图象来求
位移和速度.在图象的绘制上,一开始就要培养学生严格的
科学态度,一定要用直角或三角尺作图,标明横轴和纵轴所代
表的物理量及其单位,并选择适宜的标度.要让学生理解,在
匀速直线运动的速度图象中,可以用长方形“面积”的数值来
表示位移的大小,是因为长方形的一条边长在数值上等于物
体运动时间$t$的长短,另一条边长在数值上等于物体速度$v$
的大小,因此长方形的面积在数上恰好等于物体在时间内
的位移$s=vt$的大小;它的单位是“${\rm m/s\x s=m}$”,而不是
“${\rm m^2}$”.弄清楚这些知识,一方面有利于理解图象所表示的物
理意义,另一方面也有利于培养学生灵活运用数学知识解决
物理问题的能力.学生刚开始学习图象,要从数与形的关系
以及函数图象与物理量的关系,着重引导学生理解和领会图
象的物理意义及其应用,而不要补充利用图象去解二个物体
相向运动或同向运动、速度不同,何时相遇,或要求解释图
线相交点及负斜率的意义等比较复杂的问题.

\subsection{第三单元}

第一单元是本章的重点.教好这一章的关键,在于讲好即
时速度和加速度这两个重要的物理概念.从平均速度引入即
时速度,从即时速度的变化引入加速度,从加速度的定义式引
入匀变速直线运动的速度公式并进一步得出速度图象,从匀
变速直线运动的速度图象,引入匀变速运动的位移公式,再从
匀变速运动的速度公式和位移公式,得出两个有用的推论,是
这一单元教学的主要线索.

\subsubsection{即时速度}

讲述即时速度的概念,可以先让学生粗略
地了解运动物体在某一时刻(或某一位置)都有一定的速度,
这个速度就是物体的即时速度.然后再进一步讲述它的物理
意义.

运动物体在每一时刻(或每一位置)都有一定的速度这一
点,结合日常生活经验,学生是不难理解的.例如自行车和汽
车,在你面前驶过的瞬间,它们的快慢是不同的;百米赛跑,运
动员们在到达终点时的冲刺速度也各不相同,等等.这些事
例都可以说明运动物体在每一时刻(或每一位置)都有一定的
速度.学生难以理解的是即时速度的物理意义,这里主要应
该让学生理解即时速度也就是在足够短的时间里(或位移上)
运动物体的平均速度,因为在足够短的时间里,物体速度的变
化很小,已不能为测量仪器所分辨.在这样的条件下,物体的
运动在测量误差允许的范围内,可以认为是匀速的.

\subsubsection{加速度}

匀变速运动的加速度是本单元另一个重要一
的物理概念,教材在讲清楚什么是匀变速直线运动的基础上,
仿照匀速运动中定义速度的方法,用速度的变化和所用的时
间的比值来定义加速度,并说明了它的矢量性.教学中要强
调:
\begin{enumerate}
    \item 匀变速直线运动的速度在不断改变,而加速度是保持
不变的;
\item 加速度既是矢量,就要遵循一维矢量的运算方法,
根据事先选定的运动的正方向,用带有正负号的数值来表示
它,一般取初速度$v_0$的方向为正方向.
\end{enumerate}
因此,当$v_t>v_0$时$a=\frac{v_t-v_0}{t}$一
定是正值,表示$a$与$v_0$的方向相同;当$v_t<v_0$时,$a$
一定是负值,表示$a$与$v_0$的方向相反.

学生初学时对速度、速度的变化和加速度这几个物理量
在概念上往往混淆不清,教学中应该注意澄清.《速度和加速
度的区别》这段阅读材料,有助于弄清这些问题,应该引导学
生认真阅读,并提出问题来了解、检查学生阅读后理解的情
况.例如可以提出:在平直路面上行驶的汽车,在离开车站
和即将靠站时,汽车的加速度的方向是相同的吗?速度的方向
是相同的吗?汽车在匀加速行驶时,速度对时间的变化率以及
位移对时间的变化率都是不变的吗?等等.教师还应该引导
学生通过一些具体问题的讨论,认识加速度是表示速度变化
快慢的物理量,跟速度的大小没有直接关系.速度大的物体,
加速度不一定大;速度小的物体加速度不一定小.另外,速度
变化的大小,不仅与加速度的大小有关,还跟加速的时间长短
有关.速度变化大的物体,加速度也不一定大,速度变化小的
物体,加速度也不一定小.

\subsubsection{速度公式和速度图象}
在导出匀变速运动的速度公
式时,可先从加速度$a=\frac{v_t-v_0}{t}$
,过渡到速度的变化$v_t-v_0=at$, 然后得出速度公式$v_t=v_0+at$, 这样既可分步分别阐
明和强调公式的物理意义,并可引导学生在理解概念区别的
基础上,进一步认识匀变速运动的速度和加速度的联系.课本
中不把匀加速运动跟匀减速运动分开来处理,而统一为匀变
速运动,只是做匀减速运动时,加速度为负值,这样可以避免
把公式搞得太多,把问题看得太绝对,造成机械的记忆和硬套
公式,加速度的方向也似乎可以不考虑了.

根据速度公式$v_t=v_0+at$, 可以画出匀变速运动的速度
图象,要引导学生弄清楚:图线通过原点和不通过原点的物
理意义,纵轴上的截距和图线的斜率的物理意义,以及斜率的
大小和正负的物理意义等.把这些基本内容掌握了,便有利
于学生的认图、用图与作图.概括地说,这一节教学中,对公式
的导出,不要简单地处理为数学公式的变换,对图象教学不要
单纯地处理成数学上数与形的关系,仅仅作出公式的函数图
象.要强调公式,图象的特点及其变化所表示的物理意义.

\subsubsection{位移公式}

在引用匀速运动的速度图线和横轴之间
的面积表示位移这种方法来求匀变速运动的位移时,要讲一
讲为什么将时间轴尽量加以分割,使折线下的面积,尽量逼近
速度图线和横轴之间的面积,从而可以用它来表示匀变速运
动的位移.让学生熟悉和体会这种方法,对于培养学生的科
学思维能力是有好处的.但是限于学生的数学水平,也不宜
做过细的分析.得出位移公式以后,应通过例题说明加速度
矢量的方向如何根据一维矢的规定来表示,这也是要求学
生熟悉的.

为了使学生学会用不同的方法来解题,并在此基础上选
择简便的方法来求解,可举一些已知条件不同的题目,由同学
讨论、分析、判断、比,让他们自己去体会,不宜由教师归纳
为几条,以免代替和抑制学生的思维.但教师还应该做必要
的指导,例如为了培养良好的习惯,解题时要先弄清楚题目中
所描述的整个运动过程;对复杂的问题,要分步考虑,一步步
地分析出所求的未知量和已知量之间的关系,而这个“关系”
从数学上说是公式,从物理上说就是运动规律.在运用时要
思考它是否符合这个规律,在熟悉分步运算的同时,要引导
学生逐步学会列出方程,利用文字运算来解题,有的题目,要
注意一题多解,以提高学生掌握物理规律和分析物理问题的
能力.

\subsection{第四单元}
这一单元是用匀变速运动的知识来研究自由落体和竖直
上抛这两种常见的运动,认识这两种运动的特点和规律.这
也是对匀变速直线运动基本规律的应用和巩固.

\subsubsection{自由落体运动}

这一节教学应依次掌握好下列各个
环节:
\begin{enumerate}
\item 做好课本图2.20毛钱管演示实验,以表明在管中
空气被抽出后,重量不同的物体下落的快慢相同.
\item 把管中
空气抽出后,如果忽略余下的稀薄气体的作用,就可以近似地
看成是没有空气的空间;在没有空气的空间里,物体下落时才
是“只受重力的作用”;因此,自由落体运动是理想化的运动模
型.
\item 让同学实际测量课本图2.21频闪照片中小球在各个
相等时间里的位移,以鉴别小球自由落下时是作什么运动.有
频闪设备的学校,可以根据自己实际拍摄的频闪照片分组测
量数据;没有频闪设备的学校,也可用其他实验来代替.在得
出:“自由落体运动是初速度为零的匀变速运动”的结论以后,
紧接着要做出的第二个结论便是:不同的自由落体,它们的
运动情况相同,也就是在同一地点,一切物体在自由落体运动
中的加速度(重力加速度$g$)都相同,
\item 既然一切物体在自由
落体运动中的加速度都相同,它必然是一定值,我们可通过实
验来测定它,接着就让学生根据课本88页给出的数据表去计
算$\Delta s$和$\Delta s$的平均值,逐步引导学生重视数据处理,培养这
方面的能力.   
 \item 引导学生认真阅读《伽利略对自由落体运动
的研究》,以培养自学阅读能力和逻辑思维能力,从中学习用
外推法研究物理现象和规律的思路和方法,并且还可以使学
生获得一些物理学史的知识.
\end{enumerate}


\subsubsection{竖直上抛运动}

竖直上抛运动一直有这样两种处理
方法:一是把运动分为匀减速上升和自由下落两个过程,分
开来进行计算;另一种是把它看成是向上的匀速运动和向下
的自由落体运动这两个分运动的合成运动,前一种方法比较
直观,但是在运算时比较繁也容易错;后一种方法比较品象,
如果不进一步具体分析,就只能从运动公式的形式上来说明
两项分别表示两个分运动,学生也不容易体会.

课本在这一章不过分强调运动的合成,也避免学生不容
易理解的运动的独立性,把整个上抛运动看成是一个统一的
匀变速直线运动,这样既体现了课本中不把匀加速运动和匀
减速运动分开来的处理方法,在理解上也并不困难.而另一种
方法,把上升运动和下降运动分为两步来计算,留给同学自己
去尝试,也是可取的,在这方面的要求,对程度不同的学生可
以因材施教.

上抛运动的位移和速度的方向,仍应强调按一维矢量的
统一规定:对速度矢量来说,是以上抛运动初速度$v_0$的方向
为正方向;位移矢量是以初位置(抛出点)为原点,以初速度的
方向为正方向,物体位于抛出点下方时,位移方向向下,位移
是负值;加速度矢量也是以初速度的方向为正方向,而由于重
力加速度的方向跟初速度的方向相反,因此重力加速度$g$总
是取负值.当$g$取绝对值时,上抛运动的公式即可写成:
\[v_t=v_0-gt,\qquad s=v_0t-\frac{1}{2}gt^2\]
这样把各矢量的方向规定一并
弄清楚,就不致于混淆.

在应用上抛运动的公式讨论几个具体问题时,要引导学
生掌握它们的特征,例如物体上升到最大高度时,特征是到
达最高点时即时速度为零,据此可以很容易算出物体上升的
时间和上升的最大高度;物体落回到初位置时的特征是位移
为零,据此可以很容易得出落回原地的时间和物体着地时的
速度.引导同学理解和掌握这些特征,并不是要同学记忆几
条,而是要通过分析和练习让同学去领会.

在解题计算的过程中,还要引导同学理解方程同时有两
个解的物理意义.

\section{实验指导}
\subsection{演示实验}
\subsubsection{观察匀速直线运动和匀变速直线运动}

利用节拍器、斜面(长约1.50米)和小车观察匀速直线运
动和匀变速直线运动的实验装置如图2.1所示.事先调节好
斜面的倾斜程度,使得小车恰能沿斜面匀速下滑,做好垫木位
置的记号,然后再将垫木移右些,使小车下滑时做加速运动.
\begin{figure}[htp]
    \centering
    \includegraphics[scale=.8]{fig/2-1.png}
    \caption{}
\end{figure}

打开节拍器,当听到节拍器发出一个信号时,立即释放小
车,使它自某一固定位置$O$下滑.用一木块阻挡小车,调整阻
挡木块的位置,重复几次实验,使得小车撞击木块时发出的
声音恰巧和节拍器发出的第二个信号(以小车开始释放时的
信号作为第一个信号)重合.在阻挡木块的这一位置($A$)上,
用事先准备好的箭头标出(用胶纸把箭头贴在斜面的侧边).
用同样的方法来确定小车和木块的撞击声恰和节拍器发出的
第三个信号、第四个信号重合时的木块位置$B$和$C$, 并分别用
箭头标出.用米尺量度$OA$、$AB$和$BC$的长度,发现$OA<AB
<BC$, 然后将垫木移到事先准备好的位置,重做实验,直到测
得$OA=AB=BC$. 这表明小车在相等时间里的位移都相等,
所以小车的运动是匀速运动.

在演示匀变速直线运动时,则可以将垫木放在事先调整
好的另一位置上,用上述方法观察小车在相等时间内经过了
不相等的距离.通过调节节拍器的频率,使得小车从静止开
始释放在各相等时间里发生的位移之比$OA:AB:BC=1:3:5$
(譬如可调节到使$OA=16{\rm cm}$, $AB=48{\rm cm}$, $BC=90{\rm cm}$),在
这基础上还可进一步得出$AB-OA=BC-AB$, 即匀变速直线
运动中,在连续相等时间内的位移差是一常数.

\subsubsection{测量匀变速直线运动的即时速度}

\begin{figure}[htp]
    \centering
    \includegraphics[scale=.8]{fig/2-2.png}
    \caption{}
\end{figure}


可利用斜面、小车和节拍器采用上述的实验方法,测
出小车在各连续相等时间内的位移$OA=s_1$, $AB=s_2$, $BC=s_3$
(图2.2).节拍器发出信号的时间间隔为$T$, 根据匀变速直
线运动公式可知:
\begin{align}
    s_1&=\frac{1}{2}aT^2\\
    s_2&=v_AT+\frac{1}{2}aT^2
\end{align}

将(2.1)、(2.2)式相加,$s_1+s_2=v_A T+aT^2$.

$\because\quad v_A=aT$

$\therefore\quad s_1+s_2=v_AT+v_AT=2v_AT,\qquad v_A=\dfrac{s_1+s_2}{2T}$

也$\dfrac{s_1+s_2}{2T}$就等于小车开始运动$2T$时间内的平均速
度,所以匀变速直线运动中某一段时间的中间时刻的即时速
度就等于在这一段时间内的平均速度.

同理,可以测出当小车经过位置$B$时的即时速度$v_B=\dfrac{s_2+s_3}{2T}$

利用打点计时器来测量.
\begin{figure}[htp]
    \centering
\includegraphics[scale=.8]{fig/2-3.png}
    \caption{}
\end{figure}
如图2.3所示,在一端装有定滑轮的长木板上,放一条有
细绳的小车,通过定滑轮在细绳的另一端挂有几个钩码,固定
在小车后面的纸带和打点计时器连在一起.接通电源待打点
计时器正常工作后,释放小车.取下纸带后请一位学生选定
连续的几个计数点(可用每打五次点的时间作为时间的单
位),并要求学生毫米刻度尺测量出各相邻计数点间的距离
$s_1,s_2,s_3,s_4,\ldots$, 教师可将纸带以及选定的计数点放大后画
在黑板上,并将学生实际测得的数据标出,用跟前述相同的方
法来处理数据,求出打点计时器打下各计数点时小车的即时
速度.

\subsubsection{测量匀变速直线运动的加速度}
可采用测量匀变速直线运动的即时速度时相同的实
验装置,取得数据,然后根据匀变速直线运动$\Delta s=aT^2$来求
加速度.$a=\dfrac{\Delta s}{T^2}$

用上述装置取得数据,算出打点计时器在打下各计
数点时小车的即时速度后,然后用画$v$-$t$图象的方法求出图
线的斜率,从而得出加速度$a=\dfrac{\Delta v}{\Delta t}$

算出打点计时器在打下各计数点时小车的即时速度
后,可以任取几段不同的时间及其相应的初速度和末速度的
数据,根据加速度的定义式$a=\dfrac{v_t-v_0}{t}$
,来分别求出这几段时
间内的加速度$a_1,a_2,a_3,\ldots$, 然后再求这些加速度的平均值.

\subsubsection{空气阻力对落体运动的影响}
准备一架调节好的托盘天平,先将一个乒乓球和一个小
铁球放在托盘天平上比较它们的重量,可看到小铁球比较重.
把这两个球效在同一高度上同时下落,则铁球先落地.又把乒
乓球和一块较大的泡沫塑料平板放在天平上比较它们的重
量,可看到泡沫塑料板较重,把它们放在同一高度上同时下
落,则乒乓球先落地.再把一张纸裁成两半,把其中的一半揉
成纸团和另一半放在天平上称,它们是等重的,使它们从同一
高度同时下落,结果揉成纸团的那一半先落地.

这个演示说明了,比较重的物体可以先落地也可以后落
地,即使等重的物体落地的时间也有先后.因此使得物体落
地的时间有先后的原因不是由于重力的大小而是由于空气阻
力大小的影响.受到空气阻力大的物体总是后落地.

\subsubsection{在空气阻力很小时,不同物体同时落下}
这可以用课本图2.20所示牛顿管(又称毛钱管)的传统
实验来进行.演示时可以先不抽空气,当把管子迅速倒转来
时,金属片很快下落,羽毛则下落较慢.然后抽气(抽气要用
管壁很厚的橡皮管),抽气后再演示,发现羽毛和金属片同时
落到管子的底部,最后再将空气放入管中,则羽毛又比金属片
下落得慢.这证明了:在空气阻力很小时,一切物体在同一高
度上的落地时间都是相等的.

\subsubsection{研究自由落体的闪光照片}
课本图2.21自由落体的闪光照片表明了自由落体运动
是初速度为零的匀变速直线运动.关于这幅照片,要求学生
理解以下几点:
\begin{enumerate}
\item 这不是许多个小球,而是表明一个自由下落的小球
在经过各个相等时间
($1/30$秒)时的位置.
\item 从每隔相等时间来看,小球下落的距离越来越大,说
明小球是作变速运动.
\item 照片上小球最初几个位置比较密集,因此可选择某
一个间距较大的位置作为位置1开始测量.小球的位置都取
小球球心(也可以取小球的上缘或下缘),这样来量度相邻两
个位置间的距离$s_1,s_2,s_3,\ldots$, 再算出相邻的相等时间内的
距离之差$\Delta s_1=s_2-s_1,\Delta s_2=s_3-s_2,\Delta s_3=s_4-s_3,\ldots$指
导学生阅读课本88页的数据表,发现$\Delta s$基本上都是接近
的,因此可以证明自由落体运动是初速度为零的匀变速直线
运动.
\item 要注意课本数据表中的数据是根据照片中的刻度尺
读取的,而不是在照片上用毫米刻度尺测量的.
\item 从数据表所列的数据可以计算出自由落体运动的加
速度(即重力加速度$g$)的数值.
\end{enumerate}

\subsubsection{利用打点计时器来研究自由落体运动}
可以按照课本图10.16,利用铁架台把打点计时
器固定起来,用手提住夹有重物的纸带,接通电源,当打点计
时器正常工作后,松开纸带,让重物拖着纸带自由下落.对纸
带上记录的点的分布情况进行分析,可以证明自由落体运动
是初速度为零的匀变速直线运动,而且可以求出重力加速度
$g$的数值.

这个实验也可以让全体学生自己做,这样可以增加练习
使用打点计时器的次数,并再一次练习对实验数据的分析和
处理.


\subsection{学生实验}
\subsubsection{练习使用打点计时器}
实验前要首先弄清楚所使用的打点计时器需要多大
的工作电压,打点的时间间隔是多少.

用手拉动纸带时,速度不要过小,要水平,直到全部
把纸带拉出,这样,即可观察到纸带上被打下的一系列点.

从纸带上能看得清的某个点数起,数一数纸带上共
有多少个点,计算一下在这段距离内纸带运动的时间$t$是多少
秒?要注意如果共有几个点,已知每两个点间经过的时间是
0.02秒,则运动的总时间$t=(n-1)\x0.02$秒.


\subsubsection{研究匀变速直线运动}
这个实验对于数据处理的要求较高,内容较多,要用
两课时完成.实验的具体要求是:
\begin{enumerate}
\item 从分析纸带上的点的分布来判断小车是否做匀变速
直线运动.
\item 在确认小车是做匀变速直线运动的前提下,利用纸带
上的数据来计算出小车在各个时刻的即时速度.
\item 通过画出速度-时间图象,来计算小车做匀变速直线
运动的加速度.
\end{enumerate}

按课本图10.9的装置把实验器材装好,先不要接通
交流电源,用手挡住小车,在细绳的一端挂上三个50克的钩
码.释放后,观察小车运动时拖着的纸带通过打点计时器限
位孔的位置是否恰当.适当调整并重新固定打点计时器的位
置使得限位孔正对着小车的运动方向,然后把纸带穿好,接通
电源,待打点计时器正常工作后释放小车.

取下纸带,观察纸带上的点,会发现开始时的几个点
很密集,为了减小测量误差,可从间距较大的点(譬如相距
几个毫米)开始进行测量,选定连续的几个计数点(不少于五
个),并要求学生参照课本图10.10, 将所选定的计数点标出
$A,B,C,D,E,\ldots$, 量出各计数点间的间距$s_1,s_2,s_3,s_4,\ldots$也
标在纸带上(图2.4),以此作为原始数据记录.
\begin{figure}[htp]
    \centering
    \includegraphics{fig/2-4.png}
    \caption{}
\end{figure}

为了测量方便,可以用每打五次点的时间作为时间的单
位,这样,在两个计数点间的时间间隔$T=5\x0.02=0.1$秒.

根据课本练习九第6题,可得
\[\begin{split}
    \Delta s_1&=s_2-s_1=aT^2\\
    \Delta s_2&=s_3-s_2=aT^2\\
    \Delta s_3&=s_4-s_3=aT^2
\end{split}\]

在匀变速直线运动中,加速度$a$是恒量,因此通过对纸带
上各个相邻计数点间距离的测量,算出各相邻计数点间的距
离之差$\Delta s$均相等,则可证明小车的运动是匀变速直线
运动.

怎样求出打点计时器在打下各计数点时小车的即时
速度?

在确认小车是做匀变速直线运动的前提下,可以利用速
度图象来求小车的加速度,这就首先需要求出小车从开始计
时(即打点计时器打下$A$点时)起,经过$T,2T,3T,\ldots$也就是
打点计时器打下$B,C,D,\ldots$各点时的即时速度.
\[v_1=\frac{s_1+s_2}{2T},\quad v_2=\frac{s_2+s_3}{2T},\quad v_3=\frac{s_3+s_4}{2T}\]

要注意:从课本图10.10所示纸带上所选定的这几个计
数点,应用上述方法只能测得即时速度$v_1$、$v_2$和$v_3$, 若要测出
打下$E$点时小车的即时速度,则必须在纸带上再确定经过时
间为$4T$时的计数点$F$, 测出$EF$间的距离$s_5$, 则$v_4=\dfrac{s_4+s_5}{2T}$.
同理,若要测出打下$A$点时小车的即时速度,则必须在纸带
上的$A$点之前再确定一个计数点$O$, 然后测量$O$和$A$点间的
距离$s_0$, 则$v_A=\dfrac{s_0+s_1}{2T}$.
要注意这个$v_0$并不等于零,它表示
在实验中开始计时时刻的初速度.

求出打点计时器在打下各计数点时的即时速度后,
就可设计一个能表示时间和其对应的即时速度数值的数据表
格.在坐标纸上建立一个平面直角坐标系,用横坐标表示时
间,用纵坐标表示速度,然后在坐标平面上标出$(T,v_1),(2T,
v_2),(3T,v_3),\ldots$各数据点,数据点不得少于五个.把这些点
连结起来可以画出一条直线,画直线时应尽量使多数的点落
在这条直线上,不在直线上的各点,应使它们比较均匀地分布
在直线的两旁,这就是在这条直线两侧的点数以及这些点到
直线的平均距离应大致相等,这就得出小车的速度图线.

\begin{figure}[htp]
    \centering
    \includegraphics[scale=.8]{fig/2-5.png}
    \caption{}
\end{figure}

求出速度图线的斜率就可以得出小车的加速度.如
果画出的$v$-$t$图象如图2.5所示,应该怎样求出这条直线的
斜率呢?要从图线上选取相隔较远的两个点,如$P$和$Q$,分
别从图象上读出它们的坐标$(t_P,v_P)$和$(t_Q,v_Q)$, 即可求得加
速度
\[a=\frac{\Delta v}{\Delta t}=\frac{v_Q-v_P}{t_Q-t_P}\]

可启发学生思考以下问题:
\begin{enumerate}
    \item 如果不用速度图象来求小车的加速度,是否还有其他
的方法?
\item 实验中为什么不直接用$a=\dfrac{v_5-v_1}{4T}$
来求加速度,而要
在图线上另找$P$、$Q$两点通过求图线的斜率来得出加速度,这
样做有什么好处?
\end{enumerate}

这个实验的课时安排,可以在第一课时内完成实验
的准备、使用打点计时器打出纸带以及分析纸带上点的分布、
确定计数点、测量数据判断小车是否做匀变速直线运动等内
容.求出打点计时器在打下各计数点时小车的速度、画出$v$-$t$
图象,求得小车的加速度等内容可安排在第二课时完成.

\subsection{课外实验活动}
\subsubsection{滴水法测重力加速度}
这个实验的原理是根据自由落体运动是初速度为零
的匀变速直线运动,水滴下落的距离$h$跟运动时间$t$的平
方成正比$h=\frac{1}{2}gt^2$, 则 
\[g=\frac{2h}{t^2}\]

因此只要测出水滴下落的距离和下落的时间,便可测得
重力加速度.

这个实验中的水滴下落距离是易于测量的,比较困
难的是时间的测定.对于课本介绍的测时间的方法,要多次
耐心地调整阀门(自来水笼头)的大小,才能使水滴从阀门落
到盘子经过的时间正好等于阀门滴下水滴的时间间隔.为了
便子调整,盘子可以倒过来放在水槽里(或者用大口瓶上的金
属盖代替盘子,使盖子的顶部朝上).由于盘子下面跟水槽间
有一空腔,使得水滴在盘子上的响声比较清脆,阀门离盘
子的距离不要太近,太近了不容易区别两次滴水的时间间隔.
(譬如水滴下落的距离约为0.5米左右时,半分钟里约有
90—100个水滴从阀门滴下,这样,水滴下落的时间就正好等
于相继滴下的两个水滴之间的时间间隔).

用这一方法测定的重力加速度的数值是近似的,但
实验方法比较巧妙而且简单.

\subsubsection{用秒表测量玩具手枪子弹射出的速度}
这个实验的原理是,以初速$v_0$、竖直上抛的物体,从
开始抛出直到落回抛出点所经过的总时间$t=2v_0/g$,
只要测出
玩具手枪的子弹从发射到落回发射点的时间$t$, 当地的$g$值
可以由教师给出,即可测出玩具手枪子弹射出时的初速度
$v_0=\frac{1}{2}gt$.

由于实验条件的限制,所测得的子弹初速度是近似
的.为了能使子弹基本上做竖直上抛运动,可以设法使玩具
手枪固定起来(譬如可将手枪缚在一张方木凳的边上,使枪口
和凳面相平,并使枪管竖直向上),试着先发射一发子弹调节
枪管的位置,使子弹不做明显的斜抛运动就可以了.实验时
可以在手枪的另一侧再放一个相同高度的木凳,从扳动手枪
扳机发射子弹的同时开始计时,当子弹落到木凳时再按下秒
表,测出子弹做竖直上抛运动的总时间$t$. 如果没有秒表,也
可以用手表近似地计时.

\section{习题解答}
\subsection{练习一}
\begin{enumerate}
    \item 两辆在公路上直线行驶的汽车,它们的距离保持不
变,试说明用什么样的物体做参照物,两辆汽车都是静止的,
用什么样的物体做参照物,两辆汽车都是运动的.能否找到
这样一个参照物,一辆汽车对它是静止的,另一辆汽车对它是
运动的?为什么?


\begin{solution}
用其中任意一辆汽车里的座椅做参照物,两辆汽车
都是静止的;用车外公路旁的树木、房屋做参照物,两辆汽车
都是运动的.因为两辆车的距离不变,它们保持相对静止,所
以不可能找到一个参照物,一辆车对它是静止的,而另一辆车
对它却是运动的.
\end{solution}

\item 小孩从滑梯上滑下,钢球沿斜槽滚下,石块从手中落
下,这些物体中哪些是做平动的?

\begin{solution}
根据运动过程中物体各部分的运动是否完全相同来
判断:小孩从滑梯上滑下是平动,钢球沿斜槽滚下不是平动;
石块从手中落下时如果没有翻转则也是平动.
\end{solution}

\item 研究自行车轮的转动,能不能把自行车当作质点?研
究在马路上行驶的自行车的度,能不能把自行车当作质点?

\begin{solution}
    研究自行车轮的转动时,不能把自行车当作质点;研
    究自行车的行驶速度时,可以把它当作质点.
\end{solution}
\end{enumerate}


\subsection{练习二}
\begin{enumerate}
    \item 质点做什么运动,位移的大小才等于路程?
    
\begin{solution}
    质点始终向同一方向做直线运动时,位移的大小等
于路程.
\end{solution}
    \item 课本图2.6表示做直线运动的质点从初位置
$A$经过$B$运动到$C$, 然后从$C$返回,运动到末位置$B$, 设$AB$
长7米,$BC$长5米.求质点的位移的大小和路程.

\begin{solution}
    如图所示,初位置为$A$, 末位置为$B$, 所以位移的大
小为7米($AB$长).
\[\text{路程}=AB+BC+CB=7+5+5=17{\rm m}\]
\end{solution}
\item 在课本图2.4中汽车初位置的坐标是$-2$千
米,末位置的坐标是1千米.求汽车的位移的大小和方向.

\begin{solution}
    汽车的位移$s=1-(-2)=3$千米,由于
位移为正值,方向跟坐标轴正方向一致,即由西向东.
\end{solution}
\end{enumerate}

\subsection{练习三}

\begin{enumerate}
    \item 光在真空中沿直线传播的速度为$3.0\times 10^8$$\ms$.
\begin{enumerate}
    \item 一光年(光在一年中传播的距离)有多少千米?
    \item 最靠近我们的恒星(半人马座$\alpha$星)离我们$4.0\times 10^{13}$千米,它发出的光要多长时间才到达地球?
\end{enumerate}    

\begin{solution}
\begin{enumerate}
    \item $1\text{光年}=365\x24\x60\x60{\rm s}\x3.0\x10^5{\rm km/s}=9.5\x10^{12}{\rm km}$
    \item $t=\dfrac{4.0\x 10^{13}}{9.5\x10^{12}}=4.2\text{年}$
    
    最靠近我们的恒星发出的光要4.2年才能到达地球.
\end{enumerate}
\end{solution}
\item  在技术上常用$\kmh$作速度的单位.试求1$\ms$合多少$\kmh$.

\begin{solution}
    \[\frac{1{\rm m}}{1{\rm s}}=\frac{1/1000{\rm km}}{1/3600{\rm h}}=\frac{3600}{1000}{\kmh}=3.6\kmh\]
\end{solution}
\item 光在空气中的速度可以认为等于光在真空中的速
度.声音在空气中的速度是340$\ms$.一个人看到闪电后5
秒听到雷声,打雷的地方离他大约多远?

\begin{solution}
    设打雷的地方跟观侧者的距离为$s$, 声速$v=
340{\rm m/s}$,由于光速很快,可以认为闪电发出后,即刻被看到,于
是时间$t=5$秒即是雷声传播的时间,所以$s=vt=340
\x5=1700{\rm m}$.即打雷的地方离他1700米远.
\end{solution}

\end{enumerate}

\subsection{练习四}
\begin{figure}[htp]
    \centering\begin{minipage}[t]{0.48\textwidth}
\centering
    \begin{tikzpicture}[>=stealth,  thick, scale=.9]
    \draw [<->](0,5)node[right]{$s$(千米)}--(0,0)--(4,0)node[right]{$t$(小时)};
   
    \foreach \x in {1,2,3,...,12}
    {
        \draw(\x/4, 0) --(\x/4, .2);
    }
    
    \foreach \y in {1,2,3,...,8}
    {
        \draw(0,\y/2)--(.2, \y/2);
    }
    \node at (-.2,-.2){$0$};
    \node at (1.5,-.2){$1$};
    \node at (3,-.2){$2$};
    
    \node at (-.5,1){$200$};
    \node at (-.5,2){$400$};
    \node at (-.5,3){$600$};
    
    \draw [dashed] (1.5/2,0)--(1.5/2,1.5)--(0,1.5);
    \draw [dashed] (1.5,0)--(1.5,3)--(0,3);
    \draw [dashed] (1.5+1.5/6,0)--(1.5+1.5/6,3.5)--(0,3.5);
    \draw[ultra thick](0,0)--(1.5*1.5,4.5);
    
    \end{tikzpicture}
    
    \caption{}
\end{minipage}
\begin{minipage}[t]{0.48\textwidth}
\centering
\begin{tikzpicture}[>=stealth,  thick, scale=.9]
    \draw [<->](0,5)node[right]{$v$(千米/小时)}--(0,0)--(4,0)node[right]{$t$(小时)};
   
    \foreach \x in {1,2,3,...,12}
    {
        \draw(\x/4, 0) --(\x/4, .2);
    }
    
    \foreach \y in {1,2,3,...,8}
    {
        \draw(0,\y/2)--(.2, \y/2);
    }
    \node at (-.2,-.2){$0$};
    \node at (1.5,-.2){$1$};
    \node at (3,-.2){$2$};
    
    \node at (-.5,1){$200$};
    \node at (-.5,2){$400$};
    \node at (-.5,3){$600$};
    \node at (-.5,4){$800$};
    \draw[ultra thick](0,3)--(1.5*1.5,3);
    
    \end{tikzpicture}
    \caption{}
\end{minipage}
    \end{figure}

\begin{enumerate}
    \item 图2.6是一架民航飞机的位移图象.从这个图象求
    出
    \begin{enumerate}
        \item 飞机在30分钟内的位移;
        \item 飞行700千米所用的时间;
        \item 飞行速度并画出速度图象.
    \end{enumerate}

\begin{solution}
\begin{enumerate}
    \item 从位移图象上看出,
    时间为30分钟的点所对应的位
    移为300千米.
    \item 从位移图象上看出,位
    移为700千米的点所对应的时
    间为70分钟.
    \item 从位移图象上时间为1小时、位移为600千米的点,即
    可求得速度为600千米/小时.
    速度图象如图2.7所示.
\end{enumerate}
\end{solution}

\item  图2.8是一辆火车运动的位移图象.线段$OA$和$BC$
所表示的运动,哪个速度大?各等于多大?线段$AB$与横轴平
行,表示火车做什么运动?速度是多大?火车在3小时内的位移
是多少?通过80千米用多长时间?画出火车的速度图象.

\begin{solution}
    由于位移图象线段$OA$的斜率大于线段$BC$的斜率,
    所以$OA$线段所表示的运动速度大;从位移图象上$A$点的坐标可以求得
\[v_{OA}=\frac{90}{1.5}=60{\rm kmh}\]
从$B$、$C$两点的坐标可以求得
\[v_{BC}=\frac{140-90}{3-2}=50{\rm kmh}\]

    线段$AB$表示火车静止,速度为零.从$C$的坐标可知火
    车在3小时内的位移是140千米.在图线上取位移为80千
    米的点,可以求得火车通过这段位移需用80分钟.根据前述
    各段作出火车的速度图象如图2.9所示.
\end{solution}

\begin{figure}[htp]\centering
    \begin{minipage}[t]{0.48\textwidth}
    \centering
\begin{tikzpicture}[>=stealth,  thick, xscale=.8]
    \draw [<->](0,4.5)node[right]{$s$(千米)}--(0,0)--(4,0)node[right]{$t$(小时)};
   
    \foreach \x in {1,2,3,...,9}
    {
        \draw(\x/3, 0) --(\x/3, .2);
    }
    
    \foreach \y in {1,2,3,...,8}
    {
        \draw(0,\y/2)--(.2, \y/2);
    }
    \node at (-.2,-.2){$0$};
    \node at (1,-.2){$1$};
    \node at (2,-.2){$2$};
    \node at (3,-.2){$3$};
    
    \node at (-.5,.5){$20$};
    \node at (-.5,1.5){$60$};
    \node at (-.5,2.5){$100$};
    \node at (-.5,3.5){$140$};
    
    \draw [dashed] (1,0)--(1,1.5)--(0,1.5);
    \draw [dashed] (1+1/3,0)--(1+1/3,2)--(0,2);
    \draw [dashed] (3,0)--(3,3.5)--(0,3.5);
    \draw [dashed] (2,0)--(2,2.25);
    
    \draw[ultra thick](0,0)--(1.5 ,2.25);
    \draw[ultra thick](2,2.25)node[right]{$B$}--(1.5 ,2.25)node[left]{$A$};
    \draw[ultra thick](2,2.25)--(3,3.5)node[right]{$C$}--(4,4.75);
       
    \end{tikzpicture}
    \caption{}
    \end{minipage}
    \begin{minipage}[t]{0.48\textwidth}
    \centering
    \begin{tikzpicture}[>=latex, scale=1]
        \draw [<->](0,4.5)node[right]{$v$(千米/小时)}--(0,0)--(3.5,0)node[right]{$t$(小时)};
       
        \foreach \x in {1,2,...,6}
        {
            \draw(\x/2, 0) --(\x/2, .1);
        }
        
        \foreach \y in {1,2,3,...,8}
        {
            \draw(0,\y/2)--(.1, \y/2);
        }
        \node at (-.2,-.2){$0$};
\foreach \x in {1,2,3}
{
    \node at (\x,-.2){$\x$};
}
        \node at (-.25,1){$20$};
        \node at (-.25,2){$40$};
        \node at (-.25,3){$60$};
        \draw[ultra thick](0,3)--(1.5,3)node[above]{$A$};
        \draw[ultra thick](2,2.5)node[above]{$B$}--(3,2.5)node[above]{$C$};
        \draw[ultra thick](1.5,0)--(2,0);
\draw[dashed](1.5,3)--(1.5,0);
\draw[dashed](2,2.5)--(2,0);
\draw[dashed](3,2.5)--(3,0);
    \end{tikzpicture}
    \caption{}
    \end{minipage}
    \end{figure}

    \item  有两个物体,从同一点开始向相同方向做匀速运动,
    速度分别是3$\ms$和5$\ms$,在同一个坐标平面上画出它们
    的位移图象和速度图象,并根据这两种图象分别求出它们在5
    秒内的位移.
    
\begin{solution}
    题中所述的两个物体的位移图象如图2.10所示,速度图象如图2.11所示,从位移图象上可以看出,时间为5秒
    时,图线$A$上$P_A$点的位移为15米,图线$B$上$P_B$点的位移为
    25米.从速度图象上求得图线$A$5秒内的“面积”为$3\x5=15$米,图线$B$5秒内的“面积”为$5\x5=
    25$米,于是,从位移图象和速度图象得到的结果相同,即两个物体5秒内的位移分别是15米和25米.
\end{solution}

\begin{figure}[htp]\centering
    \begin{minipage}[t]{0.48\textwidth}
    \centering
\begin{tikzpicture}[>=latex, xscale=.7]
    \draw [<->](0,4)node[right]{$S$(米)}--(0,0)--(5.5,0)node[right]{$t$(秒)};
\foreach \x in {1,2,...,5}
{
    \draw(\x,0)node[below]{\x}--(\x,.1);
}
\foreach \x in {1,2,3}
{
    \draw(0,\x)node[left]{\x0}--(.1,\x);
    \draw(0,\x-.5)--(.1,\x-.5);
}
\draw[dashed](0,2.5)--(5,2.5)--(5,0);
\draw[dashed](0,1.5)--(5,1.5);
\draw[very thick](0,0)--(5,2.5)node[right]{$B$};
\draw[very thick](0,0)--(5,1.5)node[right]{$A$};
\node at (5,2.5)[above]{$P_B$};
\node at (5,1.5)[left]{$P_A$};
\node at (-.25,-.25){$O$};
    \end{tikzpicture}
    \caption{}
    \end{minipage}
    \begin{minipage}[t]{0.48\textwidth}
    \centering
    \begin{tikzpicture}[>=latex, scale=.7]
        \fill[pattern=north east lines](0,0) rectangle (5,5);
\fill[pattern=north west lines](0,0) rectangle (5,3);
  \draw [<->](0,6)node[right]{$v$(米/秒)}--(0,0)--(6,0)node[right]{$t$(秒)};     
\foreach \x in {1,2,...,5}
{
    \draw(\x, 0)node[below]{\x}--(\x,.1);
    \draw(0,\x)node[left]{\x}--(.1,\x);
}
\node at (-.25,-.25){$O$};
\draw[very thick](0,3)--(6,3)node[right]{$A$};
\draw[very thick](0,5)--(6,5)node[right]{$B$};
\draw[dashed](5,0)--(5,5);


    \end{tikzpicture}
    \caption{}
    \end{minipage}
    \end{figure}
\end{enumerate}


\subsection{练习五}

\begin{enumerate}
    \item 一辆汽车,起初以30$\kmh$的速度匀速行驶了30千米,然后又以60$\kmh$的速度匀速行驶了30千米.一位
    同学认为这辆汽车在这60千米中的平均速度是$1/2$(30千米/
    时+60$\kmh$)=45$\kmh$.这个结果对不对?

\begin{solution}
    按平均速度的定义应是位移与时间的比值,所以这
辆汽车的平均速度
\[\bar v=\frac{s_1+s_2}{t_1+t_2}\]
由于$s_1=30$千米,$s_2=30$千米,$t_1=1$小时,
$t_2=0.5$小时,所以
\[\bar v=\frac{60}{1.5}=40\kmh\]
因此,那位同学的看法是不对的.
\end{solution}
    \item 骑自行车的人沿着坡路下行,在第1秒内通过1米,
    第2秒内通过3米,在第3秒内通过5米,在第4秒内通过7米.求
    最初两秒内、最后两秒内以及全部运动时间内的平均速度.

    \begin{solution}
最初两秒:
\[\bar v=\frac{1+3}{2}=2\ms\]
最后两秒:
\[\bar v=\frac{5+7}{2}=6\ms\]
全部时间:
\[\bar v=\frac{1+3+5+7}{4}=4\ms\]
    \end{solution}
    \item 在一个速度是$v$的匀速直线运动中,各段时间内的
    平均速度以及整个运动的平均速度各是多大?每一时刻的即
    时速度是多大?

    \begin{solution}
        匀速直线运动是速度不变的运动,所以各段和全
部时间的平均速度以及每一时刻的即时速度都是$v$. 
    \end{solution}
    \item 火车以70$\kmh$的速度经过某一路标,子弹以
    600$\ms$的速度从枪筒射出.这里指的是什么速度?
    
\begin{solution}
    这里指的都是即时速度.
\end{solution}
\end{enumerate}

\subsection{练习六}
\begin{enumerate}
\item  加速度为零的运动是什么运动?
   

\begin{solution}
    是匀速直线运动.
\end{solution}
\item  有人说:速度越大表示加速度也越大.这话对吗?为什么?   

\begin{solution}
    不对,加速度是指速度变化的快慢,其大小取决于
    单位时间内速度变化的大小.高速运动的物体,速度虽然大,
    单位时间内速度的变化却不一定大,因而加速度也并不一
    定大.
\end{solution}
\item  汽车的加速性能是反映汽车质量的重要标志.汽车
从一定的初速度$v_0$加速列一定的末速度$v_t$,用的时间越少,表
明它的加速性能越好.下表是三种型号汽车的加速性能的实
验数据,求它们的加速度.

\begin{center}
\begin{tabular}{ccccc}
\hline
汽车型号 & 初速度$v_0$ & 末速度$v_t$ & 时间$t$ & 加速度$a$\\
& (km/h)& (km/h)& (s)& (m/s$^2$)\\
\hline
某型号高级轿车& 20& 50& 7 \\
某型号4吨载重汽车& 20& 50& 38\\
某型号8吨载重汽车& 20& 50& 50\\
\hline
\end{tabular}
\end{center}
   
\begin{solution}
    由公式$a=\dfrac{v_t-v_0}{t}$可得
\begin{enumerate}
    \item 高级轿车的加速度
    \[a_1=\frac{(50-20)\kmh}{7{\rm s}}=\frac{(50-20)\x 10^3{\rm m}}{7\x 3600{\rm s}}=1.19\msq\]
    \item 4吨载重汽车的加速度
    \[a_2=\frac{(50-20)\kmh}{38{\rm s}}=\frac{(50-20)\x 10^3{\rm m}}{38\x 3600{\rm s}}=0.22\msq\]
    \item 8吨载重汽车的加速度
    \[a_3=\frac{(50-20)\kmh}{50{\rm s}}=\frac{(50-20)\x 10^3{\rm m}}{50\x 3600{\rm s}}=0.17\msq\]
\end{enumerate}
\end{solution}
\item   以18$\ms$的速度行驶的火车,制动后经15秒停止,
求火车的加速度.
   

\begin{solution}
   \[a=\dfrac{v_t-v_0}{t}=\frac{0-18}{15}=-1.2\msq\]
\end{solution}
\end{enumerate}


\subsection{练习七}
\begin{enumerate}
\item 机车原来的速度是36$\kmh$,在一段下坡路上加
速度为$0.20\msq$,机车行驶到下坡末端,速度增加到54$\kmh$.求机车通过这段下坡路所用的时间.   

\begin{solution}
   \[a=\frac{v_t-v_0}{t},\qquad t=\frac{v_t-v_0}{a}\]
   由于$v_t=54\kmh=15\ms$,$v_0=36\kmh=10\ms$,$a=0.20\msq$,所以
\[t=\frac{15-10}{0.20}=25{\rm s}\]
\end{solution}
\item 一辆做匀变速运动的汽车,初速度是34$\kmh$,
4.0秒末速度变为42$\kmh$.如果保持加速度不变,6.0秒
末、7.0秒末的速度是多大?   

\begin{solution}
    用$v_0$和$v_4, v_6,v_7$分别表示汽车的初速度和4秒末、
    6秒末、7秒末的速度,则有
\[a=\frac{v_4-v_0}{4}=\frac{(42-34){\kmh}}{4{\rm s}}=\frac{8\kmh}{4{\rm s}}\]
所以
\[\begin{split}
    v_6&=v_0+a\x 6{\rm s}=34\kmh+\frac{8\kmh}{4{\rm s}}\x 6{\rm s}=46\kmh\\
    v_7&=v_0+a\x 7{\rm s}=34\kmh+\frac{8\kmh}{4{\rm s}}\x 7{\rm s}=48\kmh
\end{split}\]
\end{solution}
\item 匀变速运动的加速度是$-4.0\msq$.在某一时刻,
速度为$20\msq$.试求这一时刻后 4.0秒末和5.0秒末的速度.   

\begin{solution}
    由$v_t=v_0+at$,得
    4秒末的速度
    \[v_4=20{\rm m/s}+(-4.0\msq)\x4{\rm s}=4\ms\]
    5秒末的速度
    \[v_5=20{\rm m/s}+(-4.0\msq)\x5{\rm s}=0\ms\]
\end{solution}
\end{enumerate}


\subsection{练习八}
\begin{enumerate}
\item      钢球在斜槽上做初速度为零的匀变速运动,开始运动
后0.2秒内通过的路程是3.0厘米,1秒内通过的路程是多少?如
果斜面长1.5米,钢球由斜面顶端滚到底端需要多长时间?
   
\begin{solution}
由公式$s=\dfrac{1}{2}at^2$得$a=\dfrac{2s}{t^2}$.
把$t=0.2{\rm s}$,$s=3.0{\rm cm}=0.03{\rm m}$代入得
\[a=\frac{2\x 0.03}{(0.2)^2}=1.5\msq\]
由此得1秒内通过的路程
\[s_1=\frac{1}{2}at^2=\frac{1}{2}\x1.5\msq\x(1{\rm s})^2=0.75{\rm m}\]
又由公式$s=\dfrac{1}{2}at^2$得
\[t=\sqrt{\frac{2s}{a}}\]
把$a=1.5\msq$和$s=1.5{\rm m}$
代入可得钢球滚到斜面底端所需的时间
\[t= \sqrt{\frac{2\x 1.5{\rm m}}{1.5\msq}}=1.4{\rm s} \] 
\end{solution}

\item     飞机着陆后做匀变速运动,速度逐渐减小,已知初
速度是60$\ms$,加速度的大小是6.0$\msq$,求飞机着陆后5.0
秒内通过的路程.
   
\begin{solution}
\[\begin{split}
    s&=v_0t+\frac{1}{2}at^2\\
    &=60\x 5+\frac{1}{2}\x (-6.0)\x 5^2\\
    &=225{\rm m}
\end{split}\]
\end{solution}

\item     一辆汽车原来匀速行驶,然后1.0$\msq$的加速度
加快行驶,经12秒行驶了180米.汽车开始加速时的速度是多
大?
   
\begin{solution}
由公式$s=v_0 t+\dfrac{1}{2}at^2$, 可得$v_0=\dfrac{s}{t}-\dfrac{1}{2}at$.
把$s=180{\rm m}$,$t=12$s,$a=1.0\msq$代入上式得
\[v_0=\frac{180}{12}-\frac{1}{2}\x 1.0\x 12=15-6=9\ms\]   
\end{solution}

\item     骑自行车的人以5.0$\ms$的初速度登上斜坡,得到
$-40{\rm cm}/{\rm s}^2$的加速度,经过10秒钟,在斜坡上通过多长的
距离?
   

\begin{solution}
    由公式$s=v_0 t+\dfrac{1}{2}at^2$, 得:
    \[s=5\x 10+\frac{1}{2}\x(-0.4)\x (10)^2=30{\rm m}\]
\end{solution}
\item    汽车以36$\kmh$的速度行驶.刹车后得到的加
速度的大小为4$\msq$.从刹车开始,经过3秒钟,汽车通过的
距离是多少?
   
\begin{solution}
    解此题要注意两点:一是这里的加速度为负值;二是求出
    从刹车到车停止运动的时间$t$, 如果小于3秒,则求距离时用
    时间$t$; 如果大于3秒,则求距离时用的时间为3秒.
   
    由$a=\dfrac{v_t-v_0}{t}$
    ,得从刹车开始到车停止的时间$t=\dfrac{v_t-v_0}{a}$
    
    把$v_t=0$, $v_0=36\kmh=10\ms$代入上式得
\[a=\frac{0-10}{-4}=2.5{\rm s}\]
    所求的距离
    \[s=10\ms\x2.5{\rm s}+\frac{1}{2}\x (-4)\msq\x (2.5{\rm s})^2=12.5{\rm m}\]
\end{solution}

\end{enumerate}




\subsection{练习九}
\begin{enumerate}
\item 一个做匀变速运动的物体,初速度为3.0$\ms$,经过10秒钟,速度变为9.0$\ms$,它在这10秒钟内的平均速度是多大?   

\begin{solution}
由$\bar v=\dfrac{1}{2}(v_0+v_t)$,得:
\[\bar v=\frac{3+9}{2}=6\ms\]
\end{solution}
\item 从长3.0米的斜面顶端由静止滚下来的小球,末速度是2.5$\ms$,求小球滚动所用的时间.   

\begin{solution}
   由公式$s=\bar vt$和$\bar v=\dfrac{1}{2}(v_0+v_t)$,得
   \[s=\frac{v_0+v_t}{2}t\]
   所以
   \[t=\frac{2s}{v_0+v_t}=\frac{2\x 3.0}{0+2.5}=2.4{\rm s}\]
\end{solution}
\item 一辆汽车以12$\ms$的速度行驶,走到一个下坡,得到0.40$\msq$的加速度,汽车通过下坡末端的速度是16$\ms$,这个下坡的长度是多长?   

\begin{solution}
   由公式$v^2_t-v^2_0=2as$,得
   $$s=\dfrac{v^2_t-v^2_0}{2a}=\frac{16^2-12^2}{2\x 0.4}=140{\rm m}$$
\end{solution}
\item 子弹射中墙壁前的速度是400$\ms$,射到墙壁后穿进墙壁20厘米,子弹在墙内的运动可以看作匀变速运动,求子弹在墙壁内的加速度和运动时间.   

\begin{solution}
    由公式$v^2_t-v^2_0=2as$,得子弹在墙壁内的加速度
    $$a=\dfrac{v^2_t-v^2_0}{2s}=\frac{0-400^2}{2\x 0.20}=-4\x 10^5{\rm m/s^2}$$
由$s=\bar v t$,得子弹在墙壁内的运动时间
\[t=\frac{s}{\bar v}=\frac{2s}{v_0+v_t}=\frac{2\x 0.20}{400+0}=0.001{\rm s}\]
\end{solution}
\item 试证明做匀变速运动的物体在一段时间内的平均速度等于这段时间的中间时刻的即时速度.

\begin{proof}
方法一:由$s=\bar v t$和$s=v_0t+\dfrac{1}{2}at^2$, 可得
\[\bar v=\frac{s}{t}=v_0+\frac{1}{2}at =v_0+a\left(\frac{t}{2}\right)\]
即$\bar v$等于求平均速度这段时间的中间时刻的即时速度.

方法二:由$v_t=v_0+at$, 可得中间时刻的即时速度$v'$
\[v'=v_0+a\left(\frac{t}{2}\right)=v_0+\frac{1}{2}at =\frac{1}{2}(v_0+v_0+at)=\frac{1}{2}(v_0+v_t)=\bar v\]
\end{proof}
\item 做匀变速运动的物体,在各个连续相等时间$t$内的位移分别是$s_1, s_2, s_3,\ldots,s_n$.如果加速度是$a$,试证明:
\[\Delta s=s_2-s_1=s_3-s_2=\cdots=s_n-s_{n-1}=at^2 \]

\begin{solution}
    设初速度为$v_0$, 由$s=v_0t+\dfrac{1}{2}at^2$和$v_t=v_0+at$,得:
\[\begin{split}
    s_1&=v_0t+\frac{1}{2}at^2\\
    s_2&=(v_0+at)t+\frac{1}{2}at^2=v_0t+\frac{1}{2}at^2+at^2\\
    s_3&=(v_0+at+at)t+\frac{1}{2}at^2=(v_0+2at)t+\frac{1}{2}at^2=v_0t+\frac{1}{2}at^2+2at^2\\
s_{n-1}&=[v_0+(n-2)at]t+\frac{1}{2}at^2=v_0 t+\frac{1}{2}at^2+(n-2)at^2\\
s_{n}&=[v_0+(n-1)at]t+\frac{1}{2}at^2=v_0 t+\frac{1}{2}at^2+(n-1)at^2\\
\end{split}\]
$\therefore\quad \Delta s=s_2-s_1=s_3-s_2=\cdots=s_n-s_{n-1}=at^2$
\end{solution}
\end{enumerate}


\subsection{练习十}
\begin{enumerate}
	\item 为了测出井口到井里水面的深度,让一个小石块从井口落下,经过2.0秒后听到石块落到水面的声音,求井口到水面的大约深度(不考虑声音传播所用的时间).

    \begin{solution}
 由公式$s=\dfrac{1}{2}gt^2$,得: 
 \[s=\frac{1}{2}\x 9.8\x 2^2=19.6{\rm m}\]      
    \end{solution}
\item 一个自由下落的物体,到达地面的速度是39.2$\ms$,这个物体是从多高落下的?落到地面用了多长时间?

\begin{solution}
由$v^2_t=2gs$,得物体下落时的高度
\[s=\frac{v^2_t}{2g}=\frac{39.2^2}{2\x 9.8}=78.4{\rm m}\]
由$v_t=gt$,得物体落到地面所用的时间
\[t=\frac{v_t}{g}=\frac{39.2}{9.8}=4{\rm s}\]
\end{solution}
\item 一个物体从22.5米高的地方下落,到达地面时的速度是多大?下落最后1秒内的位移是多大?

\begin{solution}
    由$v^2_t=2gs$,得物体到达地面时的速度
\[v_t=\sqrt{2gs}=\sqrt{2\x 9.8\x 22.5}=21{\rm m/s}\]
由$v_t=gt$,得物体下落的时间 
\[t=\frac{v_t}{g}=\frac{21}{9.8}=2.14{\rm s}\]
前1.14秒内位移:
\[s'=\frac{1}{2}gt^2=\frac{1}{2}\x 9.8\x 1.14^2=6.4{\rm m}\]
最后1秒内位移:
\[s=22.5-6.4=16.1{\rm m}\]
\end{solution}
\end{enumerate}




\subsection{练习十一}

\begin{enumerate}
	\item 在竖直上抛运动中,$v_t$与$v_0$何时方向相同,何时相反?$v_t$与$a$何时方向相同,何时相反?

    \begin{solution}
        在上升运动中,$v_t$和$v_0$方向相同;$v_t$和$a$方向相反.
        在下降运动中,$v_t$和$v_0$方向相反;$v_t$和$a$方向相同. 
    \end{solution}
\item 竖直向上射出的箭,初速度是35$\ms$,上升的最大高度是多大?从射出到落回原地一共用多长时间?落回原地的速度是多大?

\begin{solution}
箭上升到最大高度$H$时,$v_t=0$, 由此得$v^2_0=2gH$,
所以
\[H=\frac{v^2_0}{2g}=\frac{35^2}{2\x 10}=61{\rm m}\]
由于$v_t=v_0+gt=0$, 箭的上升时间
\[t=\frac{v_0}{g}=\frac{35}{10}=3.5{\rm s}\]
由射出到落回原地共用时间$T=2t=2\x3.5=70{\rm s}$.
落回原地速度跟抛出的初速度大小相等即$35\ms$.

说明:为了计算的方便,解题时取$g=10\msq$, 以下两
题同.
\end{solution}
\item 竖直上抛的物体,初速度是30$\ms$,经过2.0秒、3.0秒、4.0秒,物体的位移分别是多大?通过的路程分别是多长?各秒末的速度分别是多大?

\begin{solution}
上升的最大高度
\[H=\frac{v^2_0}{g}=\frac{30^2}{2\x 10}=45{\rm m}\]
由$s=v_0t-\dfrac{1}{2}gt$得:
\begin{enumerate}
    \item 当$t=2.0$s时,位移
    \[s=30\ms \x2.0{\rm s}-\frac{1}{2}\x 10\msq\x(2.0{\rm s})^2=40{\rm m}<H\]
    $\therefore\quad $路程$s'=40{\rm m}$.$v_t=v_0-gt=30-10\x2.0=10\ms$.
\item 当$t=3.0$s时,位移$$s=30\ms\x3.0{\rm s}-\frac{1}{2}
\x10\msq\x(3.0{\rm s})^2=45{\rm m}=H$$
$\therefore\quad $路程$s'=45{\rm m}$.$v_t=v_0-gt=30-10\x3.0=0$.
\item 当$t=4.0$s时,位移$$s=30\ms\x4.0{\rm s}-\frac{1}{2}
\x10\msq\x(4.0{\rm s})^2=40{\rm m}<H$$
$\therefore\quad $路程$s'=45+(45-40)=50{\rm m}$.
$v_t=v_0-gt=30-10\x4.0=-10\ms$.
\end{enumerate}

\end{solution}
\item 在课文的例题中,求经过1秒后石子离地面的高度以及石子这时的速度.先分上升运动和下降运动两步来计算,再用统一的公式来计算,并加以比较.

\begin{solution}
\begin{enumerate}
    \item 石子上升时间
    \[t_1=\frac{v_0}{g}=\frac{4}{10}=0.4{\rm s}\]
    1秒内石子的下降时间$$t_2=1-0.4=0.6{\rm s}$$

    石子上升的最大高度
\[h_1=\frac{v_0^2}{2g}=\frac{4^2}{2\x 10}=0.8{\rm m}\]
0.6
秒内石子下降的高度
\[h_2=\frac{1}{2}gt^2=\frac{1}{2}\x 10\x0.6^2=1.8{\rm m}\]
石子这时的速度$v_t=gt=10\x0.6=6\ms$,方向
向下.
\item 由$s=v_0t-\dfrac{1}{2}gt^2$,
得石子在抛出1秒后的高度
\[s=4\x1-\frac{1}{2}\x10\x1^2=-1{\rm m}\]
即在抛出点下方1米处,离地面高度为$15-1=14{\rm m}$.

石子这时的速度$v_t=v_0-gt=4-10\x1=-6\ms$.
\end{enumerate}

比较:解(b)较解(a)简单,但求出的石子的速度和相对于抛
出点的位移,都是负值,必须搞清它们的物理意义,关键在于
掌握各矢量的方向.
\end{solution}
\end{enumerate}




\subsection{习题}
\begin{enumerate}
	\item 物体的加速度为零时,它的速度是否一定为零?物体的速度为零时,它的加速度是否一定为零?各举一个例子.

    \begin{solution}
        物体的加速度为零时,速度不一定为零,例如火车在
        平直轨道上匀速行驶,物体的速度为零时,它的加速度不一
        定为零,例如竖直上抛运动中当物体到达最高点时速度为零,
        加速度为$g=9.8\msq$.   
    \end{solution}
	\item 汽车以26$\kmh$的速度行驶了2小时,跟目的地还有一半路程,要想在40分钟内到达目的地,在后一半路程中汽车应该以多大速度行驶?

    \begin{solution}
        由$s_1=v_1t_1$, 得$s_1=25\x2=50{\rm km}$.
        因此,汽车在40分钟内走完另一半路程所需的速度为
    \[v_2=\frac{s_2}{t_2}=\frac{50{\rm km}}{40{\rm min}}=75\kmh\]    
    \end{solution}
	\item 矿井里的升降机,从静止开始加速上升,经过3秒速度达到3$\ms$,然后以这个速度匀速上升25秒,最后减速上升,经过2秒到达井口时,正好停下来,求矿井深度.

    \begin{solution}
        矿井的深度等于升降机各段上升高度之和,即
\[\begin{split}
     s&=v_1t_1+v_2t_2+v_3t_3\\
     &=\frac{0+3}{2}\x 3+3\x 25+\frac{3+0}{2}\x 2\\
     &=4.5+75+3=82.5{\rm m}
\end{split}\]
    \end{solution}
	\item 一架飞机以7.0$\msq$的加速度做匀加速飞行,计算它的速度由240$\kmh$增加到600$\kmh$所发生的位移和所用的时间.

    \begin{solution}
        初速$v_0=240\kmh=\dfrac{200}{3}\ms$,末速$v_t=600\kmh=\dfrac{500}{3}\ms$

由$a=\dfrac{v_t-v_0}{t}$,得所用的时间
\[t=\frac{v_t-v_0}{a}=\frac{\frac{500}{3}+\frac{200}{3}}{7.0}=14.3{\rm s}\]
这段时间内飞机的位移
\[s=vt=\frac{1}{2}\left(\frac{200}{3}+\frac{500}{3}\right)\x 14.3=1.67\x 10^3{\rm m}\]
    \end{solution}
	\item 火车制动后经过20秒停下来,在这段时间内前进120米.求火车开始制动时的速度和火车的加速度.

    \begin{solution}
由公式$s=\bar v t$和$\bar v=\dfrac{v_0+v_t}{2}$
解得开始制动时的速度
\[v_0=\frac{2s}{t}-v_t\]
由于$v_2=0$, 所以
\[v_0=\frac{2s}{t}=\frac{2\x 120}{20}=12\ms\]
火车的加速度
\[a=\frac{v_t-v_0}{t}=\frac{0-12}{20}=-0.6\msq\]        
    \end{solution}
	\item 汽车从静止开始做匀变速运动,通过一段距离,速度达到14$\ms$,汽车通过这段距离的一半时,速度是多大?

    \begin{solution}
        设汽车的加速度为$a$, 通过距离$s$获得的末速度为$v$, 通过的距离为$s$之半时,获得的速度为$v'$. 由于$v_0=0$, 所
        以有
\[\begin{split}
    v^2&=2as\\
    {v'}^2&=2a\left(\frac{s}{2}\right)=as
\end{split}\]
$\therefore\quad \dfrac{ {v'}^2}{v^2}=\dfrac{as}{2as}=\dfrac{1}{2}$,
由此得:
\[v'=\frac{\sqrt{2}}{2}v\]
已知$v=14\ms$,$\therefore\quad v'=\dfrac{\sqrt{2}}{2}\x 14=9.9\ms$
\end{solution}
	\item 一个物体从塔顶上下落,在到达地面前最后一秒内通过的位移是整个位移的9/25.求塔高.

    \begin{solution}
设塔高为$s$米,下落时间为$t$秒,因此$s=\dfrac{1}{2}gt^2$,
同理可知物体在$(t-1)$秒内落下的距离为
\[s'=\frac{1}{2}g(t-1)^2\]
由题设知
\[\frac{s'}{s}=\frac{25-9}{25}=\frac{16}{25}\]
所以
\[\frac{(t-1)^2}{t^2}=\frac{16}{25}\]
即\[\frac{t-1}{t}=\frac{4}{5}\]
解得$t=5$s.

因此塔高
\[s=\frac{1}{2}gt^2=\frac{1}{2}\x10\x25=125{\rm m}\]
    \end{solution}
	\item 自由落下的物体在某一点速度是19.6$\ms$,在另一点的速度是39.2$\ms$.求这两点间的距离和经过这段距离所用的时间.

    \begin{solution}
由公式$v^2=2gs$, 得物体速度为$19.6\ms$时下降的
高度为
\[s_1=\frac{v^2_1}{2g}=\frac{19.6^2}{2\x 9.8}=19.6{\rm m}\]
物体速度为39.2米时下降的高度为
\[s_2=\frac{v^2_2}{2g}=\frac{39.2^2}{2\x 9.8}=78.4{\rm m}\]
两点之间的距离$\Delta s=s_2-s_1=78.4-19.6=58.8{\rm m}$

由$\Delta s=\dfrac{v_1+v_2}{2}t$, 可得经过这段距离所用的时间
\[t=\frac{2\Delta s}{v_1+v_2}=\frac{2\x 58.8}{19.6+39.2}=2{\rm s}\]
    \end{solution}
	\item 一个竖直上抛的物体,经过4.0秒落回原地,经过1.0秒,2.0秒,3.0秒,物体的速度分别是多大?物体的位移分别是多大?通过的路程分别是多长?

    \begin{solution}
已知物体上抛,落回原地的时间$T=4$s,而上升时
间$t'$等于下落时间$t''$,所以$t'=t''=T/2=2$s.

由于物体上抛到最高点的速度$v_{t'}=0$, 则由公式$v_t=v0
-gt$, 得上抛的初速度$v_0=gt'=10\x2=20\ms$.

由公式$v_t=v_0-gt$和$s=v_0t-\dfrac{1}{2}gt^2$得:
\begin{enumerate}
    \item 经过1秒物体的速度
    $$v_1=20-10\x1=10\ms$$
    物体的位移
    $$s_1=20\x1-\frac{1}{2}\x 10\x1^2=15{\rm m}$$
    经过的路程也是15m.
\item 经过2秒物体的速度
\[v_2=20-10\x2=0\]
物体的位移
\[s_2=20\x 2-\frac{1}{2}\x 10\x 2^2=20{\rm m}\]
经过的路程也是20m.
\item 经过3秒物体的速度
\[v_3=20-10\x3=-10\ms\]
物体的位移
\[s_3=20\x3-\frac{1}{2}\x 10\x3^2=15{\rm m}\]
通过的路程
\[20+(20-15)=25{\rm m}\]
\end{enumerate}

    \end{solution}
	\item 气球以10$\ms$的速度匀速竖直上升,从气球上掉下一个物体,经17秒到达地面.求物体刚脱离气球时气球的高度.

    \begin{solution}
        物体从气球上掉下后到达地面时的位移为
\[\begin{split}
    s&=v_0t-\frac{1}{2}gt^2\\
    &=10\x 17-\frac{1}{2}\x 10\x 17^2\\
    &=-1275{\rm m}
\end{split}\]
所以,物体刚脱离气球时气球的高度为1275米.
    \end{solution}
	\item 初速度为零的匀变速运动,在第1秒内、第2秒内、第
	3秒内……的位移分别是$s_I,s_{II},s_{III},\ldots$.试证明:$s_I,s_{II},s_{III},\ldots$之比等于从1开始的连续奇数之比,即:
$$s_I:s_{II}:s_{III}\cdots=1:3:5\cdots$$

提示:设物体在1秒内、2秒内、3秒内……发生的位移是
$s_1,s_2,s_3,\ldots$,那么
$$s_I=s_1, s_{II}=s_2-s_1, s_{III}=s_3-s_2,\ldots$$

\begin{solution}
\[\begin{split}
    \text{1秒内位移:} & s_1=\frac{1}{2}at^2_1, \qquad t_1=1\\
    \text{2秒内位移:} & s_2=\frac{1}{2}at^2_2=\frac{1}{2}a(2t_1)^2\\
    \text{3秒内位移:} & s_3=\frac{1}{2}at^2_3=\frac{1}{2}a(3t_1)^2\\
    \vdots& \qquad \vdots
\end{split}\]
\[\begin{split}
    \text{第1秒内位移:} & s_I=s_1=\frac{1}{2}at^2_1=1\left(\frac{1}{2}at^2_1\right)\\
    \text{第2秒内位移:} & s_{II}=s_2-s_1=\frac{1}{2}a\left[(2t_1)^2-t^2_1\right]=3\left(\frac{1}{2}at^2_1\right)\\
    \text{第3秒内位移:} & s_{III}=s_3-s_2=\frac{1}{2}a\left[(3t_1)^2-(2t_1)^2\right]=5\left(\frac{1}{2}at^2_1\right)\\
    \vdots& \qquad \vdots
\end{split}\]
$\therefore\quad s_I:s_{II}:s_{III}\cdots=1:3:5\cdots$
\end{solution}
	\item 从楼顶上落下一个铅球,通过1米高的窗子用了0.1秒的时间.楼顶比窗台高多少米?

    \begin{solution}
设从楼顶到窗台的距离为$H$. 铅球从下落到通过窗
子上沿和窗台所用的时间为$t_1$和$t_2$, 即时速度分别为$v_1$和
$v_2$.

从自由落体速度公式可得$v_1=gt_1$, $v_2=gt_2$, 
设窗高为$h$($=1$米),铅球通过$h$的平均速度
\[\bar v=\frac{v_1+v_2}{2}=\frac{g(t_1+t_2)}{2}\]
从题设条件可知
\[\bar v=\frac{1}{0.1}=10\ms\]
所以
\[\frac{g}{2}(t_1+t_2)=10\ms\]
\[t_1+t_2=\frac{10\x 2}{g}=\frac{20}{10}=2{\rm s}\]
由题设条件还可知道$t_2-t_1=0.1$s,所以$t_2=1.05{\rm s}$.

因此,从楼顶到窗台的距离
\[H=\frac{1}{2}gt^2=\frac{1}{2}\x 10\x 1.05^2=5.5{\rm m}\]   \end{solution}
\end{enumerate}

\section{参考资料}
\subsection{伽利略的比萨斜塔实验}
许多著作记述了伽利略曾经做过比萨斜塔实验.

比萨是位于意大利半岛北部地区的一座古城.在流过这
座古城的阿诺河畔矗立着高56米的比萨斜塔.始建于1174
年,14世纪竣工.由于塔基问题,塔身发生了倾斜.据说,年
青的伽利略为了证明自己的论断,邀请了许多人到斜塔旁观
看他的实验.伽利略在塔上拿着两个质量相差很大而体积相
同的硬木球和铁球,让它们同时从手中自由下落,结果两个球
同时触地.于是,两千年来人们一直信奉的亚里士多德的观
点:重的物体落得快,轻的物体落得慢,终于被事实否定了,比
萨斜塔实验广为流传,比萨斜塔也随着伽利略在科学上的成
就而闻名于世.

但是,科学史家对伽利略是否在比萨斜塔上做过落体实
验持有不同的看法.从上世纪后期一直在争论着,至今仍是
悬而未决的疑案.

认为伽利略做过比萨斜塔实验的根据是维维安尼所写的
《伽利略传》(1654年出版).维维安尼是和伽利略晚年一起
生活的学生,他手中有伽利略的许多笔记和书信.另外,在伽
利略的著作中几处都提到由高塔上坠落重物的事.因此,有
人推测伽利略有可能做过比萨斜塔实验.

认为伽利略没有做过比萨斜塔实验的理由是伽利略无须
通过落体实验,只要采用逻辑推理的方法就可以否定亚里士
多德的观点.而且在伽利略的著作中找不到这个实验的记
载.在跟维维安尼同时代的其他历史资料中也没有这个实验
材料、因此,有人怀疑维维安尼有可能把别人的实验误记到伽
利略的名下,如说伽利略实验用的两个球,其中一个比另一个
重10倍,这跟比利时物理学家斯台文所做的落体实验情况
相同.

这两种看法都还缺乏充分的证据.在没有发现新的历史
资料的情况下,是难于统一认识的.

\subsection{用逐差法求加速度值}
课本练习九第6题已经证明:$$\Delta s=s_2-s_1=s_3-
s_2=\cdots=s_n-s_{n-1}=at^2$$ 

这样,在做学生实验七研究匀变速
直线运动的时候,是否可以根据相邻的距离之差$\Delta s_1,\Delta s_2,\Delta s_3,\ldots,\Delta s_{n-1}$, 分别除以$T^2$, 再取其平均值,从而得出加速
度$a$的值呢?下面看一下这个求解过程.
\[\begin{split}
    a&=\frac{\Delta s_1+\Delta s_2+\Delta s_3+\cdots+\Delta s_{n-1}}{T^2(n-1)}\\
    &=\frac{(s_2-s_1)+(s_3-s_2)+(s_4-s_3)+\cdots+(s_n-s_{n-1})}{T^2(n-1)}\\
    &=\frac{s_n-s_1}{T^2(n-1)}
\end{split}\]
可以看出,中间的各数值$s_2,s_3,\ldots, s_{n-1}$在平均过程中都已消
去,不起作用,只有首尾两个数值$s_1$和$s_n$才起作用.这样,也
就不能起到利用多个数据来减少偶然误差的作用.如果$s_1$和
$s_n$的误差很大,则求出的$a$误差也就很大了.

实际处理数据是用逐差法,把连续的数据前后对半分成
两组,将后一半的第一个数据与前一半的第一个数据相减,后
一半的第二个数据与前一半的第二个数据相减……下面我
们看一下这个求解过程.

把$s_1,s_2,s_3,\ldots,s_n$对半分为两组,每组有$m=n/2$
个数据,
前一半为$s_1,s_2,\ldots,s_m$, 后一半为$s_{m+1},s_{m+2},\ldots,s_n$, 相应的差值
是$\Delta s_1=s_{m+1}-s_1,\Delta s_2=s_{m+2}-s_2,\ldots, \Delta s_m=s_{n}-s_m$.由这些差值
求得的加速度值分别是:
\[a_1=\frac{\Delta s_1}{mT^2},\quad a_2=\frac{\Delta s_2}{mT^2},\ldots, a_m=\frac{\Delta s_m}{mT^2} \]

因此,取其平均值求得的加速度
\[\begin{split}
   a&=\frac{a_1+a_2+\cdots+a_m}{m}=\frac{\Delta s_1+\Delta s_2+\cdots+\Delta s_m}{m^2T^2}\\
   &=\frac{(s_{m+1}-s_1)+(s_{m+2}-s_2)+\cdots+(s_{n}-s_m)}{m^2T^2}\\
   &=\frac{(s_{m+1}+s_{m+2}+\cdots+s_n)-(s_1+s_2+\cdots+s_m)}{m^2T^2} 
\end{split}\]
可以看出,所有数据$s_1,s_2,\ldots,s_n$都得到了利用,因而减少了偶
然误差.

\subsection{物理学中的理想化方法、理想化模型和理想实验}

物理学研究对象受许多因素影响,但在一定条件下可以
抓住其主要因素和本质,将其他因素撇开,在此基础上进行抽
象概括,把错综复杂的问题归结为比较简单的问题进行研究,
这就是物理学研究中的理想化方法,用这种方法把研究对象
简化成的抽象模型就是物理学中的理想模型.用这种方法进
行的假想实验就是理想实验.

理想化模型在物理学研究中被广泛应用,常见的理想化
模型有:质点、刚体、弹性体、塑性体、理想气体、理想流体、弹
簧振子、单摆、点电荷、试探电荷、无限长直导线、无限大平板、
点磁荷、纯电阻(纯电感、纯电容)、光线、薄透镜、点光源等,在
各类物理书籍里,有时清楚地说明所讨论的是哪一种理想模
型(如点电荷、光线等)或者对实物附加某些说明(如“两球作
完全弹性碰撞”等).但在大多数场合,都不加说明,要我们
自己判定.不过在大多数情况下还是有一定规律可循的,如
质点力学中列举的粒子、小球、子弹、汽车、火箭、地球以至太
阳,应看作质点.在刚体力学中提到的杠杆、飞轮、圆板、皮带
轮,都看作是刚体.在流体力学中研究的空气和水,看作是理
想流体.

在用理想化方法处理问题时,考虑什么因素,舍去什么因
素不是固定不变的,随之研究对象的实际情况、研究范围和条
件的变动而变动.因此同一个物体(研究对象)可以看成不同
的理想模型,从而也会得出不同的结论.例如一个物体在不
同情况下可分别视为质点、刚体或弹性体.我们求一根搁在
墙角上的均匀铁棒里的应力时,必须先把铁棒看成是一个质
量集中于重心$O$的质点,求出重力,然后把铁棒看成刚体,利
用力的平衡条件求出各个未知力,再把铁棒看成弹性体,用材
料力学的方法求出应力.假如不这样逐步选取不同的理想模
型,这个问题是解不出的.

理想实验在物理学发展中具有极其重要的地位和意义.这是因为日常的具体事物虽然直观,但各种现象、因素、过程
交织在一起,往往掩盖事物本质的一面.运用理想实验进行
研究所揭示的特性和规律,只能以抽象的形式出现,虽然它与
具体现象远了,但离真理近了.普朗克曾说过“物理世界观之
愈益远离感性世界,无非就是与现实世界愈益接近”.例如经
典物理的鼻祖伽利略就是通过理想斜面实验揭示了惯性定律
的物理本质(课本图3.1).而爱因斯坦提出狭义相对论的基
础,同时性的相对性就是通过理想实验形象地加以说明的,此
外海森堡提出的“测不准原理”也运用了理想化实验.






 
\chapter{运动定律}
\section{教学要求}
这一章学习动力学,讨论运动和力的关系。牛顿运动定
律是动力学的基础,这一章学习的中心内容就是牛顿运动
定律。

这一章的教学要求是:
\begin{enumerate}

\item 掌握牛顿第一定律,明确力、惯性等概念的物理意义.
\item 正确理解和掌握牛顿第二定律;掌握力学单位制;了
解超重和失重。
\item 会运用运动学公式、力的合成和分解以及牛顿运动定
律分析解决综合性问题。
\end{enumerate}

下面对这一章的教学内容作些具体说明。

运动和力的关系问题是动力学的基本问题,要建立正确
的认识又是颇不容易的,学生从日常经验出发,往往产生错
误认识,有的认识同历史上前人产生过的错误有类似之处。在
讲述牛顿第一定律之前,介绍人类对这个问题的认识过程,可
以帮助学生理解和掌握运动和力的关系。课本里介绍了伽利
略的理想实验,使学生知道伽利略是怎样得出正确结论的,并
了解在实验事实的基础进行科学思维的研究方法,以培养他
们的思维能力。

为了使教学从牛顿第一定律比较自然地过渡到牛顿第二
定律,在讲过牛顿第一定律之后,用一节教材讲述怎样使物体
的运动状态发生改变。这里,把力和运动联系起来,指出力是
使物体产生加速度的原因,加深学生对力这个概念的理解。这
一节,还讲述了质量与运动状态改变的关系,明确质量在动力
学中的意义,从而为讲述牛顿第二定律做好准备。

牛顿第二定律是这一章的重点内容。采用在教师指导下
由学生自己动手做实验,最后得出结论的做法来讲述牛顿第
二定律,可以调动学生的学习积极性和主动性,培养学生通过
实验研究问题的能力,并使学生对牛顿第二定律有深刻的
印象。

在讲述加速度和力的关系时,不可忽略它们的方向总是
一致的,这对后面研究曲线运动很重要。

以后学习匀速圆周运动、简谐振动等部分时,要学到变力
和变加速度的概念。有必要指出牛顿第二定律不仅适用于恒
力作用下的运动,而且适用于变力作用下的运动;外力随时间
改变时,加速度也随时间改变。但不要求讲述即时加速度这
个概念。

讲述力的独立作用原理,可以使学生进一步掌握在有几
个力作用时如何应用牛顿第二定律。这里,要紧的是懂得加
速度的大小和方向是由合外力决定的。

牛顿第二定律的应用有两个方面:一是已知物体的受力
情况,确定物体的运动情况;二是已知物体的运动情况,确
定物体的受力情况。利用动力学知识,知道了力和加速度,
也可以确定物体的质量。习题中带解的题目,可以使学生对
用动力学方法确定质量有所了解。

在演示实验的基础上讲述超重和失重时,涉及到弹簧秤
和被测物体组成的一个连接体。这里,只要明确研究对象是被
测物体,分析这个物体的受力情况,列出方程,再应用牛顿第
三定律,问题也就解决了。在这里既不必分别列出弹簧秤和
被测物体的动力学方程,更不要系统地讲解连解体问题。通过
超重和失重这一节的学习,还应使学生知道,某些动力学问题
常常需要综合运用牛顿第二定律和牛顿第三定律才能解决。

关于牛顿运动定律的适用范围,应要求学生有初步的了
解。限于学生的水平,这个问题不可能作深入的讨论,对于刚
体、相对论、量子力学等名词,可以只稍加说明,也不要求学生
记住这些名词。

这一章的练习和习题是按照循序渐进的原则,由浅入深,
由易到难来安排的。具体的安排是:在得出$a\propto F$和$a\propto \dfrac{1}{m}$两个公式后,安排了仅限于应用上述正比、反比关系的题目;在
得出牛顿第二定律的公式$F_{\text{合}}=ma$之后,安排了由几个力的
作用使物体产生加速度的题目;在“力学单位制”一节后面,安
排了物体受一个力作用时用动力学和运动学的知识求解的题
目;在讲完全章内容后,安排了在多个力作用下用动力学和运
动学知识求解的较复杂的综合性题目。

\section{教学建议}
这一章可分三个单元进行教学。第一单元包括引言和第
一节《牛顿第一定律》;第二单元从第二节《物体运动状态的改
变》到第七节《力学单位制》;第三单元从第八节《牛顿运动定
律的应用(一)》到第十一节《牛顿运动定律的适用范围》。

\subsection{第一单元}
在这一单元的教学中,首先可以指导学生认真阅读“历史
的回顾”这段课文,认识牛顿第一定律是怎样在伽利略理想实
验的基础上总结出来的。

\subsubsection{力不是维持运动的原因}

这一内容的教学,可以从学
生的生活经验提出所要研究的问题。譬如为了使足球不致停
下来,运动员带球前进时必须不断用脚轻轻地拨动球;又如为
了自行车不致减慢速度,总要不断地用力蹬脚踏板。让学生
带着这些问题来阅读“历史的回顾”这一段课文,然后演示教
材上图3.1的斜面对接实验(可用粗铁丝弯制成导轨来代
替对接的斜面).在演示图3.1丙的实验时,也可以使一辆
小车从斜面的一定高度上下滑后,在水平面上运动,让学生观
察在阻力比较大的水平面上,小车速度减慢得比较快,在阻力
不很大的平面上,小车速度减慢得比较慢,从而认识小车在
平面上运动所以会减慢速度,恰是由于阻力的作用,阻力的
大小不同使小车速度变慢的快慢程度也不同。同样,足球在
草地上滚动速度所以会变慢,不用力蹬脚踏板自行车所以会
减慢速度,也说明有阻力的作用,因此力不是维持运动的原
因,而是改变物体运动状态的原因。

讲清这个问题,还可以使学生体会到来自生活经验的直
观感觉不一定都是正确的,要学会用科学的思想方法去分析
现象,掌握事物发展变化的规律。

在教学中要引导学生认识伽利略的理想实验是以可靠事
实为基础的科学推论。可引导学生思考,刚才所做的演示实
验中如果水平面是光滑的,那么小车在水平面上运动时,它的
运动状态是否还会发生变化呢?小车将做什么运动?接着可以
进一步演示,使小车从斜面上滑下后在一块玻璃板上运动,可
以观察到小车运动的速度几乎没有变化,以加深对伽利略理
想实验的事实依据的认识。

\subsubsection{惯性和牛顿第一定律}

要引导学生对惯性的概念有
正确的理解。有的学生认为只有运动的物体才具有惯性;也
有的学生认为只有静止的物体才具有惯性,这些看法都是不
正确的。要使他们认识惯性是物体的固有属性,它和物体的
运动状态无关;但在力的作用下,物体运动状态改变的快慢
程度却与惯性有着直接的关系(牛顿第二定律所要研究的内
容)。而牛顿第一定律则阐明了物体的匀速直线运动状态或
静止状态都不需要力来维持,外力的作用只是改变物体的运
动状态,并不能改变物体的惯性。

\subsubsection{牛顿第一定律描述的是一种理想化的状态}

不受外
力作用的物体是不存在的。要使学生认识,即使前面所做的
演示实验中,当小车在光滑的水平面上运动时,不受任何阻
力,小车做匀速直线运动,仍然不属于牛顿第一定律所描述的
理想化状态。因为小车在水平面上运动时还受到重力和水平
面的支持力的作用,只是在水平方向上不受力,所以小车在水
平方向上的运动状态并不发生改变,牛顿第一定律所描述的
是一种理想化的状态,不同于可以用实验直接验证的牛顿第
二定律。

\subsection{第二单元}
这一单元以讲解牛顿第二定律为中心,是本章的重点。
\subsubsection{物体运动状态的改变}
教学中应该让学生认识,物体的运动状态决定于物体的
速度,根据牛顿第一定律可知,物体运动状态发生改变,必定
是外力作用的结果,为了帮助学生加深理解,可以举这样的
例子:要使静止在粗糙水平面上的物体开始运动,必须有外力
的作用,而且这个外力必须等于、大于物体与水平面间的最大
静摩擦力,物体才能改变它的运动状态,从静止开始运动;接
着,如果将所加的外力撤去,则物体只受到跟运动方向相反的
滑动摩擦力的作用,所以立即又改变它的运动状态,使运动速
度逐渐减小直到停止运动。


\subsubsection{牛顿第二定律}

为了加深对这一重要定律的理解,培
养学生用实验手段研究问题的能力,教材安排先由学生进行
实验,再由实验结果得出牛顿第二定律,在这个基础上,应使
学生认识以下几点:
\begin{enumerate}
    \item 力是使物体产生加速度的原因.加速度和力存在着
这样的关系:有力作用就有加速度产生,恒定的力产生恒定的
加速度,力发生变化,加速度随即发生相应的变化,力停止作
用,加速度立即等于零,而且不论在哪种情况下,加速度和产
生它的力的方向始终是一致的。
\item 物体运动的加速度不仅决定于外力,还同时跟物体
的质量有关,在这里,质量的大小对运动状态改变的快慢程
度起了一种制约作用,所以说质量是物体惯性大小的量度。
\item 把牛顿第二定律的实验结论$a\propto \dfrac{F}{m}$
改写成等式$F=kma$时,式中的比例常数$k$的取值跟式中其他物理量所用的
单位有关。在国际单位制中,质量$m$的单位是kg,加速度
$a$的单位是${\rm m/s^2}$, 力$F$的单位是N,于是:
\[k=\frac{F}{ma}=\frac{1{\rm N}}{1{\rm kg}\x 1{\rm m/s^2}}=1\]

这样,就可以使牛顿第二定律的公式简化为
\[F=ma\]
因此在应用牛顿第二定律公式解题时,必须使$m$、$a$、$F$三个
量都选用国际单位制单位。

\item 力的独立作用原理进一步阐明了加速度对力的依存
关系。不论物体是否还受有其他力的作用,也不论物体的初
始运动状态如何,一个力总是独立地使物体产生一个加速度。
譬如在落体运动或抛体运动中,不论是否存在空气阻力,重力
产生的加速度总是等于$g$. 也正因为这样,才能应用力的矢
量合成法则,把牛顿第二定律推广为$F_{\text{合}}=ma$, 力的独立作用
原理也为以后学习运动合成、波的叠加等知识准备了基础。
\end{enumerate}

\subsubsection{力、加速度、速度以及速度改变的关系}

加速度和力
有着直接的关系,但是物体运动的速度跟力并没有直接的关
系.学生常有如下的错误看法:
\begin{enumerate}
    \item 认为作用力大,则速度
一定大,作用力小,速度就不可能大;
\item 认为作用力如果
减小,则物体的速度也将逐渐减小。
\end{enumerate}
这两种错误都要进行
纠正。作用力大产生的加速度大,速度是否大还要取决于
初速度和加速的时间。这可以匀变速直线运动为例来加以
说明:即使加速度$a=F/m$
很大,但初速度$v_0$和加速时间$t$
如果很小,则$v_t=v_0+at$也不会很大;另一种情况,$a=F/m$
并
不大,但只要初速度$v_0$或加速的时间$t$很大,则$v_t$也不一
定小。后一种错误看法可以结合课本练习四第3题
进行讨论,指出:只要力$F$的方向和物体的初速度方向相
同,即使力$F$在逐渐减小,产生的加速度$a$也逐渐减小,但是,
物体的速度还是增的,只是每隔单位时间增加的速度$\Delta v$
在逐渐减小;直到当$F=0$, $a=0$时,物体的速度就不再增
大,物体将以做变速运动时最大的末速度为速度做匀速直线
运动。关于这道题目还可用一个比喻来加以说明:如果往银
行存钱,每个月存入的钱数在逐渐减少,但存款数还是在增
加的。

\subsubsection{关于质量和重量的区别与联系}

在物理学的传统讲
法中,质量和重量是两个不同的概念,质量就其本身的含义来
说,就是组成物体的物质多少;从动力学的观点来看,是物体
惯性大小的量度。质量是标量,物体的重量则是指物体所受到
的地球的重力,是矢量。

课文中先后出现的两个式子$G=mg$和
$\dfrac{G_1}{G_2}=\dfrac{m_1}{m_2}$
表明了质
量和重量的联系。教学时对这两个式子的意义可掌握以下的
分寸:
\begin{enumerate}
    \item $G=mg$这一关系式,在初中把比值$g$作为一个比例
常数,$g=9.8{\rm N/kg}$。它的意义是质量为1千克的物体的重
量是9.8牛顿.而高中课本则是从牛顿第二定律$F=ma$推导得出的,这是指物体在重力作用下运动时,重力产生的加
速度为$g$, 物体的质量为$m$, 则物体所受的重力$G=mg$. 两
处对$G=mg$的讲法,着眼点不同,但含义是一致的。

\item $\dfrac{G_1}{G_2}=\dfrac{m_1}{m_2}$一式所表明的意义是在地球上同一地点,物体的重量正比于物体的质量。这就是说,在地球上同一地点,
一切自由落体的加速度$g$都相等,在重力作用下,$g$的大小和
方向都是确定的。即比值$g=G/m$
是与物体的质量$m$无关的
一个量。

\item 由于现在已经规定重量作为质量的同意词,不再用
来表示物体所受的重力。因此,今后中学物理教学中如何处
理有关重量的问题,尚须进一步研究。
\end{enumerate}

\subsection{第三单元}
这一单元主要是综合应用牛顿运动定律和运动学的知识
来解决一些实际问题,并了解什么是超重和失重现象,解牛
顿定律的适用范围。

\subsubsection{牛顿运动定律的应用}

教材的安排是先通过应用
(一)讨论从物体的受力情况确定物体的运动情况;再通过应
用(二)从物体的运动情况确定物体的受力情况,要引导学生
认识在分析实际问题时,这两个方面常常是结合使用的,而对
运动物体受力情况的分析则是解决任何力学问题的基本出
发点。

\subsubsection{物体受力分析}

在对物体进行受力分析时,要让学生
注意首先应明确研究的对象,考察研究对象和周围哪些物体
有联系,并要结合物体的运动状态来分析。这是分析物体受
力情况时的一条基本思路,譬如在分析课本126页例题2
时,应向学生指出:题目是要求线对小球的拉力,因此小球应
是研究的对象;跟小球发生联系只有地球和悬线两个物体,也
就是小球只受到重力和悬线拉力这两个力的作用。既然小球
跟随小车一起做匀变速运动,则小球所受的合力必然不等于
零,而且小球运动的加速度必定是由重力和悬线的拉力的合
力所产生,又由于小球是沿着水平方向做匀变速直线运动,所
以这一合力也必定是沿着水平方向。于是就可以按课本图
3.11所示进行力的合成.由于合力所产生的加速度已由题
目给出,所以就可计算出悬线对小球的拉力。

\subsubsection{超重和失重} 在讲解超重和失重现象时,可以让全体
学生利用弹簧秤做课本图3.12的实验(弹簧秤可用
量程为250克力或400克力的,物体可用100克或200克的
钩码)。

在弹簧秤静止的情况下,观察指示器所指的读数就等于
钩码的重量。使弹簧秤急剧上升,就可以看到弹簧秤指示器
所指的读数突然增大,但随即又减小了,要指导学生观察开
始上升时弹簧秤读数增大这一变化。为了看清楚这一点,可
以让学生在弹簧秤挂了钩码静止的情况下,用一小块纸片卡
在弹簧秤指示器的下面,当弹簧秤加速上升时,指示器下降将
会把纸片推到下面读数较大的位置上。这样,当弹簧秤静止
时就可以看出弹簧秤读数曾经增大到多大了。

同样,在观察物体的失重现象时,应指导学生观察当弹簧
秤突然加速下降时读数减小的现象。为了观察读数曾经减小
到什么程度,也可用一小纸片卡在弹簧秤指示器的上端,用
以记录失重时的读数。

在分析超重和失重现象时,应指出:

\begin{enumerate}
    \item 挂在弹簧秤下的砝码只受重力和弹簧拉力的作用,
当钩码在竖直方向上做变速运动时,重力和拉力的合力不可
能等于零。这一合力就是使钩码产生加速度的力。在钩码向
上做加速运动或向下做减速运动时,所需的合力方向都是向
上的,而重力的方向是竖直向下的,因此只有依靠弹簧秤的拉
力大于重力,才能使得合力方向向上。这样,弹簧秤的读数就
会大于钩码的重量,这就是产生超重现象的原因,而当钩码
向下做加速运动或者向上做减速运动时,所需的合力的方向
向下,跟物体重力方向一致,因此重力中的一部分就用来产生
向下的加速度。由于重力“用掉”一部分,因此钩码作用于弹
簧秤的力将减小,即弹簧秤的读数将减小,这就是产生失重现
象的原因。
\item 如果物体向下做加速运动的加速度等于$g$, 则物体
的重力将全部用来产生加速度,这样就没有多余的力作用于
弹簧秤,弹簧秤的读数应等于零。

可以由教师做一演示,将挂有重物(钩码)的弹簧秤用线
悬在高处,在弹簧秤的指示器上贴一小块白胶布以便于观察,
这时弹簧秤的读数等于重物的重量。当用火柴烧断悬线后,
重物和弹簧秤一起自由下落,在这过程中可以观察到弹簧秤
的读数立即减小到零(演示时在落地处准备好一个网兜以承
接弹簧秤使之不会摔坏)。指出这是完全失重的现象。

\item 要使学生明确,不论超重还是失重现象,物体所受到
的重力是没有变化的,只是在物体静止时,支承这个物体的另
一物体(如桌面、悬线等)发生形变,产生了弹力作用于物体,
使物体保持平衡。当物体加速下降时,由于重力中的一部分用
来产生向下的加速度,物体需要的支持力变小了。相反,物体
加速上升时,物体所受的重力虽然也没有变化,但重力并不能
产生向上的加速度,因此支持物提供的弹力除了克服物体的
重力外还要包括使物体产生向上加速度所需的力,因而支持
物的弹力将增大,所以通常用弹簧秤(测力计)来测量物体的
重量时,必须使物体处于静止状态,避免失重或超重的影响。
\end{enumerate}

\subsubsection{牛顿运动定律的适用范围}

只需简单地向学生介绍,
牛顿定律只适用于解决低速运动问题,而不能适用于研究高
速运动,对于微观粒子的运动,一般说来,牛顿力学也是不
适用的。可指出所谓高速是指粒子的速率已高达可与光速相
比拟的程度。关于爱因斯坦狭义相对论的时空观,只需大致
地根据课本所叙述的内容进行介绍,不必作深入的讲解。这
节教材的意图在于一方面扩大学生的知识面,更重要的是使
学生认识到任何规律都有它的适用范围。

\section{实验指导}
\subsection{演示实验}

\subsubsection{阻力是使物体运动变慢的原因}
小车自斜面滑下后在水平面上运动的演示实验,可用宽
约12—15厘米、带有档边(高约0.8—1.0厘米)的两段木板,
分别作为斜面和水平面拼接起来,作为斜面的木板长约60厘
米,作为水平面的木板长约120厘米.在水平长木板的一端
装一个用扁铁作边框的网兜,如图3.1所示.长木板平面上
贴上塑料装饰板,在上面平铺一层软棉布(灯芯绒或绒布),棉
布上面再铺放两层毛巾布,以增大阻力。将作为斜面的木板
的一端用木块垫起,倾角$5^{\circ}$左右.要注意斜面跟长木板平面
的拼接处要尽可能吻合。
\begin{figure}[htp]
    \centering
    \includegraphics[scale=.6]{fig/3-1.png}
    \caption{}
\end{figure}

实验时将小车放在斜面顶端由静止开始释放。小车到达
铺有毛巾布的平面上后,运动30厘米左右即停下来;然后把
两层毛巾布取下(注意用薄木片调节拼接处的高度使之吻合),
再把小车放在斜面顶端由静止开始释放,小车在铺有软棉布
的平面上运动的距离将增大到60厘米左右.如果把棉布取
下,小车在斜面的同一高度滑下后,可以沿着塑料平面一直运
动下去直到落入网兜里。

\subsubsection{用气垫导轨观察骑块的运动}
使用前必须将导轨调节水平(沿导轨方向和垂直于导轨
方向都应用水平仪进行调节)。用软布擦去骑块内侧以及导
轨表面的浮灰。然后接通气泵,如发现气孔有堵塞现象,可用
细铜丝(从多芯软导线中抽出一根使用,不要用钢丝,因为钢
丝太硬会使气孔边缘卷口起毛,增大骑块运动时的阻力)通一
下,务必使每个气孔保持畅通。

实验时应先开动气泵,待正常供气后,再将骑块放上导
轨,轻轻推动一下,便可观察到骑块的运动很接近于匀速直线
运动。应强调指出,所要观察和研究的是骑块已经开始运动
以后的一段过程,由于摩擦阻力很小可以忽略,重力和气垫的
托力相互平衡,合力为零,因此骑块的匀速直线运动状态并不
发生改变,当骑块到达导轨一端被弹簧片弹回反向运动时,可
指出这时由于受到外力作用,迫使骑块改变了运动方向(也就
是改变了运动状态),但在反向运动的过程中,由于骑块所受
合力为零,骑块的匀速直线运动状态仍不发生改变。

\subsubsection{惯性现象}
如图3.2所示,将小车(质量约为200克)用细线通过定滑
轮和一质量较大的钩码(譬如30克)相连,用手挡住小车,不使
它运动,将一木块(体积约为$3\x6\x10{\rm cm^3}$)竖起来放在车上。
释放后,小车突然做加速运动,木块由于惯性将向后面倾倒。
\begin{figure}[htp]
    \centering
    \includegraphics[scale=.6]{fig/3-2.png}
    \caption{}
\end{figure}

利用同一装置继续演示,开始时用手扶住木块同时施加
适当的阻力使小车的加速度不至太大。待小车和木块一起运
动后再完全放手,这时木块和小车将保持相对静止,当小车
运动到平面一端受阻突然停止运动时,木块由于惯性将向前倾倒。

\subsubsection{物体运动状态的改需要力的作用}
物体速度大小的改变.将一辆小车放在水平的玻璃
板上,使小车从静止到运动需要用手推,这就是要有力的作用
(图3.3);在小车已经具有速度的情况下,要使它停止运动,
需要用手来挡,这也就是要有力的作用,如图3.4。
\begin{figure}[htp]
    \centering
\begin{tikzpicture}[>=latex]

    \node at (.5,1)[above]{$v_0=0$};
\draw[->](3,1)--node[above]{$v$}(4,1);
    \draw(2.6,0)rectangle (4.4,.8) ;  
\draw[dashed](-.4,0)rectangle (1.4,.8) ;   
  \foreach \x in {0,1}
{
    \draw[fill=white, dashed] (\x,0) circle(.2);
    \draw[fill=white] (\x+3,0) circle(.2);
}  
\draw[->](-1.2,.4)node[left]{$F$}--(-.4,.4);
\draw(-1,-.2)--(5,-.2);
\end{tikzpicture}
    \caption{}
\end{figure}

\begin{figure}[htp]
    \centering
\begin{tikzpicture}[>=latex]
    \node at (3.5,1)[above]{$v=0$};
    \draw[->](0,1)--node[above]{$v_0$}(1,1);
    \draw(2.6,0)rectangle (4.4,.8) ;  
\draw[dashed](-.4,0)rectangle (1.4,.8) ;   
  \foreach \x in {0,1}
{
    \draw[fill=white, dashed] (\x,0) circle(.2);
    \draw[fill=white] (\x+3,0) circle(.2);
}  
\draw(-1,-.2)--(5,-.2);
\draw[->](2.2,.4)--node[above]{$F$}(1.4,.4);
\end{tikzpicture}
    \caption{}
\end{figure}

物体速度方向的改变。将一瓷球(白色的、半径较大,
可见度较大,或者用灌水的乒乓球)系在细线一端,另一端有
一金属小杯,把环套在竖直方向的固定轴上,给瓷球一个垂直
于拉线方向的速度,球就在水平面上沿着一个以线长为半径
的圆周运动,如图3.5所示(水平面可以用涂塑装饰板做).如
果水平面的阻力很小,可以认为瓷球的速率不变,但是从拉线
被绷紧的现象可以想象,线对瓷球有拉力的作用,才使瓷球的
运动方向时刻发生改变。
\begin{figure}[htp]\centering
    \begin{minipage}[t]{0.48\textwidth}
    \centering
    \includegraphics[scale=.8]{fig/3-5.png}
    \caption{}
    \end{minipage}
    \begin{minipage}[t]{0.48\textwidth}
    \centering
    \includegraphics[scale=.8]{fig/3-6.png}
    \caption{}
    \end{minipage}
    \end{figure}

物体的速度大小和方向都发生改变,如图3.6所示,
使一乒乓球沿着桌面以某一速度运动,当乒乓球离开桌子后
将沿一曲线下落。可以这样设想:乒乓球离开桌子后,如果没
有重力的作用,仍将沿着原来的水平方向作匀速直线运动。
正是由于下落过程中只受到重力的作用,乒乓球的速度大小
和方向才都发生了改变。

\subsubsection{加速度和力的关系以及加速度和质量的关系}
(演示方法见学生实验8和9)

\subsubsection{牛顿第二定律的应用}
为结合课本126页例题2的讲解可以做如下的演示:
用手提着一个挂在细绳一端的质量较大的单摆球,当球静止
时,细绳在竖直方向,当提着小球突然向右侧加速跑动时,即
可看到悬挂小球的线向左偏斜的现象。

\subsubsection{超重和失重现象}

可按课本图3.12所示的情况
进行演示,为了能把超重和失重的最大续数
固定下来,可以剪两小块厚纸(如用包装牙膏
等软管用纸盒),形状如图3.7所示.两侧狭
缝的宽度相当于弹簧秤刻度板的厚度,当弹簧秤挂着100
克钩码时,把纸片卡在弹簧秤指示器的下方和上方。卡入纸
片时,先将纸片顺着刻度板中部的隙缝嵌人,然后再扭转$90^{\circ}$,
并使它紧靠着指示器。当把弹簧秤迅速上提或迅速下降时,指
示器就会把纸片推向下方或上方,并停留在那里,这样就可
以看出超重和失重的最大读数。
\begin{figure}[htp]\centering
    \begin{minipage}[t]{0.48\textwidth}
    \centering
    \includegraphics[scale=.8]{fig/3-7.png}
    \caption{}
    \end{minipage}
    \begin{minipage}[t]{0.48\textwidth}
    \centering
    \includegraphics[scale=.6]{fig/3-8.png}
    \caption{}
    \end{minipage}
    \end{figure}

如图3.8所示,将一杠杆支放在铁架台上,在杠杆左
端悬挂一根细链条,使链条的
一端$A$固定在杠杆上,另一端
$B$用细线系在链条$A$端的一
个环上,在杠杆右端挂一砝码
使杠杆平衡。点燃火柴烧断系
住链条$B$端的细线,使这一半
链条下落。这时可观察到杠杆
左端向上倾斜,说明杠杆左侧的一半链条下落时发生了失重现象。这是因为当这一半链条
下落时,它所受的重力已用于产生向下运动的加速度,作用于
杠杆上的力减小了,因此杠杆失去了平衡,但当这一半链条
下落到被上一半链条拉住时,则杠杆又会恢复平衡。

\subsection{学生实验}
\subsubsection{研究加速度和力的关系}
实验时要注意以下几个问题:

实验前要消除摩擦阻力的影响,将长木板不带定滑
轮的一端下面垫放一木块,使木板倾斜(约$5^{\circ}$左右),将小车
放在木板上,先不要拴上砂桶。在小车后面固定一条纸带,并
让它穿过打点计时器的两个限位孔,将小车释放后,它可能
仍保持静止,也可能沿长木板加速下滑。适当改变木块的垫
放位置(即改变木板的倾斜程度),直到推动一下小车后,小车
基本上能匀速下滑(纸带上打出的点子的间隔基本上是均匀
的)。这时摩擦阻力和小车的重力沿斜面方向的分力相平衡,
消除了摩擦的影响。

砂桶和砂的质量只有远小于小车质量($m\ll M$)的条
件下,才可以认为它的重量(mg)等于对小车的拉力(还要注
意调节定滑轮的位置使得拉绳平行于斜面),一般实验用的
小车质量约为200克,因此砂和砂桶的总质量以不超过20克
为宜。如果砂桶和砂的质量太大将会产生明显误差(系统误
差)。这个问题不要在实验时讨论,以免分散学生的注意力,
可以在总结出牛顿第二定律以后再安排讨论。

这个实验中,拉力所用的单位,可以用弹簧秤直接测
量砂桶和砂的重量,然后读取用牛顿做单位的读数。

每次改变拉力,打出的纸带都要进行编号,并标明所
用拉力的大小,以免在处理数据时发生张冠李戴的混乱
情况。

在指导写实验报告时,应要求学生把每次改变拉力
时的记录纸带,作为原始资料附在实验报告中。并要写出自己
设计的能够反映数据处理过程的表格,以及与之相应的、画在
坐标纸上的$a$-$F$图象。实验报告还应包括明确的实验结
论.在最后部分还可根据课本练习二第2题的要
求,利用实验数据来求出每次实验中,加速度$a$和拉力$F$的
比值,看看它们是否大致相等,可启发学生思考这一比值跟
$a$-$F$图象中直线的斜率有怎样的关系?

\subsubsection{研究加速度和质量的关系}
这个实验的注意事项和编写实验报告的要求和实验(一)相
同。此外还应注意,在改变物体质量时,变化范围可适当放大
一些,如可分别增加50克、100克、150克、200克、250克
砝码。

实验后可引导学生讨论课本练习三第2、3题.

\section{习题解答}

\subsection{练习一}
\begin{enumerate}
	 \item 一个球以20${\rm cm/s}$的速度运动着,而且没有受到力的作用,5秒后它的速度将是多大?
	 
     \begin{solution}
        运动着的球没有受到力的作用,它将保持匀速直线
        运动状态,所以5秒后仍以20${\rm cm/s}$的速度沿着原来运动
        的方向做匀速直线运动。
     \end{solution}
	 \item 在行驶的火车里的水平桌面上放着一个小球,当小球突然相对于车厢发生向前运动或者向后运动时,火车的运动状态分别发生了怎样的改变?
	 
     \begin{solution}
        当小球突然相对于车厢发生向前运动时,说明火车
        突然制动做减速运动;当小球突然相对于车厢发生向后运动
        时,说明火车突然加速做加速运动。
     \end{solution}
	 \item 地球从西向东转,为什么我们向上跳起来以后还落到原地,而不落到原地的西边?
	 
     \begin{solution}
        因为我们站在地面时具有和地面相同的自转速度,
        当我们向上跳起时,由于惯性,在水平方向上仍要保持原来的
        速度前进,因此落下时仍落在原地而不会落到原地的西边。
     \end{solution}
	 \item 分别举出几个利用惯性和防止惯性的不利影响的例子.
	 
     \begin{solution}
        这类例子很多。锤头和木柄间有些松动,可以把锤
        子竖起来使锤柄朝下,在固定物体上撞击几下,锤柄因撞击受阻而突然停止运动,锤头由于惯性要继续向下运动,结果锤头
        和柄就套紧了。这是利用惯性的例子。为了行车安全,规定
        了城市里各种车辆的最高行驶速度,这是为了在紧急制动时
        防止由于惯性而造成的事故,这是防止惯性的不利影响的
        例子。
     \end{solution}
\end{enumerate}



\subsection{练习二}
\begin{enumerate}
	\item 根据你的记录纸带,量出有关数值,计算出每次实验的加速度,列出表格,作出$a-F$图象.
	 
    \begin{solution}
        说明:本题应根据每个人的实验结果,处理数据,填写表
        格,作出图象。
    \end{solution}
	\item 利用你自己的数据算出每次实验中$a$和$F$的比值,看看这些比值是否大致相同?利用这种办法来研究$a$和$F$的关系,比起用图象来研究,哪种办法方便?
	 
    \begin{solution}
        说明:根据每次实验中$a$和$F$的对应数据计算出来的$a$
        和$F$的比值,大致是相同的。用这种办法虽然也可以得出加
        速度$a$和力$F$成正比关系的结论,但是要进行多次计算。而
        用画$a$-$F$图象的方法,可以看出各个点基本上分布在同一条
        通过坐标轴原点的直线上,这样就可以方便地得出$a$和$F$成
        正比关系的结论。因此采用图象来研究比较方便。  
    \end{solution}
	\item 5牛的力的作用在一个物体上,能使它产生2$\msq$的加速度,要使它产生5$\msq$的加速度,需要多少牛的力?
	 
    \begin{solution}
对于一个确定的物体来说,加速度和力成正比。  
\[\frac{a_1}{a_2}=\frac{F_1}{F_2}\]
所以:
\[F_2=\frac{F_1a_2}{a_1}=\frac{5\x5}{2}=12.5{\rm N}\]      
    \end{solution}
\end{enumerate}



\subsection{练习三}
\begin{enumerate}
	\item 根据你的记录纸带,量出有关数值,计算出每次实验的加速度,列出表格,作出$a-1/m$图象.
	 
    \begin{solution}
        说明:本题应根据每个人的实验结果,处理数据,填写表
格,作出图象。
    \end{solution}
\item 利用你自己的数据算出每次实验中$m$和$a$的乘积,
看看乘积的数值是否大致相同?你由此能得出什么结论?利用这种办法来研究$a$和$m$的关系,比起用图象来研究,哪种办法方便?
	 
\begin{solution}
    说明:根据每次实验中$m$和$a$的乘积,可以看出它们的
数值大致是相同的。这说明了在相同的力作用下,小车的加
速度$a$和小车的质量$m$成反比。利用这种办法来研究$a$和$m$
的关系要进行多次计算,用图象法研究,也要多次计算$1/m$
的数值,但画出$a$-$1/m$图象后,可以直观地看出各个点基本
上分布在同一条通过坐标轴原点的直线上,说明$a$和$1/m$成
正比,从而得出$a$和$m$成反比关系的结论。

因此在研究加速度和质量的关系时,这两种处理数据的
方法对比起来,用图象法并不显得很方便,但是作为一种数据
处理方法还是应该学习的。
\end{solution}
\item 你已经用图象研究了$a$和$F$、$a$和$m$的关系,谈谈用图象处理实验数据的好处.
	 
\begin{solution}
利用图象比较直观,能为我们寻求物理量之间的定量关系提供线索,譬如$a$-$F$图象是一条直线,这就直观地说
    明了$a$和$F$成正比关系.$a$-$1/m$图象也是一条直线,说明了
    $a$和$1/m$成正比即$a$和$m$成反比关系.

    此外,图象法的优点还在于对测量数据的取舍比较方便。
    譬如在画$a$-$F$图线时,有些点并不在直线上,但只要直线能
    通过大多数的点,就可以得出$a$和$F$成正比的结论。而如果
    要计算每一次实验中$a$和$F$的比值,有的比值可能跟真实值
    相差较大,这样求得的平均比值的误差就会偏大。    
\end{solution}
\item 一辆卡车在空载时质量是$3.5\times 10^3$千克,载货时的质量是$6.0\times 10^3$千克用同样大小的牵引力,如果空载时使卡车产生1.5$\msq$的加速度,载货时产生多大的加速度?(不考虑阻力)
	 
\begin{solution}
    在不考虑阻力的情况下,对卡车的牵引力就是使卡
车产生加速度的力,在同样大小的牵引力作用下,卡车的加速
度跟它的质量成反比。
\[\frac{a_1}{a_2}=\frac{m_2}{m_1}\]
所以:
\[a_2=\frac{m_1a_1}{m_2}=\frac{3.5\x 10^3\x 1.5}{6.0\x 10^3}=0.88\msq\]
\end{solution}
\end{enumerate}



\subsection{练习四}
\begin{enumerate}
	\item 从牛顿第二定律知道,无论怎样小的力都可以使物体产生加速度,可是我们用力提一个很重的物体时,却提不动它.这跟牛顿第二定律有无矛盾,为什么?
	 
    \begin{solution}
        没有矛盾。在用力提一个很重的物体时,没有提起来,这是因为所用的力小于物体所受的重力。用力的结果只
能减小重物对支持物的压力,重物虽然受到向下的重力,向上
的支持力以及手提重物时的作用力,但它所受的合力还是等
于零,因此重物不会产生加速度,自然就继续保持静止状态
不动。
    \end{solution}
\item 要把一个箱子在地板上从这一端推到另一端,我们在全部时间内都必须用力推它,停止用力,箱子就会停下来.马必须用力拉车,车子才前进,停止用力,车子就会停下来.亚里士多德怎样解释上述现象?根据牛顿运动定律应该怎样解释?
	 
\begin{solution}
    亚里士多德认为,力是维持物体运动的原因。必须
    有力作用在物体上,物体才能运动,没有力的作用,物体就要
    静止下来,根据这种观点,人必须用力推箱子,箱子才能运动,
    停止对它用力,它就停下来;马必须用力拉车,车才能运动,停
    止对它用力,它就停下来。

    根据牛顿运动定律,力不是维持运动的原因,而是改变物
    体运动状态的原因。当人的推力大于箱子和地板间的摩擦力
    时,箱子就会在合外力的作用下产生加速度改变原来的静止
    状态开始运动;当人停止用力时,箱子又会在摩擦阻力的作用
    下产生跟原来的运动方向相反的加速度,使它做减速运动,很
    快地停了下来,马拉车,车子会前进,停止用力,车子就会停
    下来的原因,也是如此。
\end{solution}
\item 一个物体受到一个逐渐减小的力的作用,力的方向跟速度的方向相同,物体的速度怎样改变?
	 
\begin{solution}
    物体的速度将逐渐增大,直到这个力减小到零为止。这是因为虽然力在逐渐减小,只是使物体产生的加速度逐渐
    减小,但由于力的方向跟速度的方向相同,它的速度还是不断
    增大的。
\end{solution}
\item \begin{enumerate}
\item  质量是0.5千克的物体在一个恒力的作用下得到
0.1$\msq$的加速度,这个恒力是多大?
\item 10牛的力使一个物体得到2.0$\msq$的加速度,这个物体的质量是多大?
\item 质量是0.1千克的物体,在5牛的恒力作用下,得到多大的加速
度?
\end{enumerate}
	 
\begin{solution}
    根据牛顿第二定律
\begin{enumerate}
    \item $F=ma=0.5\x 0.1=0.05{\rm N}$
    \item $F=ma,\quad m=\dfrac{F}{a}=\dfrac{10}{2.0}=5{\rm kg}$
    \item $F=ma,\quad a=\dfrac{F}{m}=\dfrac{5}{0.1}=50\msq$
\end{enumerate}
\end{solution}
 \item 质量是1.0千克的物体受到互成30$^\circ$角的两个力的
作用,这两个力都是10牛,这个物体产生的加速度是多大?
	 
\begin{solution}
    这个物体所受到的合力
\[F_{\text{合}}=2F\cos\frac{\alpha}{2}=2\x 10\x \cos15^{\circ}=2\x 10\x 0.966=19.3{\rm N}\]
根据牛顿第二定律合力产生的加速度
\[a=\frac{F_{\text{合}}}{m}=\frac{19.3}{1.0}=19.3\msq\]
\end{solution}
\item 下列说法是否正确:
\begin{enumerate}
\item 物体的速度越大,表明物体所受的合外力越大.
\item 根据$F_{\text{合}}=ma$,得到$m=F_{\text{合}}/a$,所以物体的质量跟物
体所受的合外力成正比.
\item 物体所受合外力越大,速度变化越大.
\end{enumerate}

\begin{solution}	 
\begin{enumerate}
    \item 不正确。因为物体的速度大小和合外力的大小并
    无直接关系,合外力只决定了物体的加速度。
    \item 不正确。因为质量是物体惯性大小的量度,是物体固
    有的属性,跟物体是否受力无关。
    \item 不一定正确。因为速度变化$\Delta v$的大小不仅跟合外力
    产生的加速度$a$有关,还跟速度发生变化的时间$\Delta t$有关,即
    $\Delta v=a\Delta t$. 即使物体所受的合外力增大了,如果作用时间
    $\Delta t$很小,速度的变化$\Delta v$也不一定大。
\end{enumerate}
         
\end{solution}

\end{enumerate}








\subsection{练习五}
\begin{enumerate}
\item 先后在广州和北京用天平来称量同一个物体,得到
的结果是否相同?如果先后用弹簧秤来称量,得到的结果是否
相同?说明理由.
	 
\begin{solution}
    天平是利用杠杆(并且是等臂杠杆)平衡原理来称量
物体质量的,设被测物体的质量为$m_x$, 砝码质量为$m$, 力臂
为$L$, 当天平平衡时,应有$m_xgL=mgL$的关系。消去力臂$L$
和重力加速度$g$, 则$m_x=m$. 因此先后在广州和北京用天平
来称量同一个物体,得到的结果是相同的。

而弹簧秤是根据在弹性限度内,弹簧的形变跟弹力成正
比的原理制成的。作用的力越大,则弹簧的形变量也越大,弹
簧秤的示数也越大。用弹簧秤来测量物体的重量$G=mg$时,
由于广州的重力加速度小于北京的重力加速度,所以如果先
后用弹簧秤来称量,得到的结果是不相同的,在广州称量时弹簧秤的示数要小些。
\end{solution}
\item 一个学生认为半块砖的重力加速度是整块砖的重力
加速度的两倍,因为半块砖的质量是整块砖的质量的一半.另
一个学生认为半块砖的重力加速度是整块砖的重力加速度的
一半,因为半块砖的重量是整块砖的重量的一半.他们的说
法对不对?为什么?
	 
\begin{solution}
    他们的说法都不对,这是因为半块砖整块砖的重
力加速度$g$在同一地区是不变的。第一个学生的错误在于认
为半块砖和整块砖受到的重力是相等的;另一个学生的错误
在于认为半块砖和整块砖的质量是相等的,而他们发生错误
的共同原因还在于都认为重力加速度对于不同的物体可以是
不同的。
\end{solution}
\item 北京的重力加速度为980.1${\rm cm}/{\rm s^2}$.质量是1千
克的物体在北京的重量是多少牛?
	 
\begin{solution}
    根据重量和质量的关系,
\[G=mg=1\x9.801=9.801{\rm N}\]
\end{solution}
\item 有一架仪器,质量是3.0千克,把它射到月球上,这
架仪器的质量是否改变?它在月球上的重量是多少牛?月球表
面的$g$取1.6$\msq$.
	 
\begin{solution}
   \[ G_{\text{月}}=mg_{\text{月}}=3.0\x1.6=4.8{\rm N}\]
\end{solution}
\item 仔细看看课文中重力加速度的数值表,从中你可以
得到什么结论?
	 
\begin{solution}
    重力加速度的数值随纬度的增加而增大,赤道地区
的$g$值最小、北极地区的$g$值最大,但相差不很大。
\end{solution}
\item 据说以前有个商人,从荷兰那里把5000吨的货物运
往非洲靠近赤道的某个港口,发现货物少了19吨.在荷兰和
非洲,都是用托盘弹簧秤来称量货物的,你根据书中所列$g$
的数值和荷兰的地理纬度,大致估算一下货物的重量是否会
差这么多.
	 
\begin{solution}
    荷兰的地理纬度可取北纬$55^{\circ}$, 重力加速度的数值
可取$9.816\msq$.靠近赤道的某个港口的重力加速度可取
$9.780\msq$.

在荷兰,货物的重量为$G=mg,\quad g=9.816\msq$.

在赤道,货物的重量为$G'=mg',\quad g'=9.780\msq$.

重量的变化
\[\Delta G=G'-G=m(g'-g)=5000\x10^3\x(9.780-9.816)
=-0.18x10^{6}{\rm N}\]

因为质量为1吨的物体的重量为$9.8\x10^3$牛,所以
\[\Delta G=\frac{0.18\x10^6}{9.8\x10^3}=-18.4\text{吨力}\]
可见,题中所说货物少了19吨是可能的。
\end{solution}
\end{enumerate}


\subsection{练习六}
\begin{enumerate}
\item 在厘米$\cdot$克$\cdot$秒制中,力的单位达因是这样定义的:
使质量是1克的物体产生1${\rm cm}/{\rm s^2}$的加速度的力,叫做1
\textbf{达因}.试证明: $1{\rm N}=10^5{\rm dyn}$.
	 
\begin{solution}
    力的国际制单位
\[\begin{split}
    1{\rm N}&=1{\rm kg}\x 1\msq\\
    &=1000{\rm g}\x 100{\rm cm/s^2}=10^5{\rm g\cdot cm/s^2}
\end{split}\]
在厘米·克·秒制中定义力的单位
\[1{\rm dyn}=1{\rm g}\cdot 1{\rm cm/s^2}=1{\rm g\cdot cm/s^2}\]
可见$1{\rm N}=10^5{\rm g\cdot cm/s^2}=10^5{\rm dyn}$。
\end{solution}
\item 有两个力,一个是100达因,一个是20牛,哪个力
大?大的是小的多少倍?
	 
\begin{solution}
    设$F_1=10^6{\rm dyn}=\frac{10^6}{10^5}=10{\rm N}$.而$F_2=20{\rm N}$,所
以$F_2>F_1$, 即20牛的力大.
\[\frac{F_2}{F_1}=\frac{20}{10}=2\]
即$F_2$是$F_1$的2倍.
\end{solution}
\item 一个原来静止的物体,质量是600克,受到0.2牛的
力的作用,求物体在3.0秒末的速度.先用国际单位制计算,
再用厘米$\cdot$克$\cdot$秒制计算.
	 
\begin{solution}
\begin{enumerate}
    \item $a=\dfrac{F}{m}=\dfrac{0.2}{0.6}=\dfrac{1}{3}\msq,\qquad v=at=\dfrac{1}{3}\x 3.0=1\ms$
\item $a=\dfrac{F}{m}=\dfrac{0.2\x 10^5}{600}=\dfrac{1}{3}\x 10^2{\rm cm/s^2},\qquad v=at=\dfrac{1}{3}\x 10^2\x 3.0=100{\rm cm/s}$
\end{enumerate}
\end{solution}
\item 从炮筒射出的炮弹,质量是10千克,速度是$1.0\times 
10^8\ms$,炮弹在炮筒内运动的时间是$4.0\times 10^{-8}$秒.求火
药爆炸所生气体对炮弹的平均压力.

\begin{solution}	 
    炮弹在发射前原是静止的,假设炮弹在火药爆炸所
    生气体的压力作用下做匀加速直线运动。
\[v=at,\quad a=\frac{v}{t}=\frac{1.0\x 10^3}{4.0\x 10^{-3}}\msq=\frac{1}{4}\x 10^6\msq\]
气体对炮弹的平均压力
\[F=ma=10\x\frac{1}{4}\x 10^6=2.5\x 10^6{\rm N}\]
\end{solution}
\end{enumerate}



\subsection{练习七}
\begin{enumerate}
\item 一个质量是100克的运动物体,初速度是0.5$\ms$,
受到的力是2.0牛,力的方向跟速度方向相同.求3.0秒末的
速度.
	 
\begin{solution}
    由于力的方向跟物体运动的初速度方向相同,在力的
作用下物体将做匀加速直线运动。根据牛顿第二定律,加速度
\[a=\frac{F}{m}=\frac{2.0}{0.1}=20\msq\]
在3.0秒末的速度
\[v_t=v_0+at=0.5+20\x3.0=60.5\ms\]
速度的方向和初速度相同。
\end{solution}
\item 一个原来静止的物体受到互成60$^\circ$角的两个力的作
用,这两个力的大小都是50牛,物体的质量是2.0千克,求
3.0秒内物体发生的位移.
	 
\begin{solution}
    物体受到的合力
    $$F_{\text{合}}=2F\cos\dfrac{\alpha}{2}=2\x50\x\cos30^{\circ}=50\sqrt{3}{\rm N}$$
加速度
\[a=\frac{F_{\text{合}}}{m}=\frac{50\sqrt{3}}{2.0}=25\sqrt{3}\msq\]
3.0秒内物体发生的位移大小
\[s=\frac{1}{2}at^2=\frac{1}{2}\x 25\sqrt{3}\x 9.0=195{\rm m}\]
方向跟合力方向一致,即跟两个力都成30$^\circ$角。
\end{solution}
\item 一个放在桌面上的木块,质量是0.10千克,在水平
方向受到0.06牛的力,木块和桌面的滑动摩擦力是0.02牛.
求木块通过1.8米所用的时间.
	 
\begin{solution}
    物体所受的合力
\[F_{\text{合}}=F-f=0.06-0.02=0.04{\rm N}\]
加速度
\[a=\frac{F_{\text{合}}}{m}=\frac{0.04}{0.10}=0.4\msq\]

由于物体做初速度等于零的匀变速直线运动,根据公式
$s=\dfrac{1}{2}at^2$, 得通过1.8米所用的时间
\[t=\sqrt{\frac{2s}{a}}=\sqrt{\frac{2\x 1.8}{0.4}}=3{\rm s}\]
\end{solution}
\item 一物体的质量是10千克,在40牛的水平拉力作用
下沿桌面从静止开始运动,物体和桌面的滑动摩擦系数为
0.20.如果茌物体运动后的第5秒末把水平拉力撤除,算一
算,一直到运动停止,物体一共走多远.
	 
\begin{solution}
    由题意可知,在整个过程中,物体的运动分两个阶段,
第一阶段物体在合力的作用下从静止开始做匀加速直线运
动,第二阶段即在第5秒末把水平拉力撤除后,物体将在滑动
摩擦力作用下做匀减速直线运动,直到停止运动。

第一阶段的加速度
\[a_1=\frac{F_{\text{合}}}{m}=\frac{F-f}{m}=\frac{F-\mu mg}{m}=\frac{40-0.20\x 10\x 9.8}{10}=2.04\msq\]
位移
\[s_1=\frac{1}{2}a_1t_1^2=\frac{1}{2}\x 2.04\x 5^2=25.5{\rm m}\]

第二阶段的加速度
\[a_1=\frac{f}{m}=\frac{\mu mg}{m}=\mu g=0.20\x 9.8=1.96\msq\]

由于$a_2$的方向跟物体的运动方向相反,因此在代入匀变
速直线运动公式时,$a_2$应取负值,即
\[v^2_1-v^2_0=2(-a_2)s_2\]

式中$v_1=0$, $v_0=a_1t_1$, 于是第二阶段的位移
\[s_2=\frac{v^2_0}{2a_2}=\frac{(a_1t_1)^2}{2a_2}=\frac{(2.04\x 5)^2}{2\x 1.96}=26.5{\rm m}\]
总位移
\[s=s_1+s_2=25.5+26.5=52.0{\rm m}\]
\end{solution}
\item  质量为10千克的物体沿长5米、高2.5米的斜面由
静止匀变速下滑,物体和斜面间的滑动摩擦系数为0.30.物
体的加速度多大?物体从斜面顶端下滑到底端需要多长时间?
	 
\begin{solution}
    物体沿斜面下滑的过程中,共受到重力$mg$、斜面的
弹力$N=mg\cos\theta$以及滑动摩擦力$f=\mu N=\mu mg\cos\theta$三个力
的作用,垂直于斜面方向的合力为零,平行于斜面向下的合力
\[\begin{split}
    F&=mg\sin\theta-f=mg\sin\theta-\mu mg\cos\theta\\
&=mg(\sin\theta-\mu\cos\theta)\\
&=10\x9.8\left(\frac{2.5}{5}-0.30\x\frac{\sqrt{5^2-2.5^2}}{5}\right)\\
&=10\x9.8(0.5-0.26)=23.5{\rm N}
\end{split}\]
沿斜面下滑时物体的加速度
\[a=\frac{F_{\text{合}}}{m}=\frac{23.5}{10}=2.35\msq\]
根据$s=\dfrac{1}{2}at^2$, 物体从斜面顶端下滑到底端需要的时间
\[t=\sqrt{\frac{2s}{a}}=\sqrt{\frac{2\x 5}{2.35}}=2.06{\rm s}\]
\end{solution}
\end{enumerate}


\subsection{练习八}
\begin{enumerate}
\item 质量是20吨的车厢以0.2$\msq$的加速度前进,运
动的阻力是它的重量的0.02倍,牵引力是多少牛?
	 
\begin{solution}
    设牵引力为$F$, 阻力为$f$, 则车厢在运动方向上所受
    的合力$F_{\text{合}}=F-f$.

    根据牛顿第二定律$F_{\text{合}}=ma$, 于是$F-f=ma$, 则牵引力
\[\begin{split}
    F=ma+f &=20\x10^3\x0.2+0.02\x20\x10^3\x9.8\\
    &=7.9\x10^3{\rm N}
\end{split}\]
\end{solution}
\item  列车在水平铁路上行驶,在60秒内速度由82$\kmh$增加到54$\kmh$,列车的质量是$1.0\times 10^8$吨,机车对
列车的牵引力是$1.5\times 10^5$牛.求列车在运动中所受的阻力.
	 
\begin{solution}
    根据题意列车的初速度
    $$v_0=32\kmh=\dfrac{32\x 10^3}{3600}\ms=8.9\ms$$
    运动50秒时的末速度 
\[v_t=54\kmh=\frac{54\x 10^3}{3600}\ms=15\ms\]
于是加速度
\[a=\frac{v_t-v_0}{t}=\frac{15-8.9}{50}=0.12\msq\]
由于列车运动的加速度是由合力所产生的,即
$F-f=ma$, 所以阻力
\[\begin{split}
    f=F-ma&=1.5\x10^5-1.0\x10^3\x10^3\x0.12\\
&=    1.5\x 10^5-1.2\x10^5=3\x10^4{\rm N}
\end{split}\]
\end{solution}
\item  以1$\ms$行驶的无轨电车,在关闭电动机以后经
过10秒停下来.电车的质量是$4.0\times 10^3$千克.求电车所受
的阻力.
	 
\begin{solution}
    行驶着的电车关闭电动机后,在运动方向上只受到
阻力作用,因此电车是做匀减速直线运动。

根据$v_t=v_0+at$, 其中$v_t=0$, 则加速度
\[a=\frac{-v_0}{t}=-\frac{15}{10}=-1.5\msq\]
负号意义表示加速度的方向跟初速度$v_0$的方向相反。

由于这一加速度是由阻力所产生,根据牛顿第二定律,阻
力
\[f=ma=4.0\x10^3\x1.5=6.0\x10^3{\rm N}\]
阻力的方向跟电车行驶的方向相反。
\end{solution}
\item  用弹簧秤拉着一个物体在水平面上做匀速运动,弹
等秤的读数是0.40牛.然后用弹簧秤拉着这个物体在这个
水平面上做匀变速运动,测得加速度是0.85$\msq$,弹簧秤
的读数是2.10牛.这个物体的质量是多大?
	 
\begin{solution}
    由题意可知,物体与水平面间的滑动摩擦力$f=0.40$
牛。当物体作匀变速直线运动时,设摩擦力保持不变,物体的
加速度是由弹簧秤的拉力$F$和摩擦力$f$的合力所产生,于是,
$F-f=ma$. 则物体质量
\[m=\frac{F-f}{a}=\frac{2.10-0.40}{0.85}=2.0{\rm kg}\]
\end{solution}
\end{enumerate}



\subsection{练习九}
\begin{enumerate}
\item 某钢绳所能承受的最大拉力是4.0吨,如果用这条
钢绳使3.5吨的货物匀加速上升,在0.50秒内发生的速度改
变不能超过多大?
	 
\begin{solution}
    货物在竖直向上做匀变速直线运动的过程中,货物
受到钢绳拉力和重力的作用,这两个力的合力方向是竖直向
上的,即
\[F-mg=ma=m\frac{v_t-v_0}{t}\]

由于钢绳所能承受的拉力$F$有一个最大值,所以在一定
时间$t=0.50$秒内,货物的速度改变$v_t-v_0$也就受到限制,
\[\begin{split}
    v_t-v_0&= \frac{(F-mg)t}{m}\\
    &=\frac{(4.0\x10^3\x9.8-3.5\x10^3\x9.8)\x0.50}{3.5\x 10^3}\\
    &=\frac{0.5\x10^3\x9.8\x0.50}{3.5\x10^2}=0.7\ms
\end{split}\]
即在0.50秒内货物的速度变化不能超过0.7$\ms$.
\end{solution}
\item 升降机以0.30$\msq$的加速度竖直减速下降,站在
升降机里60千克的人,对升降机地板的压力是多大?他站在
升降机中的体重计上,体重计表示的他的体重是多大?如果升
降机以相同的加速度竖直减速上升,情况又怎样?在什么情况
下人对地板的压力是零?
	 
\begin{solution}
    当人跟随升降机一起减速下降时,人受到重力和地
板的支持力N的作用,这两个力的合力方向竖直向上,即
\[N-mg=ma\]
\[N=mg+ma=m(g+a)=60(9.8+0.30)=606{\rm N}\]
地板所受的压力$F=-N=-606$牛.负号表示地板所受到的压力方向和人所受到的地板弹力方向相反。

如果人是站立在体重计上,则体重计所受的压力的大小
也等于606牛,它比人所受的重力$60\x9.8=588$牛大,出现
超重现象。

当人跟随升降机一起减速上升时,人所受的合力方向竖
直向下。即
\[mg-N'=ma\]
\[N'=mg-ma=m(g-a)=60(9.8-0.30)=570{\rm N}\]
地板所受的压力$F'=-N'=-570$牛.

如果人是站立在体重计上,则体重计所受的压力的大小
也等于570牛.它比人所受的重力588牛小,出现失重现象.

如果人对地板的压力$F=0$(地板对人的支持力$N=0$)
时,根据上式可知这时,
\[mg-0=ma\quad \Rightarrow\quad a=g\]
可见当$a=g$时,即当升降机以$a=g$的加速度竖直向上做匀
减速直线运动或者说当升降机以$a=g$的加速度竖直向下做
匀加速直线运动时,人对地板的压力为零。
\end{solution}
\item 弹簧秤上挂一个14千克的物体,在下列各种情况
下,弹簧秤的读数是多大?
\begin{enumerate}
\item 以$28{\rm cm}/{\rm s^2}$ 的加速度竖直加速上升;
\item 以$10{\rm cm}/{\rm s^2}$的加速度竖直减速上升;
\item 以$10{\rm cm}/{\rm s^2}$的加速度竖直加速下降;
\item 以$28{\rm cm}/{\rm s^2}$的加速度竖直减速下降.
\end{enumerate}
	 
\begin{solution}
    物体在这四种情况下,都受到向下的重力$mg$和向
上的弹簧拉力$F$的作用。
\begin{enumerate}
    \item 物体的加速度$a=28{\rm cm/s^2}=0.28\msq$,方向向
    上,$F-mg=ma$, 所以弹簧秤读数
   \[ F=mg+ma=m(g+a)=14\x(9.8+0.28)=141{\rm N}\]
    \item 物体加速度$a=10{\rm cm/s^2}=0.10\msq$, 方向向
    下,$mg-F=ma$, 所以弹簧秤读数
\[    F=mg-ma=m(g-a)=14\x(9.8-0.10)=136{\rm N}\]
    \item 物体的加速度$a=10{\rm cm/s^2}=0.10\msq$, 方向向
    下,$mg-F=ma$. 所以弹簧秤读数
    \[     F=mg- ma=m(g-a)=14\x(9.8-0.10)=136{\rm N}\]
    \item 物体的加速度$a=28{\rm cm/s^2}=0.28\msq$,方向向
    上,$F-mg=ma$, 所以弹簧秤读数
    \[     F=mg+ma=m(g+a)=14\x(9.8+0.28)=141{\rm N}\]
\end{enumerate}
\end{solution}
\end{enumerate}


\section{习题}
\begin{enumerate}
    \item 一个小金属车可以和另外两个相同的小木车在天平
上平衡.用一个力作用在小金属车上,得到2$\msq$的加速
度,如果用相同的力作用在一个静止的小木车上,经过2秒,
小木车的速度是多大?

\begin{solution}
    设小木车的质量为$m_1$, 则小金属车的质量$m_2=2m_1$.
根据在相同的力作用下,物体的加速度跟质量成反比的关系:
\[\frac{a_1}{a_2}=\frac{m_2}{m_1}\]
则小木车的加速度
\[a_1=\frac{m_2}{m_1}a_2=\frac{2m_1}{m_1}\x 2=4\msq\]
根据$v=at$, 经过2秒小木车的速度
\[v_t=a_1t=4\x2=8\ms\]
\end{solution}
\item  一个质量是$m$克的物体沿着光滑的斜面下滑(不
计滑动摩擦),斜面的倾角是$\theta$.试证明这个物体下滑的加速
度$a=g\sin\theta$.

\begin{figure}[htp]
    \centering
\includegraphics[scale=.6]{fig/3-9.png}
    \caption{}
\end{figure}

\begin{solution}
如图3.9所示,物体
在光滑斜面上下滑的过程中,
受到重力$mg$和斜面对物体的
支持力$N$的作用。

把重力分解成平行于斜面方向和垂直于斜面方向的两个
分力$F_1$和$F_2$, 则$F_1=mg\sin\theta$, $F_2=mg\cos\theta$.
由于物体在垂直于斜面方向上没有分运动,它在这个方
向上的合力等于零,即$N-F_2=0$, $N=F_2=mg\cos\theta$. 所以物
体只受到平行于斜面方向的分力$F_1$的作用,因此合力$F_{\text{合}}=F_1=mg\sin\theta$. 所以物体沿斜面下滑的加速度
\[a=\frac{F_{\text{合}}}{m}=\frac{mg\sin\theta}{m}=g\sin\theta\]
\end{solution}
\item  一个质量是10克的物体沿着光滑的斜面从静止开
始滑下(不计摩擦),开始滑下时的竖直高度是10厘米,斜面
的倾角是30$^\circ$,这个物体滑到斜面末端时的速度是多大?另一
个质量是20克的物体也沿着光滑的斜面从静止开始滑下,开
始滑下时的竖直高度相同,斜面的倾角是45$^\circ$,这个物体滑到
斜面末端时的速度是多大?写出速度$v$的表达式,并说明物体
滑到斜面末端时的速度$v$只跟开始滑下时竖直高度$h$、重力
加速度$g$有关,跟物休的质量$m$、斜面的倾角$\theta$无关.

\begin{solution}
    质量是10克的物体沿倾角是30$^\circ$的光滑斜面下滑时的加速度
\[a_1=g\sin\theta_1=9.8\x \sin 30^\circ=9.8\x \frac{1}{2}=4.9\msq\]
已知斜面高度$h=10$厘米,则斜面长
\[\ell_1=\frac{h}{\sin 30^{\circ}}=2h=2\x 10{\rm cm}=0.20{\rm m}\]
物体滑到斜面末端时的速度
\[v_1=\sqrt{2a_1\ell_1}=\sqrt{2\x4.9\x0.20}=1.4\ms\]
质量是20克的物体沿倾角是$45^{\circ}$的光滑斜面滑下时的
加速度
\[a_2=g\sin\theta_2=9.8\x\sin45^{\circ}=4.9\sqrt{2}\ms\]
已知这一斜面高度$h=10$厘米,则这个斜面的长
\[\ell_2=\frac{h}{\sin 45^{\circ}}=\frac{2h}{\sqrt{2}}=\frac{0.20}{\sqrt{2}}{\rm m}\]
物体滑到斜面末端时的速度
\[v_2=\sqrt{2a_2\ell_2}=\sqrt{2\x4.9\sqrt{2}\x\frac{0.20}{\sqrt{2}}}=1.4\ms\]

在初速度等于零的匀变速直线运动中,即时速度$v$跟加
速度$a$以及发生的位移$s$有如下关系,$v^2=2as$. 物体从斜面
上下滑时,由于$a=g\sin\theta$, $s=\ell=\dfrac{h}{\sin\theta}$,
所以
\[v=\sqrt{2a\ell}=\sqrt{2\x g\sin\theta\x \frac{h}{\sin\theta}}=\sqrt{2gh}\]

可见物体沿光滑斜面滑下,滑到斜面末端时的速度。只
跟开始滑下时的竖直高度$h$, 重力加速度$g$有关,而跟物体的
质量$m$以及斜面的倾角$\theta$无关。
\end{solution}
\item  一个放在水平面上的物体,质量是0.50千克,在水
平方向受到6.0牛的拉力,得到10$\msq$的加速度,求这个
物体和平面间的滑动摩擦系数.

\begin{solution}
    物体在水平面上运动时的加速度是由水平拉力$F$和
滑动摩擦力$f$的合力所产生的,即
\[F-f=ma\]
摩擦力
\[f=F-ma=6.0-0.50\x10=1.0{\rm N}\]
滑动摩擦系数
\[\mu =\frac{f}{N}=\frac{f}{mg}=\frac{1.0}{0.5\x 9.8}=0.20\]
\end{solution}
\item   质量是2.75吨的载重卡车,在2900牛的牵引力作
用下开上一个山坡,沿山坡每前进1米升高0.05米.卡车由
静止开始前进100米时速度达到36$\kmh$.求卡车在前
进中所受的摩擦阻力.

\begin{solution}
 设卡车沿山坡向上做匀变速直线运动,已知卡车由
静止开始前进$s=100$米时速度$v=36\kmh=10\ms$。
根据$v^2=2as$, 则加速度
\[a=\frac{v^2}{2s}=\frac{10^2}{2\x 100}=0.5\msq\]

\begin{figure}[htp]
    \centering
\includegraphics[scale=.6]{fig/3-10.png}
    \caption{}
\end{figure}

在运动过程中,卡车受到重力$mg$、山坡的支持力$N$、牵引
力$F$以及摩擦阻力$f$的作用
(图3.10),这四个力的合力使卡车产生沿着地面向上的
加速度,因此有
\[F-mg\sin\theta-f=ma\]
所以摩擦阻力
\[f=F-mg\sin\theta-ma=2900-2.75\x10^3\x9.8\x\frac{0.05}{1}-2.75\x 10^3\x 0.5=178{\rm N}\]
\end{solution}
\item  汽车开上一段坡路.汽车的质量是1500千克,发动
机的牵引力是3000牛,摩擦阻力是900牛顿.沿坡路每前进
10米升高2米,坡长282米.汽车用20秒走完这段坡路.求
上坡前的速度和到达坡顶的速度.

\begin{solution}
汽车在坡路上行驶时,共受到重力$mg$、坡路的支持
力$N$、牵引力$F$和摩擦阻力$f$的作用,它们的合力使汽车产
生加速度,因此有
\[F-mg\sin\theta-f=ma\]
所以汽车的加速度
\[a=\frac{F-mg\sin\theta-f}{m}=\frac{3000-1500\x9.8\x\frac{2}{10}-900}{1500}=-0.56\msq\]
负号表示加速度的方向和初速度方向相反,说明汽车在坡路
上行驶时做匀减速运动。

根据$s=v_0t+\dfrac{1}{2}at^2$, 汽车在上坡前的速度
\[v_0=\frac{s-\frac{1}{2}at^2}{t}=\frac{282-\frac{1}{2}(-0.56)\x 20^2}{20}=19.7\ms\]
汽车到达坡顶的速度
\[v_t=v_0+at=19.7+(-0.56)\x20=8.5\ms\]
\end{solution}
\item  有一个质量是3.0千克的木块以速度$v_0$沿光滑的水
平面移动.一个与$v_0$方向相反的18牛的力作用在木块上,
经过一段时间,木块的速度减小到原有速度$v_0$的一半,木块移
动了9.0米的路程.这段时间有多长?$v_0$是多大?

\begin{solution}
根据牛顿第二定律,物体的加速度
\[a=\frac{F}{m}=\frac{18}{3.0}=6.0\msq\]

由于力$F$和物体初速$v_0$的方向相反,所以加速度为负
值,移动了$s=9.0$米后末速为$\dfrac{v_0}{2}$
,则有
\[\begin{split}
    \left(\frac{v_0}{2}\right)^2-v^2_0&=2(-a)s\\
    -3v_0^2&=-8as
\end{split}\]
所以
\[v_0=\sqrt{\frac{8}{3}as}=\sqrt{\frac{8}{3}\x 6.0\x 9.0}=12\ms\]
木块在这段运动中经历的时间
\[t=\frac{v_t-v_0}{a}=\frac{\frac{v_0}{2}-v_0}{a}=\frac{6-12}{-6.0}=1{\rm s}\]
\end{solution}
\item  一个物体在两个彼此平衡的力作用下处于静止状
态.现在把其中某一个力逐渐减小到零,这个物体的加速度
和速度的绝对值怎样变化?如果再逐浙把这个力恢复,这个物
体的加速度和速度的绝对值又将怎样变化?

\begin{solution}
    当把两个彼此平衡着的力中的一个力逐渐减小到零
的过程中,物体所受的合力的大小将从零逐渐增大到等于这
一个力,这样,物体的加速度将从零逐渐增大到某一个最大
值,物体运动的速度也将逐渐增大,而且速度增大得越来越快,加速度达到最大值后,速度均匀增加,物体做匀加速
运动。

如果再逐渐把这个力恢复,则物体运动的合力将逐渐减
小,加速度也将逐渐减小,但速度仍将继续增大,不过增大得
越来越慢了,当这个力恢复到原来的大小时,合力将等于零,
加速度也将等于零,这时的速度将达到某一个最大值,并且不
再发生改变。物体将做匀速直线运动。
\end{solution}
\item  一个放在水平面上的质量是5.0千克的物体,受到
与水平方向成30$^{\circ}$角的斜向上方的拉力作用,物体产生沿水
平方向的加速度是2$\msq$.物体跟平面的滑动摩擦系数是
0.1.求拉力是多大?

\begin{figure}[htp]
    \centering
\begin{tikzpicture}[>=latex]
\draw(-2,0)--(4,0);
\draw (0,0) rectangle (1,1);
\tkzDrawPoint(.5,.5)
\draw[->, thick](.5,.5)--(-.5,.5)node[left]{$f$};
\draw[->, thick](.5,.5)--(.5,2)node[above]{$N$};
\draw[->, thick](.5,.5)--(.5,-1)node[below]{$mg$};
\draw[->, thick](.5,.5)--+(30:2)node[right]{$F$};
\draw[dashed](.5,.5)--(3,.5);
\tkzDefPoints{.5/.5/O, 3/.5/A, 1.5/1.077/B}
\tkzLabelAngle[right](A,O,B){$\theta=30^{\circ}$}
\tkzMarkAngle[mark=none, size=.8](A,O,B)

\end{tikzpicture}
    \caption{}
\end{figure}

\begin{solution}
设物体所受到的拉力为$F$, 摩擦力为$f$, 水平面对物
体的支持力为$N$, 重力为$mg$, 如图3.11所示。物体在这四个力
的作用下,沿着水平面做匀变速直线运动。拉力$F$在垂直于
水平面方向的分力是$F\sin\theta$,
在平行于水平面方向的分力是$F\cos\theta$。

由于物体在垂直于水平面方向的加速度为零,
\[N+F\sin\theta-mg=0,\qquad  N=mg-F\sin\theta\]
所以摩擦力
\[f=\mu N=\mu (mg-F\sin\theta)\]
在水平方向,$F\cos\theta-f=ma$. 将$f$的表达式代入此式得
\[F\cos\theta-\mu mg+\mu F\sin\theta=ma\]
所以拉力
\[F=\frac{ma+\mu mg}{\cos\theta+\mu \sin\theta}=\frac{5.0\x 2+0.1\x 5.0\x 9.8}{\frac{\sqrt{3}}{2}+0.1\x \frac{1}{2}}=16.3{\rm N}\]
\end{solution}
\item  略(课本已作解答)。

\item    文艺复兴时代意大利的著名画家和学者达$\cdot$芬奇
提出了如下的原理:
    如果力$F$在时间$t$内使质量是$m$的物体移动一段距离
$s$,那么:
\begin{enumerate}
    \item  相同的力在相同的时间内使质量是一半的物休移动
    $2s$的距离;
\item  或者相同的力在一半的时间内使质量是一半的物体
移动相同的距离;
\item  或者相同的力在两倍的时间内使质量是两倍的物体
移动相同的距离;
\item  或者一半的力在相同的时间内使质量是一半的物体
移动相同的距离;
\item  或者一半的力在相同的时间内使质量相同的物体移
动一半的距离.
\end{enumerate}
    这些原理正确不正确?为什么?

    \begin{solution}
可根据牛顿第二定律和运动学公式来判断达·芬奇
所提出的原理是否正确。

设讨论的前提为物体原来是静止的,则
\[s=\frac{1}{2}at^2,\qquad F=ma\]
于是\[s=\frac{1}{2}\frac{F}{m}t^2\]
\begin{enumerate}
    \item 设$F_1=F,\quad m_1=\dfrac{m}{2},\quad t_1=t$, 则
\[s_1=\frac{1}{2}a_1t_1^2=\frac{1}{2}\cdot\frac{F_1}{m_1}t^2_1=\frac{1}{2}\cdot \frac{F}{m/2}t^2=2\left(\frac{1}{2}\cdot\frac{F}{m}t^2\right)=2s\]
可见原理(a)是正确的。
\item 设$F_2=F,\quad m_2=\dfrac{m}{2},\quad t_1=\dfrac{t}{2}$, 则
\[s_2=\frac{1}{2}a_2t_2^2=\frac{1}{2}\cdot\frac{F_2}{m_2}t^2_2=\frac{1}{2}\cdot \frac{F}{m/2}\left(\frac{t}{2}\right)^2=\frac{1}{2}\left(\frac{1}{2}\cdot\frac{F}{m}t^2\right)=\frac{1}{2}s\]
可见原理(b)是不正确的。
\item 设$F_3=F,\quad m_3=2m,\quad t_3=2t$, 则
\[s_3=\frac{1}{2}a_3t_3^2=\frac{1}{2}\cdot\frac{F_3}{m_3}t^2_3=\frac{1}{2}\cdot \frac{F}{2m}(2t)^2=2\left(\frac{1}{2}\cdot\frac{F}{m}t^2\right)=2s\]
可见原理(c)是不正确的。
\item 设$F_4=F/2,\quad m=\dfrac{1}{2}m,\quad t_4=t$, 则
\[s_4=\frac{1}{2}a_4t_4^2=\frac{1}{2}\cdot\frac{F_4}{m_4}t^2_4=\frac{1}{2}\cdot \frac{F/2}{m/2}t^2=\frac{1}{2}\cdot\frac{F}{m}t^2=s\]
可见原理(d)是正确的。
\item 设$F_5=F/2,\quad m_5=m,\quad t_5=t$,则
\[s_5=\frac{1}{2}a_5t_5^2=\frac{1}{2}\cdot\frac{F_5}{m_5}t^2_5=\frac{1}{2}\cdot \frac{F/2}{m}t^2=\frac{1}{2}\left(\frac{1}{2}\cdot\frac{F}{m}t^2\right)=\frac{1}{2}s\]
可见原理(e)是正确的。
\end{enumerate}    
    \end{solution}
\item   有两个物体,质量为$m_1$和$m_2$,$m_1$
原来静止,$m_2$以速度$v_0$向右运
动(图3.16).它们同时开始受到大小
相等、方向与$v_0$相同的恒力$F$的作用,
它们能不能在某一时刻达到相同的速
度?分$m_1<m_2$, $m_1=m_2$, $m_1>m_2$三种情况来讨论.
\begin{figure}[htp]\centering
    \begin{tikzpicture}[scale=.6, >=stealth]
       \draw  (0,0) rectangle (2.5, 2);
       \draw  (0,3) rectangle (2.5, 5);
    \node at (2.5/2,4){$m_1$};
    \node at (2.5/2,1){$m_2$};
    \draw[->](3,1)--node [below]{$v_0$}(4,1);
    
    \end{tikzpicture}
    \caption{}
    \end{figure}

\begin{solution}    
质量为$m_1$的物体甲做初速度等于零的匀变速直线
运动,经过时间$t$的速度$v_1=\dfrac{F}{m_1}t$。

质量为$m_2$的物体乙做初速度为$v_0$的匀变速直线运动。
经过时间$t$的速度$v_2=v_0+\dfrac{F}{m_2}t$,
由此得
\[v_2-v_1=v_0+\left(\dfrac{F}{m_2}-\dfrac{F}{m_1}\right)t=v_0+\frac{m_1-m_2}{m_1m_2}Ft\]

当$m_1\ge m_2$时,$m_1-m_2\ge 0$, 有$v_2-v_1\ge v_0>0$. 即$v_1$和$v_2$永远不会相等。

当$m_1<m_2$时,有
\[v_2-v_1=v_0-\frac{m_2-m_1}{m_1m_2}Ft\]
令$v_2-v_1=0$, 得
\[t=\frac{m_1m_2v_0}{(m_2-m_1)F}\]
这时有$v_1=v_2$, 即甲乙两物体的速度相等。

说明:由于物体的初速度(等于零)比乙物体的初速度
($v_0$)小,显然,只有甲的加速度比乙的加速度大,即
\[\frac{F}{m_1}>\frac{F}{m_2},\qquad m_1<m_2\]
时,两个物体才能达到相同的速度,这一点,不用公式也可以
得出,借助于公式则可以更明显地看出定量的关系。
\end{solution}
\end{enumerate}

\section{参考资料}
\subsection{研究加速度和力以及质量关系的实验中的误差分析}

在课本实验八和实验九中把砂桶和砂的重量看作使小车
产生加速度的力的条件是:砂桶和砂的质量$m$. 必须远小于小
车的质量$M$. 小车和砂桶应具有相等的加速度,
\[a=\frac{mg}{M+m}\]
式中的$mg$是砂桶和砂的重量,也是作用于由小车和砂桶及
砂组成的物体系的合外力。实验时为了使计算简化,取
$a\approx \dfrac{m}{M}g$。这样就产生了误差,这一误差是由于实验原理的不
完善而带来的,属于系统误差。

下面用具体数字计算一下这一误差的大小。如果实验时
所用小车的质量为200克,砂桶和砂的质量不超过20克,则
加速度的最大值
\[a=\frac{mg}{M+m}=\frac{0.02\x 9.8}{0.2+0.02}=0.89\msq\]
若忽略砂桶和砂的质量$m$, 则
\[a'=\frac{mg}{M}=\frac{0.02\x 9.8}{0.2}=0.98\msq\]
$a'$与$a$相差(即绝对误差)
\[E_a=a'-a=0.98-0.89=0.09\msq\]
相对误差(即百分误差)
\[E_r=\frac{E_a}{a}\x 100\%=\frac{0.09}{0.89}\x 100\%=10.1\%\]

相对误差在10\%左右,这在中学物理实验中是许可的。
当砂桶和砂的质量小于20克,或在小车上加砝码以增大小车
质量时,相对误差均小于10\%, 因此课本上忽略砂桶和砂的
质量来计算加速度的方法是允许的。但是如果砂桶和砂的质
量增大为50克,而小车质量仍为200克,则
\[a=\frac{mg}{M+m}=\frac{0.05\x 9.8}{0.2+0.05}=1.96\msq\]

若忽略$m$, 则
\[a'=\frac{mg}{M}=\frac{0.05\x 9.8}{0.2}=2.45\msq\]
则绝对误差
\[E_a=a'-a=2.45-1.96=0.49\msq\]
相对误差
\[E_r=\frac{E_a}{a}\x 100\%=\frac{0.49}{1.96}\x 100\%=25\%\]
这样的误差就太大了,所以在实验时所用砂桶和砂的质量以
不超过小车质量的1/10为宜.

\subsection{惯性质量和引力质量}
使物体改变运动状态,需要力的作用,在相同的力作用
下,质量越大的物体的加速度越小。这表明了质量是表示物
体所具有的阻碍运动状态改变的一种属性,质量越大,物体越
不容易改变其运动状态,所以质量是物体惯性大小的量度,物
体的这一性质跟物体是否受有重力作用完全无关(譬如放在
水平的气垫导轨上的滑块,或物体在完全失重的情况下)。因
此,牛顿第二定律的公式$F=ma$中所出现的质量$m$, 叫做惯
性质量。

根据万有引力定律可知,物体受到的地球引力的大小和
物体的质量成正比,为了使物体不致由于受到地球引力而掉
向地面,可将物体用绳子悬挂起来(或用支持物支承住)。这样,
绳子(或支持物)就发生形变,物体的质量越大,需要绳子(或
支持物)发生更大程度的形变才能产生足够大的弹力来跟物
体所受到的地球引力相平衡。因此,在这里质量的概念反映了
物体所包含的物质的多少,质量越大,物体所含的物质越多,
受到的地球引力就越大。因此,万有引力定律公式
$F=G\dfrac{Mm}{R^2}$
中所出现的物体质量$m$, 叫做引力质量。

惯性质量和引力质量是从不同的侧面来描述了物质的属
性,它们之间存在着怎样的关系呢?

设有$A,B$两个物体,它们的惯性质量分别为$m_A$、$m_B$, 引
力质量分别为$m'_A$、$m'_B$. 把$A$、$B$这两个物体放在地球上的
同一地点,则它们所受到的地球引力分别为:
\[F_A=G\frac{Mm'_A}{R^2}=G_A,\qquad F_B=G\frac{Mm'_B}{R^2}=G_B\]

若将以上两式相比,则得:
\begin{equation}
    \frac{G_A}{G_B}=\frac{m'_A}{m'_B}
\end{equation}
这表明了$A$、$B$物体所受重力的比等于它们的引力质量的比。

如果使$A$、$B$物体在重力的作用下自由下落,则根据牛顿
第二定律可知,$G_A=m_Ag_A$; $G_B=m_Bg_B$。

由于在同一地点,重力加速度都相等,即$g_A=g_B=g$. 于
是:
\begin{equation}
    \frac{G_A}{G_B}=\frac{m_A}{m_B}
\end{equation}

这表明了在地球上同一地点,物体的重量的比等于它们
的惯性质量的比。

比较(3.1)式和(3.2)式,可见物体的惯性质量$m$和引力质量
$m'$是一致的。

对单摆的振动加以讨论,也可以得出惯性质量和引力质
量是等效的结论。单摆振动在偏角很小的情况下,可看作是
有简谐振动,对于简谐振动来说,它的周期$T=2\pi\sqrt{\frac{m}{k}}$, 
式中$m$是振动系统的惯性质量,$k$是决定于振动系统的一个常数。
在单摆这一振动系统中,$k=\dfrac{m'}{\ell}g$,式中$m'$是摆球的引力质量.
代入周期公式,得单摆振动的周期公式
\[T=2\pi\sqrt{\frac{m\ell}{m'g}}\]

从实验证明,在摆角很小时,单摆的振动周期跟摆长l的
平方根成正比,跟所在地点的重力加速度$g$的平方根成反比,
而与物体质量无关,即
$$T=2\pi\sqrt{\frac{\ell}{g}}$$
这只有认为$m'=m$的情
况下才是可能的,因此物体的惯性质量和引力质量是等效的。

因此,在中学物理教学中,不必区分惯性质量和引力质量。


\subsection{狭义相对论和经典力学}
以牛顿定律为基础的经典力学认为,物体的长度和时间
间隔跟物体运动的速度没有关系,即把空间和时间看成是绝
对的。经典力学还认为物体的质量跟物体运动的速度大小也
是无关的。这些结论对于研究宏观物体的运动无疑是正确的,
跟人们的日常经验也是一致的。

但在研究微观粒子的高速运动时,发现了用经典力学无
法解释的矛盾.爱因斯坦在1905年发表了《论运动物体的电
动力学》的著名论文,提出狭义相对性原理(即第一个假设:在
所有惯性系中物理定律具有相同的形式,没有一个惯性系比
别的惯性系更优越)和光速不变原理(即第二个假设:在所有
惯性系中光速都是相同的),创立了狭义相对论,可以用来处
理微观粒子的高速运动的问题。

狭义相对论指出:有两个惯性系$s$和$s'$, 它们具有互相平
行的坐标轴$(x,y,z)$, 并且$x$-$x'$轴是公共的.若$s'$相对于$s$
以速率$v$运动,在这两个不同的惯性系中的观察者,对于同一
地点发生的两个事件之间的时间间隔的测量结果是不同的。
在$s'$中的观察者测得的时间间隔为$\Delta t'$, 而由于$s$中的观察
者认为$s'$是在运动着的,他所测得的时间间隔$\Delta t$由下式
给出:
\[\Delta t=\frac{\Delta t'}{\sqrt{1-\left(\dfrac{v}{c}\right)^2}}\]

可见$\Delta t>\Delta t'$. 即时间发生了“膨胀”,运动的钟变慢
了——通常叫做钟慢效应。

如果有一根相对于$s'$是静止的尺,平行于$x'$轴放置,在
$s'$中测得尺的长度为$\Delta x'$, 而在$s$中的观察者看来,这把尺是
运动的,他测得的尺的长度$\Delta x$由下式给出:
\[\Delta x={\sqrt{1-\left(\dfrac{v}{c}\right)^2}}\Delta x'\]

可见$\Delta x<\Delta x'$, 即长度发生了“收缩”,运动的尺变短
了——通常叫做尺缩效应。

狭义相对论还导出,跟对时间间隔和长度的测量一样,质
量$m$也是速率$v$的函数。如果质点静止时的质量为$m_0$, 则
质点相对于观察者以速率$v$运动时,它的相对论质量$m$将由
下式给出:
\[m=\frac{m_0}{\sqrt{1-\left(\dfrac{v}{c}\right)^2}}\]

可见$m>m_0$, 即质量随着速率的增大而变大,通常叫做
质-速关系。

以上三个关系式表明了狭义相对论提出的一种新的时空
观。而当速率$v\ll c$时,则$\Delta t\approx \Delta t'$, $\Delta x\approx \Delta x'$, $m\approx m_0$. 这表
明经典力学所描述的运动规律是狭义相对论的特殊情况。因
而处理宏观物体的低速运动时使用牛顿力学是足够准确的。

此外,从狭义相对论的结论中,还导出了质量和能量的等
效原理。
\[E=mc^2\]

这表明了总能量守恒和相对论质量守恒是等效的,这一
原理是研究原子核反应及其应用的一个重要理论依据。

 
\chapter{曲线运动}
\section{教学要求}
这一章讲授平抛、斜抛和匀速圆周运动,是牛顿运动定律
在“新”的领域的具体运用,速度和加速度的方向性,在曲线
运动显得非常重要,通过本章的学习,学生对这两个概念的矢
量性将得到进一步的领会。

这一章的教学要求是:
\begin{enumerate}
\item 了解曲线运动的特点和物体做曲线运动的条件,掌握
运动的合成与分解的方法。
\item 理解平抛运动和斜抛运动的规律,会用运动的合成与
分解的方法分析这两种运动。
\item 理解匀速圆周运动的角速度、线速度和周期的概念,
掌握它们之间的关系$v=r\omega=\dfrac{2\pi r}{T}$
\item 掌握向心加速度的概念,会用向心加速度的公式.掌
握向心力的概念,会分析做匀速圆周运动的物体的受力情况,
找出向心力的来源。
\item 了解离心现象及其应用。
\end{enumerate}

下面对这一章的教学内容作些具体说明。

关于曲线运动的一般特点,主要说明曲线运动中速度的
方向是时刻变化的,因此,曲线运动都是变速运动,对于曲线
运动的速度方向,教材是通过分析现象得出的,没有从数学上
进行论证,这是考虑到这种讲法比较简单易懂,便于学生接
受。关于物体做曲线运动的条件,要求学生知道,但不要求进
一步分析合外力对速度的大小和方向的影响。程度好一些的
学生,可以在学完第七节后,通过阅读材料“力的分解与曲线
运动”,来加深对这个问题的理解。

运动的合成和分解是作为研究平抛和斜抛的方法来讲解
的。因为是准备性知识,所以讲得比较简要。要求学生掌握
运动的合成和分解的方法,但不要求进行论证。教材是对同
一参照物来讲解运动的合成和分解的,不涉及相对速度、牵连
速度等概念。课本里提到的运动都是相对于地面来说的,不
会误解,所以没有把参照物写出来。

关于平抛和斜抛运动,主要是要使学生认识抛体运动可
以看作是竖直方向和水平方向两个直线运动的合运动。对于
计算飞行时间、射程和射高的公式,主要要求学生知道它们是
怎样推导出来的,掌握推导方法,而不要死记硬背。平抛和斜
抛运动的轨迹曲线,教材是用描点法画出的,不要求由已给
出的时间参数方程推导出用$x$和$y$表示的抛物线方程。

角速度和周期这两个描述匀速圆周运动快慢的物理量,
学生初次接触,应使学生理解它们的物理意义。同时还应要
求学生搞清楚线速度、角速度、周期这三个量是从不同角度
来描述匀速圆周运动快慢的,并掌握它们之间的关系。匀速
圆周运动的线速度的大小虽然不变,但方向却时刻在改变,因
而匀速圆周运动是变速运动,认识这一点是后面讲解向心加
速度的前提。

关于向心加速度,教材是用矢量方法讲的。这种讲法能
突出向心加速度的方向,有利于学生正确理解向心加速度,对
向心加速度的推导学生只要能听懂就行,并不要求学生自己
会推导。由于全书都不写矢量式,这里也没有用矢量式。为
了避免用矢量差,采用学生学过的矢量合成的三角形法,用
$v_B$等于$v_A$与$\Delta v$之和的提法代替了$\Delta \vec{v}=\vec{v}_B-\vec{v}_A$, 不同的教师
可能爱好的推导方法不一样,在教学中可以不必拘泥于课本
中的方法,根据自己的爱好选用其他的推导方法。

导出向心加速度的公式后,应使学生知道匀速圆周运动
是变加速运动,对于不同形式的向心加速度公式:$a_n=\dfrac{v^2}{r}$
和
$a_n=\omega^2 r$, 应使学生了解它们的物理意义,知道在什么条件下
$a_n$与$r$成反比,在什么条件下$a_n$与$r$成正比;并能利用$v$、$\omega$、
$r$之间关系,从一种形式推导出另一种形式,以备后面学习中
应用。

讲解向心力时,应突出牛顿运动定律的应用。向心力的
大小可由牛顿第二定律算出,向心力的方向跟加速度的方向
时刻都是一致的,总是指向圆心,在分析向心力问题时,也应
注意强调从牛顿运动定律出发进行研究。

分析向心力的来源是一个难点,又是理解和掌握向心力
的关键。为此教材写了“匀速圆周运动的实例分析”一节,就
一些实际的例子分析了向心力的来源,以及向心力的大小跟
物体做匀速圆周运动情况的关系。并且,使学生进一步明确:
使物体做匀速圆周运动的向心力就是物体受到的合外力。

“离心现象”一节主要是使学生了解物体做离心运动的原
因和离心机械的原理。

\section{教学建议}
本章可分为三个单元进行教学,第一单元是第一节《曲线
运动》,第二单元从第二节《运动的合成和分解》到第四节《斜
抛物体的运动》,第三单元从第五节《匀速圆周运动》到第九节
《离心现象》。

\subsection{第一单元}
这一单元是整章的讨论前提与研究曲线运动的基础,主
要使学生认识曲线运动中的速度方向是时刻改变的,获得曲
线运动是变速运动的印象以及正确理解物体做曲线运动的
条件。

\subsubsection{曲线运动的速度方向}

关于曲线运动的速度方向,课
本是通过图4.2所示的现象得出结论的.在分析这个现象时
要使学生认识,即使刀具不跟砂轮接触,砂粒没有落下,也没
有火星飞出,砂轮边缘上各点做圆周运动的速度方向,仍然
是在儡周各点的切线方向上。认清了这一点之后,再进一步引
导学生明确曲线运动是一种变速运动,为进一步学习物体做
曲线运动的条件做好准备。

\subsubsection{物体做曲线运动的条件}

讲这部分内容时先明确一下,
曲线运动既然是变速运动,运动
物体所受的力一定是不平衡的。
然后做一个演示实验(图4.1):
一个乒乓球在水平桌面上沿着一条直线滚动到桌子边缘的$B$
点。当乒乓球离开桌边后,由于在竖直方向上受到重力的作
用,就改变了原来的直线运动状态,沿着曲线$BCD$运动了,再
分析乒乓球的受力情况,使学生认识物体做曲线运动的条件
是:物体具有初速度,并且受到跟初速度方向成一角度(不包
括$0^{\circ}$和$180^{\circ}$这两种特殊情况)的合外力的作用。

\begin{figure}[htp]
    \centering
\includegraphics[scale=.5]{fig/4-1.png}
    \caption{}
\end{figure}

\subsection{第二单元}
这一单元学习运动的合成和分解以及平抛和斜抛物体的
运动。把复杂的运动看成由两个或几个比较简单的分运动所
组成,是这一单元中要学习的一种重要的分析研究问题的
方法。

\subsubsection{运动的合成和分解}

学生对轮船渡河的现象可能缺
乏感性认识,可以先做一个简单的演示(详见本章实验指导)
加以说明,然后再分析课本所讲的例子。

在讲解运动的合成和分解时,可以向学生说明分运动的
性质决定了合运动的性质与合运动的轨迹。如果物体的两个
分运动都是匀速直线运动,则合运动必定也是匀速直线运动,
合运动的位移和分运动位移间的关系符合平行四边形法则。
而且合运动的位移等于合运动的路程。如果物体的两个分运
动,互成一定角度,其中一个是匀速直线运动,另一个是变速
运动,则合运动是变速运动,而且运动轨迹必定是曲线。合运
动的位移仍能用平行四边形法则进行计算,但是合运动位移
并不等于合运动轨迹的长度——路程。

\subsubsection{平抛物体的运动}

讲解平抛运动时,在演示实验(详见本章实验指导)和对课本图4.10的闪光照片进行分析的基
础上,应使学生认识以下几点。

平抛运动是水平方向的匀速直线运动和竖直方向的
自由落体运动的合运动.根据课本图4.9的实验,两个小球
同时落地的事实,得出平抛运动在竖直方向上的分运动是自
由落体运动的结论,而对课本图4.10的闪光照片进行分析的
结果可以证明做平抛运动的小球在水平方向上的分运动是匀
速直线运动。

为了加深学生对平抛运动的这两个分运动的认识,也可
以再从理论上进行分析:如果小球在做平抛运动之前是在一
个光滑的水平面上运动,则根据牛顿第一定律可知,小球将保
持它的匀速直线运动状态不变,而当小球离开光滑水平面后,
在水平方向上没有受到任何力(忽略空气阻力)的作用,于是
它将继续保持做匀速直线运动。而在竖直方向上,由于小球
不再受到水平面支持力的支托,它将在重力作用下产生加速
度,做初速度等于零、加速度等于$g$的匀变速直线运动。

平抛运动的飞行时间和水平距离。在讨论平抛运动
的公式时,应使学生注意,平抛物体的飞行时间$t$受到下降距
离$y$的限制,$t=\sqrt{\dfrac{2y}{g}}$
譬如让两个身高不同的同学推铅球,假
定他们推出的铅球初速度都在水平方向上,而且大小相等,显
然他们能使铅球做平抛运动的时间是不相等的,高个同学推
出的铅球做平抛运动的时间长些,推出的铅球飞行的水平距
离也将大些。所以说,平抛运动的时间$t$决定于自由下落的
距离$y$, 而水平距离$x$则跟平抛初速$v_0$和飞行的时间$t$都有
关系。

平抛物体的落地速度.
在讨论平抛物体的落地速度
时,要让学生了解,落地速度
是水平方向分速度和竖直方向分速度的合速度,这一速度的
方向应在其运动轨迹这一点的
切线方向上,如图4.2所示.因
此是不可能垂直于地面的。如果落地速度$v$和水平方向夹角为$\theta$, 则\[\tan\theta= \frac{v_y}{v_x}=\frac{\sqrt{2gy}}{v_0} \]
在求矢量的运算中,有的同学往往只注意求出矢量的大小,忘记
了求出矢量的方向。这是初学者易犯的通病,可以结合平抛
物体落地速度的计算,提醒他们注意这个问题。

\begin{figure}[htp]
    \centering
\includegraphics[scale=.5]{fig/4-2.png}
    \caption{}
\end{figure}

还应该使同学们了解,对于落地点的即时速度的讨论同
样适用于平抛运动轨迹上的任一点。

\subsubsection{斜抛物体的运动}

在做好课本图4.14演示实验的
基础上,使学生认识以下几点:
\begin{enumerate}
\item 可以把斜抛运动看成是水平方向的匀速直线运动和
竖直方向的上抛运动的合运动。
\item 斜抛运动具有一定的对称性,这包括物体到达最高
点的时间与从最高点落回到地面的时间相等,初速度的大小
与落地时的末速度的大小相等,抛射角与到达落地点时末
速度与水平面间的夹角相等。
\item 做斜抛运动的物体在到达最高点以后的运动是平抛
运动,因此平抛运动可以看成是斜抛运动的一部分。
\item 不论是平抛运动还是斜抛运动,物体都具有跟重力
方向成一角度的初速,并且是只在重力作用下的运动。因此
它们的共同点表现为运动的加速度都等于重力加速度$g$, 运
动中的速度变化,都发生在竖直方向上。
\end{enumerate}

\subsection{第三单元}
这一单元主要讲匀速圆周运动,是本章的教学重点。
\subsubsection{描述匀速圆周运动快慢的几个物理量}
要使学生认
识用线速度来比较质点做匀速圆周运动的快慢时,质点运动
的圆周半径必须相同的。而用周期和角速度来描述匀速圆
周运动的快慢程度时,则不必考虑圆周的半径。这两种不同
的描述方法,各有它们的长处,在以后的学习中将根据讨论问
题的方便选用不同的描述方法。

课本练习五第4题提到的“每分钟的转数$n$”同
样可以描述匀速圆周运动的快慢程度。为了使学生更具体地
理解这一点,可以比较指针式手表上三个指针的运动,秒针的
转数最大,$n_1=1$转/分,分针的转数$n_2=\frac{1}{60}
$转/分,而时针的转
数最小,$n_3=\frac{1}{720}$
转/分。因此秒针上的质点(除转轴外)做匀速
圆周运动转动最快,而不论这些质点离开转轴距离的大小
如何。

\subsubsection{匀速圆周运动是变速运动}

在匀速圆周运动中,周期
和角速度这两个量是不随时间而改变的。线速度则是随时间
变化的。这一点有的同学不易接受,关键在于忽略了速度是
个矢量。线速度的大小虽然不变,但它的方向却是时刻改变
的。匀速圆周运动中的“匀速”,是指线速度的大小不变而
言的。

\subsubsection{匀速圆周运动中的向心加速度}

要使学生正确认识向
心加速度公式的两种表达式$a_n=v^2/r$
和$a_n=\omega^2 r$的物理意义,
在利用这两个公式来比较两个做匀速圆周运动的质点的向心
加速度的大小时,能搞清当线速度$v$相等时,向心加速度$a_n$, 跟
运动半径$r$成反比;当角速度$\omega$相等时,向心加速度$a_n$跟运
动半径$r$成正比。这一点可以结合自行车的传动装置来说明:
跟踏脚板连在一起的链轮边缘质点的线速度$v$和飞轮边缘质
点的线速度$v$是相等的,由于链轮的半径大于飞轮的半径,因
此链轮边缘质点的向心加速度小于飞轮边缘质点的向心加速
度。而飞轮和自行车后轮是同轴装置的,它们的角速度$\omega$相
等,所以后轮边缘质点的向心加速度大于飞轮边缘质点的向
心加速度。

\subsubsection{向心力}

向心力的教学,要使学生认识如下几个问题:
\begin{enumerate}
\item 在匀速圆周运动中必定有产生向心加速度的向心
力。要强调指出,向心力是根据力的效果来命名的,而不是根
据力的性质来命名的,因此,它不是重力,弹力、摩擦力等以外
的特殊的力,而是做匀速圆周运动的质点受到的合外力,沿着
半径指向圆心,它的方向时刻改变,因此产生向心加速度的
力——向心力也是始终沿着半径指向圆心的,它是一个变力。
\item 质点做匀速圆周运动的条件是:质点具有初速度$v$,
并且始终受到跟线速度方向垂直,大小等于
$mv^2/r$的合外力(即向心力)的作用。
\end{enumerate}

\subsubsection{做匀速圆周运动物体的受力分析}

确定物体所需的
向心力的来源,是研究匀速圆周运动的关键.课本图4.20、
4.21、4.22、4.24和4.27所分析的五个例子都是物体在水平
面上做匀速圆周运动的情况,因此它们受力情况的共同点是:
不论物体受几个力的作用,合力一定是在水平方向,沿着半径
指向圆心的,这一合力就是使物体做圆周运动所需的向心力。
在进行具体例子的分析时,要引导学生注意如下两点:

\begin{enumerate}
\item 要判断圆心的位置和
质点做圆周运动的半径,例如
在北京的物体随地球自转做匀
速圆周运动的圆心位置并不是地球的中心,而是从北京的纬
度处作地轴的垂线的垂足$O$(图
4.3). 而这一垂线的长度就是
在北京的物体做圆周运动的半
径$r=R\cos40^{\circ}$. 又如在圆锥摆的运动中(图
4.24), 小球做匀速圆周运动的圆心位置在圆锥底面的中心,
而不是悬绳上端的固定点,小球的运动半径是圆锥底面的半
径,而不是悬绳的长度。
\item 对物体进行受力情况分析时,要求作出受力图,然后
根据牛顿第二定律来确定加速度和力的关系。
\end{enumerate}

\begin{figure}[htp]
    \centering
    \includegraphics[scale=.5]{fig/4-3.png}
    \caption{}
\end{figure}


\subsubsection{离心现象}

在讲解离心运动时,要注意明确两点:
\begin{enumerate}
    \item 离心运动不是由于受到“离心力”的作用,离心运动是惯性的表现。
    \item 做匀速圆周运动的物体在合外力突然消失后,不是
    沿着半径方向“离心”而去的,而是沿着失去向心力的这一位
    置时的切线方向飞出。
\end{enumerate}

因此对于课本图4.28应有如下的理解:
\begin{enumerate}
    \item 这是一个示意图,为了便于比较,把三种情况画在同
一个图中。
\item 图中离开圆心$O$距离最近的一个小球,是能够沿着
原来的圆弧做匀速圆周运动的,所需的向心力能够得到满足,
这是正常的情况。
\end{enumerate}

离开圆心$O$距离稍远的一个小球,是表示由于向心力不
足,这一比较小的向心力将不能使小球沿着原来的半径继续
做圆周运动,在认为线速度$v$不变(由于惯性)的条件下,向心
力$F=mv^2/r$
减小后,从这一即时的情况来看,小球只能在曲率
半径$r$较大的一小段圆弧上运动。这样,对原来的圆心位置
来说,距离就远了。

离开圆心$O$最远的一个小球,是表示向心力突然消失,消
失的位置是在圆心$O$的正上方,因此小球将沿着切线方向
飞出。

后面的两种情况都是小球的离心运动。


\section{实验指导}
\subsection{演示实验}
\subsubsection{物体做曲线运动的条件}

如图4.4所示,利用投影仪观察,使一小铁球从斜槽上滚
下,小球将沿直线$OO'$运动,然后在垂直于$OO'$的方向上放
一块条形磁铁,使小球再次从斜槽上滚下后,它将偏离原来的
运动方向做曲线运动(演示时磁铁不要离$OO'$线太近,以免
小铁球被磁铁吸住)。


\begin{figure}[htp]
    \centering
    \includegraphics[scale=.5]{fig/4-4.png}
    \caption{}
\end{figure}

\subsubsection{运动的合成}
把一个注满水的乒乓球用细线系住,细线的另一端用图
钉固定在小黑板左侧的$B$点。放松细线,乒乓球静止在$A$点,
如图4.5甲所示,在小黑板上经过$B$点斜向上(方向任意)画
一直线$BB'$(注意使$BB'$的长度约等于$AB$间的距离)如图4.5
乙所示。
\begin{figure}[htp]
    \centering
    \includegraphics[scale=.5]{fig/4-5.png}
    \caption{}
\end{figure}

如果将食指勾在悬线的左侧,使手指沿着直线$BB'$移动,
便可观察到乒乓球同时参与从$A\to B$以及从$B\to B'$两个不同
方向的运动,而其实际运动轨迹的是沿着$A$点和$B'$点的连线
方向,如图4.5丙所示,从而说明两个匀速直线运动的合运
动仍是一个匀速直线运动,合运动位移和分运动位移的关系
符合平行四边形法则。

\subsubsection{自由落体和平抛物体同时落地}
课本图4.9的演示,为了使效果更好,可以使$B$球略
小于$A$球,或者设法将$B$球下方的圆孔用锉刀略为锉大些,这
样,当小锤打击弹性金属片把A球抛出的同时,$B$球就立即自
由下落,不致因碰到圆孔边缘而受到阻碍。
\begin{figure}[htp]
    \centering
    \includegraphics[scale=.5]{fig/4-6.png}
    \caption{}
\end{figure}

图4.6的装置可用来验证平抛运动在竖直方向的分
运动是自由落体运动。图中$M$是电磁铁,调节它的位置,使得
接通电路时被吸住的小铁球$B$的高度和由斜槽$S$上滚下做平
抛运动的小球$A$离开槽口时的高度相同,在斜槽的槽口用弹
簧铜丝做一个触断开关,利用铜丝的弹性使开关常闭,当小球
$A$从槽口滚出时,使电路断开,因此原来被电磁铁$M$吸住的
小球$B$同时被释放做自由落体运动,可以观察到$A$、$B$两球在
$C$处相碰。

如果将电磁铁$M$向右移过一小段距离,把$B$球仍吸住,
重复实验,则$A$、$B$两球相碰点$C'$在原来相碰点$C$的右下方。

如果使$A$球在斜槽上较高的位置释放(可用一块条形磁
铁隔着有机玻璃盖板吸引$A$球进行控制),使它做平抛运动的
水平初速增大,则可观察到$A$、$B$两球的相遇点在$C$点的正上
方,若使$A$球在斜槽上较低的位置释放,使它的水平初速减
小,则两球的相遇点在$C$点的正下方。由此可表明平抛运动
在竖直方向上的分运动是自由落体运动。

\subsubsection{斜抛物体的射程跟初速度和抛射角有关}
在课本图4.13和图4.14的演示中要注意:
\begin{enumerate}
\item 喷水管的管口位置要和桌上的接水槽在同一水平面
上。特别在改变喷射角时,要注意保持管口的水平高度不变。
\item 可以在喷水管的后面放置一块演示用的大量角器,并
在水流溅落处立一个标记,以便用来观察当抛射角为$45^{\circ}$时
射程最大以及证实抛射角互为余角时射程相等的结论。
\item 为了使水流初速度保持不变,有条件的情况下也可以
把喷水管直接接在自来水龙头上。
\end{enumerate}



\subsubsection{向心力跟哪些因素有关}
课本图4.20的实验,可以用一$k$值较小的弹簧来
代替弹簧秤。将弹簧的一端固定,另一端和尼龙线拴在一起,
当橡皮塞做圆周运动时,可以看到弹簧的明显伸长,这样可以
避免用弹簧秤时由于弹簧秤的转动而看不清示数的变化,但
是,当转速增大、弹簧明显伸长时会使转动半径变大,因此在
操作时要注意在增大转速的同时,握住笔杆的手必须适当地
抬高,使半径基本上保持不变。这个演示只能粗略地定性说
明向心力和角速度以及半径的关系。

课本图4.21的演示可用一个拴在绳端的小球(当作
质点)来代替滑块,绳的另一端用一光滑的小环套在转轴上,
给小球一个垂直于绳子方向的初速就可以观察到小球在水平
面上做圆周运动。

如果用弹簧代替上述演示中的绳子,则可观察到当
转速较大时,弹簧伸长较明显,当转速较小时,弹簧仲长得
较少。

\subsubsection{配合课本习题的演示}
课本172页第10、11题.如图4.7所示,将一圆
弧形轨道固定在弹簧磅秤(或圆盘测力计)上面,将一铁球放
在轨道底部,当球静止时,观察
秤指针所指的位置为$A$(做出
标记),使球从轨道的一侧滚
下,以某一速度经最低点时,可
观察到指针所指的位置将移到
$A'$. 结合分析铁球在竖直平面
内做圆周运动所需向心力的来
源,说明滑雪者经过凹形坡底时,雪地对滑雪者的支持力将大于滑雪者的重量。

\begin{figure}[htp]
    \centering
    \includegraphics[scale=.5]{fig/4-7.png}
    \caption{}
\end{figure}

如图4.8所示,将上面实验的圆弧形轨道换成中部凸起
的轨道,比较当球静止在坡顶时以及当它以某一速度经过坡
顶时,指针偏转角度的变化。结合分析铁球在竖直平面内做
圆周运动所需向心力的来源,说明汽车经过坡顶时,对路面的
压力小于汽车的重量。


\begin{figure}[htp]
    \centering
    \includegraphics[scale=.5]{fig/4-8.png}
    \caption{}
\end{figure}

\subsection{学生实验}
\subsubsection{研究平抛物体的运动}

做好这个实验的关键在于尽可能准确地描绘出平抛
运动轨迹,为此要注意以下几点。
\begin{enumerate}
\item 课本图10.12中的斜槽应用水平仪进行调整,使得斜
槽的下端平直部分保持水平。
\item 斜槽用夹具固定。小球滚下时不致碰到木板平面,但
也不宜离木板过远,并且要使木板平面和小球下落的竖直面
平行。在重复实验的过程中,要使木板跟斜槽的相对位置保
持不变。
\item 把小球放在槽口,在钉在木板的白纸上确定好相当于
小球球心的位置,这就是平抛运动轨迹的起点。这一点的确
定对于整个平抛运动轨迹的描绘具有关键的意义。
\item 利用有孔的纸卡片确定小球的运动轨迹时,要先使小
球从斜槽的不同位置处滚下,以选择一个适当的释放位置,使
得小球运动的轨迹大致经过白纸的右下角,而不要偏在左侧
或偏向上端,然后使小球在斜槽的这一位置处重复滚下几次,
目测小球的运动轨迹的形状。实验时可把有孔的纸卡片放在
目测的轨迹上,进行调整,以便比较顺利地描出轨迹。
\end{enumerate}

在读取和处理实验数据时,要求取三位有效数字,并
应给出当地的重力加速度的数值。在写实验报告时应附上小
球的原始轨迹描绘图,并用跟数据表中相同的符号编号标出
各点及其坐标。

在测出小球的初速度后,还可以让学生讨论以下
问题:

在所描出的平抛运动轨迹上,截取一段包括抛出点$O$在
内的轨迹,把这段轨迹在$x$轴上的投影分成四个等分,从分
点$x_1,x_2,x_3,x_4$分别作$x$轴的垂线与轨迹交于$a$、$b$、$c$、$d$四点(图4.9),经过这四点再分别
作$y$轴的垂线。
\begin{figure}[htp]
    \centering
    \includegraphics[scale=.5]{fig/4-9.png}
    \caption{}
\end{figure}

它们的垂足$y_1,y_2,y_3,y_4$在$y$轴上的截距之比应符合什
么规律?实际测量一下看看是否符合这一规律?如果存在误
差,试分析产生误差的主要原因是什么?

\subsubsection{验证向心力公式}
实验前先要让学生观察实验装置,要明确装置中的
重锤$A$是研究对象,重锤是在水平面内做圆周运动,圆心在转
轴上,圆半径就是横杆上从转轴到悬挂重锤的细绳之间的
距离。

这个实验所用的器材虽然并不复杂,但是操作时需
要一定的技巧。因此要先指导学生练习用手指搓动转轴,掌
握适当的快慢程度,使得重锤基本上能保持做匀速圆周运动,
并且每转一周重锤$A$大致都能从指示器$P$的正上方通过。然
后再开始测量时间计算角速度。

分别改变重锤的质量$m$、半径$r$和换用倔强系数不
同的弹簧重复做上述实验时,还可以让学生思考以下几个
问题:
\begin{enumerate}
\item 只改变重锤的质量$m$而使它做圆周运动的半径保持
不变,实验时搓动转轴转动的快慢程度是否要改变?
\item 如果只要求改变运动半径$r$, 应该调节什么距离?实
验时搓动转轴转动的快慢程度是否也需要改变?
\item 如果只换用原长相等而倔强系数不同的弹簧,而保持
重锤质量和运动半径不变,实验时搓动转轴转动的快慢程度
是否要改变?
\end{enumerate}

\subsection{课外实验活动}

\subsubsection{用尺测量玩具手枪子弹射出的速度}

这个实验所用的测量工具只有一把尺(米尺),要求注
意培养学生运用物理知识来解决实际问题的能力,可以先让
他们思考应怎样进行测量,而不要先急于去看课本上介绍的
原理。

在实际操作时应注意:
\begin{enumerate}
\item 用尺测量离地高度$h$时,应从枪口量到地面。
\item 发射子弹时枪管应保持水平,如果手头没有水准仪,
可以先把玩具手枪大致固定起来,用细线挂一串钥匙作为重
垂线挂在管口旁侧,再用三角板测量枪管和重垂线是否成
$90^{\circ}$角.
\item 调节好枪管水平后,应把玩具手枪固定好,然后把重
锤线移到枪口附近,把枪口(作为抛出点)在地面上的投影位
置做下标记。
\item 怎样来找到子弹的落地点?可先试射一下,看看子
弹大概的落地位置,然后在这里垫放一块深色的布(或深色的
纸),在子弹上沾上一些白粉,再发射。
\end{enumerate}

\subsubsection{估测自行车受到的阻力}
这是一个设计性的实验.首先要求学生认真阅读课
本364页这段文字,领会实验的目的要求,可由几个同学一
起讨论研究,确定测量平均阻力的原理以及所用的方法和步
骤,在确认原理是合理的,方法是可行的前提下,再进行实际
测量。

实验方法
如果把自行车从滑行到停止的过程看成是匀变速(匀减
速)直线运动,设法测出加速度$a$, 又知道自行车和人的质量
$m$, 则平均阻力$f=ma$就可以计算。根据$v_t=v_0+at$, 末速度
$v_t=0$,时间$t$可以用手表测量,如果知道初速度$v_0$, 就可以求
出加速度$a$.

怎样来确定自行车的初速度呢?车轮轮缘的线速度的大
小是和车辆行驶速度相等的,而要知道轮缘的线速度$v=\omega r$,
则必须知道角速度$\omega$和车轮半径$r$. 车轮半径$r$可以用米尺测
量,角速度$ω=2\pi/T=2\pi n/60$。
式中$n$是每分钟转数,可由每分
钟内蹬踏车板的次数算出。设用手表估测出每分钟蹬踏车板
的次数为$n_1$(它跟链轮的转数相同),且设后轮每分钟的转数
为$n_2$(它跟飞轮的转数相同).由于链条传动装置中轮缘的线
速度相等,链轮和飞轮的每分钟转数和它们的半径成反比。
因此只要知道链轮的半径$r_1$和飞轮的半径$r_2$, 就有
\[\frac{n_1}{n_2}=\frac{r_2}{r_1}\]

由于在链条传动中,链轮和飞轮缘上齿的大小间距相等,
因此轮缘上的齿数是跟半径成正比的,因此上式中半径的比
可以用齿数的比来代替,而且数齿数比量半径更为简便也比
较准确.设链轮和飞轮的齿数分别为$Z_1$和$Z_2$, 则
\[\frac{n_1}{n_2}=\frac{Z_2}{Z_1},\qquad n_2=\frac{Z_1}{Z_2}n_1\]

因此后轮轮缘的线速度为$v=\omega r=2\pi n_2r/60$, 
它也就是自行车的行驶速度。

可采用如下的步骤进行:
\begin{enumerate}
    \item 先匀速地骑行自行车,用手表测出链轮每分钟的转
数$n_1$.
\item 然后停止用力蹬踏脚板,同时看手表开始计时,尽量保持自行车在水平路面平直滑行,直到完全停止运动。测出
所经过的时间$t$.
\item 把自行车架好,数一下链轮的齿数$Z_1$和飞轮的齿数
$Z_2$, 再用尺量一下后轮的半径$r$(用米做单位),再用大磅秤称
一下自行车和你自己的质量,加在一起记作$m$(用千克做单
位)。根据前述原理和测得的就可算出自行车所受的平均
阻力。
\end{enumerate}

由于用力蹬踏脚板时所测量的链轮每分钟的转数$n$
以及自行车停止运动所用的时间$t$都不可能十分精确,测得
的平均阻力只是一个估测值,所以应引导学生把这个实验的
重点放在如何综合应用所学过的力学知识来设计好这个实验
上,上面所介绍的实验方法不是唯一的,可以让学生提出多种
设计方案,进行实验试测。考虑到有些学生可能不会骑自行
车,实际测量不一定要求每个学生都必须进行。

\subsubsection{验证向心力公式}
这个实验是课本图4.20演示实验的继续,目的在于
让学生自己动手粗略地验证向心力公式。

实验时可以先练习一下,使得基本上能掌握住使小石
块在水平面内做匀速圆周运动的技巧,然后再进行计时,为了
便于控制小石块的转动半径,可以事先在笔杆以下的尼龙线
上裹上三小块胶布,每一块胶布间相距2—3厘米,在使小石
块转动时,可以同时注意观察胶布的位置,如果使第一块胶布
接近笔杆的下端,这表示转动半径较小,增大转速使得第二块
胶布接近笔杆下端时,这表明转动半径已比原来的增加2—3
厘米,而当进一步增大转速,使第三块胶布接近笔杆的下端
时,表示转动半径最大,用这一方法也便于控制小石块以
一选定的半径运动。

\section{习题解答}


\subsection{练习一}
\begin{enumerate}
\item 汽车以恒定的速率2分钟绕广场行驶一周,汽车每
行驶半周速度方向改变多少度?汽车每行驶10秒钟速度改变
多少度?画出汽车在相隔10秒钟的两个位置处的速度矢量.

\begin{solution}
    如图4.10所示,汽车绕圆形广场每行驶半圈(例如
由$A$驶到$C$),速度方向的改变为$180^{\circ}$.

匀速行驶的汽车2分钟绕广场一周,速度的方向改变
$360^{\circ}$, 行驶10秒钟转过的角度为
\[\phi=\frac{10}{2\x 60}\x 360^{\circ}=30^{\circ}\]
如果汽车的初始位置为$A$, 10秒后的位置为$B$, 它在$A$、
$B$两点的速度矢量图如图4.11所示.
\begin{figure}[htp]\centering
    \begin{minipage}[t]{0.48\textwidth}
    \centering
\begin{tikzpicture}[>=latex, scale=.8]
    \draw[very thick](2,0)node[right]{$A$}arc (0:180:2);
\draw[dashed,very thick](-2,0)node[left]{$C$} arc (-180:0:2);
\draw[->,thick](0,0)--node[below]{$R$}(2,0)--(2,1.5)node[right]{$v_A$};
\draw[->,thick](-2,0)--(-2,-1.5)node[left]{$v_C$};
\end{tikzpicture}
    \caption{}
    \end{minipage}
    \begin{minipage}[t]{0.48\textwidth}
    \centering
\begin{tikzpicture}[>=latex, scale=.8]
    \draw[very thick, dashed](0,0)circle(2);
    \draw[->,thick](0,0)--node[below]{$R$}(2,0)node[right]{$A$}--(2,1.5)node[right]{$v_A$};
    \draw[->,thick](0,0)--(45:2)node[left]{$B$}--+(135:1.5)node[right]{$v_B$};
    \draw(.4,0) arc (0:45:.4)node[right]{$\phi$};
\end{tikzpicture}
    \caption{}
    \end{minipage}
    \end{figure}

\end{solution}
\item 举出两个实例,说明物体做曲线运动的条件.

\begin{solution}
    如图4.12所示,一个在水平面上运动着的小球,如果
    受到一个弧形挡板的阻挡,由于挡板对小球作用的弹力$F$
的方向和小球的速度不在同一条直线上,所以小球将沿着弧
形板做曲线运动。在水平桌面上做直线运动的钢珠,如果在
它的运动路线旁边放上一根条形磁铁,钢珠就会在跟它的运
动方向不在同一直线上的磁力作用下,做曲线运动。可见物
体做曲线运动的条件是:物体具有初速度,同时受到一个跟速
度方向成角度的合外力的作用。

\begin{figure}[htp]
    \centering
    \includegraphics[scale=.5]{fig/4-12.png}
    \caption{}
\end{figure}
\end{solution}
\item 某人骑着自行车以恒定的速率驶过一段弯路,自行
车进行的是匀速运动还是变速运动?为什么?

\begin{solution}
    是变速运动。因为速度是矢量,匀速运动是指物体
    速度的大小和方向都不变的运动。而这一辆自行车虽然以恒
    定的速率行驶,但在弯路上速度的方向不断改变,所以自行车
    在弯路上行驶时不是匀速运动,而是变速运动。
\end{solution}
\end{enumerate}



\subsection{练习二}
\begin{enumerate}
	\item 降落伞在下落一定时间以后的运动是匀速的,没风的时候某跳伞员着地的速度是5.6$\ms$,现庄有风,风使他以4.0$\ms$的速度沿水平方向向东移动,他将以多大的速度着地?这个速度的方向怎样?

    \begin{solution}
        设无风时跳伞员的着地速度为$v_1$, 风的作用使他获得向东移动的速度为$v_2$, 则跳伞员的着地速度$v$是$v_1$和$v_2$这两
        个速度的合速度,如图4.13所示。
\[v=\sqrt{v^2_1+v^2_2}=\sqrt{5.6^2+4.0^2}=6.9\ms\]

        设跳伞员着地时的合速度方向偏离竖直方向的角度为$\alpha$,则
\[\tan\alpha=\frac{v_2}{v_1}=\frac{4.0}{5.6}=0.7143\qquad \alpha=35^{\circ}32'\]
    \end{solution}

\begin{figure}[htp]\centering
    \begin{minipage}[t]{0.48\textwidth}
    \centering
\begin{tikzpicture}[>=latex, scale=1]
\draw[<->, very thick](0,-2.8)--node[left]{$v_1$} (0,0)--node[above]{$v_2$}(2,0);
\draw[dashed](0,-2.8)--(2,-2.8)--(2,0);
\draw[->, very thick] (0,0)--(2,-2.8)node[below]{$v$};
\draw(0,-.4) arc (-90:-54.5:.4)node[below]{$\alpha$};
    \end{tikzpicture}
    \caption{}
    \end{minipage}
    \begin{minipage}[t]{0.48\textwidth}
    \centering
    \begin{tikzpicture}[>=latex, scale=1]
\draw[<->, very thick](0,2*1.732)--node[left]{$v_y$} (0,0)--node[below]{$v_x$}(2,0);
\draw[dashed](0,2*1.732)--(60:4)--(2,0);
\draw[->, very thick] (0,0)--(60:4)node[above]{$v$};
\draw(.4,0) arc (0:60:.4)node[right]{$60^{\circ}$};
    \end{tikzpicture}
    \caption{}
    \end{minipage}
    \end{figure}

\item 炮筒与水平方向成60$^\circ$角,炮弹从炮口射出时的速度是800$\ms$.这个速度在竖直方向和水平方向的分速度各是多大?

\begin{solution}
    如图4.14所示,炮弹速度的竖直分速度
\[v_y=v\sin 60^{\circ}=800\x \frac{\sqrt{3}}{2}=693\ms\]
炮弹速度的水平分速度
\[v_x=v\cos60^{\circ}=800\x\frac{1}{2}=400\ms\]
\end{solution}
\item 小汽艇在静水中的速度是12$\kmh$,河水的流建是6.0$\kmh$.如果驾驶员向着垂直于河岸的方向驾驶,小汽艇在河水中实际行驶的速度是多大?方向怎样?

\begin{figure}[htp]
    \centering
\includegraphics[scale=.6]{fig/4-15.png}
    \caption{}
\end{figure}
\begin{solution}
    如图4.15所示,设
    小汽艇在静水中的速度$v_1=12\kmh$,由于河水的流动
    使小汽艇获得沿河流方向的速
    度$v_2=6.0\kmh$,小汽艇
    在河水中实际行驶的速度$v$是
    $v_1$和$v_2$这两个速度的合速度。
\[v=\sqrt{v_1^2+v_2^2}=\sqrt{12^2+6^2}=13.4\kmh\]
设合速度$v$和河水流动方向所成角度为$\alpha$, 则
\[\tan\alpha=\frac{v_1}{v_2}=\frac{12}{6.0}=2,\qquad \alpha=63^{\circ}26'\]
\end{solution}
\end{enumerate}

\subsection{练习三}


下面各题都不考虑空气阻力.
\begin{enumerate}
\item 从一定高度水平抛出去的物体,它在空中飞行的时间是由什么决定的?抛射的水平距离又是由什么决定的?


\begin{solution}
    由于平抛物体的运动是由水平方向的匀速直线运动
    和竖直方向的自由落体运动的合运动,则所以它在空中飞行
    的时间是由下落的距离所决定,抛射的水平距离由水平初速
    度和下落的距离决定。
\end{solution}
\item 从同一高度以不同的速度水平抛出两个质量不同的石子,下面的说法哪个对?
\begin{enumerate}
	\item 速度大的先着地;
	\item  质量大的先着地;
	\item 两个物体同时着地.
\end{enumerate}
实际做一做,看你的判断是否正确.

\begin{solution}
    由于平抛物体的飞行时间只决定于下降的高度,而
    与物体抛出的初速度和质量的大小无关。因此只有(c)说
    法是对的,即两个物体同时着地。
\end{solution}
\item 从1.6米高的地方水平射出一颗子弹,初速度是700$\ms$,求这颗子弹飞行的水平距离.

\begin{solution}
    由于下降的高度$y=\dfrac{1}{2}gt^2$. 所以飞行时间$t=\sqrt{\dfrac{2y}{g}}$。
    子弹飞行的水平距离
\[x=v_0t=v_0\sqrt{\dfrac{2y}{g}}=700\x \sqrt{\frac{2\x 1.6}{9.8}}=400{\rm m}\]
\end{solution}
\item 一个小球从1.0米高的桌面上水平抛出,落到地面的位置离开桌子的边缘2.4米,小球离开桌子边缘时的初速度多大?

\begin{solution}
小球做平抛运动的时间$t$可由公式$y=\dfrac{1}{2}gt^2$求出。
\[t=\sqrt{\frac{2y}{g}}\]
已知小球飞行的水平距离$x=2.4$米.所以
小球的平抛初速度
\[v_0=\frac{x}{t}=x\cdot \sqrt{\frac{g}{2y}}=2.4\x \sqrt{\frac{9.8}{2\x 1.0}}=5.3\ms\]
\end{solution}
\item 从15米高的楼上以1.0$\ms$的速度水平扔出一物体,此物体落地时的速度多大?方向是否与地面垂直?

\begin{solution}
    物体做平抛运动落地时的速度$v$
    是平抛初速$v_0$和在竖直方向由于重力
    作用下落15米高度时所具有的竖直分
    速度$v_y$的合速度(图4.16)。
\begin{figure}[htp]
    \centering
\begin{tikzpicture}[>=latex, scale=.7]
\draw[<->, very thick](1.5,0)node[above]{$v_0$}--(0,0)--node[left]{$v_y$}(0,-4);
\draw[->, very thick](0,0)--(1.5,-4)node[right]{$v$};
\draw[dashed] (0,-4)--(1.5,-4)--(1.5,0);
\end{tikzpicture}
    \caption{}
\end{figure}    
    
    由于$v^2_y=2gy$,所以物体落地时
    的合速度
\[v=\sqrt{v_0^2+v_y^2}=\sqrt{v^2_0+2gy}=\sqrt{1.0^2+2\x 9.8\x 15}=17.2\ms\]    

    方向不跟地面垂直,因为它具有水平分量。
\end{solution}
\end{enumerate}



\subsection{练习四}
    下面各题都不考虑空气阻力.
\begin{enumerate}
 \item 在斜抛运动中,射高$Y$和飞行时间$T$是由哪个分运
动决定的?

\begin{solution}
斜抛运动可以看成是水平方向的匀速直线运动和竖
直方向的上抛运动的合运动。它的飞行时间$T=2v_y/g$
,射高$Y=\dfrac{v^2_y}{2g}$。
所以射高和飞行时间都是由斜抛运动的竖直上抛分
运动所决定。
\end{solution}
 \item 在地面上以100$\ms$的初速度与水平面成60$^\circ$角
向斜上方扔出一石子.求石子在水平和竖直两个方向上的分
速度、石子能够到达的高度、到达这一高度所用的时间和石子
落地处到抛出处的距离.

\begin{solution}
    石子做斜抛运动。
    水平分速度
\[v_x=v_0\cos\theta=100\x\frac{1}{2}=50\ms\]
    竖直分速度
    \[v_y=v_0\sin\theta=100\x0.866=87\ms\]
    石子能到达的高度
    \[Y=\frac{v^2_0\sin^2\theta}{2g}=\frac{100^2\x \left(\frac{\sqrt{3}}{2}\right)^2}{2\x 9.8}=383{\rm m}\]
    石子到达最大高度所用时间
\[t=\frac{v_y}{g}=\frac{v_0\sin\theta}{g}=\frac{100\x \frac{\sqrt{3}}{2}}{9.8}=8.84{\rm s}\]
石子的飞行距离
\[X=v_0\cos\theta\cdot 2t=100\x\frac{1}{2}\x2\x8.84=884{\rm m}\]
\end{solution}
  \item 炮弹从炮筒中射出时的速度是1000$\ms$.比较炮
筒的仰角是30$^\circ$,45$^\circ$,60$^\circ$时,炮弹的射高和射程有何不同.

\begin{solution}
    炮弹作斜抛运动。

    当仰角是$\theta_1=30^{\circ}$时,
\[\begin{split}
     \text{射高}\; Y_1&=\frac{v^2_0\sin^2\theta_1}{2g}=\frac{(1000)^2\x \left(\frac{1}{2}\right)^2}{2\x 9.8}=1.28\x 10^4{\rm m}\\
     \text{射程}\; X_1&=\frac{v^2_0\sin 2\theta_1}{g}=\frac{(1000)^2\x \frac{\sqrt{3}}{2}}{ 9.8}=8.84\x 10^4{\rm m}
\end{split}
   \]
    
    当仰角是$\theta_2=45^{\circ}$时,
\[\begin{split}
     \text{射高}\; Y_2&=\frac{v^2_0\sin^2\theta_2}{2g}=\frac{(1000)^2\x \left(\frac{\sqrt{2}}{2}\right)^2}{2\x 9.8}=2.55\x 10^4{\rm m}\\
     \text{射程}\; X_2&=\frac{v^2_0\sin 2\theta_2}{g}=\frac{(1000)^2\x 1}{ 9.8}=1.02\x 10^5{\rm m}
\end{split}
   \]
    
   当仰角是$\theta_3=60^{\circ}$时,
   \[\begin{split}
        \text{射高}\; Y_3&=\frac{v^2_0\sin^2\theta_3}{2g}=\frac{(1000)^2\x \left(\frac{\sqrt{3}}{2}\right)^2}{2\x 9.8}=3.83\x 10^4{\rm m}\\
        \text{射程}\; X_3&=\frac{v^2_0\sin 2\theta_3}{g}=\frac{(1000)^2\x \frac{\sqrt{3}}{2}}{ 9.8}=8.84\x 10^4{\rm m}
   \end{split}
      \]

      由以上计算可知,射高随仰角的增大而增大,射程在仰角
为$45^{\circ}$时有最大值,而当仰角为$30^{\circ}$和$60^{\circ}$时,射程是相等的。
\end{solution}
  \item 一个人向着与水平面成45$^\circ$角的前上方抛出一颗手
榴弹.测出手榴弹的射程是65米,手榴弹抛出时的速度是多
大?射高是多高?

\begin{solution}
    手榴弹的运动是斜抛运动,忽略人的身长,则
\[X=\frac{v^2_0\sin2\theta}{g}\]
\[\begin{split}
    \text{初速}\;v_0&=\sqrt{\frac{Xg}{\sin2\theta}}=\sqrt{\frac{65\x 9.8}{\sin 90^{\circ}}}=25.2\ms\\
    \text{射高}\; Y&=\frac{v^2_0\sin^2\theta}{2g}=\frac{65\x 9.8\x \frac{1}{2}}{2\x 9.8}=16.3\ms
\end{split}\]
\end{solution}
\end{enumerate}


\subsection{练习五}
\begin{enumerate}
\item 对于做匀速圆周运动的物体,下面的哪种说法对,哪
种说法不对?
\begin{enumerate}
\item 速度不变;
\item 速率不变;
\item 角速度不变.
\end{enumerate}

\begin{solution}
    (a)不对。因为速度是矢量。线速度的方向时刻在改
    变,始终沿着圆周的切线方向。
    (b)、(c)对的。
\end{solution}
\item 钟表上分针的周期和角速度是多大?

\begin{solution}
    分针转一周的时间$T=1{\rm h}=3600{\rm s}$。
\[\text{角速度}\; \omega=\frac{2\pi}{T}=\frac{2\x 3.14}{3600}=1.74\x 10^{-3}{\rm rad/s}\]
\end{solution}
\item 半径10厘米的砂轮,每0.2秒转一圈,砂轮旋转的
角速度是多大?砂轮上离转抽不同距离的点,其角速度是否相
等?线速度是否相等?试求离转轴最远处的线速度.

\begin{solution}
    由题意可知,砂轮转动周期$T=0.2$秒,则砂轮旋转的
    角速度
    \[\omega=\frac{2\pi}{T}=\frac{2\x 3.14}{0.2}=31.4{\rm rad/s}\]
    砂轮上离轴不同距离的点的角速度是相等的,线速度则不相
    等,离转轴最远处,即砂轮边缘的点的线速度为最大,
\[v=\frac{2\pi r}{T}=\frac{2\x 3.14\x 0.10}{0.2}=3.14\ms \]
\end{solution}
\item 在皮带传动(图4.17)中,两皮带轮轮缘上的线速度
是相等的.如果大轮的半径是$r_1$,小轮的半径是$r_2$,求大
轮和小轮的角速度之比.如
果大轮每分钟的转数为$n_1$,
小轮每分钟的转数$n_2$是多少?


\begin{figure}[htp]
    \centering
    
    \begin{tikzpicture}[>=stealth, scale=.7]
    \draw [very thick](0,0)  circle [radius=2.5];
    \draw [very thick](8,0)  circle [radius=1.5];
    
    \draw [->, thick](210:2) arc (210:150:2);
    \draw [->, thick](9,.5) arc (30:-30:1);
    \draw (0,0)--node [above]{$r_1$}(240:2.5);
    \draw (8,0)--node [above]{$r_2$}(9.5,0);
    
    \draw [fill=black] (240:2.5) circle (2pt)node [below]{$A$};
    \draw [fill=black] (240:1.25) circle (2pt) node [right]{$C$};
    \draw [fill=black] (0,0) circle (2pt) ;
    \draw [fill=black] (8,0) circle (2pt) ;
    \draw [fill=black] (9.5,0) circle (2pt) node [right]{$B$} ;
    
    \draw [very thick] (0,2.5)--(8.5,1.45);
    \draw [very thick] (0,-2.5)--(8.5,-1.45);
    
    \end{tikzpicture}
    \caption{}
    \end{figure}

\begin{solution}    
    由于大、小两轮轮缘上的点的线速度相等,$v_1=v_2$,
    即:
    \[\omega_1 r_1=\omega_2 r_2\]
    大轮和小轮的角速度之比:
\[\frac{\omega_1}{\omega_2}=\frac{r_2}{r_1}\]

$\because\quad \omega=2\pi n$

$\therefore\quad \dfrac{n_1}{n_2}=\dfrac{\omega_1}{\omega_2}=\dfrac{r_2}{r_1}$

小轮每分钟转数$n_2=\dfrac{r_1}{r_2}n_1$.
\end{solution}
\end{enumerate}


\subsection{练习六}
\begin{enumerate}
	\item 在图4.17所示的皮带传动装置中,两轮边缘上的$A$点和$B$点的向心加速度哪个大?为什么?大轮上$A$点和$C$点的向心加速度哪个大?为什么?

    \begin{solution}
由于两轮边缘上的$A$点和$B$的线速度$v$相等,可根据$a_n=v^2/r$
式来进行比较。因为大轮半径$r_A$大于小轮半径
$r_B$, 所以向心加速度$a_B>a_A$·

由于$A$点和$C$点都是同一轮子上的点,它们的角速度$\omega$
相等,根据$a_n=\omega^2r$一式可以判断,因为半径$r_A>r_C$, 所以向
心加速度$a_A>a_C$.    
    \end{solution}
	\item 从$a_n=v^2/r$看,$a_n$跟$r$成反比,从$a_n=\omega^2r$看,$a_n$跟$r$成正比.如果有人问你:“向心加速度的大小跟半径是成正比还是成反比?”应该怎样回答?

    \begin{solution}
    任何物理量间的定量关系总是有条件的。在线速度
$v$相同的条件下,$a_n$跟$r$成反比;而在角速度$\omega$相同的条件
下,$a_n$跟$r$成正比。
    \end{solution}
	\item 由于地球的自转,地球上的物体都有向心加速度,试回答:
\begin{enumerate}
	\item “在地球表面各处的向心加速度的方向都是指向地心的”,这种说法正确吗?为什么?
	\item 在赤道和极地附近的向心加速度哪个大?为什么?
	\item 在北京的物体由于地球自转而产生的向心加速度是多大(北京的纬度取40$^\circ$,地球的半径取$6.4\times 10^8$千米)?	
\end{enumerate}

\begin{figure}[htp]
    \centering
    \includegraphics[scale=.6]{fig/4-18.png}
    \caption{}
    \end{figure}

\begin{solution}
\begin{enumerate}
    \item 不正确.如图4.18所示,在地球表面纬度为$\phi$的$P$
    处的物体,它的向心加速度的方
    向是指向做圆周运动的圆心$O$,
    而不是指向地心的。
    \item 由于在赤道上的物体和极地附近的物体随地球自转
    做圆周运动的角速度都相等,而做圆周运动的半径则是赤道大于极地附近,根据$a_n=\omega^2 r$, 所以在赤道上的物体的向心加
    速度大。
    \item 由图4.18可知,$\phi=40^{\circ}$, 圆周半径$r=R\cos\phi$, 所以
\[a_n=\omega^2r=\left(\frac{2\pi}{T}\right)^2R\cos\phi=\left(\frac{6.28}{86400}\right)^2\x6.4\x10^3\x10^3\x
    \cos40^{\circ}=2.6\x10^{-2}\msq\]
\end{enumerate}
\end{solution}
\item	 飞机由俯冲转为拉起的一段轨迹可以看作一段圆弧(图4.19).如果这段圆弧的半径$r$是800米,飞机在圆弧最低点$P$的速率为720$\kmh$.求飞机在$P$点的向心加速度是重力加速度的几倍.($g$取10$\msq$)
\begin{figure}[htp]
\centering
\includegraphics[scale=.6]{fig/4-19.png}
\caption{}
\end{figure}


\begin{solution}
    飞机在圆弧最低点$P$的速率$v=720\kmh=200\ms$.则向心加速度
   \[a_n=\frac{v^2}{r}=\frac{200^2}{800}=50\msq\]
   这时向心加速度是重力加速度的$\dfrac{50}{10}=5$倍.
\end{solution}
	\item 一个物体做匀速圆周运动,如果圆周的半径是$r$,运动的周期是$T$,试证明向心加速度$a=4\pi^2r/T$.

    \begin{solution}
        向心加速度$a=\omega^2 r$, 而角速度$\omega=2\pi/T$,
        由二式中消
        去$\omega$, 即得
        \[a=\frac{4\pi^2r}{T^2}\]
    \end{solution}
\end{enumerate}



\subsection{练习七}
\begin{enumerate}
	\item 下面的受力分析对吗?如果不对,说明错在哪里.
	\begin{enumerate}
		\item 课本图4.21中做匀速围周运动的物体受四个力的作用,这四个力是重力、支持力、绳的拉力和向心力;
		\item 课本图4.22中水平盘旋的飞机受到三个力的作用,这三个力是向心力、重力、升力.
	\end{enumerate}

    \begin{solution}
 \begin{enumerate}
     \item 不对.课本图4.21中做匀速圆周运动的物体受
     三个力的作用,这三个力是重力、支持力和绳的拉力。其中重
     力和支持力大小相等,方向相反,互相抵消,使物体做匀速圆周运动所需的向心力是绳子的拉力。
     \item 不对.课本图4.22中水平盘旋的飞机只受到重力和
     升力这两个力的作用,这两个力不在同一直线上,它们在水平
     方向的合力是使飞机水平盘旋做圆周运动所需的向心力。
 \end{enumerate}
     上面两则受力分析的共同错误是在物体所受的合外力之
     外又凭空增加了一个“向心力”。
    \end{solution}
\item 要使一个3.5千克的物体在半径是2.0米的圆周上以4.0$\ms$的速率运动,需要多大的向心力?

\begin{solution}
    物体需要的向心力
\[F=m\frac{v^2}{r}=3.5\x \frac{4.0^2}{2.0}=28{\rm N}\]
\end{solution}
\item 太阳的质量是$1.98\times 10^{30}$千克,它离开银河系中心大约3万光年(1光年$=9.46\times 10^{12}$千米),它以250千米/秒的速率绕着银河系中心转动,计算太阳绕银河系中心转动的向心力.

\begin{solution}
    太阳绕银河系中心的运动可以看成是匀速圆周运
    动,已知圆半径$r=3\x10^4\x9.46\x10^{12}\x10^3$m,$v=250\x10^3\ms$。所以向心力
\[F=m\frac{v^2}{r}=1.98\x 10^{30}\x \frac{250^2\x 10^6}{3\x 10^4\x 9.46\x 10^{15}}=4.36\x 10^{20}{\rm N}\]
\end{solution}
\item 甲乙两球都做匀速圆周运动,甲球的质量是乙球的3倍,甲球在半径是25厘米的圆周上运动,乙球在半径是16厘米的圆周上运动,在一分钟内,甲球转了30次,乙球转了75次,试比较两球所受的向心力.

\begin{solution}
    甲球的运动周期$T_1=\dfrac{60}{30}=2{\rm s}$.乙球的运动周期
    $T_2=\dfrac{60}{75}=\dfrac{4}{5}{\rm s}$。
设乙球质量为$m_2$, 则甲球质量$m_1=3m_2$。

    根据$F=m\omega^2 r=\dfrac{m4\pi^2 r}{T^2}$,
\[\frac{F_1}{F_2}=\frac{\dfrac{3m_2\x 4\pi^2\x 0.25}{2^2}}{\dfrac{m_2\x 4\pi^2\x 0.16}{\left(\frac{4}{5}\right)^2}}=\frac{3}{4}\]
    即甲球所受向心力是乙球的$3/4$
    倍。
\end{solution}
\item 线的一端拴一重物,手握线的另一端使重物在水平面内做匀速圆周运动,当每分钟转数相同时,线长易断还是线短易断?为什么?线速度相同时又怎样?

\begin{solution}
    每分钟转数相同,角速度也相等,由$F=m\omega^2r$可知-
    线长时$r$大,所需的向心力也大,所以线长容易断。而如果线
    速度相同,由$F=mv^2/r$
    可知,线短时$r$小,所需的向心力也大,
    所以线短容易断。
\end{solution}
\end{enumerate}




\subsection{练习八}
\begin{enumerate}
	\item 在课本图4.24的圆锥摆中,如果线和垂直方向成30$^\circ$角,小球在水平面内做每分钟60转的匀速圆周运动,线的长度是0.28米,计算重力加速度的值.

    \begin{solution}
        由课本图4.24可知,使圆锥摆做匀速圆周运动的向心力$F=mg\tan\theta$, 把它代入向心力公式$F=m\omega^2r$, 由于
        $r=\ell\sin\theta$, 所以$\theta$和$\omega$存在以下关系:
\[g\tan\theta=\omega^2\ell\sin\theta\]
于是,重力加速度
\[g=\omega^2\ell\cos\theta=\frac{4\pi^2}{T^2}\ell\cos\theta=\frac{4\pi^2\x 0.28\x \frac{\sqrt{3}}{2}}{1^2}=9.56\msq\]
    \end{solution}
\item 有人说:“图4.24中的圆锥摆少画了一个作用在小球上的力,这个力与$F$大小相等、方向相反,是$F$的平衡力,必须有这个力,小球才能处于平衡状态而不落向圆心.”这种说法错在哪里?

\begin{solution}
    在圆锥摆中做匀速圆周运动的小球,必须受到向心
    力的作用。正是在这个向心力的作用下产生的向心加速度,
    使速度的方向时刻发生变化,才能使小球沿着圆周运动,并不
    会使小球落向圆心,如果小球受的力是平衡的,它就不可能
    做匀速圆周运动了,物体受到的任何一个力,都不能没有施力
    物体。在圆锥摆中,小球只受到两个力,一个是地球施给它的
    重力,另一个是悬绳施给它的拉力。这两个力的合力,就是小
    球所受的向心力$F$. 此外,找不到任何其他物体能对小球施加
    一个跟$F$大小相等、方向相反的力。
\end{solution}
\item 铁路转弯处圆弧的半径是300米,轨距是1435毫米,规定火车通过这里的速度是72$\kmh$,计算内、外铁轨的高度差.
\begin{figure}[htp]
    \centering
    \includegraphics[scale=.6]{fig/4-20.png}
    \caption{}
    \end{figure}

\begin{solution}
    如图4.20所示,火车在转弯时所需的向心力由火车的重力和轨道支持力的合力F
所提供。
\[F=mg\tan\alpha=\frac{mv^2}{r},\qquad \tan\alpha=\frac{v^2}{gr}\]

由于轨道平面和水平面间的夹角$\alpha$一般较小,可以近似地认为$\tan\alpha\approx \sin\alpha=\dfrac{h}{d}$。(式中$b$为内、外轨道的高度差,$d$为
轨距)代入上式,得
\[\frac{h}{d}=\frac{v^2}{gr}\]
内、外轨道的高度差
\[h=\frac{dv^2}{gr}=\frac{1.435\x\left(\dfrac{72\x 10^3}{3600}\right)^2}{9.8\x 300}=0.195{\rm m}\]
\end{solution}
\item 一架滑翔机用180$\kmh$的速率,沿着半径为1200米的水平圆弧飞行,计算机翼和水平线的夹角(参阅课本图4.22).

\begin{solution}
    滑翔机的速率$v=180\kmh=50\ms$.滑翔机
    在水平面上做圆周运动所需的向心力是由它的重力和机翼所
    产生的升力的合力所提供.设机翼和水平面间的夹角为$\theta$, 则
    \[mg\tan\theta=\frac{mv^2}{r}\]
    \[\tan\theta=\frac{v^2}{gr}=\frac{50^2}{9.8\x 1200}=0.2126,\qquad \theta=12^{\circ}\]
\end{solution}
\end{enumerate}





\subsection{习题}

\begin{enumerate}
	\item 汽艇在静水中的速度是10$\kmh$,渡河时向着垂直于河岸的方向匀速行驶.现在河水的流速是3$\kmh$,河宽500米,汽艇驶到对岸需要多长时间?汽艇在河水中实际行驶的距离是多大?
    \begin{figure}[htp]
        \centering
        \includegraphics[scale=.6]{fig/4-21.png}
        \caption{}
        \end{figure}

    \begin{solution}
如图4.21所示,汽艇渡河时的运动是汽艇在静水中
航行的运动和由于河水冲击使汽艇沿河水流动方向的运动的
合运动,汽艇驶到对岸需要的时间
\[t=\frac{s_1}{v_1}=\frac{0.5}{10}=0.05{\rm h}=3{\rm min}\]

在这段时间里汽艇由于河水的冲击,偏向下游的距离
\[s_2=v_2t=3\x0.05=0.15{\rm km }=150{\rm m}\]
汽艇在河水中实际行驶的距离
\[s=\sqrt{s_1^2+s_2^2}=\sqrt{500^2+150^2}=522{\rm m}\]    
    \end{solution}
\item 在490米的高空,以240$\ms$的速度水平飞行的轰炸机,追击一鱼雷艇,该艇正以25$\ms$的速度与飞机同方向行驶.试问,飞机应在鱼雷艇后面多远处投下炸弹,才能击中该艇?

\begin{solution}
    飞机上投下的炸弹可看成作平抛运动。因此炸弹飞
    行时间可根据公式$h=\frac{1}{2}gt^2$来计算。
\[t=\sqrt{\frac{2h}{g}}=\sqrt{\frac{2\x 490}{9.8}}=10{\rm s}\]

由于飞机与鱼雷艇同方向行驶,所以飞机相对于鱼雷艇
的速度是$v_1-v_2=240-25=215\ms$,设飞机
在鱼雷艇后面$x$处投弹方能击中鱼雷艇,于是
\[x=(v_1-v_2)t=215\x10=2.15\x10^3{\rm m}\]
\end{solution}
\item 两人传球,如果球从一个人手里到另一个人手里经过的时间是2秒,球到达的最高点离手有多高?(设两人的手等高)

\begin{solution}
设两人间传球的运动是斜抛运动。由题意可知球做
斜抛运动的飞行时间为2秒,则球从抛出到达最高点的时间
为1秒,从最高点到达另一个人手中的时间也是1秒.由于球
到达最高点以后的运动可看成是平抛运动,于是这一最高点
离手的高度可按公式$h=\frac{1}{2}gt^2$计算。
\[h=\frac{1}{2}gt^2=\frac{1}{2}\x 9.8\x 1^2=4.9{\rm m}\]
\end{solution}
\item 从仰角是30$^\circ$的枪筒中射出的子弹,初速度是600$\ms$.求子弹在轨迹最高点和落地点的速度各是多大.

\begin{solution}
    子弹飞行到它的轨迹最高点时,由于竖直方向的速
    度已减小到零,因此这时的速度就等于斜抛初速度的水平分
    速度
    \[v_{\text{最高点}}=v_0\cos\theta=600\x \frac{\sqrt{3}}{2}=520\ms\]
    子弹到达落地点时的速度的大小等于初速度的大小,即
    \[v=v_0=600\ms\]
\end{solution}
\item 伽利略曾说过:“仰角(即抛射角)比45$^\circ$增大或减小一个相等角度的抛体,其射程是相等的.”你能证明这个说法
的正确性吗?

\begin{solution}
斜抛运动的射程
\[X=v_x T=v_0\cos\theta \x\frac{2v_0\sin\theta}{g}=\frac{2v_0^2\sin\theta\cos\theta}{g}
\]
根据伽利略的说法,设两个斜抛运动的仰角分
别为$\alpha=45^{\circ}+\theta$和$\beta=45^{\circ}-\theta$, 于是射程
\[X_1=\frac{2v_0^2\sin\alpha\cos\alpha}{g},\qquad X_2=\frac{2v_0^2\sin\beta\cos\beta}{g}\]

要证明$X_1$等于$X_2$, 只要证明$\sin\alpha\cos\alpha$ 等于 $\sin\beta \cos\beta$. 由命
题可知$\alpha +\beta =90^{\circ}$, 于是:
\[\sin\alpha\cos\alpha=\sin (90^{\circ}-\beta )\cos (90^{\circ}-\beta ) =\cos\beta \sin\beta \]
所以$X_1=X_2$. 伽利略的说法是正确的。
\end{solution}
\item 一个人站在地面上用枪踏准树上的猴子(图4.22),当子弹从枪口射出时,猴子闻声立即从树上竖直下落(初速度为零).讨论一下,猴子能否避开子弹的射击.
\begin{figure}[htp]
\centering\includegraphics[scale=.6]{fig/4-22.png}
\caption{}
\end{figure}
提示:斜抛运动可以看作是物体沿初速度方向所做的匀速直线运动和在重力作用下的自由落体运动的合运动.

\begin{figure}[htp]
    \centering\includegraphics[scale=.6]{fig/4-23.png}
    \caption{}
    \end{figure}


\begin{solution}
如图4.23所示,猴子
位于树上的$A$点,枪口$O$对准猴子。子弹射出后,如果不考
虑重力的影响,它将沿着$OA$方
向做匀速直线运动,命中$A$点。但是在重力作用下,子弹在沿
$OA$方向作匀速直线运动的同时,还要在竖直方向上作自由落
体运动。所以它在本应到达$A$点的这段时间里,只能到达$A$
点正下方的$B$点。$AB=s$就是这段时间里自由落体下降的
距离。由于猴子是在开枪的同时下落的,在同一时间里,它下
降的距离也是$s$, 恰好到达$B$点,被子弹所击中。
\end{solution}
\item  飞机从俯冲到拉起的一段轨迹是一段圆弧(参看图4.19),如果飞机在这段弧上的速率是540$\kmh$,要使它在最低点时的向心加速度不超过$5g$,圆弧的半径至少是多少米?($g$取$10\msq$).

\begin{solution}
    根据向心加速度$a_n=v^2/r$, 飞机速率$v=540\kmh=150\ms$,则圆弧半径
    \[r=\frac{v^2}{a_n}=\frac{150^2}{5g}=\frac{150^2}{5\x 10}=450{\rm m}\]
\end{solution}
\item  一个35千克的重物,系在2.0米长的悬绳下端,不断摆动.重物通过最低点时的速率是3.0$\ms$,求这时绳对物体的拉力.

\begin{solution}
    重物的摆动可以看作是在竖直平面里的圆周运动的一部分。当重物经过圆周最低
    点时,受到的重力$mg$和绳子
    拉力$T$的合力应等于重物作
    圆周运动所需的向心力(图
    4.24)。
\[T-mg=ma\]
于是,绳子拉力
\[T=mg+ma=m\left(g+\frac{v^2}{r}\right)=35\left(9.8+\frac{3.0^2}{2.0}\right)=5.0\x 10^2{\rm N}\]
\end{solution}

\begin{figure}[htp]\centering
    \begin{minipage}[t]{0.48\textwidth}
    \centering
\begin{tikzpicture}[>=latex]
\fill [pattern=north east lines](-.5,3) rectangle (.5,3.2);
\draw(-.5,3)--(.5,3); \draw(0,3)--(0,0);
\draw(0,0) arc (-90:-60:3); \draw(0,0) arc (-90:-120:3);
\draw[fill=white](0,0) circle (.2);
\draw[<->, thick](0,1)node[right]{$T$}--(0,-1)node[right]{$mg$};
\draw[->](-.2,.3)--(-.7,.3)node[above]{$v$};
\tkzDefPoints{0/0/O} \tkzDrawPoint(O)
\end{tikzpicture}
    \caption{}
    \end{minipage}
    \begin{minipage}[t]{0.48\textwidth}
    \centering
    \includegraphics[scale=.7]{fig/4-25.png}
    \caption{}
    \end{minipage}
    \end{figure}

\item  略(课本已作解答).

\item  一个滑雪者连同他的滑雪板质量共70千克,他滑到凹形的坡底时的速度是20$\ms$,坡底的圆弧半径是50米,计算在坡底时雪地对滑雪板的支持力.

\begin{solution}
    滑雪者滑到凹形坡底时所需的向心力是由坡底对滑
    雪板的支持力N和滑雪者连同滑雪板的重力$mg$的合力所
    提供,
\[    N-mg=ma\]
    所以,坡底雪地的支持力
\[N=mg+ma=m\left(g+\frac{v^2}{r}\right)=70\left(9.8+\frac{20^2}{50}\right)=1.2\x 10^3{\rm N}\]
\end{solution}
\item  一辆600千克的汽车以10$\ms$的速度通过圆弧半径是30米的山坡顶点时,汽车受到哪几个力的作用?汽车对路面的压力是多大?


\begin{solution}
    汽车通过山坡顶点时受
    到重力$mg$和山坡顶点的支
    持力$N$的作用,这两个力的合
    力等于汽车在竖直平面里做圆
    周运动所需的向心力(图4.25)。
\[mg-N=\frac{mv^2}{r}\]
\[N=m\left(g-\frac{v^2}{r}\right)=600\left(9.8-\frac{10^2}{30}\right)=3.88\x 10^3{\rm N}\]
汽车对路面的压力也等于$3.88\x10^3$牛.
\end{solution}

\end{enumerate}


\section{参考资料}
\subsection{对课本170页习题第6题提示的讨论}
这是研究斜抛运动时常用的一种分析方法。可以这样来
设想,如果不存在重力和其他任何力的作用,根据牛顿第一定
律可知,物体将以斜抛时的初速度沿着抛出方向做匀速直线
运动,又根据力的独立作用原理可知,对物体作用的任何一
个力都要使物体产生加速度,若不计空气阻力的影响,则斜抛
物体只受到重力的作用,重力使物体产生的加速度等于$g$, 从
而使物体在竖直方向上发生速度的改变,在时间$\Delta t$内,物体
的速度变化$\Delta v=g\Delta t$.

投某一斜抛物体的初速度$v_0=40\ms$,抛射角$\theta=30^{\circ}$,
$g$取$10\msq$, 则初速度的竖直分速度
$$v_y=v_0\sin\theta=40\x\frac{1}{2}=20\ms$$
物体上升到最大高度的时间
\[t=\frac{v_0\sin\theta}{g}=\frac{20}{10}=2{\rm s}\]
飞行时间$T=2t=2\x2=4{\rm s}$,即物体抛出后运动4秒就
落到地面。可以这样来想象,如果没有重力的作用,在4秒末
物体将运动到哪里呢?从图4.26可以看出,物体如果以初速
$v_0$做匀速直线运动,则4秒内的位移$OD=v_0t=40\x4=
160{\rm m}$。从图中直角三角形$ODH$来看,对边$DH=OD\sin\theta=
160\x\frac{1}{2}=80{\rm m}$,即物体应在离地面80米高处的$D$点,然
而事实上物体却在重力作用下落到了地面,因为根据自由落
体运动可知
\[y=\frac{1}{2}gt^2=\frac{1}{2}\x 10\x 4^2=80{\rm m}\]
这表明了在
这4秒时间内,由于重力作用使物体下降的距离恰为80米,
所以物体恰好落回到地面。

\begin{figure}[htp]\centering
    \begin{minipage}[t]{0.58\textwidth}
    \centering
\begin{tikzpicture}[>=latex, scale=.8]
    \draw[<->](0,7)node[right]{$y$}--(0,0)--(9.6,0)node[right]{$x$};
\draw[thick](0,0)--(8,6)node[above]{$D$};
\draw[domain=0:9, samples=100,very thick,->]plot(\x, {-5/53.33*\x*\x+.75*\x})node[right]{$v_H$};
\draw[dashed](2,1.125)node[below]{$E$}--(2,1.5)node[above]{$A$};
\draw[dashed](4,1.5)node[below]{$F$}--(4,3)node[above]{$B$};
\draw[dashed](6,1.125)node[below]{$G$}--(6,4.5)node[above]{$C$};
\draw[dashed](8,0)node[below]{$H$}--(8,6)node[above]{$D$};

\draw[thick,<->] (1.5,0)node[below]{$v_x$}--node[above]{$\theta$}  (0,0)--(0,1.5*.75)node[left]{$v_y$};
\draw[dashed](1.5,0)-- (1.5,1.5*.75)--(0,1.5*.75);
\draw[thick,->]  (0,0)--(1.5,1.5*.75)node[above]{$v_0$};
\end{tikzpicture}
    \caption{}
    \end{minipage}
    \begin{minipage}[t]{0.38\textwidth}
    \centering
\begin{tikzpicture}[>=latex, thick]
\tkzDefPoints{-1/0/O, 2/2/A, 2/1/E, 2/0/F, 2/-1/G, 2/-2/H}
\draw[->](O)--node[above]{$v_0$}(A);
\draw[->](O)--node[above]{$v_E$}(E);
\draw[->](O)--node[above]{$v_F$}(F);
\draw[->](O)--node[below]{$v_G$}(G);
\draw[->](O)--node[below]{$v_H$}(H);
\draw[->](A)--node[right]{$\Delta v$}(E);
\draw[->](E)--node[right]{$\Delta v$}(F);
\draw[->](F)--node[right]{$\Delta v$}(G);
\draw[->](G)--node[right]{$\Delta v$}(H);
\end{tikzpicture}
    \caption{}
    \end{minipage}
    \end{figure}

根据同样的道理,可以证明距离$AE$, $BF$和$CG$分别是在
1秒内,2秒内和3秒内由于重力的作用在竖直方向上下降的
距离,显然,$AE:BF:CG:DH=1:4:9:16$, 这就说明了斜抛运
动的确可以看成是物体沿初速度方向所做的匀速直线运动和
在重力作用下的自由落体运动的合运动。

此外,可以算出每1秒内的速度改变量,$\Delta v=g\Delta t=10\x
1=10\ms$,而且$\Delta v$的方向都是竖直向下的。可以把从开始抛出
时的初速度$v_0$经过4秒钟变化到落
地点的速度$v_H$的过程用图4.27表示
出来。

\subsection{关于竖直平面内的圆周运动}

在竖直平面内做圆周运动的物体,当一经过其轨迹
的最低点时,如课本171页第8题所要讨论的情况,圆心的
位置恰在物体的正上方,所以
物体所需的向心力(也就是物
体所受到的合外力)的方向,应
是竖直向上的。但由于重力方
向总是竖直向下的,在这一位
置,重力不可能起向心力的作
用,因此向心力必须由迫使物
体做圆周运动的另一物体——
悬绳来提供.如图4.28所示,如
果物体的质量为$m$, 悬绳长$\ell$, 经过最低点时的速率为$v$, 则在这一瞬时,
\[T-mg=ma_n=\frac{mv^2}{\ell}\]
绳子拉力
\[T=m\left(g+\frac{v^2}{\ell}\right)\]
表明了在这一瞬时,绳子的拉
力除了克服物体的重力外,还要承担由于提供物体做圆周运
动所需的向心力,因此绳子的拉力必然比物体静止时要大,而
且速率$v$越大,绳子拉力也越大。

\begin{figure}[htp]\centering
    \begin{minipage}[t]{0.48\textwidth}
    \centering
\begin{tikzpicture}[>=latex]
\fill [pattern=north east lines](-.5,3) rectangle (.5,3.2);
\draw(-.5,3)--(.5,3); \draw(0,3)--(0,0);
\draw(0,0) arc (-90:-60:3); \draw(0,0) arc (-90:-120:3);
\draw[fill=white](0,0) circle (.2);
\draw[<->, thick](0,1)node[right]{$T$}--(0,-1)node[right]{$mg$};
\draw[->](-.2,-.3)--node[below]{$v$}(-.7,-.3);
\tkzDefPoints{0/0/O} \tkzDrawPoint(O)
\draw[dashed](0,3)--node[right]{$\ell$}+(-70:3)[fill=white] circle (.2);
\end{tikzpicture}
    \caption{}
    \end{minipage}
    \begin{minipage}[t]{0.48\textwidth}
    \centering
    \begin{tikzpicture}[>=latex, scale=1]
\draw[very thick](0,0) circle (2);
\draw[<->](0,0)--node[fill=white]{$R$}(45:2);     
\draw[thick](0,1.8) circle (.2);
\draw[->,thick] (0,1.8)--(0,1.3)node[right]{$F$};
\draw[->,thick] (0,1.8)--(0,.5)node[left]{$mg$};
\draw[->] (0,2.25)--(.5,2.25)node[right]{$v>\sqrt{Rg}$};

    \end{tikzpicture}
    \caption{}
    \end{minipage}
    \end{figure}
在铁路和公路的立体叉道口,汽车往往是由隧道通过道
口,当汽车驶过隧道时,可以看成是在竖直平面里做圆周运
动。根据以上的分析,这时隧道底部路面受到的压力将大于汽
车重量。

当物体经过竖直圆周的最高点时,如课本图
4.33所讨论的情况,表明存在着一个最小速率$v=\sqrt{Rg}$, 式
中$R$为圆半径。如果物体的速率$v>\sqrt{Rg}$, 则所需的向心力
将大于物体所受的重力,这时,
除了物体的重力全部用作向心
力外,其不足的部分需要由圆
环顶部提供,如图4.29所示,即,
\[\frac{mv^2}{R}=mg+F\]
于是圆环顶部所受的压力
\[F'=-F=-\left(\frac{mv^2}{R}-mg\right)=-\frac{mv^2}{R}+mg\]

式中负号表示圆环顶部所受压力的方向和$F$的方向
相反。

\begin{figure}[htp]
    \centering
    \includegraphics[scale=.7]{fig/4-30.png}
    \caption{}
\end{figure}

汽车驶过一般拱形桥的顶部时,也可以看成是在竖直平
面里做圆周运动,如图4.30所示。在这一瞬时,圆心位置恰
在汽车的正下方。汽车做圆周运动经过这一位置时所需的向
心力的方向是竖直向下的,因此这一向心力就可以由汽车重
力的一部分来提供,正因为重力被用去一部分产生向心加速
度,因此作用于拱桥顶部的压力就减小了。即
\[mg-N=\frac{mv^2}{R}\]
桥顶所受的压力
\[F'=-N=-\left(mg-\frac{mv^2}{R}\right)=-mg+\frac{mv^2}{R}\]
式中负号表示拱桥顶所受的压力的方向和支持力$N$的方向
相反。

在汽车的速率$0<v<\sqrt{Rg}$的范围内,桥顶所受的压力
将比汽车静止在桥顶时要小,随着汽车行驶速率的增大,桥顶
所受的压力将减得更小,而当速率$v=\sqrt{Rg}$时,汽车做圆周
运动所需的向心力增大到恰好等于汽车所受的重力,于是桥
顶所受的压力将等于零。

如果速率$v>\sqrt{Rg}$, 则因为没有足够的向心力,汽车就
将飞离桥顶不再沿着拱桥做圆周运动。

可见以上讨论的虽然都是物体在竖直平面内做圆周运动
经过最高点时的情况,但还是有区别的,课本171页第9题
所讨论的是小球在圆弧的内侧运动,因此小球的速率必须满
足$v\ge \sqrt{Rg}$的条件,才能使小球在竖直平面里做圆周运动,
圆环顶部只有在$v>\sqrt{Rg}$的情况下才会受到压力的作用,而
在汽车驶过拱形桥顶的例子中,由于汽车在圆弧的外侧运动,
汽车的速率必须满足$v\le \sqrt{Rg}$的条件,才能使汽车沿着拱形
桥面顶部的圆弧运动,在$0<v<\sqrt{Rg}$的速率范围内,桥顶
都将受到压力的作用。

竖直平面里的圆周运动一般不是匀速圆周运动。
物体在竖直平面里做圆周运动时,由于物体所受重力的
大小和方向都是恒定不变的,因此,当物体经过圆周上的各
个不同位置时,重力对物体做圆周运动所需的向心力是否能
做出贡献、以及贡献的程度如何都是不相同的.从课本160
页的阅读材料的分析可知,假定图4.23甲所示的曲线是竖直
平面里的一段圆弧,又假定图示中的力$F$就是物体所受的重
力,这样就可想象在这一位置上,只有重力的一个垂直于圆弧
切线方向的分力$F_n$, 才起了向心力——即产生向心加速度的
作用,而重力的另一个沿着圆弧切线方向的分力$F_t$, 则起了
产生切向加速度的作用,这样,物体的线速度大小将会发生改
变,所以说一般说来,在竖直平面里的圆周运动不是匀速圆周
运动。


\chapter{万有引力定律}\minitoc[n]
\section{教学要求}
万有引力定律的发现,是人类在认识自然规律方面取得
的一个重大成果,对人类文化历史的发展有重要意义.万有
引力定律在研究天体的运动和人造地球卫星等方面有着重要
的应用.鉴于这一规律的重要,把它单独列为一章,使内容集
中,中心突出.

这一章的教学要求是:
\begin{enumerate}
\item 了解开普勒三定律,掌握万有引力定律.
\item 了解万有引力定律在天文学上的应用,了解地球上物
体的重量变化的原因.
\item 了解有关人造卫星的知识,会推导第一宇宙速度.
\end{enumerate}

下面对这一章的教学内容作些具体说明.

为讲解万有引力定律的建立作准备,第一节先介绍行星
的运动,关于人类对行星运动规律的认识过程,只要求学生
了解个梗概,知道开普勒三定律是在前人长期观察研究的基
础上总结出来的,这一节的重点是讲解开普勒三定律,使学
生对定律的内容有所了解.学生在学习本章时,还不具备椭
圆的知识,教学中需要对椭圆的焦点、半长轴作简单的介绍.

万有引力定律的教学,主要是让学生知道牛顿如何在开
普勒三定律的基础上推导出万有引力定律的思路.在介绍牛
顿建立万有引力定律之前,提到了胡克等人猜想到引力与距
离的平方成反比,是为了说明万有引力定律的建立经历了一
个过程,不是只靠个别天才人物的灵感创造的.引力与太阳
质量的正比关系,可以直接给出,不要求作进一步的讨论.地
球对月球的吸引力与地面物体所受的重力是同一种性质的
力,让学生自己通过练习计算得出,以获得较深刻的印象.如
果学生对这个题目的推理过程不很理解,也可以作些必要的
引导和说明.

万有引力定律揭示了支配天体运动的规律,把地上的运
动和天上的运动统一起来,打破了以往人们对天体运动的神
秘感,增强了人们认识自然的信心,讲述万有引力定律,应该
使学生对此有所认识.

卡文迪许实验是历史上的著名实验,它测定了万有引力
恒量的值.鉴于这个实验的重要,单独作为一节来讲述.这
个实验,不要求演示,通过介绍这个实验,使学生认识这个实
验的重要作用,领会前人是怎样进行巧妙的设计来测出万有
引力恒量值的,启发他们进一步认识培养和训练灵活运用知
识能力的重要性.

列举几个物体间引力大小的例子,是为了说明一般物体
间的引力非常小,而天体之间的引力非常大.正是这个巨大
的引力支配着天体的运动.因而万有引力定律主要用于研究
天体的运动.

天体质量的计算,说明应用万有引力定律和圆周运动的
知识,可以确定无法直接测定的天体质量.在天文学上,太
阳、地球等天体的质量就是根据行星或卫星的轨道半径和周
期来求得的.海王星、冥王星的发现,说明万有引力定律不仅
能对观察到的天体运动作出解释,而且能预言尚未观察到的
天体的存在.这是理论指导实践的典型事例.

关于人造地球卫星的教学,重点是讲述发射人造卫星的
原理,得出第一宇宙速度后,要指出卫星进入轨道的水平速
度大于7.9${\rm km/s}$,小于11.2${\rm km/s}$时,卫星绕地球运动的
轨道将不是圆,而是椭圆,进而说明速度增大到11.2${\rm km/s}$
后,不再绕地球运行,而成为围绕太阳运动的一颗行星.至于
轨道为什么会成为椭圆,限于学生的知识水平,中学阶段不能
讲解.对于第二宇宙速度和第三宇宙速度,只要求简单介绍,
使学生知道它们的意义就行了.

这一章的习题大多是综合性的,灵活性也有所提高,要
注意向学生讲明解决这类问题的思路,以培养他们灵活运用
知识,逐步提高解题能力.

\section{教学建议}
本章内容是按照人类对万有引力定律的认识过程,围绕
运动和力的关系而逐步展开的.

学习本章内容是对前几章知识的综合和提高,在教学中
应着重培养学生综合运用新旧知识对问题进行推理和分析的
能力.此外,引导学生体会建立万有引力定律程中所体现
的科学方法,以及激发学生对未知世界的探索精神和科学的
想象力,也是本章教学中不可忽视的方面.

本章分为两个单元,第一单元包括第一至第三节,概括
地介绍了万有引力定律建立的历史进程,其中包括对引力恒
量的测定.第二单元包括第四至第六节,介绍了万有引力定
律的某些应用.

\subsection{第一单元}

万有引力定律的建立过程,对于已确立的定律、新的假
说、理论推导和实验观测之间如何相互影响和补充,提供了一
个很好的范例.因此本单元的教学可以按产生这一定律的历
史背景、定律的建立和定律的实验验证这三个层次来展开,这
样可以使学生对这一理论获得一个整体的认识,从而体会到
具有突破性的重大物理理论的建立,并不是偶然的,它反映了
人类对自然界的认识不断深化和完善的过程.

\subsubsection{开普勒定律}

第一节的重点是介绍开普勒三定律,讲
述时可指出,开普勒定律是一种描述性的经验定律.开普勒
定律描述了行星运动的规律,但没有提出和解决行星为什么
这样运动的问题,这个重要的问题是牛顿在他的运动定律的
基础上解决的.

鉴于学生在学习本节时还没学过椭圆知识,因此可结合
课本图5.2作简单解释.教学中应当指出该图表示的行星椭
圆轨道是一个十分夸张的示意图,事实上大部分行星的椭圆
轨道都十分接近于圆形,因此可对开普勒第一、第二定律作近
似处理,即认为行星以太阳为圆心作匀速圆周运动,而第三定
律中椭圆的半长轴可以当作圆形轨道的半径$R$. 应使学生明
确本章所有对天体运动的分析计算都是在上述近似处理的基
础上用匀速圆周运动的动力学方法进行的.

在说明开普勒第三定律中$k$值是一个与行星无关的恒量
时,可指出$k$值只与行星所环绕的那个天体有关,至于为什么
会这样,以后将会作进一步深入讨论.最后可将练习一的第
1题作为课堂练习,让不同小组的同学分别算出各行星的$k$
值加以比较.通过这一练习学生对太阳系中的$k$值与各行星
无关便有了具体认识.对计算结果中$k$值的差异可简单指出
这是由于表中原始数据不太精确(仅三位有效数字),而且开
普勒第三定律本身也是近似的定律.(见参考资料2)

\subsubsection{万有引力定律的建立}

第二节是全章的重点也是教
学中的一个难点.通过本节学习要使学生认识牛顿所建立的
万有引力定律不仅解决了行星运动的起因,而且揭示了自然
界物体间普遍存在的一种基本相互作用.为此,教学中可围
绕地和天统一这个中心突出点:第一,牛顿如何将天体运
动规律(开普勒第三定律)和在地球上得出的力学规律联系起
来,进行演绎,从而导出平方反比定律的.分析中只需指出引
力还与太阳质量$M$成正比这一结论,不必对常数$k$和太阳质
量$M$的关系作进一步讨论.第二,牛顿如何推广平方反比定
律,将天体间的引力和地面上的重力统一起来,使之成为一条
字宙万物间的普适物理定律.教材中对后一点的陈述较简
练,并在练习二中设计了一道题,引导学生通过推导和计算来
理解这段陈述.教学中可将此题(练习二4)作为课堂练习
导学生边练习边分析.练习时,可根据保持月球在其轨道
上运动的力也就是把地面上的物体放在那个位置所受到的重
力这一思路,画出示意图来帮助学生分析.

在平方反比定律的推广中,要将$g$和$a_R$加以比较,把两
种不同运动形式的加速度联系在一起,认为它们出自同一性
质的力,学生往往感到不易理解,这主要还是由于学生仍习
惯于从运动表现形式上来比较物体的受力情况,错误地认为
物体运动形式不相同,它所受的力也一定不相同,而对物体运
动方式是由受力和初始运动状态所共同决定的这一点,缺乏
足够的认识.要让学生认识地面上的苹果和天空中的月亮虽
然受到同一性质的力——地球引力的作用,但并不因此决定
它们有相同的运动形式.苹果的初速度为零,它便自由下落;
如果给它一个水平方向的初速度,它就作平抛运动;如果这个
水平速度越来越大,苹果也有可能绕着地表作匀速圆周运动.
在练习二4中适当点明这一点,不仅能帮助学生明确牛顿推
广平方反比定律的合理性,而且也为后面“人造卫星”一节的
教学作了一定的准备.

为了培养学生演绎推理的思维能力,进行上述课堂练习
时,教师要将以下三个层次交代清楚,即:
\begin{enumerate}
    \item 提出假设:牛
顿设想使月球围绕地球运行的力和地面上的重力属于同一性
质的力,都来自地球引力;
\item 根据假设进行演绎推导:练习
二的1、2、3;
\item 用已知的观察数据验证推导的结论,从
而证实假设是否成立:练习二的4、5.
\end{enumerate}

在归纳时,教师
可指出这种研究方法与第二章第十节阅读材料中所介绍的伽
利略研究匀变速运动的方法是一致的,从而引导学生对这一
物理学基本研究方法有更深一步的体会.

\subsubsection{对万有引力定律公式的理解}

引入万有引力定律公
式后要引导学生认识以下两点:

第一,平方反比定律公式的
形式是学生在学习物理中第一次遇到,在以后的学习中还要
接触,可引导学生注意这一公式在数学形式上的特点,并点明
这种与距离平方成反比的数学形式反映了自然界物质相互作
用所遵循的一种重要方式.此外必须明确对两个相距不太远
的非球形物体,不可简单地把两物中心间距作为R代入公式
来计算,这样只能作出粗略的估算.

第二,在说明引力与两个物体质量的乘积成正比时,要指
出两个不接触物体间的相互引力作用也是服从牛顿第三定律
的,即使很大质量和很小质量之间的相互吸引力也是大小相
等的.这一点似乎与学生的普通常识相矛盾,由于学生一般
遇到的都是卫星绕行星、行星绕恒星、地球表面物体自由下落
之类的问题,所以往往容易产生似乎只是质量大的物体吸引
质量小的物体,或者质量大的物体对质量小的物体的引力大
的错误观念,为了帮助学生理解这一点,可将练习二中的1、2作为课堂讨论题进行分析,并且举潮汐为例说明不仅地球
吸引月球,而且月球也吸引地球,潮汐就是质量小的物体也吸
引质量大的物体的具体例证.

\subsubsection{万有引力恒量} 
$G$是学生接触到的不多几个具有重
要地位的物理学普适恒量之一,要向学生指出,$G$作为万有
引力定律中的比例常数,不能单纯从数学角度去理解,要充分
认识它所表征的物理意义.要使学生理解比例常数是在描述
某种物理规律时经常出现的,各个常数有其特定的物理意义.
让学生回忆一下过去有比例常数的公式,如第一章中的胡克
定律$f=kx$中的 $k$, 表示某种材料在弹性限度内的力学性质,
因材料而异,不带普适性.而$G$表征质点间引力作用的
性质,它的数值等于两个质量各为1千克的质点相距1米的
相互吸引力,是适用于任何物体的普适恒量.以上这些比例
常数都有单位,单位由相关物理量决定.可以让学生自己确
定一下$G$的单位.其次要注意让学生对$G$的数值非常小有
个感性认识,防止学生产生一种错误观念:诸如认为固体之
所以成形,主要是由于物质颗粒间的万有引力使它们结合在
一起等,本章中许多数据用指数表示,并出现了不少指数运
算,为此可简单向学生介绍一下什么是数量级和怎样进行数
量级的估算.可结合本节教材中最后一段让学生自己估算一
下几种不同情况下引力的数量级,从而对一般物体之间和
天体之间引力大小的巨大差异有一个鲜明的认识.并建议用
以下的板图(或投影片)形象化地表示,也可
要求学生自己画在笔记本上,这比只用文字表示更易留下较
深的印象.

\subsubsection{卡文迪许实验}

在介绍卡文迪许实验装置时,可绘制
扭秤装置俯视图,也可做一个扭秤模型,说明扭秤装置中的
$T$形架增大了引力$F$的力臂,从而使石英细丝在$m$、$m'$两球
间微小的引力作用下产生一定的扭转形变,而$T$形架上的小
镜又利用光的反射定律把这一微弱的形变效应放大,加大标
尺与小镜间距离又能增大标尺上光点的偏转距离.在此可提
醒学生回忆一下第14页阅读材料:“显示微小形变的装置”.
正是这种“三次放大”的作用,扭秤才能较准确的测定微小的
作用力,例如,现代形式的卡文迪许装置能测出的引力约为
$6\x10^{-10}$牛,一根人发的重量是它的一万倍.这类利用“光杠
杆”作用的扭秤装置,是所有机械装置中最灵敏的装置之一.

\begin{center}
\begin{tabular}{p{.35\textwidth}p{.1\textwidth}p{.4\textwidth}}
    \hline
    两个物体 &引力的数量级 &相当于\\
    \hline
相距1米的两个1千克物体 & $10^{-10}$N  &  一粒砂子重量的1千万分之一 或 一根头发重量的十万分之一\\
相距10厘米的两个100克苹果 & $10^{-8}$N  &  一粒砂子重量的10万分之一\\
相距1米的两个成人 & $10^{-7}$N  &  一粒砂子重量的万分之一\\
相距100米的两艘万吨轮 & $10^0$N  &  两只鸡蛋的重量\\
相距$4\x 10^8$米的地球和月球 & $10^{20}$N  & 拉断钢索\\
相距$10^{11}$米的太阳和地球 & $10^{22}$N  &  可将直径为几千米的钢柱拉断\\
    \hline
\end{tabular}
\end{center}

\subsection{第二单元}
本单元运用的公式及相应的物理量较多,问题的综合程
度和灵活性又较前一章有所提高,学生往往不注意作有条理
的分析而惯于套用现成公式,单纯作公式代换,并易犯单位
和运算的错误,因此在教学过程中需注意帮助学生掌握综合
运用万有引力定律和匀速圆周运动的动力学方法分析具体问
题的基本思路.有关单位统一、指数运算等也要注意作出示范.

\subsubsection{万有引力定律在天文学上的应用} 

在引入“天体质量
算”这一课题时,可先提出能不能用简单的实验方法直接测
定地球或太阳的质量的问题,启发学生思考,并引导他们自己
用万有引力定律和圆周运动的知识,一步步导出计算天体质
量的公式.然后指出,计算某天体质量时只需知道围绕该天
体运行的行星(或卫星)的轨道半径$R$和周期$T$, 因这两个量
是可以测定的.

\subsubsection{地球上物体重量的变化}

通过本节学习应使学生了
解影响地球上物体重量变化的三个因素:纬度、离地高度和地
质结构,其中纬度(即地理位置)的变化是主要因素,关键是
使学生明确由于地球的自转,重力仅是引力的一个分力,而且
引力本身又从两极到赤道逐渐变小,此外应向学生指出课本
图5.4仅仅是一个示意图,地球的椭球状以及向心力相对于
引力的大小都是夸大了的,引力和重力之间的夹角也是极
小的.

本节教学中还有必要向学生指出处理某一物理量的变与
不变是相对的,必须根据所研究问题的要求来决定,由于地
球上的$g$随纬度、高度变化的相对数值很小,在一般计算中并
不考虑$g$的变化,而将它作为常数处理.


\subsubsection{人造地球卫星}

由于人造卫星问题的综合性较强,所
涉及的概念较多,学生往往搞不清其中的关系,常犯的错误
是把卫星绕地球运行的速率和第一宇宙速度(环绕速度)相混
淆.应该使学生明确:卫星绕地运行速率的表示式$v=\sqrt{\dfrac{GM}{r}}$,
对所有在圆形轨道上的地球卫星普遍适用,$v$的大小随$r$而
改变.而环绕速度表示式$v=\sqrt{gR_{\text{地}}}$仅适用于在近地圆形轨
道上运行的卫星,式中$g=9.8\msq$, 环绕速度的值为$7.9{\rm km/s}$,是个定值.

根据计算式$v=\sqrt{\dfrac{GM}{r}}$
,离地越远的卫星,$r$越大,$v$越
小.学生往往感到这一结论与课本图5.6中卫星进入轨道的
水平速度越大,轨道偏离地球越远的情况相矛盾,应该使学
生认识,卫星在椭圆轨道上运行时,它在各点的运动速度是不
同的.根据开普勒第二定律,卫星在近地点速度大,在远地点
速度小,在轨道上的平均速度也比在近地点的速度小,卫星进
入轨道的水平速度,只是卫星在近地点的速度,并不能反映出
它在椭圆轨道上各点的实际速度.根据公式$v=\sqrt{\dfrac{GM}{r}}$,
可
以用练习三1的方法,以卫星在近地点和远地点到地心距离
的平均值作为平均轨道半径,近似地求出卫星在轨道上的平
均运行速度.

教学中也可将练习四3作为课堂讨论练习题,通过对这
一问题的具体分析,进一步引导学生明确以上几点,在解答
该题时学生往往会将$2\pi R$, 除以周期80分钟,得出运行速率
$v\approx 8.4{\rm km/s}$,又根据这一速率大于$7.9{\rm km/s}$,便断定可
发射这样一颗卫星,学生之所以会得出这错误结论,是由
于不明确卫星的$T$、$r$和$v$之间有着确定的关系,因而错误地
将$R_{\text{地}}$作为轨道半径$r$来求运行速率$v$. 其次又将运行速率
错误地与进入轨道的最低水平速率混同起来.遇到这类错误,
可将它写在黑板上让学生共同来分析产生错误的原因.

关于第二、第三宇宙速度,只需指出只有当卫星获得足够
大的速度时它才能摆脱地球,甚至太阳的引力羁绊,而不必作
其他补充.

卫星中的超重、失重问题,主要抓住加速度向上还是向下
这个关键进行分析.对卫星在轨道上的失重情况,应使学生
理解此处所谓向下的加速度就是指向地心的向心加速度.

\section{实验指导}
\subsection{演示实验}

\subsubsection{天文挂图}

太阳系八大行星围绕太阳运动的示意
图,以及八大行星大小比较示意图,使学生对太阳系的结构有
一定性的形象了解.

\subsubsection{卡文迪许扭秤实验模型}
可根据课本图5.3的卡文迪许扭秤实验示意图,自制模
型,使学生了解扭秤装置的构造原理,以及如何利用光杠杆的
放大作用,读出石英丝的微小扭转形变.同时可以画出俯视
示意图(图5.1).模型中射到平面镜$M$上的光是从平行光管
$S$射出的,反射光点投到圆弧形刻度盘上.

\begin{figure}[htp]
    \centering
    \begin{tikzpicture}[>=latex]
\draw[dashed](-.05,-3) rectangle (.05,3);
\draw[rotate=-15](-.05,-3) rectangle (.05,3);
   \draw(45:3.5) circle(5pt)node[right=3pt]{$m'$};
\draw(45+180:3.5) circle(5pt)node[left=3pt]{$m'$};
\draw(75:3)[fill=white] circle(3.5pt)node[above=3pt]{$m$};
\draw(75+180:3)[fill=white] circle(3.5pt)node[below=3pt]{$m$};
\draw(90:3)[fill=white] circle(3.5pt);
\draw(90+180:3)[fill=white] circle(3.5pt);
\node at (0,0)[left]{$M$};
\draw[very thick] (40:5) arc (40:-40:5);
\draw[thick](5,0)node[right=.5]{$S$}--(0,0)--(-30:5)node[right]{$P$};
\draw[->,thick](5,0)--(2.5,0);
\draw[->,thick](0,0)--(-30:2.5);
\draw[dashed](0,0)--(-15:2);
\draw(-15:1) arc (-15:0:1);\node at (-7.5:1.3){$\theta$};
\draw (-15:.8) arc (-15:-30:.8);\node at (-22.5:1.2) {$\theta$};
\draw(5,-.1)--(5.5,-.1);
\draw(5,.1)--(5.5,.1);
\tkzDefPoint(75:3){A}  \tkzDefPoint(75+180:3){B}
\tkzDefPoint(45:3.5){A1}  \tkzDefPoint(45+180:3.5){B1}
\tkzDrawLines[->, add = 0 and -.7](A,A1 A1,A B,B1 B1,B)
\draw(-90:1.2) arc (-90:-105:1.2);
\node at (-97.5:1.5){$\theta$};
\draw[decorate,decoration={brace,raise=8pt}] (A) --node[above=9pt]{$r$}node {$F$} (A1);
\draw[decorate,decoration={brace,raise=8pt}] (B) --node[below=9pt]{$r$} node {$F$}  (B1);

    \end{tikzpicture}
    \caption{}
\end{figure}



\subsubsection{挂图——人造地球卫星、宇宙飞船}

目的在于使学生了解人造地球卫星和宇宙飞船在轨道上
运行的原理,以及通讯卫星为什么可以实现全球电视转播的
原理.

有条件的可以播放人造地球卫星的发射、空间站、宇
宙飞船的运行(包括字航员的失重状态)、航天飞机的发射与
返航等资料影片或录象,以增加感性认识,提高学习兴趣.




\section{习题解答}
	
\subsection{练习一}
\begin{enumerate}
	\item 下表给出了太阳系九大行星平均轨道半径和周期的数值.从表中任选三个行星验证开普勒第三定律,并计算恒
	量$k=R^3/T^2$的值.
\begin{center}
	\begin{tabular}{ccc}
\hline
行星    &  平均轨道半径(m)  & 周期(s)\\
\hline
水星      &  $5.79\times 10^{10}$    & $7.60\times 10^8$ \\ 
金星    &  $1.08\times 10^{11}$    &  $1.94\times 10^7$ \\ 
地球    &  $1.49\times 10^{11}$    &  $3.16\times 10^7$ \\ 
火星    &  $2.28\times 10^{11}$    &  $5.94\times 10^7$ \\ 
木星    &  $7.78\times 10^{11}$    &  $3.74\times 10^8$ \\ 
土星    &  $1.43\times 10^{12}$    &  $9.30\times 10^8$ \\ 
天王星    & $2.87\times 10^{12}$     &  $2.66\times 10^9$ \\ 
海王星    &  $4.50\times 10^{12}$    &  $5.20\times 10^9$ \\ 
冥王星    & $5.9\times 10^{12}$     &  $7.82\times 10^9$ \\ 
\hline
	\end{tabular}
\end{center}


\begin{solution}
	取地球、火星、木星为例来验证开普勒第三定律.
地球:
\[k=\frac{R^3}{T^2}=\frac{(1.49\x 10^{11})^3}{(3.16\x 10^{7})^2} =3.31\x 10^{18}  {\rm m^3/s^2}\]

火星:\[k=\frac{R^3}{T^2}=\frac{(2.28\x 10^{11})^3}{(5.97\x 10^{7})^2} =3.36\x 10^{18}   {\rm m^3/s^2}\]
木星:\[k=\frac{R^3}{T^2}=\frac{(7.78\x 10^{11})^3}{(3.74\x 10^{8})^2} = 3.37\x 10^{18}  {\rm m^3/s^2}\]
故:\[\bar k=\frac{(3.31+3.36+3.37)\x 10^{18} }{3}=3.35\x 10^{18}  {\rm m^3/s^2}\]
\end{solution}

\item 有一个名叫谷神的小行星(质量$1.00\times 10^{21}$kg),它的轨道半径是地球的2.77倍,求出它绕太阳一周需要多少年.

\begin{solution}
	由开普勒第三定律$k=\dfrac{R^3}{T^2}$,取$k=3.35\x 10^{18}  {\rm m^3/s^2}$,则
\[T=\sqrt{\frac{R^3}{k}}=\sqrt{\frac{(1.49\x 10^{11}\x 2.77)^3}{3.35\x 10^{18}}}=1.45\x 10^8{\rm s}\]
\end{solution}
说明:题中给出的谷神小行星的质量与解题无关,目的是
使学生明确行星运行周期仅取决于轨道半径,同时培养学生
合理利用已知数据的能力.
\end{enumerate}





\subsection{练习二}
\begin{enumerate}
	\item 你能说出你对地球的引力是多少吗?

	\begin{solution}
		人对地球的引力大小等于地球对人的引力大小,也
		就是大的体重.
	\end{solution}
	
\item “我们说苹果落向地球,而不说地球向上运动碰到苹果,是因为地球的质量比苹果大得多,地球对苹果的引力比苹果对地球的引力大得多.”这种说法对吗?为什么?

\begin{solution}
	这种说法不对.地球对苹果的引力大小与苹果对地
	球的引力大小相等,是一对作用力和反作用力,因为地球质量
	比苹果大得多,所以它产生的加速度就比苹果产生的加速度
	(即重力加速度)小得多,几乎等于零,所以我们说苹果落向地
	球,面不说地球向上运动碰到苹果.
\end{solution}

\item 两个质量都是4千克的铅球,相距0.1米远,它们之间的引力是多少?

\begin{solution}
	据万有引力公式
\[F=G\frac{Mn}{R^2}=6.67\x 10^{-11}\x \frac{4^2}{0.1^2}=1.07\x 10^{-7}{\rm N}\]
\end{solution}

\item 用$M$表示地球的质量,$R$表示地球的半径,$T$表示月球到地球的距离.试证明,在地球引力作用下,
\begin{enumerate}
	\item 地面上物体的重力加速度$g=\dfrac{GM}{R^2}$;
	\item 月球的加速度$a_{\text{月}}=\dfrac{GM}{r^2_{\text{月地}}}$;
	\item 已知$r_{\text{月地}}=60R$,利用(a)(b)求$a_{\text{月}}/g$;
	\item 已知$r_{\text{月地}}=3.8\times 10^8$米,月球绕地球运行的周期$T=27.3$天,计算月球绕地球运行的向心加速度$a_{\text{月}}$.
	\item 已知重力加速度$g=9.8\msq$.用(d)中算出的$a_{\text{月}}$,求$a_{\text{月}}/g$.
	
	比较(c)(e)中求出的$a_{\text{月}}/g$是否相等.如果相等,则表明地球
	对月球的引力和对地面物体的引力都遵守平方反比定律,因
	而是同一种性质的力,牛顿就是根据这一结果证明地球对月球的引力和地面上物体所受的重力是同一种力的.
\end{enumerate}

	\begin{proof}
\begin{enumerate}
	\item 地球表面物体的重力等于地球对物体的引力,有
\[mg=G\frac{Mm}{R^2}\quad \Rightarrow\quad g=G\frac{M}{R^2}\]
\item 假设月球绕地球运行所需的向心力就是地球对月球
的万有引力.$a_{\text{月}}$是月亮绕地球运行的向心加速度,方向指向地
球.则有
\[ma_{\text{月}}=G\frac{Mm}{r^2_{\text{月地}}}\quad \Rightarrow\quad a_{\text{月}}=\frac{GM}{r^2_{\text{月地}}}\]
\item \[\frac{a_{\text{月}}}{g}=\frac{GM\cdot R^2}{r^2_{\text{月地}}\cdot GM}=\frac{R^2}{r^2_{\text{月地}}}\]
将$r^2_{\text{月地}}=60R$代入上式,得
\[\frac{a_{\text{月}}}{g}=\frac{R^2}{(60R)^2}=\frac{1}{3600}=2.8\x 10^{-4}\]
\item 由匀速圆周运动向心力公式得
\[\begin{split}
	a_{\text{月}}=\omega^2r_{\text{月地}}=\left(\frac{2\pi}{T}\right)^2\cdot r_{\text{月地}}
&=\frac{4\pi^2}{T^2}\cdot r_{\text{月地}}\\
&=\frac{2\x (3.14)^2}{(27.3\x 86400)^2}\x 3.8\x 10^{8}\\
&=2.69\x 10^{-3}\msq
\end{split}\]
\item \[\frac{a_{\text{月}}}{g}=\frac{2.69\x 10^{-3}}{9.8}=2.8\x 10^{-4}\]
\end{enumerate}
比较(c)、(e)的结果相等,说明牛顿的假说是正确的.
	\end{proof}
	
\end{enumerate}




\subsection{练习三}
\begin{enumerate}
	\item 应用人造地球卫星可以测定地球的质量.我国1970年4月24日发射的第一颗人造地球卫星,其周期是114分,它的近地点是439千米,远地点是2384千米,以卫星在近地点和远地点时到地心距离的平均值作为卫星轨道的平均半径,试计算地球的质量.

	\begin{solution}
		取地球半径为6370千米,则卫星轨道的平均半径为		
	\[R=\frac{439+2384}{2}+6370=7782{\rm km}\]
		卫星绕地球运动的向心力等于地球对卫星的引力,即有
\[m\omega^2 R=G\frac{Mm}{R^2},\qquad \left(\frac{2\pi}{T}\right)^2 R=G\frac{Mm}{R^2}\]
则:
\[M=\frac{4\pi^2\cdot R^3}{GT^2}\]
代入数据得
\[M=\frac{4\x 3.14^2\x (7.782\x 10^6)^3}{6.67\x 10^{-11}\x (114\x 60)^2}=5.96\x 10^{24}{\rm kg}\]
	\end{solution}
	
	\item 登月密封舱在离月球表面112千米的空中沿圆形轨道运行,周期是120.5分钟,月球的半径是1740千米,根据这些数据计算月球的质量和平均密度.

	\begin{solution}
密封舱绕月球所需的向心力就是月球对密封舱的引
力.

由$m\cdot \dfrac{4\pi^2}{T^2}\cdot R=G\dfrac{Mn}{R^2}$,得$M=\dfrac{4\pi^2 R^3}{GT^2}$

密封舱轨道半径
\[R=112+1740=1852{\rm km}\]
代入数据
\[M=\frac{4\x3.14^2\x(1.852\x10^8)^3}{6.67\x10^{-11}\x(120.5\x60)^2}
=7.19\x10^{22}{\rm kg}\]
平均密度
\[\rho=\frac{M}{V}=\frac{M}{\frac{4}{3}\pi R^3}=\frac{7.19\x 10^{22}\x 3}{4\x 3.14\x (1.740\x 10^{6})^3}=3.26\x 10^{3}{\rm kg/m^3}\]
	\end{solution}
	
\end{enumerate}


\subsection{练习四}

在下列各题中,地球质量取$M=6.0\times 10^{24}{\rm kg}$.
\begin{enumerate}
	\item 图5.8 $A$、$B$、$C$是在地球大气层外圆形轨道上运行的三颗人造卫星,$A$、$B$的质量相同,它们的轨道速率是否也相同?$B$、$C$的质量不同,它们的轨道速率是否也不同?

\begin{figure}[htp]
\centering\begin{tikzpicture}[>=latex]
\draw (0,0) circle (15pt);
\node at (0,0){地球};
\draw (0,1.5) arc (90:15:1.5);
\draw (0,3) arc (90:15:3);
\draw [<-](0,2.2) arc (90:70:2.2);
\node at (0,2.2)[left]{卫星运行方向};
\draw[fill=white] (60:1.5) circle (2pt) node[above]{$A$};
\draw[fill=white] (75:3) circle (2pt) node[above]{$B$};
\draw[fill=white] (40:3) circle (5pt);
\node at  (40:3.2) [right]{$C$};
\end{tikzpicture}
\caption{}
\end{figure}


\begin{solution}
	由卫星绕地球运行所需向心力即为地球对卫星的引力,即
\[\frac{mv^2}{r}=G\frac{M_{\text{地}}m}{r^2}\]
得卫星的轨道速率
\[v=\sqrt{\frac{GM_{\text{地}}}{r}}\]
由于
式中$G$和$M_{\text{地}}$为常量,所以,卫
星的轨道速率只与其轨道半径的平方根成反比.由于卫星$A$
方向
和$B$不在同一轨道上,即$r_A\ne r_B$,即速率不等.

卫星$B$和卫星$C$同在一条
轨道上运行.即$r_B=r_C$, 则$v_B=v_C$,即速率相等,跟卫星$B$、$C$的质量无关.
\end{solution}


\item  假定一颗人造地球卫星正在离地面700千米高空的圆周轨道上运转,计算它的速率和周期.

\begin{solution}
	同上题由$\dfrac{mv^2}{r}=G\dfrac{Mm}{r^2}$,得速率
$v=\sqrt{\dfrac{GM}{r}}$,代入数据:
\[v=\sqrt{\frac{6.67\x 10^{-11}\x 6.0\x 10^{24}}{(6400+700)\x 10^3}}=7.5{\rm km/s}\]
周期
\[T=\frac{2\pi R}{v}=\frac{2\x 3.14\x (6400+700)\x 10^3}{7.5\x 10^3}=5.95\x 10^3{\rm s}\approx 99{\rm min}\]
\end{solution}

\item 能否发射一颗周期是80分钟的人造地球卫星?说明你的理由.

\begin{solution}
	解法1:
	若卫星的周期为80分钟,则
\[\frac{m\left(\frac{2\pi}{T}\right)^2}{R}=G\frac{Mm}{R^2},\qquad \frac{4\pi^2}{T^2}=\frac{GM}{R^3}\]
因此:
\[R^3=\frac{GMT^2}{4\pi^2}=\frac{6.67\x 10^{-11}\x 6.0\x 10^{24}\x (80\x 60)^2}{4\x 3.14^2}=2.34\x 10^{20}{\rm m^3}\]
\[R\approx 6.2\x 10^{6}{\rm m}\]
取地球半径为$6.4\x10^6$m,则$R<R_{\text{地}}$.

由于卫星飞行的圆周半径不可能小于地球半径,故不可
能发射这样一颗卫星.

解法2:根据$T^2=\dfrac{4\pi^2R^3}{GM}$,卫星轨道半径$R$越大,周期$T$
越长.

靠近地球表面以第一宇宙速度运行卫星的周期最短,为
\[T=\frac{2\pi R}{v}=\frac{2\x 3.14\x 6.4\x 10^{6}}{7.9\x 10^3}=5.1\x 10^3{\rm s}=85{\rm min}\]

而题中卫星周期为80分钟,所以不可能发射一颗运行周
期比以第一宇宙速度运行的卫星还短的地球卫星.
\end{solution}

\end{enumerate}





\subsection{习题}
\begin{enumerate}
	\item 在一次测定引力恒量的实验里,已知一个质量是0.80千克的球,以$1.0\times 10^{-10}$牛的力吸引另一个质量是$4.0\times 10^{-3}$千克的球.这两个球相距$4.0\times 10^{-2}$米.地球表面的重力加速度是9.8$\msq$,地球的半径是6400千米.根据这些数据计算地球的质量.

	\begin{solution}
		因为地球对物体的引力就是物体所受的重力,所以$mg=G\dfrac{Mm}{R^2}$,由此 得:
		\[M=\frac{gR^2_{\text{地}}}{G}\]
		先根据$F=G\dfrac{m_1m_2}{r^2}$,求出$G=\dfrac{Fr^2}{m_1m_2}$,代入上式得:
	\[M=\frac{gR^2_{\text{地}}m_1m_2}{Fr^2}=\frac{9.8\x (6.4\x 10^6)^2\x 0.80\x 4.0\x 10^{-3}}{1.3\x 10^{-10}\x (4.0\x 10^{-2})^2}=6.2\x 10^{24}{\rm kg}\]
	\end{solution}
	
	\item 行星的质量为$M$,一个围绕它作匀速图周运动的卫星的轨道半径是$R$,周期是$T$.试用两种方法求出卫星轨道上的向心加速度.

	\begin{solution}
		解法1:根据向心加速度的公式有
	\[a_n=\omega^2 R=\left(\frac{2\pi}{T}\right)^2\cdot R=\frac{4\pi^2 R}{T^2}\]
		解法2:根据卫星所需向心力等于行星对它的引力有
	\[ma_n=G\frac{Mm}{R^2}\]
	所以:$a_n=\dfrac{GM}{R^2}$
	\end{solution}
		
	\item 应用通讯卫星可以实现全地球的电视转播,这种卫星位于赤道的上方,相对于地面静止不动,犹如悬在空中一样,叫做同步卫星.同步卫星的周期是多大?计算它的高度和速率.

	\begin{solution}
		同步卫星的周期与地球自转的周期相同.
		\[T=24\x3600=86400{\rm s}\]
		根据
$m\dfrac{4\pi^2}{T^2}\cdot r=G\dfrac{Mm}{r^2}$,得:
\[r^3=\frac{GM}{4\pi^2}\cdot T^2\]
故卫星轨道半径为
\[r=\left(\frac{GMT^2}{4\pi^2} \right)^{1/3}=\left(\frac{6.67\x 10^{-11}\x 6.0\x 10^{24}\x 86400^2}{4\x 3.14^2}\right)^{1/3}=4.23\x 10^7{\rm m}\]
上式也可由下式得出:
\[v=\sqrt{\frac{GM}{r}},\qquad \frac{2\pi r}{T}=\sqrt{\frac{GM}{r}}\]

卫星高度
\[h=r-R_{\text{地}}=4.23\x 10^{7} - 6.4\x 10^6=3.59\x 10^7{\rm m}\]
卫星运行速率
\[v=\frac{2\pi r}{T}=\frac{2\x3.14\x4.23\x10^7}{86400}=3.07\x10^3{\rm m/s}\]
或者:
\[v=\sqrt{\frac{GM}{r}}=\sqrt{\frac{6.67\x 10^{-11}\x 6.0\x 10^{24}}{4.23\x 10^7}}=3.07\x 10^3\ms\]

		说明:此题说明卫星的轨道半径$r$、运行周期$T$和速率
$v$之间存在着确定的关系.对同步地球卫星讲,$T$是确定的,
因此所有的同步卫星均在赤道平面的同一轨道上,以相同速
率运行.
	\end{solution}
	
	\item 试用万有引力定律证明:对于某个行星的所有卫星来说,$R^3/T^2$是一个恒量.其中$R$是卫星的轨道半径,$T$是卫星的运行周期.

	\begin{proof}
	卫星围绕行星运行所需的向心力就是它们之间的
万有引力.即有
\[G\frac{Mm}{R^2}=m\omega^2 R,\qquad G\frac{M}{R^2}=\frac{4\pi^2}{T^2}\cdot R,\qquad \frac{GM}{R^3}=\frac{4\pi^2}{T^2}\]
由此得
\[\frac{R^3}{T^2}=\frac{GM}{4\pi^2}\]

对某个行星的所有卫星来说,$M$恒定,是该行星的质量,$\dfrac{GM}{4\pi^2}$
是恒量,所以$R^3/T^2$也是一个恒量.
	\end{proof}
	
	\item 行星的密度是$\rho$,靠近行星表面的卫星运行周期是$T$.试证明$\rho T^2$是一个普遍适用的恒量,即它对任何行星都相同.

	\begin{proof}
球体积
$V=\frac{4}{3}\pi R^3$, 所以行星的密度
\[\rho=\frac{M}{\frac{4}{3}\pi R^3}\]
由$m\omega^2 R=G\dfrac{Mm}{R^2}$
得
\[\frac{M}{R^3}=\frac{4\pi^2}{GT^2}\]
将上式变形得
\[\frac{M}{\frac{4}{3}\pi R^3}=\frac{3\pi}{GT^2}\]
即
\[\rho T^2=\frac{3\pi}{GT^2}\]
由此得
\[\rho T^2=\frac{3\pi}{G}\]
由于$G$是普适恒量,所以$\rho T^2$对于任何一个行星都相同.
	\end{proof}
	
	\item 一艘宇宙飞船飞近某一个不知名的行星,并进入靠近该行星表面的圆形轨道,宇航员着手进行预定的考察工作.宇航员能不能仅仅用一只表通过测定时间来测定该行星的密度?说明理由.

	\begin{solution}
		根据题意,字航员可用表测定该飞船在行星表面附
		近绕行星运行一周所需要的时间$T$, 利用第5题的公式
		\[\rho=\frac{3\pi}{GT^2}\]
		将$T$值代入,即可算出密度.
	\end{solution}
	
	\item 不考虑地球的自转,求出用地球半径$R$、地面重力加速度$g$和引力恒量$G$表示的地球密度的公式.

	\begin{solution}
不考虑地球的自转,可认为物体重量就是地球对它
的引力,即
\[mg=G\frac{Mm}{R^2}\]
将$M=\frac{4}{3}\pi R^3\rho$代入上式得
\[g=\frac{4}{3}\pi\rho GR\]
则地球的密度
\[\rho=\frac{3g}{4\pi GR}\]
	\end{solution}
	
	\item 用火箭把宇航员送到月球上,如果他已知月球的半径,那么他用一个弹簧秤和一个已知质量的砝码,能否测出月球的质量?应该怎样测定?	

	\begin{solution}
只要已知月球半径,便可测出月球质量.具体步骤
如下:
\begin{enumerate}
\item 可以先将质量已知的砝码挂在弹簧秤上,测出读数
$W$, 由$W=mg_{\text{月}}$, 可求出$g_{\text{月}}=W/m$;
\item 然后由砝码重量等于月球对它的引力得
\[mg_{\text{月}}=G\frac{M_{\text{月}}m}{R^2_{\text{月}}},\qquad g_{\text{月}}=\frac{GM_{\text{月}}}{R^2_{\text{月}}}\]
故
\[M_{\text{月}}=\frac{g_{\text{月}}R^2_{\text{月}}}{G}=\frac{WR^2_{\text{月}}}{G_m}\]
月球半径$R_{\text{月}}$已知,$W$由弹簧测
出,$G$是普适恒量,$m$已知,因此可以计算出月球质量$M$.
\end{enumerate}


	\end{solution}
	

\end{enumerate}
	
	
\section{参考资料}
\subsection{太阳系中的最大和最小}

太的体积约是地球体积
的130万倍,地球的体积约是月球体积的50倍.太阳的质量
约是地球质量的33万倍,地球质量约是月球质量的81倍.下
表列出了太阳系八大行星及冥王星距太阳的平均距离、体积、质量、表
面平均重力加速度、平均密度、自转周期最大和最小的星球.

\begin{center}
\begin{tabular}{p{.45\textwidth}p{.25\textwidth}p{.2\textwidth}}
\hline
&  最小  &最大\\
\hline
距太阳平均距离(以地球与太阳平均距离为单位) & 水星(约为0.4)&冥王星(约39.5)\\
体积(以地球体积为单位) & 水星(约为0.056)&木星(约1316)\\
质量(以地球质量为单位) & 水星(约为0.055)&木星(约318)\\
表面平均重力加速度($\msq$)&  水星(3.6)&木星(约26)\\
平均密度(克/厘米)& 土星(0.7)&地球(5.5)\\
自转周期&木星(9时50分)& 金星(244.3日)\\
\hline
\end{tabular}
\end{center}

\subsection{开普勒定律}
开普勒研究所根据的资料都是凭肉眼
观测的.随着望远镜等精密仪器的出现,发现开普勒定律只
是近似的,行星实际运行的情况与开普勒定律有少许偏离.造
成这种情况的有以下两个原因:由于太阳也受到行星的吸引,
它也有加速度,并不是静止不动的,实际上太阳和行星都绕它
们的质心各自沿椭圆轨道运动,此时行星椭圆轨道半长轴(平
均半径)立方与运行周期平方之比已不再是常数,而应修正为
\[\frac{R_1^3}{T^2_1}\cdot \frac{R^3_2}{T^2_2}=\frac{M+m_1}{M+m_2}=\frac{1+\dfrac{m_1}{M}}{1+\dfrac{m_2}{M}}\]
式中的$R_1$和$R_2$分别是质量为
$m_1$
和$m_2$的行星轨道半长轴,$T_1$和$T_2$分别是它们的运行周
期,$M$是太阳的质量,实际上太阳系中质量最大的行星是木
星,它的质量是太阳质量的$1/1047$, 上式之比与1相差极微.
所以开普勒第三定律虽然只是近似的,但近似程度是相当高
的.以上结论只考虑了行星与太阳间的相互吸引,在理论力
学中称为二体问题,如果要考虑任一行星还受到其他行星的
吸引,则成为多体问题,此时只能用微扰法来近似求解.

\subsection{万有引力定律建立的历史进程}

在古代和中世纪,引
力被认为是位置的一种性质.亚里士多德认为“宇宙中的万
物都有它的指定位置,一旦脱离原位,就要回复回去”以此来
解释石头落地的问题.哥白尼设想太阳、月球和各个行星都有
自己的引力体系,地球上空的石头会落向最近的引力体系,即
落向地面.伽利略提出惯性概念时虽已意识到约束行星沿闭
合轨道需要力的作用,但没指出这力的性质.开普勒在探索
行星运动的规律时,也产生了寻求行星运动原因的思路,他认
为是太阳发出的磁力推动着行星的公转.

英国物理学家胡克提出了一切天体都具有倾向于其中心
的吸引力,它不但吸引其本身的各个部分,还吸引其作用范围
的其他天体.这就是行星绕太阳作椭圆运动的原因,他还提
出了这个引力反比于距离的平方,但他一直未能从理论上证
明这一点.

牛顿在1665—1666年间想到“把推动月球在轨道上运行
的力和地面上的重力加以比较”,可是由于在计算上遇到的困
难,他的研究迟迟没有进展.

1685年,牛顿从理论上解决了把太阳、月球、地球都当成
一个个质点的问题,采用了地球半径的新数据,证明地面上物
体坠落和月球沿闭合轨道运行是出于同一原因,并把这一结
论推广到所有的行星运动中去,从而提出了著名的万有引力
定律.

\subsection{卡文迪许}

亨利·卡文迪许(1731—1810)是近代著名
英国科学家,他一生从事大量的化学、电学实验,不疲倦地埋
头于实验研究工作达50年之久.大约在库仑确定著名的静
电学基本定律的同时,他独自发现并测得电荷间的作用力跟
距离平方成反比的规律,还独立提出了电势的概念,1798年,
已近垂暮之年的卡文迪许运用构思巧妙的精密“扭秤”实验技
巧测定出地球的平均密度为$5.481{\rm g/cm^3}$(现代公认值为
$5.517{\rm g/cm^3}$),由此可推算出万有引力常数是$6.754\x10^{11}{\rm N\cdot m^2/kg^3}$
(现代公认值为$6.668\pm 0.005\x10^{11}{\rm N\cdot m^2/kg^3}$).他被公认为是最伟大的实验科学家之一.英国科学家
坡印廷盛赞他“开创了测量弱力的新时代”.

\subsection{三种宇宙速度}


\subsubsection{第一宇宙速度} 课本中已经讲过了第一宇宙速度的
推导过程和数值,即$v_1=\sqrt{Rg}=7.9{\rm km/s}$.

\subsubsection{第二宇宙速度} 即物体能够脱离地球引力而不再回
到地球所需的最小发射速度,通常用$v_2$表示.

如果知道从地球上射出的物体至少需要有多大的动
能,才能够克服地球的引力逃到无限远处(即脱离地球的引力
范围),就可以求出$v_2$.要知道这个动能就必须求出物体从地
面移到无限远处反抗地球引力所做的功.

由于把物体移到无限远处的过程中,地球对物体的引力
是变化的,所以不能照恒力做功那样简单地用$W=Fs\cos\theta$
来求,而要用到积分,用$W$表示反抗地球引力从地面到无限
远处所做的功,则
\[W=\int^{\infty}_R \frac{GMm}{r^2}\dd r=\frac{GMm}{R} \]
由于在地面上
\[G\frac{Mm}{R^2}=mg,\qquad G\frac{M}{R^2}=g\]
所以
\[W=mgR\]

根据动能定理知道,物体需要具有的动能应该等于这个
功,即
\[\frac{1}{2}mv^2_2=mgR\]
所以
\[v_2=\sqrt{2Rg}\]
由于$\sqrt{Rg}=v_1=7.9{\rm km/s}$,所以
\[v_2=\sqrt{2}v_1=11.2{\rm km/s}\]

\subsubsection{第三宇宙速度}

使物体不但挣脱地球的引力,而目.
挣脱太阳的引力,逃到太阳系以外去,物体所需要的速度叫第
三宇宙速度,通常用$v_3$表示.

我们知道,地球以约$30{\rm km/s}$的速度绕太阳运动,地球
上的物体也随着地球以这个速度绕太阳运动.正象物体挣脱
地球引力所需的速度等于它绕地球运动的速度的2倍那
样,地球上的物体挣脱太阳引力所需的速度为$30{\rm km/s}$的
2倍,所以地球上的物体挣脱太阳引力所需要的速度为
$30\x\sqrt{2}=42.3{\rm km/s}$.

由于地球上的物体已经具有绕太阳运动的$30{\rm km/s}$的
速度,要使它相对于太阳的速度达到$42.3{\rm km/s}$,只要使它
在沿着地球运行方向增加$12.3{\rm km/s}$的速度就行了,但是,-
要使物体脱离太阳,首先要使它脱离地球,因此,除了给予物
体$\frac{1}{2}mv^2$($v$代表$12.3{\rm km/s}$的速度)的动能以外,还必须给
予物体
$\frac{1}{2}mv^2_2$($v_2$代表第二宇宙速度)的动能.也就是说,必
须给予物体的动能为
$\frac{1}{2}mv^2+\frac{1}{2}mv^2_2$.如果用$v_3$代表第三宇
宙速度,这个动能就等于$\frac{1}{2}mv^2_3$.所以
\[\frac{1}{2}mv^2_3=\frac{1}{2}mv^2+\frac{1}{2}mv^2_2\]
\[v_3=\sqrt{v^2+v^2_3}=\sqrt{12.3^2+11.2^2}=16.7{\rm km/s}\]

\subsection{不同高度上卫星的环绕速度}

\begin{center}
\begin{tabular}{ccccccccc}
\hline
高度& 0&300&500&1000&3000&5000&35900& 380000\\
(km) &&&&&&&(同步轨道)  &(月球轨道)\\
\hline
环绕速度 &7.91&7.73&7.62&7.36&6.53&5.29&2.77&0.97\\
(km/s)\\
周期&84.4&90.5&94.5&105&150&201&23h 56min&28d\\
(min)\\
\hline
\end{tabular}
\end{center}

\subsection{五个国家第一颗卫星比较}
\begin{center}
\begin{tabular}{cccccc}
\hline
&中国&苏联&美国&法国&日本\\
\hline
发射日期&1970.4.24&1957.10.4&1958.2.1&1965.11.26&1970.2.11\\
质量(kg)&173&83.6&13.97*&40&38*\\
\hline
\end{tabular}

* 包括最后一级运载火箭壳体.
\end{center}

\subsection{我国发射的十七颗卫星简况}

\begin{center}
	\begin{tabular}{ccp{.3\textwidth}p{.25\textwidth}}
		\hline
		名称	&	发射时间	&	工作情况	&	其他\\
		\hline
		人造地球卫星&	1970.4.24&	播送《东方红》乐曲\\
		科学实验卫星&		1971.3.3&		向地面播送科学实验数据\\
		人造地球卫星&		1975.7.26&		星上各种仪器工作正常\\
		人造地球卫星&1975.11.26&卫星各种系统工作正常&三天后,按计划返回地面\\
人造地球卫星&1975.12.16&卫星工作正常\\
人造地球卫星&1976.8.30&卫星工作正常\\
人造地球卫星&1976.12.7&卫星工作正常&按预定计划准确返回地面	\\
人造地球卫星&1978.1.26&卫星运行良好,完成了科学实验任务&按预定计划成功地返回地面\\
空间物理探测卫星&1981.9.20&各系统工作正常,不断向地面发送各种科学探测和试验数据&用一枚火箭发射三颗卫星\\
科学试验卫星&1982.9.20&卫星运行良好,仪器工作正常&运行五天后,按预定计划返回地面\\
科学试验卫星&1983.8.19&卫星运行良好,各系统工作正常&按预定计划准返回地面\\
试验卫星&
1984.1.29&取得了重要成果\\
试验通信卫星&1984.4.8&进入预定轨道,设备工作正常\\
科学实验卫星&1984.9.12&卫星运行和工作正常&按预定计划准确返回地面\\
实用通信广播卫星&1986.2.1&进入预定轨道,设备工作正常\\
		\hline
	\end{tabular}
\end{center}











\chapter{物体的平衡}
\section{教学要求}
这一章讲述静力学的基本知识,主要是讲在共点力作用
下物体的平衡和有固定转动轴的物体的平衡。

这一章的教学要求是:
\begin{enumerate}
\item 理解平衡和平衡条件的概念,掌握在共点力作用下物
体的平衡条件。
\item 理解力矩的概念,掌握有固定转动轴物体的平衡
条件。
\item 了解力偶和力偶矩的概念,知道力偶的作用是使物体
只发生转动。
\item 了解平衡的种类和稳度。
\end{enumerate}

下面对这一章的教学内容作些具体说明。

在动力学之后讲述静力学,有可能把静力学知识当成动
力学的特殊情况来理解,课本在实验的基础上得到共点力作
用下物体的平衡条件后,再从牛顿第二定律推导出这个平衡
分件,其目的就是要加深学生对平衡条件的理解。

讲述力对物体的转动作用,是为了讲解有固定转动轴的
物体的平衡做准备,这里讲述的转动的特点、转动快慢的描
述以及匀速转动和变速转动的概念,要求学生了解即可。关
于力矩,重点是要求学生理解力矩的概念,至于力矩的代数
和等于零或不等于零时物体怎样转动,使学生有个了解就可
以了。

关于有固定转动轴的物体的平衡条件的教学,还是以实
验为基础,但应注意实验与推理相配合,以加深学生的认识。

力偶是一个重要概念,而且以后讲通电线圈在磁场中运
动时会用到,因此应使学生对力偶知识有所了解。这里主要
是使学生明确知道力偶与一个力对物体的作用不同,--个力
可以使物体同时发生转动和平动,而力偶的作用是使物体只
发生转动而不发生平动。对有固定转动轴的物体来说,虽然
一个力的作用和力偶的作用都可以使物体发生转动,但效果
是有区别的,教材中说明这一点,是为了加强力偶的作用是
使物体只发生转动的认识。

关于平衡的种类和稳度的教学,都只要求作一般的介绍,
目的是扩展学生的知识面,把他们的静力学知识用来分析这
一常见现象。平衡的种类的区分,要让学生知道区分的标志
是以物体的平衡遭到破坏之后能否自行回到原来的平衡
位置。讲解稳度,要使学生知道在必要时增大稳度的方法。

理论联系实际,是物理教学的一条重要原则,静力学知
识在实际中很有意义,这一章的教学更应注意与生产和生活
实际的联系,所选习题力求有实际意义,而不要补充那些过难
而又缺少实际意义的题。

\section{教学建议}
本章教学建议分成四个单元,第一单元(全章引言和第
一节)主要讲述什么是物体的平衡状态和平衡条件,并得出
在共点力作用下物体的平衡条件。第二单元(第二、三节)
通过力对物体的转动作用的讨论,引入力矩这个重要概念,并
进一步得出具有固定转动轴的物体的平衡条件。第三单元
(第四节)讲述力偶的初步知识。第四单元(第五、六节)是对
物体平衡状态的进一步分析,说明物体平衡还有个稳定不稳
定的问题,稳定平衡还有个稳定程度的问题。

\subsection{第一单元}
本单元教学应该以实验为基础,做好三个共点力平衡的
演示实验。但同时要注意实验与推理相结合,这里主要是两
方面的推理:一是将三个共点力的平衡条件推广到三个以上
共点力作用下的平衡条件。虽然教材说“用实验还可以证明”,
但教学时一般不必再做三个以上共点力平衡的实验了。另一
个是从牛顿第二定律推出共点力作用下物体的平衡条件。

\subsubsection{对平衡状态的理解}

在全章引言中,教材明确指出:
“如果一个物体既不做平动,也不做转动,即保持静止,或者
做匀速直线运动或匀速转动,我们就说这个物体处于平衡状
态。”这是平衡状态的定义。比起以前教材中关于物体平衡的
说法,含义更广了。以前所说的平衡,仅指物体处于静止或匀
速直线运动状态。而这里的定义说,匀速转动状态也是平衡
状态。为什么说匀速转动也是平衡状态呢?因为它与静止和
匀速直线运动有共同之处,即运动状态保持不变。前者是物
体的即时速度保持不变(加速度为零),后者是物体转动的角
速度保持不变,这样,既使学生对平衡状态概念的理解进一步
深化,而且为利用牛顿第二定律推导出共点力平衡条件,也为
以后推导有固定转动轴的物体的平衡条件,作了准备,教材
中把物体的平衡安排在牛顿运动定律之后讲述,也为加深对
平衡这一概念的理解提供了条件。

\subsubsection{什么是共点力}

在教学中可以提一提,所谓共点力并
不一定是几个力都作用在同一点上,还应包括几个力的作用
线相交于一点的情况。因为作用在物体上的力沿着力的作用
线平移,其作用效果是相同的。此外还应注意,我们在这里所
讨论的共点力,仅限于在同一平面上的共点力,有的书上称为
共面力。如果不是共面力,情况就要复杂得多。

\subsubsection{三力平衡实验}

证明三个共点力作用下的平衡条件
是合力为零的实验并不难做,但要得出平衡条件则必须把实
验做得尽量准确一点.建议在教学中注意两点:
\begin{enumerate}
\item 课本图
6.1甲中的整个装置应是水平放置的.如果竖直放置,则可
能因下面一个弹簧倒置而产生较大误差。    
\item 如果教师演示
时,不便将装置水平放置,则可以将下面那个弹簧秤牵拉的力
改为挂上砝码,当然,受力物体应该选择重量很小的物体,或
者就可以把绳子的结点作为受力平衡的研究对象。
\end{enumerate}

\subsubsection{平衡条件的应用}

共点力的平衡条件$F_{\text{合}}=0$原是一
个矢量方程,由于教材不介绍正交分解法,所以必须设法将
平衡条件简化,例如本章后习题1的解答中,先将$F_1$和$F_2$
合成为$F'$, 接着就把方程$F_{\text{合}}=0$简化为$F'$与$F$大小相等、方
向相反的结论,再通过几何关系分别求出$F_1$和$F_2$. 如何将
平衡条件$F_{\text{合}}=0$简化成既符合题意,又便于研究的形式,是学
生学习静力学时常常感到困难的问题,教学中要有意识加以
指导。

\subsubsection{静力学问题宜用平衡条件解}

学生在第一章学习力
的分解时,实际也解过类似本章后习题1的题目.所以学生
可能习惯于用力的分解来解题。在比较简单的情况下,从力
的作用效果出发,用力的分解来解题也是可以的,例如本章后
第1题,但如果题目复杂一些,用力的分解来解则可能发生
差错。因为有时某个力在某方向上的分力是没有实际意义的。
所以学了这一单元以后,建议学生以后可尽量用平衡条件来
解静力学问题。


\subsection{第二单元}
本单元讲述力对物体的转动作用、力矩的概念和有固定
转动轴物体的平衡条件。
\subsubsection{物体转动与物体平动的对应关系}

物体转动与质点
运动虽然是两种不同的运动,但它们的许多概念和规律都有
类似的关系,可在适当的时候进行对照比较。例如课本中所
说的,平动物体上各点的速度都相同,任何一点的速度可以
代表整个物体的速度。同样,转动物体上各点的角速度都相
同,任何一点的角速度都可以代表整个物体的角速度。这样,
平动物体的速度与转动物体的角速度相对应,另外还有,平动
的位移和转动时转过的角度(称为角位移)相对应,匀速直线
运动与匀速转动相对应,共点力的平衡条件与力矩平衡条件
相对应,等等。教学中教师可以启发学生进行这样的对比,有
利于学生对转动知识的理解。

\subsubsection{力矩的概念和计算}
学习力矩的概念,关键在于掌握
力臂的概念。关于力臂的意义和计算,虽然在初中学习杠杆
平衡条件时已经反复练习过,但在高中阶段学生还是常常容
易搞错。所以这里仍要多举几个例子让学生学会确定力臂的
长度。在此基础上,高中还要学生掌握力矩的单位和正负。
力矩的单位是${\rm N\cdot m}$,不是${\rm J}$,要求学生不要同功的单位混
淆。力矩有正负,但在教学中不要提力矩的方向。因为力矩
方向的规定涉及矢量乘法,已超出高中物理范围。

\subsubsection{关于有固定转动轴物体平衡的实验和推理}

通过实验
和推理得出有固定转动轴物体的平衡条件是本单元的重点。
课本是先通过推理得出平衡条件,然后用实验来验证的。教
学中如果教师采用先实验再说理的办法也是可以的。经过实
验,学生对力臂的计算和有固定转动轴物体的平衡条件将有
较深刻的理解.在将结论$M_1+M_2=M_3$改写成$M_{\text{合}}=0$的过
程中,教师还得作一些解释。前者等号两侧都是绝对值,等号
左侧为使物体向顺时针方向转动的力矩,右侧为使物体向逆
时针方向转动的力矩。后者$M_{\text{合}}$中包括了正负两种力矩。写
成$M_{\text{合}}=0$的形式便于同共点力平衡条件$F_{\text{合}}=0$对照.$F_{\text{合}}=0$
为一矢量方程,而$M_{\text{合}}=0$中的力矩不是正,就是负,是个代数
方程.因此在运用上$M_{\text{合}}=0$要比$F_{\text{合}}=0$容易掌握一些。

\subsubsection{关于固定转动轴的理解}

课本在对固定转动轴的理
解上作了一些扩展,即不限于研究确有实际的固定转轴的情
况,而把在研究问题时该物体绕某线转动的那条线看作是固
定转动轴,本章末习题第4题和第9题就是这样,解这类
题可以培养学生灵活应用所学知识的能力,但课本对一般平
面力作用下物体平衡问题,即须同时使用平动平衡条件和转
动平衡条件的问题一概不作要求,教师应加以注意。

\subsubsection{三角支架问题在什么情况下要用力矩平衡来解}

在支架本身重量可以忽略不计的情况下,三角支架问题可以用
共点力平衡条件来解,也可以用具有固定转动轴的物体的平
衡条件来解,但当支架本身重量不能忽略时,如本章后习题
第3题,则必须以横梁$BO$为研究对象,以$B$为转动轴,利用
力矩平衡条件来解.如果此时我们仍以$O$点为研究对象,$O$
点除受钢绳$AO$的拉力$T$和重物的拉力$F$以外,一定还有一
个$BO$对$O$点的支持力$N$. $T$、$F$、$N$三个力是共点的,$O$点又处
于静止状态,因此这三个力也一定满足$F_{\text{合}}=0$这个条件,但
是,当$BO$梁的自重不能忽略时,它对$O$点的支撑力$N$不再
沿着$BO$的方向.$N$的方向未定,用$F_{\text{合}}=0$来解,条件就不充
足了,如果我们用有固定转动轴的平衡条件先求出$T$的大小,
然后再利用共点力平衡条件可以求得$BO$对$O$点的支撑力
$N$, 很容易看出这个力不在$BO$的方向上。对程度较好的学
生,启发他们作以上的分析,对培养学生分析问题的能力很有
好处。


\subsection{第三单元}
这一单元介绍力偶这个概念,教学中应突出两个主要问
题:第一是使学生掌握力偶、力偶臂、力偶矩的意义,力偶矩大
小的计算和正负的规定;第二是要使学生明确,力偶的作用
是使物体只发生转动而不发生平动。
\subsubsection{关于力偶的作用}

力偶的作用是使物体只发生转动
而不发生平动,这个结论是通过力偶对物体的作用和一个力
对物体的作用的对比实验来使学生认识的.课本中图6.11和
图6.12的实验可利用书写幻灯进行投影增大可见度,被拉
动的物体可用一块圆形的有机玻璃板,上面画一些辐条,这
样在书写幻灯的投影下,可以清楚地看到平动和转动。课本
第三段说到用一只手板套筒容易磨损螺纹,用两只手板套筒,
轴上就不会受到压力,目的也是要突出力偶只使物体发生转
动.这一点,也可以利用课本图6.11和图6.12的装置加以
说明。具体做法见本章实验指导。

\subsubsection{力偶矩也是力矩}

教学中要使学生明白,力偶矩实际
上就是力偶的两个力的力矩之和。力偶矩和力矩的作用都是
使物体的转动状态发生变化,因此,不要把力偶矩与力矩这
两个概念孤立开来。教学中,要通过力矩代数和的计算推导出
力偶矩$M=Fd$的关系式。通过推导,还要启发学生理解力
偶矩的大小不随所取的转动轴位置的改变而改变。至于在力
偶的作用下物体究竟以哪一点为轴转动的问题,课本没有提,
教学中也应回避。


\subsection{第四单元}
本单元对物体平衡的问题作进一步的分析,目的是扩展
学生的知识面,把物理知识与实际联系得更紧密。教师在教
学过程中,应多举些实例,多做些演示,以加深对教材的理解,
培养学生观察和思考问题的能力。

本单元的重点是要学会怎样分析稳定平衡、不稳平衡和
随遇平衡,而不是光记忆这些结论,对有支点的物体来说,是
根据它偏离平衡位置后,重力和支持力的合力是否能
使物体回到平衡位置来判断的;对有支轴的物体来说,是
根据它偏离平衡位置后,合力矩能否使物体回到平衡位置来
判断的。或者两者都用重心的升高或降低来判断。教师在分
析各种实例时,都要从这个角度来分析。

在所举的实例和演示中,既要注意简単叨了,便于分析,
联系实际,也要注意生动活泼,活跃课堂气氛,例如不倒翁的
例子就比较生动。此外,杂技团演员走钢丝、顶碗等实例都有
个平衡和稳定问题。有条件的学校,可以放映一些科技电影,
如“杂技团的秘密”等。

\section{实验指导}
\subsection{演示实验}
\subsubsection{共点力作用下物体的平衡条件}
课本图6.1所示的演示实验是在水平面上做的,可
以通过投影仪显示,物体可用
有机玻璃做,这样,可以看到作
用点。

\begin{figure}[htp]
    \centering
   % \includegraphics[scale=.7]{fig/6-1.png}
    \caption{}
\end{figure}

还可以利用两个弹簧
秤和一组钩码来进行演示,如
图6.1所示,要使学生认识这
个实验中的研究对象是绳的结
点$O$(也可以在$O$点固定一块用泡沫塑料锯成任意形状的薄
板作为象征性的物体),$O$点受三个共点力的作用而保持平
衡。如果调节好两个弹簧秤的位置,使得$F_1=120$克力.
$F_2=90$克力,$F_1$和$F_2$的夹角为$90^{\circ}$, 所用的钩码的总重量为150
克力。就可以方便地得出在共点力作用下,物体的平衡条件
是合力等于零。

如图6.2所示,一辆小车放在光滑斜面上,小车一端
通过细绳和弹簧秤相连,弹簧秤的另一端固定在斜面顶端,当
小车静止时,小车受到三个力而平衡,使斜面倾角改变,可以
看到弹簧秤的示数也发生相应的变化,达到某一数值时,小车
又处于平衡状态。

\begin{figure}[htp]
    \centering
   % \includegraphics[scale=.7]{fig/6-2.png}
    \caption{}
\end{figure}

\subsubsection{力矩的作用}

\begin{figure}[htp]
    \centering
   % \includegraphics[scale=.7]{fig/6-3.png}
    \caption{}
\end{figure}

可以利用教室的门做如图6.3所示的演示,把橡皮绳的
一端系在门把手上,另一端捏在手中,先将门关上,沿着垂
直于门面的方向拉动橡皮绳,观察到橡皮绳稍有伸长,门就
能被拉开(如图中橡皮绳的拉力为$F_1$);然后重新把门关上,
改变拉像皮绳的方向,使得拉力仍沿水平方向但不与门面
垂直。这时可观察到橡皮绳必须拉得更长,才能使门拉开(如
图中橡皮绳的拉力增大为$F_2$)。启发学生思考,为什么会出
现这种现象?说明什么问题?然后演示橡皮绳在水平方向沿着
门面拉动,直到橡皮绳被拉断,还是拉不开门,让学生思考这
又是为什么?

\subsubsection{有固定转动轴的物体的平衡条件}
正力矩和负力矩的规定:
根据课本图6.8, 用力矩盘进行演示,指出力$F_1$和$F_2$所产
生的力矩是负的,它们的作用效果是使有固定转动轴的物体
向顺时针方向转动、力$F_3$产生的力矩是正的,它的作用效果
是使物体向反时针方向转动。

仍利用课本图6.8所示的力矩盘进行演示,并可改
变细绳下端所挂钩码的个数,改变作用点的位置。当力矩盘
平衡时,从各个力和力臂的大小可以得出:有固定转动轴的物
体的平衡条件是力矩的代数和等于零。

\subsubsection{应用有固定转动轴的物体的平衡条件解题}
课本213页习题第4题,可用模拟演示来加以说
明,如图6.4所示,将一根粗细
不均匀的木棍(如教棒),平放
在讲台上,在棍的两端各系一个细绳套,用弹簧秤先后勾住
棍的细端和粗端的绳套,稍稍提起,使该端脱离桌面,分别读
得弹簧秤的示数为$F_1$和$F_2$. 再用弹簧秤勾住绳套,把木棍整个
提起来,读出弹簧秤示数为$F$。从实验结果可知:$F=F_1+F_2$


\begin{figure}[htp]
    \centering
   % \includegraphics[scale=.7]{fig/6-4.png}
    \caption{}
\end{figure}

课本213页第9题可用铁架台及顶部装有定滑轮
的木杆制成模型进行演示(图6.5)。


\begin{figure}[htp]
    \centering
   % \includegraphics[scale=.7]{fig/6-5.png}
    \caption{}
\end{figure}

当在定滑轮下所挂的重物
$G$的重量过大时,铁架台将向
右倾倒,可以减小重物$G$的重
量或缩短拉紧撑杆的细绳长
度,使得撑杆的倾角变大,直到
铁架台恰巧不倾倒。应该指出,
在这个模拟演示中是将整个铁
架台连同撑杆和所挂重物看成
一个物体来进行研究的,课本213页第9题的起重机也应
该看作是一个整体,各个力对前轮$O$所产生的力矩代数和应
等于零,来求出起重机至多能提起多重的物体。

\begin{figure}[htp]
    \centering
   % \includegraphics[scale=.7]{fig/6-6.png}
    \caption{}
\end{figure}

如图6.6所示,将均
匀米尺的一部分伸出水平桌面
外,在米尺伸出部分的顶端放
一小砝码(10克或20克),调节
米尺伸出部分的长度,使得米尺仅对桌边有压力,这时就可以
把米尺跟桌边接触的地方看成是固定转动轴。只要读出伸出
部分的米尺长度,计算从均米尺的重心位置到桌边的距离,
从已知砝码的重量,根据有固定转动轴物体的平衡条件,即可
求出米尺的重量。

\subsubsection{力偶}
课本图6.11和图6.12的演示可以通过投影仪来显
示,圆盘可以用厚一些的有机玻璃板来做,在圆盘的侧边车制
一凹槽,以便绕线。为了使效果明显,在有机玻璃圆盘上沿着
半径方向可用透明漆画几条有色条纹。

\begin{figure}[htp]
    \centering
   % \includegraphics[scale=.7]{fig/6-7.png}
    \caption{}
\end{figure}

如图6.7所示,在铁架台上通过复夹和试管夹安装
一块水平放置的中央有孔的有
机玻璃板,在平板上放一个准
备好的有机玻璃圆盘。在盘心
位置开一个圆孔,将一细竹针
穿过圆孔,并用复夹使竹针的
两端固定。然后在圆盘的一侧
通过缠绕的细线拉圆盘,可以
观察到圆盘转动的同时,竹针弯曲,这说明力矩的作用可以使有固定转动轴的物体发生转
动,但转动轴是受力作用的。如果不存在固定转动轴(将竹针
抽去),在力矩的作用下圆盘将同时发生转动和平动。如果对
圆盘作用一个力偶则可观察到圆盘转动时竹针并不弯曲,说
明竹针并不受力。因此即使抽去竹针,在力偶作用下,圆盘也
只发生转动,不发生平动。

\subsubsection{物体平衡的种类}
课本图6.16有支点的物体的平衡的演示,可利用小球放
在离心轨道(间距小于小球直径的两根平行铁丝)上来进行演
示.课本图6.17有支轴的物体的平衡,可用一均匀薄木板
(厚度约为2—3mm)来演示,中间的小孔要开在薄木板的重
心上,孔内侧要粗糙些,这样,演示随遇平衡的效果会好些。


\subsubsection{稳度跟重心的高低和支面的大小有关}
可制成一个如图6.8所示的形状可变的框架(稳度演示
器)来进行演示,当把它由长
方体改变成斜方体时,只要系
在它的中心(表示重心位置)的
重垂线不超出支面,则斜方体
就不会倾倒。
如果重垂线超出
支面,斜方体就会倾倒。

\begin{figure}[htp]
    \centering
   % \includegraphics[scale=.7]{fig/6-8.png}
    \caption{}
\end{figure}

\subsection{学生实验}
\subsubsection{研究有固定转动轴物体的平衡条件}

实验时,先用胶纸在力矩盘上粘贴一张白纸,用手指
隔着白纸在转动轴部位按一下,在纸上留下转动轴的痕迹,然
后按课本图10.15所示的装置把力矩盘装好.要注意转动轴
应在承平方向,使力矩盘位于竖直平面内。在盘上任意选择四
个位置,各插一根大头针(要插深些),再按课本的要求(课本图10.15)在三根针上用细线悬挂钩码,悬挂细线时要造当
靠近大头针的根部,但又不要使细线和盘面发生摩擦。在第
四根针上用细线钩在弹簧秤的钩上,要注意调节固定在横杆
上的弹簧秤的位置,使得力矩盘平衡时,弹簧秤的拉线不要通
过转动轴。

当力矩盘在这四个力作用下处于平衡状态时,记下
这四个力的大小。用削细的铅笔沿着四根细线的方向,在离开
大头针较远的地方,分别画上一个“$\x$”号,并在悬挂这些细线
的大头针的针孔周围画一小圆,以便确定这四根拉力的方向。

取下钩码和弹簧秤,拔去大头针,取下白纸;用直尺
将做过记号的针孔和相关的“$\x$”号用虚线连接起来,并根据
自己选定的标度(要在记录纸上明确标出),按力的图示法分
别画出这四个拉力$F_1$、$F_2$、$F_3$和$F_4$.

根据事先在白纸上所做的记号,画出固定转动轴$O$,
然后分别作出从$O$点到四个拉力作用线的垂线,并用毫米刻
度尺量出这些垂线的长度,这就是力臂,在原始记录纸上标出
各个力臂$L_1$、$L_2$、$L_3$和$L_4$.

这个实验中也可以不
用横杆来固定弹簧秤,而用另
一个铁架台,通过复夹用试管
夹把弹簧秤背面的铁壳夹紧
(图6.9),这样做的好处在于可
以方便地调节弹簧拉长的方
向,使它跟细线的方向一致。

启发学生讨论课本最后提出的问题,认识到这是为
了便于使力矩盘平衡。因为弹簧秤在它的量程范围内,拉力的。
大小可以连续变化,这样就可以自行改变弹簧秤拉力的大小,
使得力矩盘平衡。

\subsection{课外实验活动}
\subsubsection{制作杆秤}
在练习制作时,可注意并思考以下几点:
\begin{enumerate}
\item 制作杆秤所用的细木棍要挑比较结实一些的材料,但
不要太粗,如果找不到合适的材料,找一根较长的竹筷或木筷
也可以。提纽要用强度大一些的细绳,秤钩可以用粗铁丝制
成,(也可以用几股细铁丝绞起来做),秤锤用质量小于1千克
的物体也可以。
\item 在秤钩不挂物体的情况下,把秤锤挂在秤杆上确定
秤杆的零刻度(课本图6.26的$A$点)时,要注意秤杆
和秤钩(看成一体)的重心在提纽$O$点的哪一侧?
上
\item 证明秤杆上刻度间的距离为什么是均匀的。可以结
合课本214页第10题的计算进行讨论.
\item 要增大杆秤的称量范围,可以再装一个提纽(称做二
纽),想想看这个二纽的位置应该离秤钩远一些还是近一些?
\end{enumerate}









































\section{参考资料}
\subsection{静力学中物体受力的方向}
可以在空间作任意运动的物体叫自由体,如飞行中的飞
机,发射出去的炮弹等都是自由体。因受到周围物体阻碍、限
制而不能作任意运动的物体称为非自由体,限制非自由体运
动的其他物体,称为该非自由体的约束体 例如,摆动着的单
摆,摆绳是摆球的约束体,它限制摆球只能在不大于绳长的范
围内运动,而通常是在以绳长为半径的圆弧上运动。

促使物体运动(或有运动趋势)的力称为主动力 非自由
体在主动力作用下,将向某一方向运动,此如果受到约束体
的限制,非自由体将给约束体一个作用力,同时约束体给非自
由体一个反作用力。这个反作用力称为约束反力 因为约束
反力是限制物体运动的,所以它的作用点应在约束体与非自
由体相互连接或接触处。它的方向应与约束体所限制的运动
方向相反,这是确定约束反力方向和作用点的基本依据。

工程中常见的约束有以下几种基本类型,下面着重说明
约束反力的方向。

\subsubsection{柔性体约束}

由绳索、链条、胶带等形成的约束是柔
性体约束,这些约束体的特点是只能承受拉力,不能承受压力
和抗拒弯曲,因而只能限制物体沿着柔性体伸长的方向运动。
。所以柔性体的约束反力只能是拉力,作用在连接点(或假想的
截断处),方向沿着柔性体的轴线而背离物体。课本中悬挂重
物的绳、牵引悬臂的钢素,对物体的作用力都属于这种情况。

\subsubsection{光滑接触面的约束}

在忽略摩擦,把接触表面作为
理想的光滑面的情况下,无论支承接触表面的形状如何,约束
体只能限制非自由体沿接触处公法线向接触体内的运动,不
能限制非自由体向任何其他方向的运动。所以光滑接触面的
约束反力只能是压力,作用在接触处,方向沿着接触表面在接
触处的公法线而指向物体。

\subsubsection{光滑圆柱形铰链约束}

圆柱形铰链是连接两个构
件的圆柱形零件,普通称为销钉 门窗上的合页,机器上的轴
承都属于圆柱形铰链。铰链约束只限制两个非自由体的相对
移动,而不能限制它们的相对转动。课本中塔式起重机的悬
臂与塔身的连接,三角形支架两支杆与墙壁的连接都作为铰
链连接处理,属于铰链约束,这种情况下的约束反力在垂直于
圆柱销轴线的平面内,通过圆柱销中心,方向(是压力或拉力)
视具体情况而定。

\subsection{力偶的性质}
力偶是两个具有特殊关系的力的组合,虽然力偶中的每
个力仍具有一般的力的性质,但在作为一个整体考虑它们对
刚体的作用时,则出现了与单个力不同的性质:
\begin{enumerate}
\item 力偶既没有合力,本身又不平衡,是一个基本的力
学量。
\item 力偶对于作用面内任一点之矩与矩心位置无关,恒
等于力偶矩。因此力偶对刚体的效应用力偶矩量度。
\item 作用在同一平面内的两个力偶,若其力偶矩大小相
等,转向相同,则该两个力偶彼此等效。这就是平面力偶的等
效定理。
\end{enumerate}

由等效定理可以得出两个推论:
\begin{enumerate}
\item 力偶可以在其作用面内任意转移,而不影响它对刚
体的效应。
\item 只要力偶矩大小和转向不变,可以任意改变力偶中
力的大小和相应地改变力偶臂的长短,而不影响它对刚体的
效应。
\end{enumerate}







\chapter{机械能}\minitoc[n]
\section{教学要求}
通过本章的学习,要使学生认识到:各种形式的能可以
相互转化,而且在转化过程中保持守恒;功是能的转化的
量度。

能的转化和守恒,是贯穿全部物理学的基本规律之一。解
决力学问题,从能量的观点人手分析,往往是很方便的。因此,
在本章的教学中,要特别注意培养学生从能量的观点来分析
问题。

这一章的教学要求是:
\begin{enumerate}
\item 理解功的概念,掌握功的计算公式,了解正负功的
意义。
\item 理解功率的概念,掌握功率的公式$P=Fv$.
\item 理解动能的概念,掌握动能的公式,掌握动能定理,会
运用动能定理解决力学问题。
\item 理解重力势能的概念,掌握重力势能的公式,掌握重
力做功与重力势能变化的关系。了解重力做功与路径无关是
引入重力势能的前提。
\item 了解弹性势能的概念,知道弹力做功与弹性势能
方关。
\item 掌握机械能守恒定律,会运用这个定律解决力学
问题。
\item 理解功和能的关系,知道功是能的转化的量度,学习
用能的观点来分析决力学问题。
\end{enumerate}


下面对这一章的教学内容作些具体说明。

为了给讲解动能和重力势能做好准备,教材安排了“能
量”一节,让学生对什么是能量以及功和能的关系有一个概括
的了解。什么是能量?许多课本中常常引用的“能是物体做功
的本领”,这个定义,严格说来有些不妥。从状态函数的角度
给出能量的普遍定义,中学生又难于接受 因此,教材沿用初
中物理的提法:一个物体能够做功,就说这个物体具有能量。
这并不是给能量下定义,只是使学生初步认识能量。

动能的公式实际上是动能定理的特殊情形,可以把动能
和动能定理合在一起来讲,直接推导出动能定理,同时给出动
能的定量表示。这样的讲法看起来简洁,但对初学者来说知识
太密集了 从便于接受的角度来考虑,还是把动能和动能定
理分作两节来讲为宜。

严格地说,应在重力做功与路径无关的基础上得出重力
势能的概念,这样讲,学生接受起来有一定的困难,因此教
材的讲法是,先通过克服重力做功引入重力势能,然后说明重
力做功与路径无关,正因为这样,才能引入重力势能的概念
这一点不能要求过高,使学生大体上有个印象就可以了,讲
解重力势能的相对性时,要使学生明确知道,参考平面的选择
可视研究问题的方便而定。重力势能的正负,使学生理解它
的意义即可。负的重力势能在力学中基本不用,在电学中要
讲到负的电势能。

关于势能属于系统,也不要求细讲;以后本章提到重力
势能时,仍认为是物体所具有的,因为采取这样的讲法,学生
容易理解,可以避免因过分严格而造成学习上的困难。但初
步了解势能属于系统,对今后学习内能有好处。

机械能守恒是指机械能在转化中守恒,因此在讲守恒定
律之前,应让学生对机械能的相互转化获得深刻的印象。这
里,要求学生对功这个概念有进一步的理解。在本章第一节
给出功的概念及其量度;第三节把功和能这两个概念联系起
来,使学生知道做功的过程总伴随着能量的改变,而且做多
少功,能量就改变多少,但未提及能的转化;这一节讲了机械
能的相互转化,并把做功和能的转化联系起来,使学生对功的
概念的理解进一步加深。

关于机械能守恒定律,在证明中,先得到结论,重力做多
少功,就有多少重力势能转化成等量的动能,然后移项得出机
械能守恒定律的表达式。这样处理,目的是强调等量转化,以
期学生对机械能守恒的理解能够具体些;同时有助于在本章
最后概括出功是能量转化的量度这一结论。教材是就自由落
体这个具体例子来证明机械能守恒定律的,没有作一般性的
证明。但是要求学生明确知道,在只有重力做功的情形下,不
论物体做直线运动还是曲线运动,守恒定律都成立。由于弹性
势能不作定量讨量,因此在守恒定律的表达中没有提到弹力
做功:面是在定律的表达之后提到在有弹性势能参加的相互
转化中,如果只有弹力做功,机械能也保持守恒。

第十节通过例题讲解如何应用机械能守恒定律时,应着
重说明以下几点。第一,说明用守恒定律求得的答案与用动
力学和运动学求得的答案相同,使学生确信守恒定律的有效
性。第二,说明运用守恒定律只涉及起始和终了状态,不涉
及中间过程的细节,因此用它来处理问题相当简便。第三,说
明有的问题只用守恒定律还不能完全解决,还需要用其他知
识,希望学生体会这一点,培养自己综合运用知识的能力。第
四,强调从能量观点分析问题的重要性,本节最后说明寻求
“守恒量”是物理研究工作的一个重要方面,希望学生能对守
恒定律的重要有所了解。

最后一节功和能的教学,主要是在前面的知识基础上,进
一步明确其他力做机械功的过程实际上是机械能与其他形式
的能相互转化的过程,而且做了多少机械功,机械能就改变多
少。最后得出结论:能量在转化中保持守恒,功是能量转化
的量度。

为了简明易懂,在最后一节的讲述中没有涉及弹性势能,
其他力是指重力以外的其他力。用绳子拉物体,绳的拉力属
于弹力,但作为外力来处理。这类问题不要引导学生去研究,
它已超出了本书的要求。

\section{教学建议}
本章是从做功和能量变化的角度,来研究物体在力的作
用下运动状态的改变的。为此,引入功、功率、动能、势能等概
念,介绍了动能定理和机械能守恒定律两条规律。这些概念
和规律在物理学上占有重要地位。

这一章可分两个单元。第一单元(第一、二节)讲述功和
功率这两个概念,第二单元(第三至十一节)讲解能的概念以
及不同情况下功和能的关系。本章教学的重点是使学生掌握
动能定理和机械能守恒定律这两条重要的物理规律。

\subsection{第一单元}
\subsubsection{功的概念的建立}

在物理学里,许多概念一建立起来
就能体会它明确的物理意义,如速度是用来反映运动快慢的
物理量,加速度是用来反映速度变化快慢的物理量等。但功这
个概念不是这样 教材先给功下一个明确的定义,然后要在
研究功和能的关系时才能逐步体会到功是物体能量转化的量
度,这一认识,要逐步渗透并贯穿于整章教学的过程中。因
此,在第一单元的教学中,首先是要让学生准确掌握功的定义
和计算 对功的意义的理解,不要急于求成,要在整章教学中
一步-步地引导,使学生逐步理解。

在介绍功的计算公式,研究力的方向与运动方向成$\alpha$角
的情况时,教材是将力$F$分解成两个分力$F_1$和$F_2$, 然后计算
出分力的功,从而得出$W=Fs\cos\alpha$的。这一分析过程包含
了分力所做的功的总和等于合力所做的功这一重要思路。教
师在教学中可适当地启发学生思考和体会这一点,以便在以
后的学习中应用。

得出公式$W=Fs\cos\alpha$后,应该通过对不同的$\alpha$角度的
讨论,理解什么情况下力$F$做功,什么情况下力$F$不做功,从
而与前面所说的功的两个不可缺少的因素相呼应。要使学生
明白,只要夹角$\alpha$不等于$90^{\circ}$, 力对物体就做了功.

对程度较好的学生,还可以进一步分析一下什么叫“物体
在力的方向上的位移”,如图7.1所示,$s\cos\alpha$就是物体在力
方向上的位移,也就是位移$s$在力的方向上的投影,用它乘
上力$F$, 也可以得到$W=Fs\cos\alpha$.
\begin{figure}[htp]
    \centering
\begin{tikzpicture}[>=latex]
\fill[pattern=north east lines](-1.5,-.2) rectangle (5.5,0);
\draw(-1.5,0)--(5.5,0);
\draw(-.7,0) rectangle (.7,1);
\draw[dashed](-.7+4,0) rectangle (.7+4,1);
\draw[dashed](0,.5)--(4,.5)--+(120:2);
\draw(0,.5)--(0,-1); \draw(0+4,.5)--(0+4,-1);
\draw[<->](0,-.75)--node[fill=white]{$s$}(4,-.75);
\draw[->, thick](0,.5)--+(30:1.5)node[below]{$F$};
\draw(.5,.5) arc (0:30:.5)node[right]{$\alpha$};
\draw[dashed](0,.5)--+(30:5);
\draw[decorate,decoration={brace,raise=5pt}] (0,.5) --node[above=10pt]{$s\cos\alpha$}+ (30:2*1.732);
\end{tikzpicture}
    \caption{}
\end{figure}

\subsubsection{正功和负功}

正确地理解正功和负功对以后的学习
很重要。由于公式$W=Fs\cos\alpha$ 中$F$和$s$均为绝对值,所以$W$
的正负完全决定于$\cos\alpha$ 的正负,要使学生体会到,$\cos\alpha >0$($F$
与$s$的夹角为锐角)意味着力$F$对物体产生位移$s$有一定贡
献,可以理解为力$F$对物体确实做了功.而当$\cos\alpha <0$时,$F
$与$s$的夹角为钝角,力$F$对物体产生位移$s$实际上起了阻碍
作用,所做的是负功。这时,物体要继续产生位移,必须克服
力$F$的阻碍,所以力$F$对物体做负功,又可以表达为物体克服
力$F$做功(取正值)。学了动能定理之后,学生就能进一步从
能量改变的角度体会正功和负功的意义。

\subsubsection{汽车牵引力与速度的关系}

教材推导了公式$P=Fv$。
要提醒学生注意,这是在力的方向与位移方向相同的情况下
推出来的,一般的实际问题都属于这种情况。

利用公式$P=Fv$来讨论牵引力与速度的关系,是以汽车
的输出功率一定为前提的,当然,汽车开动时实际功率不一定
保持不变,驾驶员可以用控制混合气体流量的办法来控制功
率,然后再用换档的办法来调节速度,从而改变牵引力的大
小,因此,公式$P=Fv$是有实际意义的。

为了使学生熟悉公式$P=Fv$的应用,课本安排了一个例
题。在讲解这个例题时,如何分析卡车由开始到匀速行驶的过
程是很重要的,它可以培养学生养成分析物理过程的习惯,避
免简单地套用公式。这一段分析实际上包含了三个内容,一是
在功率一定的条件下,利用公式$P=Fv$讨论$F$与$v$的关系;第
二是在不同$v$的情况下讨论$F$与阻力$f$的合力如何变化;第
三是当$F_{\text{合}}$减小时,加速度减小,但速度继续增大,当$F_{\text{合}}$减小
到零时,加速度为零,速度不再增加。此时的速度就是汽车在
上定功率下的最大速度。教学
时把这三个内容分析清楚了,
学生就比较容易理解了,为了
教学方便起见,在分析这一问
题时,建议教师在黑板上画出示意图,如图7.2所示.
\begin{figure}[htp]
    \centering
    \includegraphics[scale=.8]{fig/7-2.png}
    \caption{}
\end{figure}


\subsubsection{第二单元}
这一单元从第三节一直到第十一节,内容很多、第三节
相当于这一单元的引言。在这一节里所说的人们对功能关系
的基本认识——做功总是伴随着能量的改变,而且做多少功,
能量就改变多少一是整个单元教学的主线。第十一节是单
元的结束语。第三节和第十一节给出了这一章的基本思路,
应予以足够的重视。

\subsubsection{关于能量的概念}

要对能量下一个严格的定义是困
难的,因此,在新课本中不给出能量的定义,只沿用了初中课
本中的一个粗浅的说法:一个物体能够对外界做功,我们
就说这个物体具有能量。为了使学生能够比初中有更进一步
的理解,教材讨论了怎样定量地确定能量的变化问题,从而得
出用做功的多少来确定能量变化的多少这样一个基本认识。

讲解这一节教材的关键是要分析好一些实例:要分析流
动的河水、举高的铁锤等物体怎样对别的物体做功,自己的能
量又怎样变化。这样多次重复,就能使学生对上述基本认识
有深刻的印象。

这里有一个问题教学时应加以注意。本章前两节只讲力
对物体做功,即做功的主体是力。而这里说物体做功,怎么理
解?实际上这就是指物体克服阻力做功(即阻力对物体做负
功),例如流动的河水克服水轮机的阻力而做功;举高的铁锤
克服木桩的阻力而做功,等等,有了这一认识,就能对教材224
页中说的什么时候物体能量增加,什么时候能量减少有正确
的理解。要使学生明白,物体克服阻力做功,能量要减少,外
力对物体做正功,物体的能量要增加。

\subsubsection{动能概念和动能定理的建立}

教材分成三个层次来
建立起动能这个概念,第一步,先说运动的物体能够做功,因
而具有能量,称为动能。这一点是直接根据上一节所讲的能
量的概念来讲的第二步,讨论如何通过做功来定量地
确定动能的方法。这是以上一节中所讲的功和能的关系的基
本认识为依据的。第三步,具体推导动能的定量表达式。前
两步由于直接引用上一节关于能量的结论,所以容易被学生
所接受,第三步的推导应用了牛顿第二定律和运动学的一些
知识,其推导方法同下一节动能定理的推导方法是一样的,
只不过情况较为简单(假设物体原来是静止的)。所以这一节
教材同前后各节的联系是很紧密的。教学中要引导学生注意
这种联系,使学生对功和能的关系的认识能一环扣一环,逐步
加深理解。

动能定理是一条适用范围很广的物理规律,但教材在推
导这条规律时是由特殊到一般逐步扩大的。先把第四节的推
导扩大到初速不为零的情况,得到外力对物体所做的功等于
物体动能的增加的结论。然后将此结论推广到外力与
运动方向相反的情况,最后再推广到物体受到几个力作用的
情况。但是,尽管经过几次推广,教学中还是应该引导学生注
意这条规律的适用范围。在中学范围内,动能定理只应用在
研究对象是一个物体(质点)、作用力是恒力的情况,对作用力
是变力的情况,动能定理也是适用的,但学生没有学过变力作
功的计算,无法应用动能定理来解决实际问题。对于几个物
体组成的物体系,动能定理必须改变形式,否则不能适用(见
参考资料)

\subsubsection{“动能的增量”和“动能的增加”}

课本在表述动能定
理时没有用“动能的增量”这种提法,而说“动能的增加”,学生
比较容易理解,但对程度较好的学生也可介绍“增量”这种提
法。不管使用何种提法,都要使学生理解,变化后的动能与变
化前的动能之差,即
\[\frac{1}{2}mv^2_2-\frac{1}{2}mv^2_1\]
可以是正的,也可以
是负的。

\subsubsection{动能定理的应用}

讲解动能定理的例题要达到两个
目:一是学会应用动能定理解题的一般方法,即首先要分
析物体受力情况,并研究整个过程中外力所做的总功。然后
看初末状态的动能。最后列出方程,解出未知量,二是要让
学生明白同一例题用牛顿第二定律和运动学知识也可以解。
但在不要求研究过程中加速度和时间的情况下,用动能定理
要比用牛顿第二定律解题方便得多。

教师在教学过程中如果感到书上例题不够,可适当补充。
例如可以补充一道物体初速不为零的例题。但例题和习题不
必做得太多,这不仅是因为要减轻学生的负担,而且是对动能
定理这个教学重点的处理要适当。如果仅仅从算题的角度来
理解动能定理的重要,认为用它可以计算本章的各种问题,那
就必然会削弱其他知识的学习和理解。例如有的同学就认为
连势能这样的概念也没有必要引入,这种看法显然对培养学
生从能量角度考虑问题是不利的。


\subsubsection{如何使学生正确理解重力势能的概念}

重力势能是
一个重要的概念,对学生来说,掌握这个概念又是一个难点。为
了使学生掌握好这个难点,课本先从功和能的基本认识出发
引入重力势能概念,然后再讲重力做功的特点,并将重力势能
的一些性质(相对性、正负的意义、属于系统共有等)分散在第
六、七节中讲述。这样编排有利于学生理解重力势能的概念。
教学中应按教材的顺序逐步讲解,使学生的认识步步深入。
为了帮助学生正确理解重力势能的概念,再作如下几点建议:
 
重力势能的引入同第四节中动能的引入思路是一样
的。先讲被举高的物体能够做功,所以具有能量,称为重力势
能;然后从功和能的基本认识出发研究如何定量地确定势能
的大小;最后通过简单的例子(匀速举高一个物体)来引入重
力势能的计算式$E_p=mgh$. 因此讲解这部分内容时,可以与
动能的引入对照起来讲,使学生觉得并不陌生。

要使学生明白,课本231页第二小节里说的“克服重
力做多少功(即重力做负功),重力势能就增加多少;重力对物
体做多少功,重力势能就减少多少”这个结论始终成立,与物
体是否还有其他力做功以及物体的动能变化与否无关。

如果把这个结论同课本第三、四节中讨论功和能的关系
时说的“外力对物体做多少功,物体的动能就增加多少”进行
对照,会发现表述形成上不一致。为什么外力对物体做功能使
物体的能量增加,而重力对物体做功会使物体能量减少呢?这
是学生在学习过程中很容易混淆的一个问题。许多学生就是
由于在这里产生了混淆又不能及时澄清,采取了死记硬背的
学习方法,影响了整章的学习效果,其实,这两句话是不矛盾
的。外力对物体做功,使物体增加的是动能;量力对物体做正
功,所减少的是重力势能。此时物体动能增加与否要看合外
力是否对物体作正功。例如自由落体运动,物体只受到重力
作用,重力作为外力对物体作正功,物体的动能增加。但同时
重力势能减少。从这里也可以看出,重力做功使物体动能增
加,这是以减少重力势能为代价的,因此,重力作功的问题总
是伴随着能量的转化。这就把重力势能的问题同第九节机械
能守恒定律联系起来了,为学习第九节作了准备。

用这个观点再来分析课本图7.9的例子,可以看到,外力
$F$对物体做功,本来应该使物体动能增加,但重力作为外力对
物体作负功,能使物体动能减少,结果物体动能不变。不管动
能变或不变,重力作了负功,重力势能一定增加、这一结论仍
然成立。

关于重力做功的特点,教材通过物体从$A$点到$C$点的
不同途径(折线、直线、曲线)的计算,得出重力对物体所做的
功只跟起点和终点的位置有关,而跟物体运动的路径无关的
结论。这一结论的得来并不困难,但后面一段讨论,正因为
重力做功有这样的特点,才能引人重力势能的概念,就比较难
理解了,教材是用反证法来论证这一点的。教学中能使学生
通过阅读和讲解对这个问题有个印象就可以了。这对今后学
习静电学等知识有好处。教学中不要提保守力、耗散力的概
念,但应把重力做功和摩擦力做功的情况对照起来讲。

第六、七节内容较多,在两节学完以后建议小结一
下,突出两个重点。一个是重力势能的计算式$E_p=mgh$; 另
一个是重力做功与重力势能变化的关系。至于“克服重力做
多少功,重力势能就增加多少;重力对物体做多少功,重力势
能就减少多少”这一结论是否要再抽象为“重力做的功等于
物体重力势能增加的负值”,则要根据学生的接受能力而定。
对程度较好的学生,这样做也未尝不可。


\subsubsection{弹性势能的定性研究}

关于弹性势能,课本没有作定
量讨论,只作定性介绍.这一节包括三个方面的内容:
\begin{enumerate}
\item 什
么是弹性势能?教学中可采取与重力势能相对照来引人的方
法,并通过实例来加以解释,其中特别要注意弹性势能是属于
发生弹性形变的系统的。
\item 弹力作功与弹性势能变化的关系,这是达一节教材的主要内容,要使学生明白,克服弹力
做多少功,弹性势能就增加多少;弹力对其他物体做多少功,
弹性势能就减少多少,这个规律与重力做功跟重力势能变化
的关系是一样的,这个关系始终成立,跟物体(被弹力作用的
物体)是否还受其他外力、动能增加与否无关。
\item 弹簧的弹
性势能大小与形变大小和倔强系数有关,这一点不作定量计
算,但定性掌握它们的关系对于学习振动的知识是很有用的。
\end{enumerate}

\subsubsection{机械能的转化和守恒}

要讲机械能的守恒,先要讲机
械能的转化。没有动能和势能的相互转化,就无所谓机械能
的守恒,因此,首先要通过一些实例的分析,研究物体的动能
和势能的转化,为了使教学更充实些,除了教材所举的自由
落体、光滑斜面、竖直上抛及弹簧、弓箭等例子外,还可以举出
平抛、斜抛、光滑曲面等例子,来说明这些过程中机械能是如
何转化的。

要启发学生注意,势能的变化是由于重力和弹力做功而
引起的。但重力作为外力,又可以改变物体的动能(动能定
理)。如果重力做正功,重力势能将减少,动能将增加,意味着
重力势能转化为动能。反之也一样,这样不仅可以帮助学生理
解为什么课本中说“机械能的相互转化是通过重力或弹力做
功来实现的”,也为推导机械能守恒定律提供了思路。

在得到机械能守恒定律后,一定要强调条件。课本在表达
时只说“只有重力做功”,这是因为弹性势能不作定量讨论,只
限于定量研究重力势能与动能的转化问题。但在这一节的最
后也提到了如只有弹力做功,机械能也是守恒的。这一点,让
学生了解一下就可以了,需要强调的是,“只有重力做功”不
等于“只受重力作用”。物体可以受其他外力作用,只要这些
力不对物体做功,机械能就是守恒的。

应用机械能守恒定律解题时,首先要检验是否符合守恒
的条件,如果符合,则明确初状态和末状态后就可以列方程解
出未知量了,如果除重力和弹力外还有其他外力做功,则机械
能不守恒,但这种情况仍可以应用动能定理来解题。不管用
动能定理还是用机械能守恒定律解题都只涉及起始和终了状
态,不涉及中间过程的细节,因此相对于用牛顿第二定律和
运动学来解题要简便得多。此外,用守恒定律解题还有更深
刻的意义,因为自然界的许多规律就是通过寻找“守恒量”而
找到的。教学中讲一讲这个问题对开拓学生的视野是有好处
的。

\subsubsection{除重力、弹力外还有其他力做功的情况}

教材第十一
节是前几节的自然延伸,也是整章的小结。这一节的教学处理
得当,能把全章知识联系起来,做到融会贯通。建议教师在教
学时注意以下几点:

教材通过一系列实例来分析,除重力和弹力外如果
还有其他外力做功,则物体的机械能要增加,且其他力做多少
功,物体的机械能就增加多少。反之,如果物体克服其他外力
做功,机械能将减少。为了使学生对这一段分析理解得更具
体,可采取两种办法,对一般程度的同学,可以举一个有具体
数字的例子,证明其他力(如汽车牵引力)做的功等于物体机
械能的增加;对程度较好的同学,还可以从动能定理出发进行
推导.推导时不要涉及弹力做功.例如对课本245页上所举
的第二个实例(图7.3)应用动能定理可以写出如下的式子:
\[W_{\text{牵}}+W_G=\frac{1}{2}mv^2_2-\frac{1}{2}mv^2_1\]
由于重力做功等于重力势能增加的负值,即
$W_{\text{牵}}=mgh_1-mgh_2$,代入上式,得
\[W_{\text{牵}}+mgh_1-mgh_2 =\frac{1}{2}mv^2_2-\frac{1}{2}mv^2_1 \]
整理后,可得:
\[W_{\text{牵}}=\left(\frac{1}{2}mv^2_2+mgh_2\right)-\left(\frac{1}{2}mv^2_1+mgh_1\right)=E_2-E_1\]
\begin{figure}[htp]
    \centering
\begin{tikzpicture}[>=latex, scale=1.3]
    \fill[pattern=north east lines](-1,-.2) rectangle (1.5,0);
\draw(-1,0)--(1.5,0);
\draw (0,1.5) rectangle (.5,2);
\draw[dashed] (0,3.5) rectangle (.5,4);
\draw[|<->|](-.2,1.5)--node[fill=white]{$h_1$}(-.2,0);
\draw[|<->|](-.7,3.5)--node[fill=white]{$h_2$}(-.7,0);

\draw[<->](.25,1.75-.7)node[right]{$G$}--(.25,1.75)--(.25,1.75+.7)node[right]{$F$};
\draw[<->](.25,3.75-.7)node[right]{$G$}--(.25,3.75)--(.25,3.75+.7)node[right]{$F$};
\tkzDrawPoint(.25,1.75) \tkzDrawPoint(.25,3.75)
\draw[->](.75,1.5)--node[right]{$v_1$}(.75,2);
\draw[->](.75,3.5)--node[right]{$v_2$}(.75,4);

\end{tikzpicture}
    
    \caption{}
\end{figure}


采用推导的方法比较简捷明
了,而且有利于把其他力做功引起
的机械能增加同动能定理中的外力做的总功使物体动能增加
区别开来。但对基础不太好的同学,这样推导不易真正理解,
因此不宜采用。

得出上述其他力做功使机械能变化的结论后,课本
接着阐述了机械能的变其实是机械能同其他形式的能的转
化。这部分叙述是以这样两句话为依据的:“增加了的机械能
并不是凭空产生的,”“减少的机械能也不能无影无踪地消
失。”这实际上就是能量的转化和守恒的思想,由于前面已
学过机械能守恒定律,又讲过寻找“守恒量”的意义,所以这里
这样说学生是能够接受的。

在这一节的教学中,可以也应该将整个第二单元所
讨论的功和能的关系作一次整理,这一单元讨论功和能的关
系大致有这样几个层次:

\begin{enumerate}
    \item 先定性讨论做功能位能量发生变化。
    \item 定量研究合力做功(或外力做功的代数和)能使物体
    动能增加(动能定理)。
    \item 在这些外力中有两个力是特殊的;重力和弹力,重力
    做功,重力势能减少;克服重力做功,重力势能增加。弹力做
    功,弹性势能减少;克服弹力做功,弹性势能增加。
    \item 如果除重力和弹力外,其他力不做功,则机械能守恒。
    \item 如果除重力和弹力外,还有其他力做功,则机械能与
    其他形式的能发生转化,但能量仍守恒,
\end{enumerate}

总之,做功总是跟能量变化相联系,而且两者在数量上是
相等的,也就是说,功是能量转化的量度,以上几个关系可用
下面的示意图表示出来,由于课本不定量研究弹性势能,所
以示意图中也可以把弹力和弹性势能略去。

\begin{center}
\includegraphics[scale=.8]{fig/1.pdf}
\end{center}

\section{实验指导}
\subsection{演示实验}
\subsubsection{物体的动能跟它的质量和速度有关}
如图7.4所示,使小车沿一光滑斜面下滑.斜面倾角约
$10^{\circ}$—$15^{\circ}$即可,斜面底端接着一个水平木板,在水平木板上
铺一层毛巾布。在斜面和平面的接合处,用毛巾布堵塞以减少
小车下滑到接合处时发生撞击或弹跳。

















































































\chapter{动量}

\section{教学要求}

动量是力学中的重要概念,动量守恒定律是自然界最重
要的普遍规律之一。因此在甲种本中单设一章来对有关知识
作较深入的讨论。

这一章的教学要求是:
\begin{enumerate}
\item 理解冲量和动量两个概念,掌握动量定理并会用它解
释一些物理现象。
\item 掌握动量守恒定律并会用它分析、计算有关的问题.
\item 理解弹性碰撞和非弹性碰撞,会计算一维弹性碰撞的
有关问题。
\item 了解反冲运动的原理及其在火箭技术中的应用。
\end{enumerate}

下面对这一章的教学内容作些具体说明。

冲量和动量是两个不容易理解的概念。教材在分析具体
事例的基础上,引用公式进行讨论,得出这两个概念,这样做,
对初学者来说,概念的物理意义可能清楚些,容易理解些。动
量的矢量性在研究动量定理和动量守恒定律时都很重要。但
是学生初学时往往对此认识不够,需要一开始讲解动量时,就
强调它的矢量性。

第二节讲解动量定理。要使学生明确:动量定理$Ft=
mv'-mv$表示的是力在一段时间内连续作用的累积效果与物
体动量变化间的关系;在变力的情况下,公式中的$F$表示变
力在时间$t$内的平均值。

动量守恒定律也可以用牛顿第二定律和第三定律推导出
来。这种讲法的好处是比较简便。然而,动量守恒定律是一
个独立的实验定律,而且它的适用范围比牛顿运动定律更广
泛。因此,教材以实验为基础总结出动量守恒定律,然后说明
这一定律与牛顿运动定律的一致性。

第三节讲述相互作用的物体的动量变化的实验,是为第
四节讲动量守恒定律作准备的。在归纳出动量守恒定律时应
向学生说明,这个定律是在分析研究大量实验事实的基础上
建立起来的,而不是仅靠几个实验得出来的。还应使学生清
楚系统动量守恒的条件,防止解题时不问条件,乱套公式的
倾向。

第五节讲解动量守恒定律和牛顿运动定律的关系。应当
使学生清楚地了解这两个定律在牛顿运动定律运用的范围内
是一致的,这两个定律的区别在于适用范围的不同。牛顿运
动定律只适用于宏观物体的低速运动情况,而动量守恒定律
的适用范围更普遍,不论是宏观物体还是微观粒子,低速运动
还是高速运动,相互作用是什么性质的,都遵守动量守恒
定律。

第六节讲述如何运用动量守恒和动能守恒来研究碰撞问
题。应该使学生明确,对于弹性碰撞需要同时运用动量守恒
和动能守恒来解决。讲解弹性碰撞,要注意培养学生运用两
个守恒定律分析解决问题的能力,进一步认识守恒定律的重
要性,例题中得出的两个钢球发生弹性正碰后的速度表达式,
应要求学生会自行推导,而不是记住它们。同时,要求学生能
够利用表达式讨论一些具体情况.例如,$m_1>m_2$, $m_1=m_2$,
$m_1<m_2$时,两钢球碰撞后的运动情况;$m_1\ll m_2$, $m_1=m_2$, 
$m_1\gg m_2$时,两钢球碰撞中能量的传递情况.并能根据讨论的
结果,解释一些弹性碰撞中发生的现象。

第七节反冲运动中讲述的内容都属于常识性的介绍,目
的是使学生了解物理知识在现代科学技术中的应用,介绍一
些有关的科学技术常识,开阔学生的眼界。

\section{教学建议}
全章教学可分三个单元。第一单元(第一、二节)讲授冲
量和动量的概念以及动量定理。第二单元(第三至五节)讲授
动量守恒定律,这是本章的重点。第三单元(第六、七节)讲
授碰撞和反冲运动,这是动量守恒定律的应用。

\subsection{第一单元}
\subsubsection{冲量和动量概念的引入}

冲量和动量这两个概念也
同功和能一样,不是一引入就能体会它的物理意义的。对这
两个概念的理解,要通过整章的教学过程逐步加深。所以第
一节的教学既要使学生初步了解冲量和动量的意义,特别是
为什么要引入这两个概念以及两个物理量的定义、计算式、单
位和矢量性等,又不能操之过急,要求过高。

课本在引入冲量和动量的概念时,通过日常生活中常见
的事实,以开动汽车为例,说明汽车获得一定的速度不仅同它
的牵引力有关,而且同力的作用时间有关。然后应用牛顿运
动定律和运动学的知识,得出了速度的变化跟力和作用时间
的关系$Ft=mv$. 再通过对这个式子的讨论给出冲量和动量
的定义。这样引入,除了物理意义比较明显、较易理解外,还
可以使学生从一开始就认识这两个概念之间的密切联系,了
解冲量(它表现力的作用)的效果是使物体获得动量。

教学中还应该使学生了解速度和动量的联系和区别。速
度和动量都可以作为运动的量变。但是速度只告诉我们物体
运动的慢快和方向,没有告诉我们使物体运动或者停止运动
需要多大的冲量。动量没有告诉我们物体运动速率的大小,
却能告诉我们使物体运动或停止运动所需的冲量。所以速度
是一个运动学的量,只能用来描述运动,而动量则是一个动力
学的量,它跟冲量即物体运动变化的原因相联系。这里,可以
用课本上以相同的速度运动的铅球和乒乓球的例子来说明。

\subsubsection{动量的变化}

学生初学动量时,往往忽略动量的矢量
性,只注意它的大小,不注意它的方向,以为只要物体的速度
大小不变,动量也就不改变.教学时可利用课本252页的例
题纠正这种错误认识。要想学生考虑钢球从坚硬的障碍物上
弹回,动量的大小并没有变,动量的变化为什么不等于零?原
因是动量是矢量,它的方向改变了,所以动量发生了变化。

教材明确指出,所谓动量的变化就是变化后的动量减去
变化前的动量,因此这里所说的动量的变化实际上就是上一
章所说“增加”或“增量”。具体计算时,由于我们只研究一
直线上的动量变化,所以只要先选定一个方向为正,就可以
把矢量运算简化为代数运算。

\subsubsection{关于打击时的平均作用力}
动量定理可以看作是牛顿第二定律的变形,在这里没有
必要作过多的讲解,可只限于用来解释一些这个定理便于解
释的现象,如为什么茶杯掉在地上要碎,而掉在软的东西上就
不易摔碎等。在计算方面,只限于计算打击、碰撞等问题中的
平均作用力,如课本255页中的例题那样。

课本255页例题后面有一段讨论.讨论中说“如果把铁
锤的重量也考虑在内,那么,这时道钉所受的打击是上面算
出的打击力加上铁锤的重量。”如果学生对此提出怀疑,教学
中可加以说明。应该指出,动量定理$Ft=mv'-mv$中的$F$是
物体所受的合力。所以在这个例题中应该列出如下方程:
\[F=\frac{mv'-mv}{t}=2.5\x10^3{\rm N}\]
而$F=N-G$, 所以
\[N=F+G=2.5\x10^3+49=2549{\rm N}\]

至于什么时候可以忽略铁锤的重量$G$, 这要看$G$与$F$的
差别有多大.象书上例题中$G$为$F$的2\%, 就可以忽略了。

\subsection{第二单元}
本单元是全章的重点。整个单元就是围绕一个中心——
动量守恒定律展开的。

\subsubsection{相互作用物体的动量变化}

第三节教学的关键是要
做好研究相互作用物体的动量变化的演示实验。实验可分为
两部分.第一部分为定性研究,如课本图8.2那样,目的是说
明两个相互作用的物体动量会发生变化,在课堂上可用两个
压紧弹簧的小车代替图8.2中的两个人,也可以用玩具小车
在可动的平板上的运动来演示。

第二部分为定量研究,要求分析课本插页的气垫导轨上
相互作用滑块的闪光照片,得出动量变化总是大小相等方向
相反的结论,教学中要注意引导学生学会如何根据闪光照片
计算滑块速度,并测量动量变化的方法,培养分析原始实验资
料的能力。根据学校的实验设备条件,也可以在课堂上当场
演示,得出同样结论。例如用光电计时器对气垫导轨上滑块
的速度测定来研究共动量的变化;用打点计时器研究相互作
用小车的动量变化;利用平抛原理测定相互作用小球的动量
变化等,有条件的学校还可以让学生自己动手来得出这个结
论。这一节的实验做得越好,对下一节动量守恒定律的学习
就更有利,教学中要十分重视。

\subsubsection{动量守恒的语言表达}

从“相互作用的物体的动量变
化总是大小相等、方向相反”的结论到“系统的动量守恒”的推
理过程,对语言表达的要求较高,教学中教师除应注意自己
授课语言的准确外,还应注意培养学生学习物理的语言表达
能力。推理过程有三个层次:第一是实验结论,“相互作用的
物体的动量变化总是大小相等,方向相反的”;第二是引入系
统总动量的概念,得出“系统的总动量的变化为零”;第三,再
得到系统的总动量守恒,以上三个层次,用公式表达,可与语
言表达对照如下(表中的$p_1$、$p'_1$、$p_2$、$p_2'$都是矢量。如果两个
物体作用前后都在一直线上运动,则上述四个量均代表带正
负号的动量数值。其中最后一式常常写成$m_1v_1+m_2v_2=m_1v'_1
+m_2v_2'$)。

\begin{center}
\begin{tabular}{p{.35\textwidth}c}
    \hline
    用语言表达&用公式表达\\
    \hline
    相互作用物体的动量变化总是大小相等,方向相反的&$p'_1-p_1=-(p'_2-p_2)$\\
    总动量的变化为零&$(p'_1+p'_2)-(p_1+p_2)=0$\\
    系统的总动量守恒&$p'_1+p'_2=p_1+p_2$\\
    \hline
\end{tabular}
\end{center}

\subsubsection{动量守恒定律的条件}
课本明确指出,系统动量守恒
的条件是“系统不受外力或所受外力的合力为零”。教学中要
重视培养学生在应用动量守恒定律时先检验是否符合守恒定
律条件的习惯,防止随意乱套公式,但是有一点应该向学
说明,在有些问题中,系统虽然受到外力作用,而且合力不为
零,但合外力同内力相比非常小,可以忽略不计时,动量守恒
定律仍然可以适用.例如课本260页例题,在列车间相互作
用时,重力和轨道支持力虽然合力为零,但列车和轨道间总有
摩擦力存在。由于这个摩擦力同列车间的内力相比很小,所
以仍可用动量守恒定律。又如手榴弹爆炸时,虽然整个系统
受到重力作用,但比起炸药的爆炸力来重力很小,可以忽略,
因此还是可以用动量守恒定律来计算爆炸后碎块的速度。

\subsubsection{动量守恒定律是普遍适用的物理规律}

动量守恒定
律是自然界最重要的普遍规律之一。教材在第三节末说了
相互作用物体的动量变化总是大小相等方向相反的结论不仅
适用于正碰,而且适用于斜碰。在第四节第一段,教材又说到
这个结论“在任何情况下”都成立,在第四节中,又特别提出三
点说明,对上述“任何情况”作了具体的解释。为了使学生对动
量守恒定律的普遍性有深切的了解,教学中可以多举一些实
例,特别是对说明中第1点可多举例说明,例如,子弹射入木
块、火车车厢连接在一起,课本插页图8.6的碰撞,属于粘合
在一起的例子,炮弹和手榴弹的爆炸属于分裂成碎块的例子。
此外,枪炮发射弹丸,人在船上行走等相互作用的例子,也可
用动量守恒定律来研究.对第2、3点,由于学生知识水平
有限,不可能理解得很具体,只要有一个印象就可以了。

\subsubsection{动量守恒定律中物体的速度}

在两个物体相互作用
对应用动量守恒定律可用表达式
\[m_1v_1+m_2v_2=m_1v'_1+m_2v'_2\]
式中涉及四个速度,要向学生指出,这四个速度必须是相对于
同一参照系的,一般都以地面为参照系。教师在引导学生解决
如人在船上行走之类的问题时,也要注意不要涉及相对速度,
而应把问题局限在相对于同一参照系研究其动量守恒。

\subsubsection{寻找“守恒量”的一个例子}

教材在第四节后安排了
一段阅读材料,要引导学生认真阅读,使学生了解到十六、七
世纪的哲学家如何从观察宇宙间各种物质的不断运动得出宇
宙运动的总量不会减少的看法,笛卡儿和牛顿又怎样去努力
寻找一个物理量来量度这种永恒的运动。把这一段阅读材料
同上一章第十节所说的寻求“守恒量”的重要意义联系起来
(课本244页),可以使学生体会到守恒定律在物理学里的重
要地位。这一段阅读材料中关于笛卡儿寻找量度运动的合适
物理量时的失误和对他的功绩的评述对学生也是有启发的。

\subsubsection{动量守恒定律和牛顿运动定律的关系}

教材第五节
主要说明两个问题,
\begin{enumerate}
 \item 动量守恒定律和牛顿运动定律是一致
的.但现在已认识到,动量守恒定律具有更大的普遍性;
\item 由于动量守恒定律不涉及相互作用的中间过程,所以在处理
某些问题时会更简便。
\end{enumerate}

教学中为了说明这两点,除了按教材讲述外,还可以选择
一道不要求研究中间过程的例题(例如课本260页的例2, 用
动量守恒定律和牛顿运动定律两种方法来解,说明两者的一
致性,同时又可以比较用动量守恒定律解题得更简便一些。

\subsection{第三单元}
\subsubsection{寻找“守恒量”的又一例子}

教材第六节从两个质量
相等的小球作弹性碰撞的演示开始,讨论只有动量守恒定律
还不能解释为什么现象是唯一的。最后得出在这种碰撞中还
应满足动能守恒。这是寻找“守恒量”的又一个例子。通过学
习,学生可以体会到,在一个物理过程中多一个“守恒量”,就
多一个制约因素,只有存在足够的制约因素,现象才能是唯
一的。在哪些现象中有什么守恒量,这就是自然界的规律,学
习物理学就要掌握这些规律。因此寻找“守恒量”是有重要意
义的。这一段教学内容很有启发性,对培养学生分析问题的
能力很有帮助.教学中首先要做好课本图8.7所示的演示实
验。在分析这一实验时,要使学生弄清初末状态的情况,所谓
碰撞的初末状态,指的是两球接触的短暂时间的前后,两
个状态的小球位置都是在最低点,而不是在弹起的最高点。
虽然初末状态的小球位置一样,但它们的速度不一样。在定
义了弹性碰撞以后,教材接着介绍了非弹性碰撞和完全非弹
性碰撞。教师也可在堂上利用课本图8.7的装置演示一下完
全非弹性碰撞.用课本图8.7演示弹性碰撞和完全非弹性碰
撞只能是定性的。有条件的学校可以用气垫导轨和光电计时
器演示,这样就可以作定量的测量,来验证弹性碰撞中动能守
恒,而非弹性碰撞中动能有损失。

\subsubsection{弹性碰撞末速度公式的推导}

课本第六节例题是根
据动量守恒和动能守恒列出方程组解弹性碰撞问题的一个例
子。例题中推导了碰撞后两钢球的速度,教学中要对推导方
法作一示范,並要求学生能自行推导,而不要简单地记忆末
速度公式。对例题后面的讨论,教学中也要加以重视。在讨
论中,可以将练习四的第4题一起进行讨论.从方程解中讨
论几种情况的物理意义,是一种重要的能力。对于弹性碰撞
的讨论是培养这方面能力的好时机。

对于弹性碰撞问题,课本局限在讨论正碰,并且两球中有
一球原来是静止的。对于学有余力的同学,两球初速均不为
零的弹性正碰可以作为练习题让他们自己推导和讨论,只要
他们掌握了推导方法,这样做是不太困难的。至于斜碰则不
要让学生去做了。

\subsubsection{反冲运动的方程}

反冲运动问题可以用动量守恒定
律来解决,如果反冲运动发生前物体是静止的,则动量守恒定
律可写成$MV+mv=0$的形式,其中$M$、$m$分别为向两个方向
运动的物体的质量,$V$、$v$是相应的速度,其正负号由假设的正
方向决定,可以让同学根据动量守恒定律自己写出这一方程。
但同学在写这一方程时,常常有两个容易出错的地方。第一,
如果$M$表示反冲运动发生前的总质量,则方程应改写为
$(M-m)V+mv=0$. 第二,有的同学根据反冲的两部分动量
大小相等写出$MV=mv$, 算出答案往往也是对的。但此式中
$V$和$v$均为绝对值。这种写法与本章前面的做法不一致,要
特别小心。如果反冲系统原来的动量不为零,则不能用这个
办法了。

\subsubsection{关于火箭的教学}

第七节的主要篇幅是介绍反冲运
动在火箭技术中的应用,主要是为了扩大知识面。教材不要
求作定量计算,所以关于火箭的最终速度与喷气速度、质量比
的关系不必介绍计算公式。至于火箭技术的一些原理、应用
及我国火箭技术发展情况,可鼓励学生自己查阅科普小册子
和杂志中的有关资料。这既可以激发学生的学习兴趣,又可
以培养学生自己查阅资料的能力。

\section{实验指导}
\subsection{演示实验}
\subsubsection{冲量的引入}

\begin{figure}[htp]
    \centering
    \includegraphics[scale=.6]{fig/8-1.png}
    \caption{}
\end{figure}

可利用图8.1的装置来演示冲量的作用效果,使小球自
静止开始从斜槽上的某一位置$O$滚下(在$O$处做一记号),斜
槽的底端部分是水平的。观察小球的落地点,在落地点$P$处
可放一塑料小桶,重复实验,使小球恰能落到桶中。使斜槽的
倾角变小,即使小球所受的合力变小,如果仍在$O$点让小球
自静止滚下,则小球将不会落在桶内.若将小球移到$O$点上
方的$O'$点处开始释放,使力的作用时间增长,则小球仍能落
在桶内。这说明了可以用较大的力作用较短的时间,也可以
用较小的力作用较长的时间,使原来静止的物体获得相同的
速度。

实验所用的斜槽,可用两
条平行的粗铁丝焊制而成,或
利用铝材商店出售的用铝合
铁片
金制成的U形铝材来制作
(图8.2)。
\begin{figure}[htp]
    \centering
    \includegraphics[scale=.6]{fig/8-2.png}
    \caption{}
\end{figure}

\subsubsection{动量}
如图8.3所示,使一个乒乓球从一光滑斜槽的顶端自静
止滚下,在光滑桌面上运动一段距离后,被一个固定着的装有
橡皮膜(从气球上剪下,装在架子上不要绷紧)的架子$R$阻挡
后,停止了运动,再用一个大小相同的金属球,把它从斜槽的
同一位置上释放,当它到达桌面时具有相同的速度,但被架子
$R$阻挡时,将会观察到橡皮膜被拉得很长后(阻力$F$和力作用
的时间$t$都比较大)才停止运动。这说明乒乓球和金属球虽
然具有相同的速度,但由于金属球的质量大,动量也大,因此
要使它停止下来,需要更大的冲量。
\begin{figure}[htp]
    \centering
    \includegraphics[scale=.6]{fig/8-3.png}
    \caption{}
\end{figure}


\subsubsection{动量传递和动量守恒}
用气垫导轨演示课本插页图8.3, 图8.4的实验时,可将
一根轻质弹簧先固定在一个滑块上,然后用棉线扎紧使弹簧
处于压缩状态,再把另一个滑块紧靠着这个滑块装有弹簧的
一侧。在两个滑块都静止的情况下,点燃火柴将棉线烧断,即
可看到两个滑块同时分离向相反方向运动的现象。如果两滑
块的质量相等,则它们分离的速度大小相等,若两个滑块的质
量不相等,则质量小的滑块速度大,质量大的滑块速度小。

演示课本插页图8.5, 图8.6的实验时,可在静止滑块靠
近运动滑块的一端,事先粘上一块较软的橡皮泥(如橡皮泥比
较干、硬,可把它切成小块后加些缝纫机油调软),这样,当运
动滑块跟它碰撞时能较好地粘合在一起。

\subsubsection{弹性碰撞}
课本图8.7的演示实验,
也可以用瓷球代替钢球。为了
保证使两个小球在同一竖直平
面内摆动,可采用双线摆的结
构(图8.4).为了增加可见度,
也可以用注满水的乒乓球来代
替钢球进行演示。
\begin{figure}[htp]
    \centering
    \includegraphics[scale=.6]{fig/8-4.png}
    \caption{}
\end{figure}


乒乓球中注水的方法是这样的:用注射器针头先在乒乓
球上戳一小孔,再在小孔的近处插入针头,接上针筒注满水,
拔去针头后可在两个针孔间用弯曲的细铜丝穿入细线,再用
橡皮胶布将小孔封住。

\subsubsection{反冲运动}
反冲小车:
如图8.5所示,在小车上用铝皮做一个支架,上面固定一
根试管,试管略倾斜,管内盛一些水,试管口用橡皮塞(质量可
比软木塞大)塞紧,试管底部安装一个盛有酒精棉花的小盘。
点燃酒精棉花,待试管中的水沸腾后,产生大量蒸汽将橡皮塞
冲出的同时,小车就发生后退现象。

\begin{figure}[htp]\centering
    \begin{minipage}[t]{0.48\textwidth}
    \centering
\includegraphics[scale=.6]{fig/8-5.png}
    \caption{}
    \end{minipage}
    \begin{minipage}[t]{0.48\textwidth}
    \centering
\includegraphics[scale=.6]{fig/8-6.png}
    \caption{}
    \end{minipage}
    \end{figure}

将一长形气球打足气后,用手指捏住打气口,放开后
可观察到气球由于放出气体而发生的反冲现象(图8.6)。

\subsection{学生实验}
\subsubsection{研究弹性碰撞}
这个实验安排两课时完成.第一课时可以结合仪器
的调节与使用来熟悉处理数据的方法,从而进一步理解实验
的设计原理。第二课时进行实际的测量与研究。

这个实验的设计原理比较复杂,应要求学生弄明白
后再行操作。

在调节仪器和实验操作时,要注意以下几点:
\begin{enumerate}
    \item 调节仪器的水平和支放被碰小球的小柱高度和位置。
使两球能在同一高度上发生正碰。
\item 初步调节好支放被碰小球的支柱的高度和位置后,要
进行试测,观察时入射小球的落点位置$P$, 以及发生碰撞时,
入射小球的落点$M$和被碰小球的落点$N$, 看看$P$、$M$、$N$三
个点是否大致在一条直线上,如果偏离很大,则应进一步调节
支放被碰小球的小柱在水平面上的位置,直到碰撞后,$P$、$M$、
$N$三个点看起来基本上在同一直线上,才可以正式做实验。
\item 确定斜槽底部水平部分槽口中心在水平面上的投影,
即入射小球的抛出点在水平面上的投影$O$点时,重垂线的长
度要恰当,应控制在即将接触到纸面的高度上。在这个地方
不要覆盖复写纸,实验时要使斜槽固定使重垂线始终指在
$O$点.
\item 要使学生理解课本图10.19中$O$点的位置,即槽口
重垂线所指的位置,而$O'$点的所在位置,应在原始记录纸上
沿着$OP$直线量度$2r$($r$是小球半径)的距离来确定,如果入
射小球和被碰小球的半径不相等,则距离$OO'$应等于两个小
球的半径之和,小球半径可以用游标卡尺来测量。
\end{enumerate}

利用等式$m_1(OP)=m_1(OM)+m_2(O'N)$研究动量
守恒时,对于式中相同的量取相同的单位。譬如质量的单位
都用千克(或克),距离单位都用米(或厘米)就可以了,不一定
都要用动量的单位进行计算。因为在这个实验中,是用距离
来表示速度的,实际上是
\[v_1=\frac{OP}{\sqrt{2h/g}},\qquad v'_1=\frac{OM}{\sqrt{2h/g}},\qquad v'_2=\frac{O'N}{\sqrt{2h/g}}\]
式中的$h$为小球下落的高度。只要用
$t=\sqrt{2h/g}$
来除上面的等式,式中的各项仍具有动量的单位。

在利用等式$m_1(OP)^2=m_1(OM)^2+m2_(O'N)^2$研究能量
守恒时,同样是相同的量取相同的单位就可以了。在这里要
使学生了解,通常我们总是用国际单位制,但有时为了研究问
题简便,是可以更为方便的单位的。不要把单位问题搞得
那么死板。

实验后还可以让学生思考以下问题:
\begin{enumerate}
\item 在这个实验中为什么入射小球每次都必须从斜槽的
同一高度滚下?
\item 如果入射小球的质量小于被碰小球,将会发生什么现
象?是否同样可以进行研究?
\end{enumerate}

\subsubsection{用冲击摆测弹丸的速度}

对于这个实验的原理,要使学生理解为什么在弹丸
射入摆锤过程中动量守恒而动能不守恒;在摆锤(连同弹丸一
起)向上摆动的过程中机械能守恒而动量不守恒。这是由于
弹丸进入摆锤的过程中,弹丸受到的冲力和摆锤受到的冲力,
这一对相互作用力是属于系统(弹丸和摆锤所组成)的内力,
而它们的重力由悬线的拉力所平衡,因此在弹丸和摆锤相互
作用时,它们所受到的合外力为零,所以可用动量守恒定律来
计算。在这过程中,弹丸由于克服摩擦阻力做功,一部分动能
将转化为内能,所以不能用动能守恒来计算碰撞后的共同
速度。

当摆锤获得速度和弹丸一起运动后,可以把弹丸和摆锤
看成是一个整体,在它们高度上升的过程中,悬绳拉力不做
功,只有重力做功,所以机械能守恒,因此,它们在获得速度、
开始运动时的动能,在到达最高位置时将完全转化成重力
势能。

在进行实验时,要注意以下几点:
\begin{enumerate}
\item 实验前应先将冲击摆装置调水平,要注意在调节悬挂
摆锤的四根细线时,必须使它们的长度相等,这样才能使摆锤
在上升时保持平动。在调节时并且要使摆锤的上沿与刻度盘
上画出的水平虚线对齐,右侧边与偏转角度的零刻度线对齐,
以保证弹丸能够射入摆锤孔内。
\item 应提请学生注意,利用公式$h=\ell(1-\cos\theta)$计算摆锤
的上升高度时,式中的摆长$\ell$是摆线的长度,即悬点与摆锤上
沿之间的距离,而不是悬点与摆锤中心之间的距离。对于为
什么要这样计算摆长的道理,可让学生自己考虑。
\item 为减少机械能损失,调节好摆锤的起始位置后,可使
用弹簧枪的第一档发射速度先试射一二次,观察指针偏转的
最大角度,实验时,使指针预先停留在较小角度上(譬如最大
偏角为$40^{\circ}$, 可以把指针先放在$35^{\circ}$的位置上),然后再进行
发射,读出指针的最大偏角$\theta$. 在使用弹簧枪另外两档发射
速度时也应先进行试射。
\end{enumerate}

对有兴趣的学生可以启发思考下面两个问题
\begin{enumerate}
\item 使用弹簧枪的不同档来发射弹丸时,为什么弹丸的速
度会不相同?
\item 测出弹丸的速度后,如何来计算由弹丸和摆锤组成的
系统的机械能损失(用百分数表示),用测得的数据具体计算
一下,把所算出的结果跟比值$\dfrac{M}{M+m}$
比较一下,看看它们之间
有什么联系?
\end{enumerate}

\subsection{课外实验活动}
\subsubsection{观察反冲现象}
把包装香烟用的铝箔浸湿后,将它反面的一层薄纸用干
布揩去,剪成宽约5厘米、长约20厘米的一块,卷在圆珠笔的
笔芯上做成一个空心铝管。在
封口处浆糊粘牢,用二根细
线将它水平地悬挂起来,在铝
管两端分别插入两根火柴(有
火药的一端向里),使它们刚刚
接触(图8.7),然后点燃一根火柴,在铝管中部加热,当火柴即将燃尽时,由于铝管温度升高,
管内的火柴已达燃点,燃烧产生的气体推动两根火柴向相反
方向从管子两端飞出,若只有一根火柴飞出(另一根与铝管
烧结在一起),则铝管将向反方向摆动。应当注意,做实验时,
人要站在面对空心铝管侧面的位置。

\begin{figure}[htp]
    \centering
    \includegraphics[scale=.6]{fig/8-7.png}
    \caption{}
\end{figure}



\section{习题解答}

\subsection{练习一}
\begin{enumerate}
    \item 用4牛的力推动一个物体,力的作用时间是0.5秒,力的冲量是多少?

    \begin{solution}
        冲量$Ft=4\x0.5=2{\rm N\cdot s}$
    \end{solution}
    \item 使质量为4吨的汽车,从静止达到20$\kmh$的速度,需要多大的冲量?

    \begin{solution}
        汽车动量的变化
\[ p'-p=mv-0=4\x10^3\x\frac{20\x10^3}{3600}
        =2.22\x10^4{\rm kg\cdot m/s}\]       
        所以根据$Ft=mv$, 需要的冲量为$2.22\x10^4{\rm kg\cdot m/s}$.
    \end{solution}
    \item 质量是25千克以0.5$\ms$的速度步行的小孩和质量是0.02千克以800$\ms$的速度飞行的子弹,哪个动量大?

    \begin{solution}
小孩的动量
\[p_1=m_1v_1=25\x0.5=12.5{\rm kg\cdot m/s}\]
子弹的动量
\[p_2=m_2v_2=0.02\x800=16{\rm kg\cdot m/s}\]
所以飞行的子弹动量较大。
    \end{solution}
    \item 质量为8克的玻璃弹球以3$\ms$的速度向左运动,碰到一个物体后弹回,以2$\ms$的速度沿同一直线向右运动,弹球的动量改变了多少?

    \begin{solution}
        若以向右运动的方向为正,则动量的改变:
       \[ p'-p=mv'-mv=8\x10^{-3}\x2-8\x10^{-3}\x(-3)=4\x10^{-2}{\rm kg\cdot m/s}\]
       $ p'-p$的值为正,说明动量的改变方向向右。
    \end{solution}
    \item 以相同的速度分别向竖直和水平方向抛出两个质量相等的物体,抛出时两个物体的动能是否相等?动量是否相等?

    \begin{solution}
        动能相等,动量不等。因为动能是标量,与方向无关,而动量是矢量,方向不同,动量就不等。
    \end{solution}
\end{enumerate}


\subsection{练习二}
\begin{enumerate}
    \item 10千克的物体以10$\ms$的速度作直线运动,在受到一个恒力作用4.0秒钟后,速度变为反向2.0$\ms$.求:
     \begin{enumerate}
        \item 物体在受力前和受力后的动量;
        \item 物体受到的冲量;
        \item 力的大小和方向.
    \end{enumerate}

    \begin{solution}
物体原来速度$v_1=10\ms$,则受到力$F$作用后,速
度变为$v_2=-2.0\ms$.
\begin{enumerate}
\item 受力前物体的动量
\[p_1=mv_1=10\x10=100{\rm kg\cdot m/s}\]
受力后的动量
\[p_2=mv_2=10\x(-2.0)=-20{\rm kg\cdot m/s}\]
\item 根据动量定理,物体受到的冲量
\[Ft=p_2-p_1=-20-100=-120{\rm kg\cdot m/s}-120{\rm N\cdot s}\]
负号表示冲量的方向与原来的速度方向相反。
\item $F=\dfrac{Ft}{t}=\dfrac{-120}{4.0}=-30{\rm N}$
负号表示力的方向与原来的速度方向相反。
\end{enumerate}

    
    \end{solution}
    \item 列车的质量是$2.5\times 10^6$千克,受到的牵引力是$4.0\times 
    10^5$牛,它的速度由10$\ms$增加到24$\ms$需要用多少时间?

    \begin{solution}
        根据动量定理$Ft=mv'-mv$,
\[t=\frac{mv'-mv}{F}=\frac{ 2.5\x10^6\x24-2.5\x10^6\x10}{4.0\x 10^5}= 87.5{\rm s}\]      
列车的速度从10$\ms$增加到24$\ms$需要87.5秒。
    \end{solution}
    \item 一个质量是65千克的人从墙上跳下,以7$\ms$的速度着地,与地面接触后0.01秒停了下来,地面对他的作用力是多大?如果他着地时弯曲双腿,用了1秒钟才停下来,地面对他的作用力又是多大?

    \begin{solution}
若取向上为正方向,则应用动量定理,可得出
\[Ft=0-mv\]
式中$v=-7\ms$.如果落地时力的作用时间为$t=0.01$秒,
地面对人的作用力为
\[F=\frac{-mv}{t}=\frac{-65\x (-7)}{0.01}=4.55\x 10^4{\rm N}\]
说明:上述解答忽略了人所受的重力,这是由于人所受的重力
为$mg=65\x9.8=637{\rm N}$,与上面算出的作用力相比,还不
到2\%, 所以是可以忽略的。但在下面的情况下,人所受的重
力不能忽略。

当人着地时双腿弯曲,力的作用时间为$t'=1$秒,此时
必须考虑到是合力的冲量使动量发生变化,所以若以$F'$表示
地面作用力,则
\[(F'-mg)t'=0-mv\]
\[F'=\frac{-mv}{t'}+mg=\frac{-65\x (-7)}{1}+65\x 9.8=1.1\x 10^{3}{\rm N}\]
    \end{solution}
    \item 跳远时,为什么跳在砂坑里比跳在混凝土路面上安全?钉钉子时,为什么要用铁锤而不用橡皮锤?


    \begin{solution}
    跳远时,从着地到停止下来所经过的时间,跳在沙坑
里比跳在混凝土路面上要长,根据动量定理可知,在动量变化
一定的情况下,跳在沙坑里的平均作用力就较小,比较安全。

钉钉子时,铁锤的质量较大,可以有较大的动量。又因铁
锤比较坚硬,与钉子接触的时间短.根据动量定理,
\[Ft=0-mv,\qquad F=\frac{-mv}{t}\]
因为$mv$较大,$t$较小,所以$F$就很大,容易把
钉子钉入木块等物中去。相反,若用橡皮锤,作用力就较小。
    \end{solution}
    \item 质量为4千克的铅球和质量为0.1千克的皮球以相同的速度运动着,要使它们在相同的时间内停下来,作用在铅球上的力和作用在皮球上的力哪个大?为什么?


    \begin{solution}
    铅球与皮球的速度相同,因为铅球的质量大,所以它
的动量大。要使它停下来,动量的变化也大。根据动量定理,冲
量也必须大。又由于作用时间相同,所以对铅球的作用力应该
比对皮球的作用力大。
    \end{solution}
\end{enumerate}


\subsection{练习三}
\begin{enumerate}
    \item 两个原来静止的在水平面上挨在一起的小车,质量分别是0.5千克和0.2千克,在弹力作用下分开,较重的小车以0.8$\ms$的速度向右运动,求较轻的小车的速度.

    \begin{solution}
因为系统的合外力为零,所以动量守恒。
\[m_1v_1+m_2v_2=0\]
如果以向右的方向为正方向.则根据题意,$m_1=0.5$千克,
$m_2=0.2$千克,$v_1=0.8$米/秒.代入方程,可解得,
\[v_2=\frac{-m_1v_1}{m_2}=\frac{-0.5\x 0.8}{0.2}=-2\ms\]
负号表示较轻的小车的速度与假设的正方向相反,即向左。
    \end{solution}
    \item 在气垫导轨上,一个质量为600克的滑块以15${\rm cm}/{\rm s}$的速度赶上另一个质量为400克速度为10${\rm cm}/{\rm s}$的滑块而发生碰撞,碰撞后两个滑块并在一起,求两个滑块碰撞后的速度.

    \begin{solution}
根据题意,滑块质量分别为$m_1=600{\rm g}=0.6{\rm kg}$,
$m_2=400{\rm g}=0.4{\rm kg}$,$v_1=15{\rm cm/s}=0.15\ms$,$v_2=10{\rm cm/s}=0.10\ms$.设碰撞后的共同速度为$v$. 则根据动量
守恒定律,
\[m_1v_1+m_2v_2=(m_1+m_2)v\]
\[v=\frac{m_1v_1+m_2v_2}{m_1+m_2}=\frac{0.6\x 0.15+0.4\x 0.10}{0.6+0.4}=0.13\ms\]
方向跟原来的方向一致。
    \end{solution}
    \item 一个小孩从静止的小船上水平抛出一个球,球的质量是2.0千克,抛出的速度是20$\ms$.如果小孩和船的总质量为100千克,球抛出时船得到的速度是多大?

    \begin{solution}
    设小孩和船的总质量为$M$, 小球的质量为$m$, 小球
抛出的速度$v$即小球抛出时相对于地面的速度。则小孩和船
的速度$V$可由动量守恒定律求出。设$v$的方向为正。
\[mv+MV=0\]
\[V=-\frac{mv}{M}=\frac{-2.0\x 20}{100}=-0.40\ms\]
负号表示$V$的方向与小球抛出方向相反。
    \end{solution}
    \item 质量为10克速度为300$\ms$的子弹,打进质量为
    24克静止在光滑水平面上的木块中,并留在木块里,子弹进入木块后,木块运动的速度多大?如果子弹把水块打穿,穿过木块后子弹的速度为100$\ms$,这时木块的速度多大?

    \begin{solution}
    设子弹质量为$m$, 速度为$v_1$, 木块质量为$M$. 当子
弹打入木块并留在木块内时,子弹和木块有共同速度$V$,则根
据动量守恒定律,有
\[mv=(m+M)V\]
\[V=\frac{mv}{m+M}=\frac{0.01\x 300}{0.01+0.024}=88.2\ms\]
如果子弹穿过木块后有速度$v'=100\ms$,则木块速度
$V'$可由下式求得
\[mv= mv'+ MV'\]
\[V'=\frac{mv-mv'}{M}=\frac{0.01\x300-0.01\x100}{0.024}=83.3\ms\]
    \end{solution}
    \item 光滑的水平面上停着一辆平车,有两个人在车上相向而行,在什么情况下平车保持静止?在什么情况下平车要运动,运动的方向由什么决定?

    \begin{solution}
        两人动量的大小相等时,平车不动;两人动量的大小
        不等时,平车就要运动。平车运动的方向跟动量小的人的运动
        方向相同。这是因为,两人动量与平车的动量之和应该守恒,
        即为零,因此,平车的动量应与两个人的合动量的大小相等方
        向相反,而两人合动量的方向决定于哪个人的动量大,所以平
        车的动量方向应与动量小的人的动量方向相同。
    \end{solution}
\end{enumerate}


\subsection{练习四}
\begin{enumerate}
\item 两个质量都是3千克的球,各以6$\ms$的速率相向运动,发生正碰后每个球都以原来的速率向相反方向运动.它们的碰撞是弹性碰撞吗?为什么?


\begin{solution}
    是弹性碰撞.两个小球虽然碰撞前后运动方向都发
    生变化,但速度大小不变,所以动能不变,由于动能守恒,所以
    是弹性碰撞。
\end{solution}
\item 一个1.5千克的物体原来静止,另一个0.5千克的以0.2$\ms$的速度运动的物体与它发生弹性正碰,求碰撞后两个物体的速度.

\begin{solution}
根据题意,$m_1=0.5$千克,$v_1=0.2$米/秒,$m_2=1.5$千
克,$v_2=0$. 因为两球发生弹性碰撞,因此满足动量守恒和动能
守恒:
\[\begin{cases}
    m_1v_1=m_1v_1'+m_2v_2'\\
    \frac{1}{2}m_1v_1^2=\frac{1}{2}m_1{v'_1}^2+\frac{1}{2}m_2{v'_2}^2\\
\end{cases}\]
解得,
\[\begin{split}
v'_1&=\frac{m_1-m_2}{m_1+m_2}v_1=\frac{0.5-1.5}{0.5+1.5}\x 0.2=-0.1\ms\\
v'_2&=\frac{2m_1}{m_1+m_2}v_1=\frac{2\x 0.5}{0.5+1.5}\x 0.2=0.1\ms\\
\end{split}\]
$v'_1$为负值,说明质量较小的物体碰撞后速度的方向与原来
相反。
\end{solution}
\item 甲乙两物体在同一直线上同向运动,甲物体在前,乙物体在后,甲物体质量为2千克,速度是1$\ms$;乙物体质量为4千克,速度是3$\ms$.乙物体追上甲物体发生正碰后,两物体仍沿着原来的方向运动,而甲物体的速度变为3$\ms$,乙物体的速度变为2$\ms$,这两个物体的碰撞是弹性碰撞吗?为什么?

\begin{solution}
    设甲物体的质量为$m_1$, 它在碰撞前后的速度为$v_1$、$v_1'$
    乙物体的质量为$m_2$, 它在碰撞前后的速度为$v_2$、$v'_2$. 要
    判断是否弹性碰撞,可检验其碰撞前后的总动能是否守恒。
    
    碰撞前的总动能:
\[\frac{1}{2}m_1v_1^2+\frac{1}{2}m_2v_2^2=\frac{1}{2}\x 2\x 1^2+\frac{1}{2}\x 4\x 3^2=19{\rm J}\]
碰撞后的总动能:
\[\frac{1}{2}m_1{v'_1}^2+\frac{1}{2}m_2{v'_2}^2=\frac{1}{2}\x 2\x 3^2+\frac{1}{2}\x 4\x 2^2=17{\rm J}\]
可见,总动能不守恒,不是弹性碰撞。
\end{solution}
\item 在课文第六节的(8.6)式中,如果$m_2\gg m_1$,就得到$v'_1\approx -v_1,\; v'_2\approx 0$.这组解的物理意义是什么?


\begin{solution}
    在弹性碰撞方程组中解得的末速度公式[(8.6)式]是:
    \[\begin{cases}
        v'_1=\dfrac{m_1-m_2}{m_1+m_2}v_1\\
        v'_2=\dfrac{2m_1}{m_1+m_2}v_1\\
        \end{cases}\]
当$m_2\gg m_1$时,可得$v'_1\approx -v_1$, $v'_2\approx 0$. 这说明:当质量很小的物
体去与质量很大的静止物体发生正碰时,小物体将原速弹回,而大物体几乎不动。
\end{solution}
\end{enumerate}



\subsection{习题}
\begin{enumerate}
    \item 质量为1千克的手榴弹以60$^\circ$角斜抛出去,抛出的速度为10$\ms$,手榴弹到达最高点时炸成两块,一块的质量是0.6千克,以15$\ms$的速度沿原方向运动,求另一块的速度大小和方向.

\begin{figure}[htp]
    \centering
    \includegraphics[scale=.4]{fig/8-8.png}
    \caption{}
\end{figure}

    \begin{solution}
手榴弹以60$^\circ$角斜抛出去,达最高点时的速度
\[v=v_0\cos\theta=10\x\cos60^{\circ}=5\ms\]
(图8.8)。此时手榴弹炸
成两块,爆炸前后的动量应该守恒。(它们所受的重力与爆炸
力相比可忽略不计)。设爆炸后沿原方向运动的一块质量为
$m_1$, 速度为$v_1$, 另一块的质量为$m_2$, 速度为$v_2$. 以原运动方向
为正方向,则
\[mv=m_1v_1+m_2v_2\]
\[v_2=\frac{mv-m_1v_1}{m_2}=\frac{1\x 5-0.6\x 15}{0.4}=-10\ms\]
负号表示方向与手榴弹在最高点的速度方向相反。
    \end{solution}
    \item 对于在一直线上运动的两个物体组成的系统,动量守恒定律的一般表达式为:
\[m_1v_1+m_2v_2=m_1v'_1+m_2v'_2 \]
    在不同情况下,这个表达式往往可以简化为不同形式,试写出下列各种情况下得出的简化的表达式:
\begin{enumerate}
    \item 两个物体原来静止,发生相互作用后分开;
    \item 一个物体原来静止,另一个物体跟它碰撞后粘合在一起并共同沿原来的方向运动;
    \item 一个物体原来静止,另一个运动物体与它正碰后,两物体以不同的速度在原来的直线上运动;
    \item 两个相向运动的物体,相碰后都静止下来.
\end{enumerate}

\begin{solution}
\begin{enumerate}
    \item $0=m_1v_1'+m_2v_2'$
\item 设$m_2$原来静止,则
\[m_1v_1=(m_1+m_2)v\]
式中$v$为粘合后的共同速度。
\item $m_1v_1=m_1v_1'+m_2v'_2$
\item $m_1v_1+m_2v_2=0$
以上几式中的$v_1,v_2,v_1',v_2',v'$等速度的正负要根据与假定正
方向的一致或相反来确定。

\end{enumerate}
\end{solution}
\item 试证明:两个物体碰撞后,它们的速度变化$\Delta v_1=v'_1-v_1$和$\Delta v_2=v'_2-v_2$跟它们的质量成反比,即
\[\frac{\Delta v_1}{\Delta v_2}=-\frac{m_2}{m_1}\]
并利用所得结果来讨论:很轻的物体(如乒兵球)跟一个很重的物体(如课桌)碰撞后,它们的速度变化有什么特征.

\begin{proof}
    两物体碰撞,动量守恒。据两个物体组成系统
    的动量守恒定律一般表达式
    \[m_1v_1+m_2v_2=m_1v'_1+m_2v'_2\]
    移项得
   \[\begin{split}
       m_1(v'_1-v_1)&=-m_2(v'_2-v_2)\\
       \frac{v'_1-v_1}{v'_2-v_2}&=-\frac{m_2}{m_1}
   \end{split} \]
    即
\[\frac{\Delta v_2}{\Delta v_1}=-\frac{m_2}{m_1}\]
\end{proof}
\item 质子的质量是$1.67\times 10^{-27}$千克,速度为$1.0\times 10^7\ms$,与一个静止的氦核碰撞后,质子以$6.0\times 10^6\ms$的速度反弹回来,氦核以$4.0\times 10^6\ms$的速度向前运动.
   \begin{enumerate}
       \item 你能否求出氦核的质量?如果能,是多少?
       \item 你能否求出碰撞时的相互作用力?为什么?
   \end{enumerate}

   \begin{solution}
\begin{enumerate}
    \item 能。可以用动量守恒定律求出。设质子质量为
    $m_1=1.67\x10^{-27}$千克,速度$v_1=1.0\x10^7$米/秒.碰撞后质
    子反弹,速度为$v'_1=-6.0\x10^6$米/秒.氦核的质量为$m_2$, 碰
    撞后速度$v'_2=4.0\x10^6$米/秒.则
\[m_1v_1=m_1v_1'+m_2v_2'\]
\[\begin{split}
    m_2&=\frac{m_1v_1-m_1v_1'}{v_2'}\\
    &=\frac{ 1.67\x10^{-27}\x1.0\x10^7-1.67\x10^{-27} \x(-6.0\x10^6)}{4.0\x10^6}\\
    &=6.68\x10^{-27}{\rm kg}
\end{split}
\]
    \item 不能。因为根据质子的动量变化可以求得质子受到
    的冲量,但由于作用时间未知,所以无法求得作用力。
\end{enumerate}
   \end{solution}
   \item 两个球以相同的速度相向运动,其中一个球的质量是另一个的三倍,相碰后重球停止不动,轻球以二倍的速率弹回,试证明它们发生的是弹性碰撞.

\begin{solution}
    设轻球质量为$m$, 则重球质量为$3m$. 碰撞前速率
    都是$v$, 碰撞后轻球速率是$2v$, 重球静止,则碰撞前总动能为
   \[\frac{1}{2}mv^2+\frac{1}{2}\x 3mv^2=2mv^2\]
    碰撞后总动能为
\[\frac{1}{2}m(2v)^2=2mv^2\]
    可见,碰撞前后动能守恒,为弹性碰撞。
\end{solution}
   \item 在光滑水平面上一个质量为0.2千克的小球以5$\ms$的速度向前运动,途中与另一个质量为0.3千克静止的小球发生正碰.假设碰撞后第二个小球的速度为4.2$\ms$,你算出的第一个小球的速度是多大?想一想,这种情况真的可能发生吗?这道题的毛病出在哪里?

   \begin{solution}
    如果这种情况真的发生,则碰撞前后一定满足动量
    守恒。因此可用动量守恒定律求得第一个小球碰撞后的速度
    $v'_1$. 设第一个小球的质量为$m_1$, 碰撞前速度为$v_1$, 第二个小球
    的质量为$m_2$, 碰撞后的速度为$v'_2$. 则
\[m_1v_1=m_1v_1'+m_2v_2'\]
\[v'_1=\frac{m_1v_1-m_2v_2'}{m_1}=\frac{0.2\x 5-0.3\x 4.2}{0.2}=-1.3\ms\]
但实际上这种情况是不可能发生的。因为碰撞前的总
动能
\[E=\frac{1}{2}m_1v_1^2=\frac{1}{2}\x 0.2\x 5^2=2.5{\rm J}\]
而碰撞后的总动能为
\[\begin{split}
    E'&=\frac{1}{2}m_1{v'_1}^2+\frac{1}{2}m_2{v'_2}^2\\
    &=\frac{1}{2}\x 0.2\x (-1.3)^2+\frac{1}{2}\x 0.3\x 4.2^2\\
    &=2.8{\rm J}
\end{split}\]
碰撞后的总动能大于碰撞前的总动能是不可能的。这道题的毛病在所给的数据不符合实际情况。
   \end{solution}
   \item 一个质量$M=0.2$千克的小球放在高度$h=5$米的直杆顶端(图8.11),一颗质量$m=0.01$千克的子弹以$v_0=500\ms$的速度沿水平方向击中小球,并穿过球心,小球落地处离杆的距离$s=20$米.求子弹落地处离杆的距离.子弹的动能有多少转化成了热能?
   \begin{figure}[htp]\centering
    \begin{tikzpicture}[>=latex, scale=.8]
    \draw (-1,0)-- (6,0);
    \draw (-0.1,0) rectangle (.1, 5);
    \node at (0, 5.6){$M$};
    \draw [<->](-.75,5.2)--node[fill=white]{$h$}(-.75,0);
    
    \draw[fill=gray] (-2, 5.3) --(-1.7, 5.3) to [bend left=15] (-1.45, 5.2) to [bend left=15] (-1.7,5.1)--(-2,5.1)--(-2,5.3); 
    \node at (-1.7, 5.5){$m$};
    \draw[->](-1.1, 5.2)--node[above]{$v_0$}(-.3, 5.2);
    \draw [|<->|](0,-.3)--node[fill=white]{$s$}(2,-.3);
    \draw [|<->|](0,-.8)--node[fill=white]{$s'$}(5.2,-.8);
    \draw (0,0)--(0,-1);
    \draw (2,0)--(2,-.5);
    \draw (5.2,0)--(5.2,-1);
    
    \draw [dashed] (5.2,0) arc (0:87:5.2);
    
    \draw [dashed] plot[domain=0:2, samples=100] function{-1.3*x*x+5.2} ;
    \draw [fill=gray] (0,5.2) circle (.2);
    \draw [dashed, fill=white] (2,.2) circle (.2);
    \draw [dashed, fill=white] (4.9+.2,.55)--(5.1+.2,.55)--(5.1+.2,.25) to [bend left=15](5+.2,0) to [bend left=15](4.9+.2,.25)--(4.9+.2,.55);
    \end{tikzpicture}
    \caption{}
    \end{figure}

    \begin{solution}    
    根据小球落地点离杆
的距离$s$, 利用平抛运动规律,
可求出小球在碰撞后的速度$V$.
\[V=\frac{s}{t}=\frac{s}{\sqrt{2h/g}}=\frac{20}{\sqrt{\dfrac{2\x 5}{9.8}}}=20\x\sqrt{0.98}=19.8\ms\]
再根据动量守恒定律求得子弹在穿过小球后的速度$v'$.
\[mv_0=mv'+MV\]
\[v'=\frac{mv_0-MV}{m}=\frac{0.01\x500-0.2\x19.8}{0.01}=104\ms\]
再根据平抛运动规律求出子弹落地点离杆的距离$s'$, 
\[s'=v't=v'\x\sqrt{\frac{2h}{g}}=104\x \sqrt{\frac{2\x 5}{9.8}}=105{\rm m}\]
设转化为热能的能量为$E$, 则根据能量守恒:
    \[\begin{split}
E&=\frac{1}{2}mv_0^2-\frac{1}{2}m{v'}^2-\frac{1}{2}MV^2\\
&=\frac{1}{2}\x 0.01\x 500^2-\frac{1}{2}\x 0.01\x 104^2-\frac{1}{2}\x 0.2\x 19.8^2\\
&=1.16\x 10^3{\rm J}        
    \end{split}
        \]


    \end{solution}

   \item 略(课本已作解答)。

\item 在上题中,如果宇航员想以最短的时间返回飞船,他开始最多能释放出多少氧气?这时他返回飞船所用的时间是多少?

\begin{solution}
要使返回时间最短,就要使开始释放的氧气最多。这
样反冲速度大,返回时间短,但释放氧气后的剩余氧气又必
须足够字航员在途中呼吸所用。其极端的情况就是所剩的氧
气正好够宇航员途中呼吸,即$m+m_{\text{吸}}=m_{\text{总}}$, 式中$m$就是开始
喷出的氧气质量。根据上题分析,宇航员的反冲速度为$V$,
而$V=-mv/M$。
返回时间
\[t=\frac{d}{V}=-\frac{Md}{mv}\]
在这段时间内宇航员吸氧气
\[m_{\text{吸}}=m_{\text{总}}-m=Rt\]
所以,
\[m_{\text{总}}-m=Rt=-R\frac{Md}{mv}\]
整理得
\[m^2-m_{\text{总}}m-\frac{RMd}{v}=0\]
所以,
\[m=\frac{1}{2}\left(m_{\text{总}}+\sqrt{m^2_{\text{总}}+\frac{4RMd}{v}}\right)\]
(因$m$应取较大值,所以舍去根号前的负号解)。代入数据$m_{\text{总}}=0.5{\rm kg}$
\[\frac{RMd}{v}=\frac{2.5\x 10^{-4}\x 100\x 45}{-50}=-0.0225{\rm kg^2}\]
解得:$m=0.45{\rm kg}$

根据题意,要使宇航员返回时间最短,开始时应释放氧气0.45
千克。宇航员返回时间为
\[t=\frac{d}{V}=-\frac{Md}{mv}=-\frac{100\x 45}{0.45\x (-50)}=200{\rm s}\]

说明:上述返回时间的答案可以验证。看看在这段时间
里宇航员呼吸氧气有没有问题。如果从吸完氧气所需的时间
来计算,则
\[t=\frac{m_{\text{总}}-m}{R}=\frac{0.5-0.45}{2.5\x 10^{-4}}=200{\rm s}\]
与上述答案是一致的。
\end{solution}
\item 速度为$10^5{\rm cm}/{\rm s}$的氦核与静止的质子发生正碰,氦核的质量是质子的4倍,碰撞是弹性的,求碰撞后两个粒子的速度.

\begin{solution}
设氦核质量为$m_1$, 速度为$v_1$, 质子质量为$m_2$. 已知
$m_1=4m_2$. 根据弹性碰撞的动量守恒和动能守恒,列出方程:
\[\begin{cases}
  m_1v_1=m_1v_1’+m_2v_2'\\
\frac{1}{2}m_1v_1^2=\frac{1}{2}m_1{v_1'}^2+\frac{1}{2}m_2{v_2'}^2 
\end{cases}\]
解得碰撞后氦核速度
\[v'_1=\frac{m_1-m_2}{m_1+m_2}v_1=\frac{3m_2}{5m_2}v_1=\frac{3}{5}v_1=\frac{3}{5}\x 10^5=6\x10^4{\rm cm/s}\]
质子速度 
\[v'_2=\frac{2m_1}{m_1+m_2}v_1=\frac{8m_2}{5m_2}v_1=\frac{8}{5}v_1=\frac{8}{5}\x 10^5=1.6\x 10^5{\rm cm/s}\]
$v_1'$、$v_2'$都与$v_1$的方向相同.
\end{solution}
\item 一个质量是$m_1$,动能是$E_K$的物体与一个质量是$m_2$的不动的物体正碰,假定发生的是弹性碰撞,在$m_1=0.01m_2$,$m_1=m_2$,$m_1=100m_2$的情况下,$m_1$传递给$m_2$的动能各是多少?

(有兴趣的同学还可以进一步讨论$m_1$传递给$m_2$的动能最大或最小的条件).

\begin{solution}
    设$m_1$的原速度为$v_1$, 碰撞后两物体的速度分别为
    $v_1'$、$v_2'$。则根据弹性正碰的特点列出方程:
\[\begin{cases}
  m_1v_1=m_1v_1’+m_2v_2'\\
\frac{1}{2}m_1v_1^2=\frac{1}{2}m_1{v_1'}^2+\frac{1}{2}m_2{v_2'}^2 
\end{cases}\]
解得:
\[v'_1=\frac{m_1-m_2}{m_1+m_2}v_1,\qquad v'_2=\frac{2m_1}{m_1+m_2}v_1\]
$m_1$传递给$m_2$的动能
\[\begin{split}
    E'_{k_2}-0=E'_{k_2}&=\frac{1}{2}m_2{v_2'}^2=\frac{1}{2}m_2\left(\frac{2m_1}{m_1+m_2}\right)^2 v_1^2\\
    &=\frac{1}{2}m_1v_1^2\cdot \frac{4m_1m_2}{(m_1+m_2)^2}\\
    &=\frac{4m_1m_2}{(m_1+m_2)^2}E_k
\end{split}\]

\begin{itemize}
\item 当$m_1=0.01m_2$时,
\[E'_{k_2}=\frac{4\x 0.01m_2^2}{1.01^2m_2^2}E_k=0.039E_k\]
说明传递给$m_2$的动能只占$m_1$原动能的3.9\%.
\item 当$m_1=m_2$时,$E'_{k_2}=E_k$. 说明$m_1$的动能全部传递给$m_2$.
\item 
当$m_1=100m_2$时,
$$E'_{k_2}=\frac{400m^2_2}{101^2\cdot m^2_2}E_k=0.039E_k$$
说明传递给$m_2$的动能也只占$m_1$原动能的3.9\%.
\end{itemize}

$m_1$传递给$m_2$的动能为最大的情况,就是将自己的动能
全部传给$m_2$的情况,即上面所说的$m_1=m_2$的情况。

从式子$E'_{k_2}=\dfrac{4m_1m_2}{(m_1+m_2)^2}E_k$ 可见,
\[E'_{k_2}=\frac{4m_1m_2}{m_1^2+2m_1m_2+m_2^2}E_k=\frac{4}{\dfrac{m_1}{m_2}+2+\dfrac{m_2}{m_1}}E_k\]

\begin{itemize}
    \item 当$m_1\gg m_2$时,$\dfrac{m_2}{m_1}\to 0$, 而$\dfrac{m_1}{m_2}\to \infty$, 所以$E'_{k_2}\to 0$. 
    \item 当
$m_1\ll m_2$时,$\dfrac{m_1}{m_2}\to 0$, 而$\dfrac{m_2}{m_1}\to \infty$, 所以$E'_{k_2}\to 0$
\end{itemize}

所以,当$m_1\gg m_2$或$m_1\ll m_2$时,$m_1$传递给$m_2$的动能最小(等于零)。
\end{solution}
\item 在有些原子反应堆里,要让中子与原子核碰撞,以便把中子的速率迅速降低下来.为此,是选用较重的还是较轻的原子核效果较好?为什么?


\begin{solution}
    要使中子速率迅速降低下来,就要使中子与原子核
    碰撞的过程中将动能传递给原子核。根据上题的讨论,被撞
    原子核的质量越接近中子质量,传递动能越多,中子的速率就
    降低得越快。所以选用较轻的原子核效果较好。
\end{solution}
\end{enumerate}


\section{参考资料}
\subsection{机械运动中动量及动能的区别}

课本里的阅读材料:《笛卡儿和动量守恒定律》中已经提
到动量这个概念是笛卡尔、牛顿先后提出的,并且笛卡儿明确
地把物体的质量和速度的乘积作为物体“运动量”的量度。在
历史上由于他们的影响,在十七世纪四十年代至八十年代,科
学界普遍承认$mv$是机械运动唯一的量度。

在这同一时期内,由于惠更斯对完全弹性碰撞的研究,得
出了“各个质量和各个速度的平方乘积之和,在碰撞前后不
变”的结论。

1686年德国数学家莱布尼兹通过对落体运动的分析,认
为物体的质量和速度平方的乘积——活力——才是机械运动
的真正量度,从而与笛卡尔的主张展开了争论。

关于两种运动量度的争论,持续了近二百年,许多著名的
数理学家参加到争论中。后来随着力学本身的发展,人们对
这两种量度取得了清楚的认识。

牛顿第二定律
\[\frac{\dd(mv)}{\dd t}=F\]
所表现的只是运动的原因(力)
和结果(动量变化)之间的瞬时关系。如果考察力在一段时间
内的累积效应,可由上式得出:
\[\dd(mv)=F\cdot \dd t\]
\[\therefore\quad mv_2-mv_1=\int^{t_2}_{t_1}F\cdot \dd t\]
这就是动量定理:在一段时间内物体动量的变化,等于物体在
同一时间内所受外力的冲量,如果要根据物体在力的作用下
所通过的距离来考察力的作用效果,即力的空间积累效应,则
可得出:
\[\begin{split}
    F\cdot \dd s&=\frac{\dd(mv)}{\dd t}\cdot \dd s\\
    \int^{t_2}_{t_1}F\cdot \dd s&=\int^{t_2}_{t_1}v\cdot \dd (mv)
\end{split}\]
\[\therefore\quad \frac{1}{2}mv^2_2-\frac{1}{2}mv^2_1=\int^{t_2}_{t_1}F\cdot \dd s\]
这就是动能定理:物体动能的增加,等于外力对物体所做
的功。

所以,在力学中动量的变化表现着力的时间累积效应,动
量的变化与外力的冲量相等;动能的变化表现着力的空间累
积效应,动能的变化与外力的功相等。动量是与冲量密切联
系着的,动量决定物体反抗阻力能够移动多么久;动能是与
功密切联系着的,动能决定物体反抗阻力能够移动多么远。


\subsection{动量守恒定律的适用范围比牛顿运动定律广}
动量守恒定律比牛顿运动定律的适用范围要广。近代的
科学实验和理论分析都表明:在自然界中,大到天体的相互作
用,小到质子、中子等基本粒子间的相互作用都遵守动量守恒
定律。

在天文学中发现过这样一种现象:在太空的某个地方有
时会突然发出非常明亮的光,这就是超新星,可是它很快就
暗淡下来,经过几十个昼夜亮度就会减弱一半,光要从这样
一颗超新星出发到达地球需要几百万年,而相比之下超新星
从发光到熄灭的时间就显得太短了,在光到达我们这里以前,
超新星早已烧光了。

当光从超新星到达地球时,它给地球一个轻微的推动,而
与此同时地球却无法给超新星一个轻微的推动,因为它已消
失了,因此,如果我们想象一下超新星与地球之间的相互作
用力,在同一瞬间也就不是什么大小相等,方向相反了。此时,
牛顿第三定律显然已不适用了。

虽然如此,动量守恒定律还是正确的。不过,我们必须把
光也考虑在内。当超新星发射光时,星体反冲,得到动量,同
时光也带走了大小相等、方向相反的动量。经过几百万年光
到达地球时,光把它的动量给了地球。这里要注意的是:动量
不仅可以为实物所携带,而且可以以辐射的方式传递动量,当
我们考虑到这点时,动量守恒定律还是正确的。

\subsection{相对论的动量}

在牛顿力学里,动量定义为$mv$, 质量$m$是个不变的量。根
据牛顿第二定律,一个恒定的力,持续作用于一个物体,可以
使该物体有任意大的高速度。但是在现实中,真空中的光速是
极限速度,并且在任何条件下物体的速度都不可能超过真空
中的光速。因此,在高速运动时,认为质量,以及动量,是与速
度无关的,是不正确的。

相对论告诉我们,在高速运动时质量不再是一个不变的
量,而是随着运动的速度接近光的速度$c$而增大,如果用$m_0$
表示静止物体的质量,则以速度$v$运动的物体的质量$m$可以
用下式表示:
\[m=\frac{m_0}{\sqrt{1-\dfrac{v^2}{c^2}}}\]
相对论的动量仍定义为,
\[p=mv=\frac{m_0v}{\sqrt{1-\dfrac{v^2}{c^2}}}\]
在采用这样定义的情况下,牛顿本人所用的第二定律的表
达式
\[\frac{\dd p}{\dd t}=F,\qquad F=\frac{\dd}{\dd t}\cdot\frac{m_0v}{\sqrt{1-\dfrac{v^2}{c^2}}}\]
在接近光速的情况下也同样适用了,因为随着运动速度的增
大,决定物体惯性大小的质量也增大。当$v\to c$时,$m\to \infty$, 所以加速度趋于零,不论力作用多长时间,速度也不会超过
光速。

对于静止质量$m_0=0$, 而速度为$c$的光子来说,它的动量$p=E/c$
可以这样推得:

因为$E=mc^2$,所以
\[\begin{split}
    E^2=m^2c^4&=m^2c^4-m^2v^2c^2+m^2v^2c^2\\
    &=m^2c^4\left(1-\frac{v^2}{c^2}\right)+p^2c^2=m^2_0c^4+p^2c^2
\end{split}\]
在$v\to c$, $m_0\to 0$时,$E^2=p^2c^2$, 所以
\[p=\frac{E}{c}\]




\chapter{机械振动和机械波}
\minitoc[n]
\section{教学要求}


这一章在前面学过的知识基础上讲解机械振动和机械
波.为了清楚起见,本章教材分为三部分:第一部分讲机械振
动,第二部分讲机械波,第三部分讲声学初步知识.

这一章的教学要求是:
\begin{enumerate}
\item 了解振动产生的条件,理解回复力的概念.
\item 理解振幅、周期和频率等概念的意义.了解相和相差
的概念,知道什么是同相和反相.
\item 从受力情况、速度和加速度、能量几个方面明确简谐
振动的特点;掌握简谐振动的周期公式.
\item 了解振动图象的物理意义.
\item 了解自由振动和受迫振动的意义,明确产生共振的
条件.
\item 理解机械波是怎样产生的,知道什么是横波和纵波.
\item 了解波动图象的物理意义,知道振动图象和波动图象
的区别.
\item 了解波的干涉和波的衍射.
\item 了解声波的产生,了解声波的反射、干涉、衍射以及声
音的共鸣.
\item 了解音调、响度、音品的意义,知道它们各是由什么
决定的.
\item 知道什么是乐音和噪声,了解噪声的危害和控制,知
道什么是超声波和次声波,了解超声波的应用.
\end{enumerate}

下面对这一章的教学内容作些具体说明.

讲解产生振动的条件时,要使学生很好地理解回复力的
概念,知道它是根据力的效果命名的.介绍表征振动的物理
量即振幅、周期和频率时,应注意说明振动有它的特点,需要
:引入新的物理量来描述这种特点,讲解周期的概念时,要着
重说明什么是一次全振动,这是正确理解周期这个概念的
基础.

讲解简谐振动时,应该让学生理解为什么要先研究简谐
振动,再一次说明理想化的方法,要求学生对这种研究方法进
一步有所领会.

简谐振动的周期公式,虽然是就弹簧振子给出的,但对任
何简谐振动都适用;只是对不同的简谐振动,由于受力的性质
不同,$k$的含义也不同.对于单摆,要使学生明确:只有摆角
很小时回复力才跟位移成正比,单摆才做简谐振动.单摆的
周期公式要求学生能够根据简谐振动的一般周期公式自己推
导出来.

相这个概念比较抽象,学生不容易体会它的意义.因此
教材没有给出相的定义,只要求了解:两个简谐振动的振动步
调不一致,就表示它们的相不同,或者说存在着相差.这里,
不讲初相的概念,只要求学生知道什么是同相什么是反相.

由于没有讲振动方程,更需要学生理解振动图象的物理
意义;它表示了振子对平衡位置的位移怎样随时间而变化.还
要求学生明确知道:在振动图象上可以表示出周期和振幅;利
用振动图象还可以比较振动的相.

单摆中能量的转化,在机械能一章中已经讨论过,这里着
重说明单摆的能量跟振幅有关,振幅越大,能量越大,但振动
中能量的转化不要求定量讨论,对于阻尼振动,只要求学生知
道:什么是阻尼振动;在什么条件下可以把阻尼振动作为简谐
振动来处理.

波的概念初学者较难理解,要做好演示,使学生清楚地看
出波是振动的传播,媒质本身并不随波迁移.要求学生对波
的形成有明确的认识,知道振动为什么不会局限在媒质的一
个地方,而要传播出去;知道在振动的传播中,后一个质点总
比前一个质点迟一些开始振动,相邻质点振动的相不同,因而
在整体上看才形成波向前传播.对于横波和纵波,只要求学
生知道,什么是横波,什么是纵波,不要求对它们的传播过程
作过细的分析.

波速的公式$v=\lambda f$是对各种波普遍适用的公式,要求学
生掌握.这里,首先是对波长的概念要有清楚的理解,其次是
知道在一个周期的时间内振动传播的距离等于一个波长.

关于波的图象,要求学生了解它的物理意义,它表示的是
某一时刻各个质点的位移,它是数学图象,只是对横波来说才
直观地表示波形,对纵波的图象,要求学生理解图象的意义
即纵坐标所表示的是各个质点离开平衡位置向左或向右的位
移,不要求仔细地讲述怎样得出这个图象,要求学生能够区
分振动图象和波动图象二者的不同的意义,不要求综合利用
两种图象来分析问题.

关于波的干涉,要求学生知道干涉现象是怎样产生的,即
波峰和波峰(波谷和波谷)相遇处,叠加的结果,振动最强,波
峰和波谷相遇处,叠加的结果,振动最弱.要求学生知道什么
是相干波源.这里提到相差恒定,可以要求学生从波源总保
持同相这种特殊情形来理解.关于波的衍射,只作简单介绍.

第三部分介绍声学的初步知识,重点是介绍乐音的三种
特性,说明它们各是由什么决定的,目前,噪声已成为污染城
市环境的公害之一,教材单列一节讲述噪声的危害和控制,
希望引起学生注意,并知道这方面的简单常识.

“超声波”一节是选讲内容,即使不讲授这节内容,也应该
使学生知道什么是超声波和次声波.

\section{教学建议}
全章分为三个单元.第一单元从第一节至第七节,介绍
了机械振动的基础知识,主要讨论了简谐振动的规律,介绍了
受迫振动和共振的知识,本单元是全章的基础和重点.

第二单元从第八节至第十二节,介绍了机械波的产生、传
播形式,描述手段和波的两个主要特性-干涉和衍射.
第三单元从第十三节到第十七节,介绍了声波的产生、基
本声现象、乐音的三要素和噪声的危害.

\subsection{第一单元}
\subsubsection{机械振动的定义和产生的条件}

为了培养学生科学
的观察和分析能力,可以先学生举一些他们在生活中观察
到的振动的例子,接着教师可展示一组不同物体的振动,例如
可选取弹簧振子、单摆、钟摆的摆轮、一端夹紧的锯条、内燃机
汽缸模型中的活塞、水中的浮子、不倒翁以及天平指针等.启
发他们归纳出这些运动的共同特点即物体或物体的一部分在
某位置附近沿着直线或弧线作往复的周期性运动.接着再让
学生仔细观察一下竖直放置的弹簧振子的振动,分析一下振
子在某位置附近作往复运动的这个“某位置”有什么特点,从
而帮助学生认识课本上指的平衡位置就是振动停止时物体所
在的位置.这样引导学生通过观察,掌握机械振动这种运动
形式的特点.同时也为后面引入产生振动的两个条件作了
准备.

在引入回复力概念时,可先提出前面讲过振动是一种作
用力大小、方向都变的运动.那末振动的物体所受的作用力
有什么共同的特点呢?要求学生分别观察弹簧振子以及水中
浮子的运动,思考物体为什么会作往复运动.在分析物体受
力基础上得出振动物体离开平衡位置后就受到一个指向平衡
位置的力的作用,因此称这个力为回复力,而这个力可以是不
同性质的力或者它们的合力.这样通过典型振动实例的受力
分析来引入回复力的概念,有助于学生认识回复力同向心力
一样,是根据力的效果命名的.

在引入振动的第二个条件时,可先提出为什么前面演示
中的振动物体最终都要停下来?在学生回答有阻力存在后,
可进一步提出如果阻力越来越大会怎样?接着便让学生观察
摆在水中和油中的运动,由此说明阻力过大单摆根本无法实
现往复运动,只有阻力足够小时,才能多次往复运动.

\subsubsection{表征振动的物理量}
在引入振幅概念时,要让学生通
过观察明确振子或摆在振动时,以平衡位置为原点,有一个最
大位移.这个最大位移的绝对值叫做振幅,所以课本中说振
幅是振动物体离开平衡位置最大的距离,而不说最大位移.

学生对周期这一概念并不陌生,首先应该指出振动最基
本的特点是它的周期性,在此基础上,着重帮助学生理解全
振动这一概念,教学中特别要交代清课本中的“位移和速度回
到原来的数值”,所指的“数值”不仅表示它们的大小而且包括
正负.为了使学生掌握振动物体的位移和速度这两个矢量经
过一次往复运动均返回到原先值,就完成一次全振动,有必要
让学生做一次观察练习.用一硬纸板做一红色箭头标志,将
此标志先后放在振子或单摆的不同位移处,让学生反复观察、
明确一次全振动的意义.这样,周期和频率这两个概念和其
相互关系就不难掌握了.为了使学生明确周期和频率是两个
表征振动快慢的物理量,还可让学生观察比较两个摆长相差
较大的单摆的振动,要求学生用脉搏或秒表计时比较一下这
两个摆的周期和频率,从而认识频率越大,周期越小,它们
之间的关系是互为倒数.

教学中还要注意防止学生将“振动的快慢”和“振动物体
运动的快慢”这两种表述混淆起来.因此要点明前者用周期
或频率来描述,对一个确定的振动物体讲是恒定的,而后者用
速度来描写,它是随时间而变的,由此使学生认识振幅、周期、
频率都是从整体上描述振动特点的物理量.

\subsubsection{简谐振动的回复力}

在第二节和第三节中应使学生
对简谐振动回复力的特点、来源以及分析方法有一个逐步深
入的理解.第二节通过弹簧振子的实例引入简谐振动回复力
的表达式$F=-kx$后,应该指出式中回复力与位移的比例常
数,是由振动系统本身结构决定的物理量,应该指出,如果
物体除受回复力作用外,在振动方向上还受其它不平衡力的
作用(如阻力),物体的振动就不是简谐振动了.

\subsubsection{简谐振动的运动规律}

分析简谐振动的规律是教学
中的难点,学生对加速度最大时速度为零,加速度最小时速
度最大往往不易接受,错误地认为,在平衡位置处加速度最
大,在位移最大处加速度最小,这主要在于学生仍没有真正
理解加速度的物理意义以及速度和加速度之间的关系.在分
析简谐振动的规律时,要帮助学生澄清以上错误.

为了培养学生的观察、分析能力,建议在分析振子运动规
律时将课本练习二3作为课堂练习,让学生一边认真观察弹
簧振子(最好选一个$k$较小、$m$较大的振子)的速度随位移变
化的情况,一边将观察结果先填人表中第一项和第四项(回复
力和速度随位移变化的规律).其中振子速度换向时速度为
零,可提醒学生回忆一下竖直上抛物体达到最高点时的速度
等于什么,然后要求学生分别根据已填好的第一和第四项来
判断第三项(加速度随位移变化规律)和第二项(加速度的方
向和大小变化规律)应怎样填写,在以上观察、填表、分析的
基础上,最后再让学生阅读课本283页对简谐振动的分析,作
为小结.

关于$k$和$m$对振动周期的影响可以进行定性演示.演
示时可将振子放在气垫导轨上,让学生用秒表测出多次振动
的平均$T$值,通过比较用同一弹簧不同质量振子的$T$值和
同一振子不同$k$值弹簧的$T$值,使学生具体认识周期$T$随
的增大而减小,随$m$的增大而增大,为学生理解和接受周期
公式做好准备.


还要指出,$T=2\pi\sqrt{\dfrac{m}{k}}$
是一个适用于一切简谐振动系统
的表达式,只是对不同的振动系统,因回复力的性质不同,式
中$k$的形式也不同.对于固有周期与振幅无关,也要通过演
示使学生信服.

\subsubsection{单摆的周期公式}

单摆周期公式可通过实验观察、设
疑、释疑的方式引入,以培养学生探求和分析新问题的能力并
加深对公式中$k$的物理意义的理解.为此,可以首先介绍一
下伽利略发现教堂吊灯振动规律的故事,并用演示说明摆的
振动周期$T$与振幅、摆锤质量无关,而仅与摆长$\ell$有关.这
样,一方面能进一步加深学生对前面讲过的固有振动周期与
振幅无关的认识,另一方面由于$T$与$m$无关的实验结论和上
节学过的简谐振动周期公式中$T\propto\sqrt{m}$形式上的不一致,可以
提出一引导学生进一步探索的新问题.然后指出解决这一
表面上“矛盾”的关键,是找出单摆振动系统的$k$取决于什么
因素.接着通过对单摆回复力的分析,得出单摆的$k=mg/\ell$,
推导出单摆周期公式$T=2\pi\sqrt{\dfrac{\ell}{g}}$,
解决了前面提出的“矛盾”,
并说明理论上的分析推导与演示实验得出的结论是一致的.


\subsubsection{相和相差}

“相”对学生来说是一个抽象和陌生的新概
念,教学时主要应通过演示实验引导学生观察振动的步调是
否一致来认识相和相差的物理意义,而不必引入相的定义.可
先让学生观察两个相同的单摆作振幅相同但步调不一致的振
动.要求学生指出这两个单摆的振动有什么相同和不同的地
方.从分析这两个摆振动的不同之处,重点启发学生认识振
动步调是否一致就是指是否能保持“同时、同向”(同时向一
个方向运动).在此基础上指出为了能对这两个摆的振动情
况分别加以描述,就必须引入一个表示振动步调的物理
量——相.接着可分别演示频率相同的摆同相和反相的振
动.演示反相时可先将两单摆从平衡位置左右两侧同时放
手,然后再让学生考虑一下如果两个单摆从同一侧放手,怎
样实现反相,并试着做一下.

\subsubsection{简谐振动图象}

做好绘制振动图象的演示,使学生理
解振动图象的物理意义,是本节教学的关键.为了增加演示
的可见度,便于边演示边分析,建议对课本图9.6的实验作
适当改进,用投影仪来演示.

将振动曲线视为振动质点的运动轨迹,认为振动物体的
速度方向沿着曲线的切线方向是学生中常见的错误.为了帮
助他们理解振动图线的物理意义,关键是使学生搞清沿着拉
动玻璃板方向的横轴所表示的是时间而不表示振动物体的位
移.演示时,可先使摆振动但不拉动下面的玻璃板,让砂或笔
头在它上面来回划出一条直线.说明振动物体仅仅只在平衡
位置两侧来回运动,但由于各个不同时刻的位移在玻璃板上
留下的痕迹相互重叠而呈现为一条直线.在此可让学生思考
一下如何将不同时刻的位移分别显示出来,接着匀速拉动玻
璃板,结果原先成一条直线的痕迹展开成一条曲线,这样便
清楚显示了不同时刻振动物体的位移,从而说明横轴表示的
是时间.教师还可指出匀速拉动(或转动)记录纸来记录参量
随时间变化的技术,被广泛应用于各种仪器中,例如脑电图、
心电图、温度、压力、地震波记录仪所记录的曲线的横坐标都
表示时间,条件允许还可让学生看一下这些仪器的实物和记
录下的曲线图.

\subsubsection{简谐振动的能量}

在从能量角度对简谐振动进行描
述前,可要求学生复习一下机械能守恒定律及其守恒条件,接
着可让学生阅读课文和观察摆或振子的振动.将练习五作为
课堂练习,让学生当堂巩固、并加深对振动过程中能量转化规
律的理解.

教学中应该说明振动系统的总能量取决于外界提供给振
动系统的能量大小,而与振动系统本身的结构无关.还应该指
出,只有与外界没有能量交换的系统作简谐振动时机械能守
恒,才遵循上述能量转化规律,与外界有能量交换的系统,情
况则不一定如此.

\subsubsection{受迫振动和共振}

在受迫振动演示实验中,要指出
只有当受迫振动达到稳定状态后,其频率才等于策动力的频
率.此时策动力对振动系统做功所传递给系统的能量恰好补
偿系统因阻力而损耗的能量,系统的机械能保持不变,成为等
幅振动.

共振的演示实验,除了可用课本图9.12和图9.13两个
装置来演示外,还可增加几个简单的演示,结合练习六2
布置几个课外实验习作,以扩展学生对共振现象的认识.课
本上关于对共振产生原因的定性解释,也可用单摆代替“秋
千”,分别施加跟单摆振动“合拍”和“不合拍”的推力,让学生
观察振幅的变化,启发学生从能量的角度,根据策动力做正功
和负功去认识共振的成因.

至此本单元已介绍了简谐振动、阻尼振动、自由振动、受
迫振动的概念,学生往往搞不清它们的区别和关系,教师可作
一简单的归纳.

\subsection{第二单元}
\subsubsection{机械波的基本特征}

在引入媒质概念时,可让学生
观察下面的演示:在钟罩中置一发声的电铃,将罩中空气抽空
便无法听到铃声,接着将钟罩中的电铃换成一个大功率的灯
泡,抽空罩中空气接通电源,在罩外仍可看到灯泡发光和感到
灯泡发出的热量,由此模拟太阳光波把光和热送到地球上是
不需要任何媒质的,通过以上演示的比较,再举水波和地震波
的例子说明机械波的基本特征是必须依靠某种媒质来传播.

\subsubsection{机械波为什么会在媒质中传播}

可首先用发波水槽
演示一下槽中水面上的浮子不随水波向前运动,使学生对机
械波是振动的传播而不是媒质的迁移获得初步的感性认识.
接着提出振动为什么不会局限在媒质内一个地方的问题让学
生思考,再慢慢转动波动演示器的手柄,让学生观察沿着波的
传播方向相邻质点依次振动的过程,并从媒质各部分之间存
在相互作用力来分析机械波的成因.

\subsubsection{机械波是怎样在媒质中传播的}

左右甩动放在光滑
地面上的长弹簧和推动水平悬挂的长弹簧,让学生从整体上
初步观察一下弹簧中凹凸相间波和疏密相间波的传播,并提
出这两种波怎么形成的问题加以研究.然后在弹簧上的某处
作出醒目的记号,重复以上演示,要求学生注意观察在波的传
播过程中该处质点作什么运动,看它是否一直向前迁移.接
着在弹簧上不同处分别做不同颜色的记号,要求学生从整体
上仔细观察这几处质点振动时步调是否一致,频率是否相同.
然后再用波动演示器或活动投影幻灯片模拟波的传播,放慢
披的传播速度,再现课本上图9.15和图9.16的动态过程,
让学生进一步证实他们在实际观察中得到的初步结论,通过
以上步步深入的观察,引导学生认识弹簧上每个质点只在它
自己的平衡位置附近振动,不同质点的振动频率相同,但相位
不同,它们在传播方向上依次落后,就形成了在整体上所观察
到的凹凸相间和疏密相间的波.

在以上分析的基础上,最后可引导学生归纳一下振动和
波的区别与联系.明确它们的研究对象不同,振动是波的起
因,波是振动在媒质中的传播,并由此得出形成机械波的两个
条件——波源和媒质.

\subsubsection{机械波也是能量在媒质中的传播}

在阐明机械波的
传播实质上就是能量在媒质中传播时可先提出以下问题让学
生思考:波源质点的振动是否可能是一种无阻尼的自由振动?
通过对这一问题的讨论、分析,帮助学生认识媒质质点间的
相互作用力对波源讲是阻力,对波传播方向上的质点讲它既
是动力又是阻力,因此只有不断供给波源能量,它的振动才能
保持下去,并不断地向外输出能量.而沿着振动传播方向的
媒质质元则起了能量传递者的作用,它不断从前面的质元
获取能量,又把这能量传给后面的质元,(它的能量是随时间
作周期性变化的,这与前面第六节中介绍的孤立质元作无阻
尼振动时机械能守恒的情况不同.)于是波在传递振动形式
的同时,也将波源的能量传递开去.

\subsubsection{波长、波速与频率的关系}
掌握波长的定义的关键,是让学生弄清在波的传播方向
上哪两个点是相邻的同相质点.可以结合课本上图9.15和
用手摇波动器 做演示让学生找一下哪些点是相邻的同相质
点.引入波长定义后,还可要求学生进一步集中注意力观察
一下某时刻波动演示器上横波相邻波峰(或波谷)上的两个质
点或者纵波相邻密部(或疏部)中央的两个质点在振动过程中
是否同相,并可将练习七4作为课堂练习,让学生当堂巩固
对波长定义的理解.

掌握波长、波速和频率关系的关键是要把波长等于一个
周期内振动在媒质中传播的距离交代清楚.为此除了用图
9.15进行分析外,仍可在波动演示器上边演示边说明以达到
加深学生印象的效果.此外还可将波长、周期、波速与步行时
的步长,走一步的时间和步行的速度类比(将频率理解为单位
时间内走几步),以帮助学生理解和掌握这个公式.

\subsubsection{横波图象和纵波图象}

应使学生明确,无论是横波还
是纵波图象,都是表示某一时刻媒质各质点离开平衡位置位
移的函数图象.横波图象直观地表示了波的形状,犹如在某
时刻给波传播方向上全体质元拍的一张“照片”,故称横波图
象为波形曲线.学生较难理解的是纵波图象,因为它不太直
观.教学时,可将练习八3作为课堂练习,此外还可补充一
个问题:纵波密部中央质点和疏部中央质点的位移有何特征?
是否与横波中的波峰和波谷相对应.

\subsubsection{波的图象与振动图象} 

比较波动和振动图象的教学,
可在引入简谐波的概念基础上,引导学生分析讨论以下问题:
这两种图象描述的对象是否相同?纵、横坐标的意义是否相
同?相邻两个位移最大值之间距离所表示的意义是否相同?判
断质点在某时刻运动趋势的方法是否相同?由此将它们在物
理意义上的本质区别作一简要归纳.

\subsubsection{波的独立传播和叠加} 

课本图9.21的演示可用长弹
簧(或灌满铅粒的细橡皮管)代替绳子,将它们放在光滑地面
上同时甩动一下两端,要求学生观察两个脉冲波在相遇时振
幅怎样变化,相遇后是否会消失或改变方向和形状,为了使学
生对波的这一基本性质获得更为具体生动的认识,可在波纹
槽或盛水面盆中用两个手指同时轻点水面,就能看到两列水
波互相穿过,这说明了水波的独立传播.在以上演示基础上
再用运动和位移的合成来解释波的叠加现象,学生便较易接
受和理解了.

\subsubsection{波的干涉} 

波的干涉的教学可采用两种方式处理,第
一种方式可按课本的顺序,在前面介绍波的独立传播和叠加
原理的基础上,先从理论上分析两列频率和相都相同、振动方
向一致的水波叠加后会出现什么现象,然后再用演示实验验
证预期的现象,并得出结论.在演示水波的干涉图样时,用音
叉作为两个相干波源,也可取得较好效果.敲击音叉,将它的
两臂接触水面,用投影仪或直接利用阳光的反射都可将水面
上的干涉图样清晰地显示在屏幕或墙壁上.第二种方式可先
用波纹槽演示两列频率和相都相同的水波叠加后产生的干涉
现象,以激起学生探索兴趣,然后再根据波的传播和叠加原
理从理论上分析形成所观察到的现象的原因.

\begin{figure}[htp]
    \centering
\includegraphics[scale=.6]{fig/9-1.png}
    \caption{}
\end{figure}

在用波的叠加原理分析干涉现象时,最好先让学生用两
条截下的瓦楞纸表示两列横波,如图9.1所示,用两个大头针
将它们的端部分别固定在木板上两点表示两个波源,使两条
瓦楞纸在波源前方交叉,交叉点沿着平行于连接两个波源的
一条直线移动.分别用两种不同记号表示交叉处波峰与波峰
(或波谷与波谷)以及波峰与波谷的叠加.这样,可使学生形
象地认识两列同频率的波迭加所产生的振动加强和削弱互相
间隔的效果,还可启发学生通过仔细的观察思考一下,在振
动最强处和最弱处质点所参与的两个振动的相位关系.讲述
时还可用两张印有许多同心圆的投影片的重叠所产生的视觉
形象来清晰显示课本图9.22所示的两个相干波源产生的干
涉条纹.

\subsubsection{产生波的干涉以及衍射的条件}

首先要求学生明确
产生干涉现象的两列波的振动方向必须一致,关于相干波源
的条件的教学,可用演示实验来说明,两个波源的频率必须相
同.关于要求两个波源相差恒定这一点只需从演示干涉现象
时两个波源保持同相这一特殊情况推广即可,不必再引伸解
释.如果时间允许还可用上面介绍的瓦楞纸教具结合章末习
题9做一下,两个波源频率相同但振动反相是否也会出现干
涉现象的演示(用两枚大头针分别将一条瓦楞纸端部的波峰
和另一条端部的波谷固定在木板上两点,代表两个反相的
波源).

有关波的衍射条件,只需用发波水槽通过演示课本图9.24的情况给出,而不必作理论上的分析解释.应该向学生指
出课本上讲的障碍物或孔的大小尺寸与波长差不多是指产生
明显衍射现象的起码条件,若障碍物或孔的尺寸远小于波长,
无疑是可以产生衍射的,这一点可以通过演示水波通过的孔
缩小至比波长还小时衍射现象越来越明显来说明.

在指出波的干涉、衍射是一切波所特有的现象时可举无
线电波衍射的例子和让学生观察光通过指缝时的衍射现象,
从而使学生获得初步的感性认识.有条件的学校还可用厘米
波发生接收器来演示无线电波通过双缝时的干涉和通过单缝
时的衍射现象.

\subsection{第三单元}
\subsubsection{声源发声时的振动}

在介绍声源是振动物体时,要做
好几个不同的演示,除了课本图9.25的实验外,还可以选用
以下几个演示实验.
\begin{enumerate}
\item 用敲响的音叉接触蒸发皿中的水面,可看到水向外
飞溅.
\item 在鼓面或向上放置的喇叭纸盒上撒一些碎的硬质泡
沫塑料屑,敲响鼓面和使喇叭发声,可看到碎屑的跳动.
\item 弹拨绷紧的橡皮筋或弦可看到橡皮筋和弦的轮廓变
模糊,发音时用手摸咽喉可感到声带的振动.
\end{enumerate}

\subsubsection{声波}

说明声波是声源的振动在媒质中的传递时,可
在一发出低频信号声的喇叭纸盆前置一烛焰,观察烛焰随着
信号声而抖动的现象,从而形象地显示出图9.26所示的声
波是纵波.教学中还可选用以下几个随堂小实验来显示声波
也可在固体、液体中传播以及在不同的媒质中传播的速率
不同.
\begin{enumerate}
\item 将耳朵贴在桌面上,在离耳朵不同距离处用指甲轻敲
桌子.
\item 如图9.2所示,将匙子系在绳子的中间,把绳子的两
端分别用两具手的拇指按在两个耳孔上,敲响匙子,接着松开
按住耳朵的手指,比较前后听到的两种声音有何不同.
\begin{figure}[htp]
    \centering
\includegraphics[scale=.6]{fig/9-2.png}
    \caption{}
\end{figure}

\item 把耳朵紧贴在一个盛满水的塑料袋上,能听到贴在水
袋另一边表的滴嗒声.把塑料水袋取走,比较前后两次听到
的声音有何不同?
\item 将一敲响的音叉放在鱼缸水面的上方,观察缸内鱼的
反应.
\end{enumerate}

\subsubsection{声音的现象} 

讲声波的反射时,要注意讲清人耳能分
清回声的间隔时间与建筑物内交混回响时间的区别,免得学
生把两者混淆起来.

讲解声波的干涉和共鸣现象时,可把音叉发给学生,让他
们自己做课本图9.29的实验,此外还可将面对学生的两个
扬声器(相距约1米)接到同一音频信号发生器上,让学生在
座位上左右晃动身体,便可听到有的地方声音增强,有的地
方声音减弱.在做声音共鸣实验时可用扩音机把共鸣声放大,
使全班同学都能听到.


\subsubsection{乐音和乐音的波形曲线}

学生往往容易把一些声学
名词和概念混淆起来,因此教学中应尽可能使学生通过自身
的听觉和观察来认识和描述人耳对乐音感觉的三个特征以及
它们取决于什么因素.

在介绍课本图9.30乐音的周期性波形曲线时,应说明是
由单一乐器(钢琴)所发出的,并非表示许多乐器或同一乐器
演奏一首乐曲时的波形曲线.此外可以点明这里的波形曲线
实质上也是波源或媒质质点的振动曲线,它的横轴是时间轴,
纵轴是位移轴.后面课本图9.33(乙)和图9.34(乙)所表示的
波形曲线也是如此.有条件的话,最好用示波器演示乐音和
噪声的波形曲线.

\subsubsection{音调和响度}

课本图9.32的演示实验可见度大,效
果较明显,但发声较轻,可用扩音机放大声音.在说明声源振
幅对响度的影响时,应设法增加声源振幅大小的可见度,例如
可轻拨和重拨拉紧的弦或一端夹紧的锯条,使学生看到声音
强弱不同时振幅不同.用课本介绍的音叉或鼓做实验时,可将
敲响的音叉接触通草球或碟子中的水面,在鼓面上撒一些爆
米花来间接显示振幅大小.

在引入声强概念时应使学生明确响度是人们主观上感觉
到的声音强弱,它跟客观上的声强有密切关系.但是,人耳能
听到的最低声强随频率而异.所以,不同频率的声音,即使声
强相同,响度也是不一致的.

\subsubsection{音品}

学生对音品这一概念较难理解,而且容易将它
与音调混淆起来,为此在引入音品这一概念时可播放一段事
先准备好的用不同乐器(如钢琴、小提琴、笛子等)以同一音调
演奏的同一乐曲的录音,让学生辨别 这样能使学生对人耳
之所以能辨别出不同乐器的声音,是由于音品不同,留下一个
鲜明生动的印象.为了使学生了解音品取决于什么因素,关
键是要讲清基音和泛音这两个概念.讲解时最好用示波器来
显示音叉发出的声波是简谐波,而其他乐器发出的声波是类
似图9.33、9.34所示的周期性波.不同乐器发出声音的音调
相同即是指其基音频率相同,而人耳之所以能分辨出不同乐
器发出的声音,是因为它们发出的泛音的多少、频率、振幅不
同所致.

\subsubsection{噪声的危害和控制、超声波}

这两节课可以学生阅读
讨论为主,提高他们的阅读和表达能力.例如让学生列出一
张生活环境中各种噪声源的表,考虑一下有什么控制的方法,
采用讨论的方式,互相交流、补充.此外还可组织看有关超声
波和噪声的教学电影并选择一些杂志上有关的科普文章推荐
给学生看,以开拓他们的知识面.对有兴趣的同学还可组织
他们对周围环境中的噪声(噪声的强弱及噪声源)作些调查,
对噪声治理提出一些建议和办法.

\section{实验指导}
\subsection{演示实验}
\subsubsection{机械振动}
(1)课本图9.1的实验可用倔强系数较小的弹簧来做,
重物的质量可以稍大些,这样可使振动较慢.如果没有$k$值
较小的弹簧,也可以把弹簧取得长一些,这样就相当于$k$值变
小.在演示时可在铁架台上用标记指示重物的平衡位置,以
便于观察重物以这一位置为中心上下往复运动.

(2)也可以用图9.3的装置来演示机械振动,在一乒乓
球的两侧,分别用橡皮胶粘贴
一段细橡皮绳(航模材料),将两
端的橡皮绳固定在两个铁架台
上,调节两个铁架台间的距离,
使得两段橡皮绳都被绷紧,将
乒乓球向上(或向下)拉开一段
距离,释放后,乒乓球就以平衡位置$O$点为中心做上下往复运
动,如果把两个铁架台前后放置,也可以观察乒乓球以平衡位
置为中心的左右往复运动.
\begin{figure}[htp]\centering
    \begin{minipage}[t]{0.48\textwidth}
    \centering
\includegraphics[scale=.7]{fig/9-3.png}
    \caption{}
    \end{minipage}
    \begin{minipage}[t]{0.48\textwidth}
    \centering
\includegraphics[scale=.7]{fig/9-4.png}
    \caption{}
    \end{minipage}
    \end{figure}

    (3)如图9.4所示,在大量筒(800—1000毫升)里盛满
酒精(粘滞系数比水小),将一密度计(比重计——轻表)浮在
酒精中,注意不要让密度计与筒壁相接触.当它静止后,把它
向下压(或向上提起)一小段距离,释放后,可观察到密度计在
酒精中以原来的平衡位置为中心上下振动.

\subsubsection{产生振动的条件}

(1)当物体离开平衡位置后必须受到回复力的作用.如
图9.5所示,弹簧$A$和由普通铁丝绕成的形状和$A$相仿的“弹
簧”$B$, 在它们的下方各挂一个50克的钩码.并使两个钩码
都在同一高度,这一位置就是两个钩码的平衡位置.将弹簧$A$
下的钩码再往下拉一小段距离,释放后,可观察到钩码做上下
振动;而将“弹簧”$B$下的钩码也往下拉一小段相同的距离,释
放后钩码根本不运动.以此来说明物体离开平衡位置后必须
受到回复力的作用,才能产生振动.

(2)将一单摆放在空气中观察其振动,然后再把一盛
有粘性很大的油(如10号或15号机油)的玻璃缸放在桌上,
缸内油面的深度要能使摆球经过平衡位置时全部被浸没(图
9.6).将摆球拉出油面,释放后,摆球只能比较缓慢地运动到
平衡位置附近,而不能继续做往复运动.这说明产生振动的
第二个必要条件是:阻力要足够小.

\begin{figure}[htp]\centering
    \begin{minipage}[t]{0.48\textwidth}
    \centering
\includegraphics[scale=.7]{fig/9-5.png}
    \caption{}
    \end{minipage}
    \begin{minipage}[t]{0.48\textwidth}
    \centering
\includegraphics[scale=.7]{fig/9-6.png}
    \caption{}
    \end{minipage}
    \end{figure}

    (3)将一钢球放在离心轨道(可用两条粗铁丝自制,如图
9.7所示)的底部$O$点,指出这是钢球的平衡位置,然后将钢
球移到位置$B$, 释放后,可观察到钢球将以$O$点为中心沿着圆
弧做往复运动.这说明钢球离开平衡位置后,受到重力和弹
力的作用(阻力可以不计),这两个力的合力,是使钢球回到平
衡位置的回复力.

\begin{figure}[htp]\centering
    \begin{minipage}[t]{0.48\textwidth}
    \centering
\includegraphics[scale=.7]{fig/9-7.png}
    \caption{}
    \end{minipage}
    \begin{minipage}[t]{0.48\textwidth}
    \centering
\includegraphics[scale=.7]{fig/9-8.png}
    \caption{}
    \end{minipage}
    \end{figure}

\subsubsection{简谐振动}
(1)可利用气垫导轨,在
滑块上固定一根倔强系数较小
的细弹簧(弹簧的直径约1.5—2.0厘米),弹簧的另一端固定在
气垫导轨的一侧,按课本图9.2进行演示和分析.演示时可在
导轨旁放一大的刻度板,用来指示滑块原来的平衡位置,并可
观察滑块在平衡位置两侧的位移变化.

(2)也可利用图9.8所示的装置.在小车的一端固定一
根值较小的弹簧,小车下面垫放一块玻璃板,放在水平桌面
上进行演示.

(3)利用课本图9.1的演示,说明挂在弹簧下端的重物
的振动是简谐振动时,应指出如重物在平衡位置弹簧的形变
为
$x_0$, 则弹簧的弹力$kx_0$和重力$mg$相平衡的位置即是振动
中的平衡位置(图9.9甲).当重物向下距平衡位置的距离为
$x$ ($x<x_0$) 时,它受的合力$F=k(x_0+x)-mg=kx$, 方向向上
(指向平衡位置),如图9.9乙所示.在重物向上运动的过程
中,这一合力将逐渐变小,当经过平衡位置时,合力$F=0$.
而当重物继续向上运动离开平衡位置的距离为$x$时,重物所
受合力$F=mg-k(x_0-x)=kx$, 方向向下(指向平衡位置,
图9.9丙),这一合力将随着上升距离$x$的增大而增大,因此
从整个振动过程来分析,重物所受的合力跟它离开平衡位置
的位移$x$成正比,而方向始终跟位移相反,所以挂在弹簧下端
的重物在竖直方向上的振动是简谐振动.

\begin{figure}[htp]
    \centering
    \includegraphics[scale=.7]{fig/9-9.png}
    \caption{}
\end{figure}

\subsubsection{简谐振动的周期和频率}
做课本图9.1的实验时,如果把重物的质量减小为原来
的1/4, 振动周期约减小一半,也就是频率约增加一倍.重物
质量不变的情况下,还可以改用$k$值较大或较小(可采用不同
长度的同种弹簧,弹簧原长增加一倍,相当于$k$值减小一半,
弹簧原长减少一半,相当于$k$值增加一倍)的弹簧,重新测定
振动周期,以获得弹簧振子的周期跟它的质量和弹簧$k$值大
小有关的认识.


\subsubsection{简谐振动的图象}
(1)在做课本图9.6的演示实验时,要使用烘干的并用
细端筛过的砂粒(若用金刚砂则效果更好),平板可以用玻璃
板(或透明的有机玻璃板),整个装置放在投影仪上进行演示,
让学生看到振动图象的描绘过程.为了使摆锤的摆动稳定,
可以采用双线摆的结构(图9.10).
\begin{figure}[htp]
    \centering
    \includegraphics[scale=.7]{fig/9-10.png}
    \caption{}
\end{figure}

(2)或可将一毛笔头缚在锯条的一端,锯条的另一端用
夹具夹住(图9.11),把毛笔尖蘸上少许墨水,将锯条拨动使
之发生振动,同时将与笔尖刚接触的纸片,沿着垂直于笔尖的
振动方向匀速拉动,在纸片上就描绘出笔尖做简谱振动的
图象.
\begin{figure}[htp]
    \centering
    \includegraphics[scale=.7]{fig/9-11.png}
    \caption{}
\end{figure}

(3)也可以像图9.12那样,在两根轻质弹簧之间拴一具
有较大质量的金属螺母,在螺母上固定一个蘸有墨水的毛笔
头,将两根弹簧的另一端$A$和$B$固定起来,使两根弹簧适当绷
紧,弹簧基本上在水平方向,将金属螺母向右(或向左)拉过一
小段距离,释放后带动笔尖在水平方向做简谐振动.调节两
根弹簧固定端$A$、$B$的高度,使得笔尖刚能和放在下面的一张
纸片接触,将纸片沿着垂直于笔尖的振动方向匀速拉动,在纸
片上就描绘出笔尖做简谐振动的图象.

(4)在上述(2)、(3)两个演示实验中,如果适当降低锯条
和弹簧的高度,使得笔尖和纸片间的压力增大,从而增大了振
动时的摩擦力,则可描绘出明显的阻尼振动的图象.
\begin{figure}[htp]\centering
    \begin{minipage}[t]{0.48\textwidth}
    \centering
\includegraphics[scale=.7]{fig/9-12.png}
    \caption{}
    \end{minipage}
    \begin{minipage}[t]{0.48\textwidth}
    \centering
\includegraphics[scale=.7]{fig/9-13.png}
    \caption{}
    \end{minipage}
    \end{figure}

\subsubsection{受迫振动和共振}
(1)课本所示的受迫振动实验,可以在振子下方放一个
盛着水的烧杯(图9.13),适当调节烧杯的高度,使得振子在水
中振动,以增大阻尼,使它的固有振动尽快得到衰减,呈现受
迫振动,但要控制摇手柄的转速,使策动力的频率不要远大于
振子的固有频率,因为由于水的阻力,振子会来不及振动,如
果摇手柄的转速较小,策动力的频率小于振子的固有频率,演
示效果较好.为了使转速均匀,可以把手柄拆下,在曲轴上安
装一个直径约为10—12厘米的皮带轮(可用厚有机玻璃片车
制),利用玩具电动机(2—6伏)通过皮带传动装置来带动曲
轴转动(图9-13),而且通过调节串联在电动机电路中的滑动
变阻器,可以方便地改变电动机转速(相当于改变策动力的
频率).

(2)摆的共振:用课本图9.13的装置观察摆的共振时,$A$摆的摆球质量
要大些,以仅贮存较多的能量,在张紧的绳上不一定要同时放
上七个单摆,例如除$A$外保留$B$、$D$、$E$三个摆,能观察到跟$A$
摆摆长相等的$B$摆振动的振幅最大也就足以说明问题了.

\begin{figure}[htp]
    \centering
    \includegraphics[scale=.7]{fig/9-14.png}
    \caption{}
\end{figure}

(3)如图9.14所示,将一长木片(航模材料,$50\x1\x300{\rm mm^3}$
)用铁块压在桌子边,长木片露出桌边的长度约为12—14厘米.木片的端部用螺丝固定一个小的玩具电动机(2—6伏),在机轴上缠绕一段焊锡丝并留出一段(约2厘米)不绕上
去.这样,玩具电机运转时就对长木片产生一个周期性的
策动力.在电动机电路中串联一个滑动变阻器,当变阻器的
阻值较小时,电动机转速较大,形成的策动力频率也较大,露
出桌边部分的木片就按相同频率做受迫振动,可观察到这时
木片的振幅较小,调节变阻器,使电阻逐渐增大,电动机转速
逐渐变小,当形成的策动力频率等于露出桌边部分木片的固
有频率时,便可观察到木片做受迫振动的振幅达到最大——
即产生共振的现象.

\subsubsection{横波的形成和传播}
(1)课本图9.15的演示,如果把绳子放在地上做,由于
地面支持力抵消了重力作用,效果较好.但最好选择象磨光
水泥地那种摩擦较小的地面来做.

如果用弹簧来代替绳子,则效果更好.可采用直径为
12—15毫米左右钢丝直径约为0.30—0.35毫米的密绕弹簧.
如果买来的弹簧比较短,则可以用几段接到一起使总长度约
为4—5米.联接的方法:可以用铜丝把两段弹簧一端的一圈
钢丝绞合在一起.
\begin{figure}[htp]
    \centering
    \includegraphics[scale=.7]{fig/9-15.png}
    \caption{}
\end{figure}

(2)如图9.15所示的简单装置,可用来演示横波的形
成.一根拉长到发生均匀范性形变的弹簧,根据需要可截取
数圈装置在有机玻璃支架上.当缓慢转动弹簧时,通过投影
仪,在屏幕上可形象地观察横波的传播过程,改变弹簧的转
向,波的传播方向也随之改变.

这类装置制作简便,但要注意每圈弹簧间的间隔应相等,
并且转动轴在弹簧的轴线上,为了观察横波中质点的振动方
向,可在弹簧的任一圈上滴上一滴熔融的火漆成一小球.当
转动弹簧时,可观察到凹、凸相间的横波波形向前推进的同
时,质点以平衡位置为中心作振动(图9.16).

\begin{figure}[htp]
    \centering
    \includegraphics[scale=.7]{fig/9-16.png}
    \caption{}
\end{figure}

为比较同相点或反相点的振动,可在弹簧的相应位置上
滴制二个或四个火漆小球(图9.17).

\begin{figure}[htp]
    \centering
    \includegraphics[scale=.7]{fig/9-17.png}
    \caption{}
\end{figure}

为了不同的观察目的,这类装置可以制作一批,例如弹簧
的不同圈距可以表示不同波长,弹簧的不同直径则可表示不
同的波幅.

\subsubsection{波长、频率和波速的关系}

\begin{figure}[htp]\centering
    \begin{minipage}[t]{0.48\textwidth}
    \centering
\includegraphics[scale=.6]{fig/9-18.png}
    \caption{}
    \end{minipage}
    \begin{minipage}[t]{0.48\textwidth}
    \centering
\includegraphics[scale=.7]{fig/9-19.png}
    \caption{}
    \end{minipage}
    \end{figure}

(1)可利用发波水槽和投影仪进行演示.如图9.18所
示.在玩具电动机的机轴上用焊锡丝缠绕成一个偏心装置,
将玩具电动机固定在具有弹性的金属片的一端,金属片固定
在发波水槽的边上,它的伸出部分的长度可以调节,在金属
片的伸出端下方装-个用来跟水面接触的平板型振子(图9.19).振子可用1.5—2毫米厚的木板制成,用螺丝固定时要注
意调节安装位置,使得电动机转动时它只做上下振动而没有
横向摆动(还可以通过改变金属片的伸长部分的长度来调
节).这样,通过投影就可以在屏幕上观察到平板型振子所产
生的平面波.从波纹明暗相间的间隔宽度可以大致后出波长.
改变电动机电路中串联的滑动变阻器的阻值,使电动机转速
变快,相当于使波源的频率变大,这时可以观察到,波纹间的
间隔变窄,即波长变小;使变阻器电阻增大,电动机转速变慢,
波源频率变小,则波纹间距变宽,即波长变大.这就定性地说
明了在波速一定时(机械波的波速是由媒质性质所决定的),
波长跟频率成反比关系.

(2)用圆盘频闪观察器观察当频率一定时,波长随波速
的增大而增大.
\begin{figure}[htp]
    \centering
    \includegraphics[scale=.7]{fig/9-20.png}
    \caption{}
\end{figure}

如图9.20所示,在发波水槽的一边装一个框架,把带有
偏心装置的玩具电动机安装在一块木条上,木条用两根橡皮
筋(或弹簧)吊在框架上,使木条下部稍许浸入水中,当电动机
转动时,使木条振子发生上下振动,在水面上形成一系列平面
水波.在水槽的另一边叠放几块矩形玻璃板,调节玻璃板的
厚度,使水层变得很浅.这样,当平面波进入到这一薄层的区
域时,相当于传播振动的媒质性质发生了改变,于是波速有了
改变(变小),由于波源的频率不变,因此波长也有了相应改变
(变短).上述现象可以投影在屏幕上直接观察(为了消除槽
边反射波的影响,可在水槽四周垫放一些塑料回丝).

\begin{figure}[htp]
    \centering
    \includegraphics[scale=.7]{fig/9-21.png}
    \caption{}
\end{figure}

屏幕上的波纹可以利用手动式圆盘频闪观察器来进行
观察.频闪观察器可以自制,在一块直径约25厘米的圆盘状
硬纸板或薄木板上开6条或12条等距对称、成辐射状的条形
观察孔,如图9.21所示,在圆盘中心开孔,用螺栓做轴,使
圆盘能绕轴转动,在轴上用螺母固定一个手柄,并在盘上靠
近中心处开一个直径约1.5—1.8厘米的指孔,用来拨动圆盘,
这便制成了圆盘频闪观察器.使用时一手持柄,另一手用食指
插入指孔转动圆盘,眼睛通过盘上的辐射状观察孔注视屏幕
上的波纹,逐渐增大圆盘转速,当圆盘上的观察孔经过眼前的
频率与水波频率同步时,所观察到的波纹就会“固定”下兼静
止不动(否则波纹不是前进便是后退).这时可看到玻璃板上
方的波纹和前面部分的波纹同时被“固定”住.但这两个区域
内相邻两条波纹的间距大小是不相等的,即波长不同.这就
直观地显示了波进入不同媒质时频率不变,波速和波长发生
了变化的情况.

\subsubsection{波的叠加}
(1)课本图9.21所示的在绳子上两个相遇的波互相穿
过的实验效果不是很明显.可以改用前述的4—5米长的弹
簧进行演示.演示时让学生围在四周观察,请一位学生当助
手.教师与学生各拿着弹簧的一端,把弹簧平放在地板上先
让学生把弹簧的$A$端按在地上,教师把弹簧的$B$端迅速向上
抖动一下,这时可观察到一个凸起的(振动平面垂直于地面)
半波,从$B$向$A$传播.分别从弹簧的$A$、$B$端发出一个波,可
观察到这两列波相遇,互相穿过后,仍然各自保持原有的状态
继续向前传播的现象,这个实验也可以贴着地面抖动弹簧,
使波动在水平方向上发生.

(2)利用发波水槽也可以观察两列水波互相穿过的现
象.演示时可用两支口径粗细不同的滴管,在水槽中同时滴
下两颗水滴,大水滴激起的水面波的能量较大,从屏幕上看,
圆面的波纹较粗,小水滴激起的水而波的波纹较细.可以看
到粗、细两个圆形波互相穿过后,仍保持各自原有的粗细程度
向前传播.

(3)还可利用课本图9.16的装置,在细长弹簧的左端推
动摆球振动一次,发出一个波的同时,在右端用手掌推动一
下弹簧形成一个频率较小的波,可以观察到这两列纵波相互
穿过的现象.

\subsubsection{水波的干涉}
利用图9.18的发波水槽和投影装置,可以观察课本图
9.23的水面波的干涉图样,演示时可以在弹性金属片下安装
两个相隔一定距离的金属小球(直径约4—5毫米),作为两个
频率和相都相同的波源(相干波源).观察时要调节聚光镜的
位置,使水面的像最为清晰.

发生干涉时只要求两个波源频率相同、相差恒定,两个波
源的振幅不一定要相等,这一点可以通过演示证明.把安装
双球振子的固定片稍稍倾斜,使一个球接触水面的深度深些,
另一个浅些,这样,发出的两列波的振幅就不相等,但还是能
看到稳定的干涉图样.

\subsubsection{声波的干涉}
将正在发声的音叉放在耳旁徐徐转动,就能辨别出声音
忽强忽弱的现象.也可以将正在发声的音叉放在一个话筒前
转动,把信号放大后接在扬声器上,可以听出声音忽强忽弱.
还可以把示波器并联在扬声器两端进行观察,可以看到随着
音叉的转动,所形成的声波波幅的变化.

以上几个现象都说明了声波的干涉.但对如何形成声波
干涉的具体过程,建议不必行分析,因为这是比较复杂的,
不象发波水槽中水面波产生干涉那样的单纯.

\subsubsection{声音的共鸣}
(1)利用课本所述的共振音叉演示声音的共鸣时,要使
两个共鸣箱的开口端互相对着,比较靠近些,并且使两个音
叉的振动方向在同一平面上.当用橡皮锤敲击一个音叉时,要
稍待一会儿,使得通过空气的振动把能量较多地传给另一个
音叉,然后用手按住被敲出的音叉,去听另一音叉发出的
声音.

(2)也可利用弦的共振来进行演示(沈括在《梦溪笔谈》当
中所介绍的方法),把两根弦固定在弦音计上,调整到相同的
频率,拨动一根弦时,可以看到骑放在另一根弦上的小纸片会
发生弹跳飞落的现象.

(3)课本图9.29所示的空气柱的共鸣实验中,所用的玻
璃管的直径约为2厘米左右.如果选用频率为520赫兹的音
叉,则玻璃管的长度应不小于20厘米,因为声波在这一频率
时它的四分之一波长约为16厘米,这样才能上下移动玻璃管
调节气柱的长度.

\subsubsection{音调跟频率有关}
(1)按课本图9.32的装置进行演示时,所用的纸片应选
用薄而硬的材料(譬如可用一小块新的牛皮纸效果很好).

(2)敲击不同频率的音叉,由话筒通过放大器用示波器
观察它们的波形.若以384赫的波形为标准(譬如调到出现
三个全波),再换上256赫兹或520赫兹的音叉,可以明显地
看到波数变少或变多,说明频率越大,音调越高.

(3)利用音频信号发生器,当音调连续(或不连续)改变
时,可观察到示波器上的波数出现相应的改变.

\subsubsection{响度跟振幅有关}
(1)音叉插在共鸣箱上,用橡皮锤轻敲,音叉发出比较轻
的声音,同时用悬挂在支架上的小木球靠近音叉的一个叉股,
观察小球被弹开的角度.然后再用橡皮锤较重地敲击音叉,
音叉发生较响的声音,用小木球靠近时,可观察到小木球被弹
开的角度要大得多.

(2)利用示波器进行观察,轻敲音叉时波形的波幅较小,
较重地敲击音叉时,可观察到波幅明显增大.

\subsubsection{音品}
用示波器观察,对纯音(可用频率为256赫兹的音叉)和
其他乐器(或人声)所发出的中央C音的波形进行对比.

\subsection{学生实验}
\subsubsection{用单摆测定重力加速度}
(1)单摆是一个理想化的振动系统,选择材料时应用较
细的蜡线或尼龙丝,小球应选用体积较小、密度较大的金属
球,这样,才能比较符合一根“不能伸长,设有质量的线的下
端系一质点”的要求.

(2)为了使摆角不超过5°,摆球从平衡位置拉开的距离
应不超过$\ell\sin5^{\circ}$. 如果摆长$\ell=1.000$米,$\sin5^{\circ}=0.0872$, 则
摆球从平衡位置拉开的距离$A=\ell\sin5^{\circ}=0.0872{\rm m}=8.7{\rm cm}$.

(3)测量单摆的振动周期时,在摆球的平衡位置下面做
一记号(譬如放一支铅笔,如图9.22所示).将摆球从平衡位
置拉开一小段距离,由静止释放后,观察摆球的振动,同时以
平衡位置为标准,默数摆球完成全振动的次数,使数数的快
慢能跟振动周期同步,然后再来计时,当观察到摆球又一次
经过平衡位置时,采取倒数的方法,默数“四——、三——、
二——、一——、零”,数到零时开始计时,接着顺数单摆完成
全振动的次数,记下摆动30—50次全振动的时间.

\begin{figure}[htp]\centering
    \begin{minipage}[t]{0.48\textwidth}
    \centering
\includegraphics[scale=.7]{fig/9-22.png}
    \caption{}
    \end{minipage}
    \begin{minipage}[t]{0.48\textwidth}
    \centering
\includegraphics[scale=.7]{fig/9-23.png}
    \caption{}
    \end{minipage}
    \end{figure}

(4)实验中应注意的几个问题
\begin{enumerate}
\item 摆线的悬点一定要固定好,要用比较紧一些的铁夹
(文具夹)夹牢,以免在摆动过程中摆长发生改变.不允许将
悬线随意绕在铁架台的复夹上,如图9.23所示的固定悬线的
方法是错误的,这将使摆长在振动过程中时刻发生变化.使
用铁夹固定悬点,还可以根据实验需要方便地改变摆长.

\item 测量周期时,开始计时要以平衡位置为标准,而不是
以摆球到达最大位移处为标准,这是因为摆球经过平衡位置
时的速度最大,误差可以小一些;而当摆球到达最大位移处
时,摆球的速度最小,人的眼睛不易分辨摆球的速度是否恰好
等于零.
\end{enumerate}


(5)几个可以让同学思考的问题
\begin{enumerate}
\item 为什么要测出摆动30—50次的时间,再算出平均摆
动一次的时间,而不是只测一次全振动的时间?
\item 为了更接近于单摆这一理想化的模型,实验中所用的
悬线长度长一些好还是短一些好?摆球的体积大一些好还是
小一些好?
\item 在研究周期与摆长的关系时,是否可以计算出$T^2$的
数值,然后跟对应的摆长$\ell$来计算出每一组比值$T^2/\ell$, 看看
它们的关系如何?或者是否可以画出$T^2$-$\ell$图象来进行研究?
\end{enumerate}

\section{习题解答}

\subsection{练习一}
\begin{enumerate}
    \item 设图9.1中振幅是2厘米,完成一次全振动,振动物体通过的路程是多少厘米?如果频率是5赫,振动物体每秒通过的路程是多少厘米?
    
    \begin{solution}
        完成一次全振动,振动物体通过的路程
       \[ s=4\x 2=8{\rm cm}\]
        已知频率$f=5$赫,每秒钟通过的路程
       \[ s'=8\x 5=40{\rm cm}\]
    \end{solution}
    \item 设图9.1中振幅是2厘米,取竖直向上的方向作为正方向,物体运动到点$C$时对平衡位置的位移是多大?运动到点$B$时对平衡位置的位移又是多大?
    
    \begin{solution}
        物体运动到$C$点时,位移$x_C=2$厘米.
        物体运动到$B$点时,位移$x_B=-2$厘米. 
    \end{solution}
\end{enumerate}

	
\subsection{练习二}

\begin{enumerate}
    \item 物体在任意回复力作用下振动,一定是做简谐振动吗?为什么?


    \begin{solution}
        不一定,因为做简谐振动的物体,受到的回复力不能
        是任意的,它一定是跟对平衡位置的位移成正比而方向相反
        的,即$F=-kx$, 否则物体的振动就不是简谐振动.
    \end{solution}
    \item 用手拍球,使球在硬地上来回跳动,球的运动是简谐
    振动吗?为什么?


    \begin{solution}
        上下跳动的球,除了运动到最低点,即跟地面发生碰
        撞的很短一段时间外,只受到重力的作用.重力是一个跟球的
        位移无关的恒力,不是回复力.所以球的跳动不是简谐振动.
    \end{solution}
\item 分析课本图9.2中弹簧振子的运动,并填好下表:	
\begin{center}
    \begin{tabular}{p{.3\textwidth}|c| c |c |c}
        \hline
        振子的运动  & $C\to O$ & $O\to B$ & $B\to O$ & $O\to C$\\
        \hline
        回复力的方向怎样?大小如何变化?&向右变小&向左变大&向左变小&向右变大\\
        运动的性质(加速或减速)&加速&减速&加速&减速\\
        加速度的方向怎样?大小如何变化?&向右变小&向左变大&向左变小&向右变大\\
        速度的方向怎样?大小如何变化?&向右变大&向右变小&向左变大&向左变小\\
        \hline
    \end{tabular}
\end{center}
	\item 课本图9.2所示的弹簧振子的质量是100克,频率为2赫,求弹簧的倔强系数.

\begin{solution}
由公式$T=2\pi\sqrt{\dfrac{m}{k}}$和$f=\dfrac{1}{T}$,得
\[f=\frac{1}{2\pi}\sqrt{\frac{k}{m}}\]
所以
\[k=4\pi^2 f^2m=4\x 3.14^2\x 2^2\x 0.1=15.8{\rm N/m}\]
    \end{solution}
    \item 一个如课本图9.2所示的弹簧振子的质量是200克,弹簧的倔强系数是16牛/米,振幅是2厘米,取水平向右的方向作为正方向.当振子运动到右方最大位移时,回复力和加速度的数值各是多大?当振子运动到左方最大位移时,回复力和加速度的数值又各是多大?这个弹簧振子的周期和频率各是多大?


    \begin{solution}
振子运动到右方最大位移时,回复力的数值
\[F=-kx=-16\x0.02=-0.32{\rm N}\]
加速度
\[a=\frac{F}{m}=-\frac{0.32}{0.2}=-1.6\msq\]
振子运动到左方最大位移时,回复力的数值
\[F=-kx=-16\x(-0.02)=0.32{\rm N}\]
加速度
\[a=\frac{F}{m}=\frac{0.32}{0.2}=1.6\msq\]
这个弹簧振子的周期
\[T=2\pi\sqrt{\frac{m}{k}}=2\x3.14\x\sqrt{\frac{0.2}{16}}=0.70{\rm s}\]
频率
\[f=\frac{1}{T}=\frac{1}{0.70}=1.4{\rm Hz}\]
    \end{solution}
\end{enumerate}


\subsection{练习三}
\begin{enumerate}
    \item 假如把单摆和弹簧振子都从地球移到月球上,它们的振动频率是否改变?为什么?
    
    \begin{solution}
弹簧振子的周期不会改变,由于周期$T=2\pi\sqrt{\dfrac{m}{k}}$,
把弹簧振子从地球移到月球上后$m$和$k$都不变,所以周期不
变,振动频率也不变.

单摆的周期会发生变化,由于周期$T=2\pi\sqrt{\dfrac{\ell}{g}}$,
而月球
上的重力加速度$g_{\text{月}}$与地球上的重力加速度$g$不等,所以周期要变,振动频率也要变.
    \end{solution}
    \item 两个单摆,它们的摆长的比是1:4,求它们的周期的比.两个单摆,它们的频率的比是1:4,求它们的摆长的比.

    \begin{solution}
由单摆周期公式$T=2\pi\sqrt{\dfrac{\ell}{g}}$,得周期之比
\[\frac{T_1}{T_2}=\sqrt{\frac{\ell_1}{\ell_2}}=\sqrt{\frac{1}{4}}=\frac{1}{2}\]
由单摆频率公式$f=\dfrac{1}{2\pi}\sqrt{\dfrac{g}{\ell}}$,得
\[\frac{f_2}{f_1}=\sqrt{\frac{\ell_1}{\ell_2}}\]
所以
\[\frac{\ell_1}{\ell_2}=\left(\frac{f_2}{f_1}\right)^2=\left(\frac{4}{1}\right)^2=16\]
    \end{solution}
    \item 测某地的重力加速度时,用了一个摆长为2米的单摆,测得100次全振动所用的时间是4分44秒,这个地方的重力加速度多大?

    \begin{solution}
        100次全振动所用的时间
\[t=4\x 60 +44=284{\rm s}\]
周期$T=\dfrac{t}{100}=2.84{\rm s}$,由$T=2\pi\sqrt{\dfrac{\ell}{g}}$,得:
\[g=\frac{4\pi^2\ell}{T^2}=\frac{4\x 3.14^2\x 2}{2.84^2}=9.78\msq\]
    \end{solution}
    \item 假如把上题中的单摆拿到月球上去,月球的重力加速度是1.6$\msq$,摆的周期将变为多少秒?

    \begin{solution}
        在月球上的周期
        \[T=2\pi\sqrt{\frac{\ell}{g_{\text{月}}}}=2\x3.14\x\sqrt{\frac{2}{1.6}}=7.0{\rm s}\]
    \end{solution}
\end{enumerate}




\subsection{练习四}


\begin{enumerate}
    \item 图9.24是一个简谐振动的图象.根据图象确定它的振幅和周期.
    \begin{figure}[htp]\centering
        \begin{tikzpicture}[>=stealth, xscale=.7, domain=0:3*pi, samples=200]
    \draw [->](0,0)node [left]{$0$}--(11,0) node [below]{$t$(秒)};
    \draw [->](0,-1.2)--(0,1.5) node [right]{$x$(厘米)};
    
    
            \draw [very thick]  plot (\x,{cos(\x r)});
    
     \draw [dashed](0, 1) -- (2*pi, 1);  
     \draw [dashed](0,-1) -- (3*pi, -1);     
    
    \foreach \x in{1,2,...,5}
    {
        \draw(\x*pi/2, 0)node [below]{\x}--(\x*pi/2, 0.2);
    }
    
    \node at (-.5,1){$+10$};\node at (-.5,-1){$-10$};
    
    
        \end{tikzpicture}
        \caption{}
    \end{figure}

    \begin{solution}
        振幅$A=10$厘米,周期$T=4$秒.
    \end{solution}
    \item 图9.10的振动图象是一条余弦曲线,你能不能应用学过的数学知识算出下列时刻振子对平衡位置的位移?
    \begin{enumerate}
        \item $t=0.5$秒;
        \item $t=1.5$秒.
    \end{enumerate}

    提示:想一想在图9.24中,1秒、2秒等时刻相当于余弦函数的多大的角度.

    \begin{solution}
因为振动曲线是余弦函数图象,位移的函数表达式
可写为$x=A\cos\theta$, 由图象可以看出周期$T=4$秒.当$t_1=0.5$
秒时,相当$\theta_1=0.5\x\dfrac{2\pi}{4}=\dfrac{\pi}{4}$,
当$t_2=1.5$秒时,相当$\theta_2=1.5\x\dfrac{2\pi}{4}=\dfrac{3\pi}{4}$.
故有
\[\begin{split}
    x_1&=A\cos\theta_1=10\x\cos\frac{\pi}{4}=10\x \frac{\sqrt{2}}{2}=7.07{\rm cm}\\
    x_2&=A\cos\theta_2=10\x\cos\frac{3\pi}{4}=10\x \left(-\frac{\sqrt{2}}{2}\right)=-7.07{\rm cm}\\
\end{split}\]
    \end{solution}
\end{enumerate}



\subsection{练习五}
分析弹簧振子(课本图9.2)和单摆(课本图9.3)在振动中能量的转化情况(增多或减少),填好下表:

\begin{center}
    \begin{tabular}{c|c|c|c|c}
        \hline
        振子的运动  & $C\to O$ & $O\to B$ & $B\to O$ & $O\to C$\\
        \hline
动能的变化  &增大&减小&增大&减小\\
势能的变化&减小&增大&减小&增大\\
总能量的变化&不变&不变&不变&不变\\
\hline
    \end{tabular}
\end{center}



\subsection{练习六}
\begin{enumerate}
    \item 除了书上讲过的自由振动和受迫振动的例子外,再各举两个实例.


    \begin{solution}
        琵琶的弦被拨动后的振动,鼓面被击后的振动是自
        由振动.

        蚊子、蜜蜂、蜻蜓等飞行时翅膀的振动,人挑担行走时扁
        担的振动都是受迫振动.
    \end{solution}
    \item 仿照课本图9.13所示的研究共振现象的装置,自己利用手边的材料来做实验,观察受迫振动的振幅跟策动力频率之间的关系.


    \begin{solution}
    (解答略)
    \end{solution}
    \item 汽车的车身是装在弹簧上的,如果它的固有周期是1.5秒,汽车在一条起伏不平的路上行驶,路上各凸起处相隔的距离都大约是8米,那么汽车以多大的速度行驶时车身的起伏振动最激烈?


    \begin{solution}
汽车受迫振动频率由$f=v/\ell$决定,式中$v$为汽车速
率,$\ell$是路面凸起处的间隔,当$f=f_{\text{固}}$时发生共振,车身振动最剧烈,故有
\[\frac{v}{\ell}=\frac{1}{T_{\text{固}}}\]
则
\[v=\frac{\ell}{T_{\text{固}}}=\frac{8}{1.5}=5.33\ms\]
    \end{solution}
\end{enumerate}



\subsection{练习七}
\begin{enumerate}
    \item 在某一地区,地震波的纵波和横波在地表附近的传播速率分别是9.1${\rm km}/{\rm s}$和3.7${\rm km}/{\rm s}$.在一次地震时,这个地区的一个观测站记录的纵波和横波的到达时刻相差5秒,那么地震的震源距这个观测站多远?

    \begin{solution}
        设震源距观测站距离为$s$, 由于地震波的纵波和横
        波由震源处同时发出,它们的速度分别为$v_1$、$v_2$, 则先后到达
        观测站的时间差
        \[\Delta t=\frac{s}{v_2}-\frac{s}{v_1}\]
        所以
\[s=\left(\frac{1}{v_2}-\frac{1}{v_1}\right)\Delta t=\left(\frac{1}{3.7}-\frac{1}{9.1}\right)\x 5=31{\rm km}\]
    \end{solution}
    \item 一只船停泊在岸边,如果海浪的波峰间的距离是6米,海浪的波速是1.5$\ms$,求船摆晃的周期是多少?

    \begin{solution}
        海浪波峰间距离即海浪的波长$\lambda$.船摇晃的周期就
        是它上下浮动,完成一次全振动的周期$T$. 
        \[T=\frac{\lambda}{v}=\frac{6}{1.5}=4{\rm s}\]
    \end{solution}
    \item 甲乙二人分乘两只船在湖中钓鱼,两船相距24米.有一列水波在湖面上传播开来,每只船每分钟上下浮动20次,当甲船位于波峰时,乙船位于波谷,这时两船之间还有一个波峰,水波的波速是多大?

    \begin{solution}
        船每分钟上下浮动20次,即完成20次全振动,故水
        波周期
        \[T=\frac{t}{20}=\frac{60}{20}=3{\rm s}\]
        甲、乙两船相距1.5个波长,故波长
\[ \lambda=\frac{s}{1.5}=\frac{24}{1.5}=16{\rm m} \]
        波速
\[v=\frac{\lambda}{T}=\frac{16}{3}=5.33\ms\]

        此题也可按以下方法求解:因为甲、乙两船相距1.5个
        波长,故水波通过两船时间为1.5个周期,于是波速
        \[v=\frac{s}{t}=\frac{24}{1.5T}=\frac{24}{1.5\x 3}=5.33\ms\]
    \end{solution}
    \item 仔细研究课本图9.15,说明:两个相邻的反相质点间的距离等于波长的多少.

    \begin{solution}
        由于两个相邻的同相质点间的距离等于一个波长,
        则两个相邻的反相质点间的距离等于半个波长.
    \end{solution}
\end{enumerate}


\subsection{练习八}
\begin{enumerate}
    \item 振动图象和波的图象各表示的是什么内容?振动图象中相邻两个最大值之间的间隔等于什么?波的图象中相邻两个最大值之间的间隔等于什么?

    \begin{solution}
        振动图象表示的是某振动质点对平衡位置位移怎
        样随时间而变化.波的图象表示的是某一时刻沿波传播方向
        上媒质各质点对平衡位置的位移.振动图象中相邻两个最大
        值之间的间隔等于振动周期$T$, 波动图象中相邻两个最大值
        之间的间隔等于波长$\lambda$.
    \end{solution}
    \item 有一列波,它在某一时刻的波形曲线如课本图9.18中的实线所示,这列波经过$T/4$后的波形曲线是什么样?经过$2T/4$,$3T/4$后又是什么样?

    \begin{solution}
        经过$T/4$、$2T/4$、$3T/4$后的波形曲线分别如图9.25
        的甲、乙、丙所示.
\begin{figure}[htp]
    \centering
    \includegraphics[scale=.6]{fig/9-25.png}
    \caption{}
\end{figure}

    \end{solution}
    \item 横波的图象直观地表现了横波在某一时刻的波形,纵波的图象却不能直观地表现纵波的情况,但是,我们看到纵波的图象应该很自然地想出纵波的情况,就象我们看到振动图象应该很自然地想出弹簧振子或单摆的振动情况一样.仔细考察课本图9.19,回答下面的问题.
    \begin{enumerate}
        \item 在图象的什么地方质点向右的位移最大?
        \item 在图象的什么地方质点向左的位移最大?
        \item 在图象的什么地方质点的位移为零?
        \item 密部中央两侧质点的位移有什么特征?
        \item 疏部中央两侧质点的位移有什么特征?
    \end{enumerate}

    \begin{solution}
    在图线的最高点处质点向右的位移最大;在图线的
最低点处质点向左的位移最大;在图线与横轴的交点处质点
的位移为零;密部中央左侧的质点位移向右,右侧的质点位移
向左;疏部中央左侧的质点位移向左,右侧的质点位移向右.
    \end{solution}
\end{enumerate}




\subsection{练习九}

\begin{enumerate}
\item 第一次测定声音在水中的传播速率是1827年在日
内瓦湖上用下面的方法进行的:在一只船上实验员向水里放
下一个钟,当敲这个钟的时候,使船上的火药同时发光;在另
一只船上,另一实验员向水里放下一个听音器,他测量从看到
火药闪光到听到钟声所经过的时间.

两船相距14千米,看到火药闪光后10秒钟听到声音,求
声音在水中的传播速率.


\begin{solution}
    光在空气中的传播速率很大,它在两只船间的传播
时间可以忽略不计,因此声在水中的传播速率
\[v=\frac{s}{t}=\frac{14000}{10}=1400\ms\]
\end{solution}
\item 第一次测定铸铁里的声速是在巴黎用下面的方法进
行的:从铸铁管的一端敲一下钟,在管的另一端听到两次响
声,第一次是由铸铁管传来的,第二次是由空气传来的.管长
931米,两次响声相隔2.5秒,如果当时空气中的声速是340
$\ms$,求铸铁中的声速.

\begin{solution}
    设管长为$s$, 声波在空气中传播速度为$v_1$, 在铸铁中
    传播速度为$v_2$, 则在管端听到两次声音的时差为
\[\frac{s}{v_1}-\frac{s}{v_2}=t\]
所以\[v_2=\frac{s}{s-v_1t}\cdot v_1\]
代入数值计算可得:$v_2=3.9\x 10^3\ms=3.9{\rm km/s}$
\end{solution}
\item 为了听到回声,反射声波的障碍物至少应该离开我
们多远?猎人在射击后6秒钟听到射击的回声,障碍物离猎人
有多远?设空气中的声速是340$\ms$.

\begin{solution}
    设从声源(人)发出声波到接收到从障碍物反射回来
    的声波的时间为$t_0$, 只有当$t_0\ge 0.1$秒时才能将回声和原来
    的声音区分开来.如果声源(人)到障碍物距离为$s$, 能听到
    回声的条件是
    \[\frac{2s}{v}\ge 0.1{\rm s}\]
    所以
 \[s\ge   \frac{v}{2}\x0.1=\frac{340}{2}\x 0.1=17{\rm m}\]
    得$s\ge 17$米,即反射物离人至少应有17米远.

    设障碍物离猎人的距离为$L$, 因射击后6秒钟听到回
    声,故
    \[\frac{2L}{v}=t,\qquad L=\frac{vt}{2}\]
    代入数值计算可得
\[L=\frac{340\x 6}{2}=1020{\rm m}\]
\end{solution}
\item 人能听到的声膏的最高频率是20000赫.狗能听到
的声音的最高频率是50000赫.蝙蝠能发出并且能听到的声
音频率高达120000赫.分别求出人、狗、蝙蝠能听到的,在
0$^{\circ}$C空气中传播的声波的最短波长.

\begin{solution}
    在0$^{\circ}$C空气中,人能听最短声波波长
    \[\lambda_1=\frac{v}{f_1}=\frac{332}{20000}=0.0166{\rm m}=1.66{\rm cm}\]
    狗能听到最短声波波长
    \[\lambda_2=\frac{v}{f_2}=\frac{332}{50000}=0.00664{\rm m}=6.64{\rm mm}\]
    蝙蝠能听到的最短声波波长
    \[\lambda_3=\frac{v}{f_3}=\frac{332}{120000}=0.00227{\rm m}=2.77{\rm mm}\]
\end{solution}
\item 在一次如课本图9.29所示的空气柱共鸣的实验中,测得
共鸣时空气柱的最短长度为19厘米,声波的波长有多长?已
知音叉的频率是440赫,空气中的声速有多大?


\begin{solution}
    声波的波长
    \[\lambda=4L=4\x19=76{\rm cm}\]
在空气中的声速
\[v=\lambda f=0.76\x440=334\ms\]
\end{solution}

\end{enumerate}




\subsection{习题}
\begin{enumerate}
    \item 一座摆钟走得慢了,要把它调准,应该怎样改变它的摆长?是增长还是缩短?为什么?

    \begin{solution}
        应该缩短摆长.因为摆长短了,周期变小,频率加
        快,钟比原来走得快,这样原先走得慢的钟就能调准了.
    \end{solution}
    \item 周期是2秒的单摆叫做秒摆,试根据测得的当地重力加速度$g$的数值自制一个秒摆.

    \begin{solution}
        略.
    \end{solution}
    \item 使悬挂在长绳上的小球偏离平衡位置一个很小的角度,然后放开它;使另一个小球以初速度为零从长绳的悬挂点自由落下,如果两球同时开始运动,哪一个球先到达第一个球的平衡位置?

    \begin{solution}
        单摆小球到达平衡位置所需时间
        \[t_1=\frac{T}{4}=\frac{1}{4}\x2\pi\sqrt{\frac{\ell}{g}}=\frac{\pi}{2}\sqrt{\frac{\ell}{g}}=1.57\sqrt{\frac{\ell}{g}}\]
         自由下落的小球到达摆球平衡位置所需时间
\[t_2=\sqrt{\frac{2\ell}{g}}=\sqrt{2}\cdot \sqrt{\frac{\ell}{g}}=1.41\sqrt{\frac{\ell}{g}}\]
        因为$t_1>t_2$, 所以自由下落的小球先到达单摆的平衡位
        置.
    \end{solution}
    \item 一位物理学家通过电视机观看宇航员登月球的情
况,他发现在发射到月球上的一个仪器舱旁边悬挂着一个重
物,在那里摆动,悬挂重物的绳长跟宇航员的身高相仿.这
位物理学家看了看自己的手表,测了一下时间,于是他测出了
月球表面上的自由落体加速度的数值,他是怎么测出的?

\begin{solution}
    据公式
    \[T=2\pi\sqrt{\frac{\ell}{g_{\text{月}}}}\]
    只要用手表测出摆的振动周
    期估测出宇航员的身高,即得
    \[g_{\text{月}}=\frac{4\pi^2 \ell}{T^2}\]
\end{solution}
\item 一个准备装到人造卫星上的小型电子计算机将承受
10$g$的加速度,为了试验它是否承受得了这样大的加速度,
将它装到一个在水平方向上做简谐振动的试验台上.试验台
的频率是10赫,要使试验台的最大加速度达到10$g$,它的振
幅必须多大?

\begin{solution}
    据作简谐振动回复力公式$F_m=kA$, 最大加速度的
    大小$a_m=\dfrac{F_m}{m}=\frac{k}{m}A$.

由$f=\dfrac{1}{2\pi}\sqrt{\dfrac{k}{m}}$,得:$k=4\pi^2f^2m$,代入上面的$a_m$表达式$a_m=4\pi^2f^2A$,得:
\[A=\frac{a_m}{4\pi^2f^2}=\frac{10g}{4\pi^2f^2}=\frac{10\x 9.8}{4\x 3.14^2\x 10^2}=0.0248{\rm m}=2.48{\rm cm}\]
\end{solution}
\item 火车车轮经过接轨处时要受到震动,因而使车厢在
弹簧上上下振动.已知弹簧每受1吨的力,被压缩1.6毫米.
三厢和载重共重55吨,每段铁轨长12.5米,火车沿轨道做
匀速运动时,它的危险速度是多少$\kmh$?

\begin{solution}
    据胡克定律,车厢弹簧的倔强系数
    \[k=\frac{F}{x}=\frac{9800}{0.0016}{\rm N/m}\]
可计算出火车行驶时振动的固有周期
\[T_{\text{固}}=2\pi\sqrt{\frac{m}{k}}=2\x 3.14\x \sqrt{\frac{55000\x 0.0016}{9800}}=0.60{\rm s}\]
    火车受迫振动周期
    \[T=\frac{\ell }{v}\]
    式中$v$为火车的行驶速度,$\ell$是铁轨的长度.

    只有当$T=T_{\text{固}}$时火车才发生共振,车身振动最剧烈,故
    危险速度
\[v=\frac{\ell}{T_{\text{固}}}=\frac{12.5}{0.60}=21\ms=75.6\kmh\]
\end{solution}
\item 已知0$^{\circ}$C时空气中的声速是332$\ms$,水中的声速
是1450$\ms$,声波由空气传入水中时波长变化了多少倍?

\begin{solution}
    因为声音由空气传入水中时,频率不变,故有
\[\frac{v_{\text{水}}}{\lambda_{\text{水}}}=\frac{v_{\text{空}}}{\lambda_{\text{空}}}\]
所以
\[\frac{\lambda_{\text{水}}}{\lambda_{\text{空}}}=\frac{v_{\text{水}}}{v_{\text{空}}}=\frac{1450}{332}=4.37\]
    即:水中波长增大至空气中波长的4.37倍.
\end{solution}
\item $A$、$B$、$C$三点分别距声源$S$ 40厘米、52.5厘米、65
厘米,从$S$传出的声波波长是25厘米,分别求出$A$、$B$两点
和$A$、$C$两点相的关系.

\begin{solution}
    设在波的传播方向上有两点,若振源到这两点的距
    离的差等于波长的整数倍,这两点同相;若振源到这两点的距
    离的差等于半波长的奇数倍,这两点反相.故有
    声源距$A$、$B$两点的距离的差为
\[\Delta s_{AB}=s_B-s_A=52.5-40=12.5{\rm cm}=\frac{\lambda}{2}\]
所以$A$、$B$两点反相.

声源距$A$、$C$两点的距离的差为
\[\Delta s_{AC}=s_C-s_A=65-40=25{\rm cm}=lambda\]
所以$A$、$C$两点同相.
\end{solution}
\item 图9.26中的$S_1$和$S_2$是两个同相、同频率的波源:
$S_1$和$A$点的距离是$\ell_1$,$S_2$和$A$点的距离是$\ell_2$,如果$\ell_2-\ell_1$等
于一个波长,两列波到达$A$点时同相,波峰和波峰相遇(或波
谷和波谷相遇),$A$点的振动加强;如果$\ell_2-\ell_1$等于半个波长,
两列波到达$A$点时反相,波峰和波谷相遇,$A$点的振动减弱.

试证明:当$\ell_2-\ell_1$为半波长的偶数倍时,$A$点的振动加强;当
$\ell_2-\ell_1$为半波长的奇数倍时,$A$点的振动减弱.

\begin{figure}[htp]\centering
    \includegraphics[scale=.6]{fig/9-26.png}
    \caption{}
    \end{figure}

\begin{proof}
    由于两个波源$S_1$和$S_2$的频率相同而且是同相
    的,$A$点距离两个源的距离之差$\ell_2-\ell_1$只要是波长的整数
    倍,两列波到达$A$点时就是同相的,振动互相加强;如果$\ell_2-\ell_1$比波长的整数倍相差半个波长,两列波到达$A$点时就是反相的,振动互相减弱.

    所以,如果$k$为任意的正整数,当$\ell_2-\ell_1=2k\cdot \dfrac{\lambda}{2}$
    时,有
    $\ell_2-\ell_1=k\lambda$, 为波长的整数倍,两列波到达$A$点时互相加强.
    当$\ell_2-\ell_1=(2k+1)\dfrac{\lambda}{2}$
    时,有$\ell_2-\ell_1=k\lambda+\dfrac{\lambda}{2}$,
    比波长的
    整数倍相差半个波长,两列波到达$A$点时互相减弱.
\end{proof}
\end{enumerate}



\section{参考资料}
\subsection{周期运动、振荡和振动}
周期运动是任何一种在相等的间隔中完全重复的运动.
设$X(t)$代表系统在时刻$t$在某坐标系中的位移,则对于时间
变量的每一个$t$值周期运动都具有方程$X(t+T)=X(t)$所
定义的性质.一个循环持续的时间$T$称为运动的周期.钟
表的擒纵机构的运动,地球绕太阳的公转以及发动机在匀速
运转时曲柄、连杆和活塞更复杂的运动都是周期运动的例子.
任何周期运动都可以表示为傅里叶级数,即一些正弦项或余
弦项之和,各项的频率是整个周期运动频率$f$的整数倍.例
如通过傅里叶分析任何一个复杂的周期波都可以表示为一系
列正弦波分量的级数之和,分量的周期是这个复合波的基本
周期的$1/n$, $n$为正整数.

振荡是任何一种往复变化的现象,振荡的例子包括声波
中压力的变化,以及数学函数的起伏,即某数值在某一平均值
上下的重复交替变化.振荡这个词在很多场合与振动同义,
虽然后者主要指机械振动.例如为了检测产品抗震性能的机
器称为振动台,而使电流方向周期性反复交变的电子器件通
常称为振荡器.电磁场的交替变换称为电磁振荡.如给系统
以某一初扰动,然后让它自己进行振荡,这种现象称为\textbf{自由振
荡}.系统对恒定作用的周期激扰的响应称为\textbf{受迫振荡}.振幅不
断减小的任何振荡称为\textbf{衰减振荡}.通常是由于系统有能量输
出.振幅保持不变的振荡称为非衰减振荡,这通常是由于外
部能源补充能量.

振动这个术语描述相对于某一规定的中心基准(平衡位
置)位移的持续周期变化,这种周期运动可以包括从摆的简
单来回摆动、钢板受锤击后的较为复杂的振动,直到大型结构
的极其复杂的振动.例如汽车在粗糙路面上行驶时发生的振
动.所有的原子、分子与核子也都在不断振动,一个机械系
统要能自身维持其自由振动必须具有质量与弹性的特性或者
与此相当的特性,具有弹性是指系统从正常形态的任何偏离
将引起回复力促使系统返回正常形态.具有质量或惯性是指
系统在回到正常形态时所获得的速度又可使系统超越这一形
态.正是由于质量和弹性的相互作用,机械系统才有可能发
生周期振动.

\subsection{谐运动和谐振子}
谐运动是以时间的正弦(或余弦)函数表示的一种周期运
动,通常称为简谐运动,它是最简单形式的振动,这种运动对
它的平衡位置是对称的,在平衡位置处速度最大,加速度为零;
在最大位移(或转向点)处速度为零,加速度最大.这种运动的
特征是由单一的频率(无泛音)来表示的.谐运动可出现于非
常简单的机构中,例如匀速圆周运动物体上任一点在固定直
线上的投影.谐运动也可以是振动系统对某一周期正弦力的
响应.谐运动是许多简单系统的典型运动,只要使系统偏离
其稳定平衡位置然后释放并略去阻力得到的即是简谐运动.

谐振子是被回复力或回复力矩束缚在稳定平衡位置的任
何物理系统,其中回复力或回复力矩与离开平衡位置的线位
移或角位移成正比.如果这样一个物体从它的平衡位置被扰
动后释放并且阻尼可以忽略,则由此引起的振动是简谐振
动而没有谐波.振动频率是振子的固有频率,取决于它的惯
性(质量)和回复力的劲度(倔强系数).谐振子并不限于力学
系统也可以是电学系统.(然而典型的电子振荡器只是近似
的谐振子)

\subsection{声压、声强}
声压:没有声波传播时媒质中的压强为$p_0$, 当声波在媒
质中传播时,某点的压强为$p_1$, 定义$p=p_1-p_0$为该点的声
压.随着质点位移的周期性变化,声压也作周期性的变化,声
压与媒质中质点的速度成正比,而且两者的位相相同.频率越
高,越容易获得较大的声压.正是由于声压的变化,引起耳膜
的振动才产生了听觉.在室内大声说话的声压大约为$10^5$帕.

声强:声波的能流密度称为声强,平面简谐波声强的大
小为
\[I=\frac{1}{2}\rho uA^2\omega^2\]
其中$\rho$为媒质密度,$u$为波速,$A$为振
幅,$\omega$为振动角频率).声强的单位是${\rm W/m^2}$.
对于能引起听
觉的声波,不仅有频率的限制,而且声强也必须满足一定的要
求,对每一频率的声波,人能所到的声强都有一个上限值和
一个下限值.低于下限值或高于上限值都不能引起听觉.

\subsection{乐器的基音与泛音}

乐器的种类很多,按其发声部分的性质,可分为弦乐器和
管乐器,凡是振动部分是弦线的,称为弦乐器;凡是利用空气
柱的振动发声,称为管乐器.

\begin{figure}[htp]
    \centering
    \includegraphics[scale=.7]{fig/9-27.png}
    \caption{}
\end{figure}

弦乐器:入射波和反射波的叠加形成驻波.入射波
和反射波叠加在固定端形成波节,在自由端形成波腹.弦乐
器的两端形成波节,最简单的弦振动如图9.27甲上图,波长
是弦长的2倍,或弦长等于半波长,若波的传播速度为$v$, 弦
长为$\ell$, 则所发声音的频率
\[f=\frac{v}{\lambda}=\frac{v}{2\ell}\]
这是基音.弦的振动
还有如图9.27乙、丙……上图所示的情况,即弦长分别等于
半波长的2,3……倍,声音的频率分别为$2f$、$3f$……这些频
率较高的声音叫泛音.

管乐器:管乐器可分为两端开口的开管和一端关闭
的闭管.吹入气流,引起管内气柱的振动,形成驻波,发生乐
音.开管的两端都是波腹,它的基音波长是管长的2倍,如图
9.27甲下图,乙、丙下图是开管的泛音.

闭管末端的空气不能振动,总是波节,所以它的基普波长
等于管长的4倍(如图9.28所示),或者说管长等于基音波长
的1/4.它也产生泛音,管长等于泛音波长的3/4,5/4……

\begin{figure}[htp]\centering
    \begin{minipage}[t]{0.48\textwidth}
    \centering
\includegraphics[scale=.7]{fig/9-28.png}
    \caption{}
    \end{minipage}
    \begin{minipage}[t]{0.48\textwidth}
    \centering
\includegraphics[scale=.7]{fig/9-29.png}
    \caption{}
    \end{minipage}
    \end{figure}

\subsection{弦乐器上共鸣箱的作用}
弦乐器上共鸣箱里的空气柱,只是在弦振动周期力的作
用下,发生受迫振动.一定形状、体积和密度的空气柱的固有
频率是一定的,而弦振动的频率是变化的,故空气柱不可能跟
弦振动的各种频率都发生共振,(声音的共振现象也称共鸣).
实际上弦乐器在演奏时应避免尖锐的共振,因为它会使这一
频率的声音特别响.弦乐器上装“共鸣”箱,要避免共鸣箱对
某一频率过分增强的现象,使琴体空气柱的固有频率偏离演
奏时弦振动的频率,并应有适当的阻尼作用.共鸣箱的主要
作用在于增大发声体辐射面的面积,提高向周围空气辐射声
能的效率,因此把琴筒、琴体等称之为共鸣箱是不确切的,这
“共鸣”二字,只能从增强声音的辐射效率方面去理解.

\subsection{我国区域环境噪声标准}

\begin{center}
    \begin{tabular}{ccc}
        \hline
        适用区域&昼间分贝&夜间分贝\\
        \hline
特殊住宅区&45&35\\
居民、文教区&50&40\\
一类混合区&55&45\\
商业中心区、二类混合区&60&50\\
工业集中区&65&55\\
交通干线道路两侧&70&55\\
        \hline
    \end{tabular}

    单位:等效声级log分贝
\end{center}

说明:“特殊住宅区”指当地人民政府指定的特别需要安
静的住宅地区(如休养地、高级宾馆等).“居民、文教区”指纯
居民和文教、机关地区.“一类混合区”指工业、商业、少量交通
和居民混合区.“商业中心区”指商业集中的繁华地区,“工业
集中区”是指当地政府指定的工业地区.“交通干线道路两
侧”是指车流量每小时一百辆以上的道路两侧.(引自国家劳
动总局主编《噪声控制技术》,上海科技出版社1983年1月第
1版)
















\end{document}