\chapter{前~~言}

为了帮助教师使用好高中物理(甲种本)第一册教材,我
们编写了这本教学参考书,内容包括全册书的说明,引言和
各章的教学说明和资料.

全册书的说明对这册教材的内容安排、全书的一些重要
问题以及课时安排,作了简要的说明。

各章的教学说明和资料,包括教学要求、教学建议、实验
指导、习题解答、参考资料五项内容。在“教学要求”中对各章
的教学内容提出了具体的要求和说明,在“教学建议”中对怎
样进行教学提出了参考性意见。在“实验指导”中提出了演示
实验、学生实验及课外实验活动中应当注意的问题,还提供了
自制简单仪器的制作方法和不同的实验方法,补充了一些实
验内容,供教师选用。在“习题解答”中给出了课本中全部练习
和习题的解答,供教师参考。在“参考资料”中提供了一些教
学中可供参考的材料,这些参考资料只供教师参考,个别的也
可在教学中引用。

本书第一章和第二章“教学建议”、“习题解答”由吴孟
明编写,第三章和第四章的“教学建议”、“习题解答”、“参考资
料”及全书的“实验指导”由袁哲诚编写,第五章和第九章的
“教学建议”、“习题解答”及部分“参考资料”由曹磊编写,第六
章、第七章和第八章的“教学建议”、“习题解答”及第七章的
“参考资料”由唐锦顺编写,其余部分由郭连璧编写。全书由
郭连璧统稿,统稿中刘克桓、董振邦给予了很多帮助,最后经
雷树人审阅修改。

欢迎教师对本书提出宝贵意见。


\chapter{高中物理甲种本第一册的说明}

(1)高中物理课本(甲种本)第一册全部讲力学知识.第
一章讲解力的基本知识,第二章讲解怎样描述运动。在这两
章的基础上第三章讲解运动和力的关系。前两章是必要的预
备知识,而牛顿定律是力学的核心,第四章讲解曲线运动,是
运动定律知识的具体应用和扩展,第五章讲解万有引力定
律,把已学知识应用到天体运动中去,第六章讲解静力学的
基本知识,可视为动力学的特殊情形,第七章和第八章分别
讲解机械能和动量,是牛顿力学的进一步展开,特别是得出守
恒定律,为解决力学问题开辟了新的途径。最后一章是利用
以前学过的知识来分析更为复杂的运动——振动和波。

上述安排,第一章不涉及力的平衡,第二章不讲述平抛和
斜抛物体的运动,这样,使得教学难点后移,头一、二章的内
容相对集中,只包括了学习牛顿定律所必需的预备知识,有助
于降低初高中的台阶,另外,万有引力定律单独设章,使每一
章的中心突出,便于教学。

(2)高中物理难学主要在力学部分,原因之一是某些内
容要求偏高。为了减轻学生的学习困难,使他们对基础知识
掌握得更好些,教学要从实际出发,要求要适当。这里,对大
家一直很关心的几个问题作些说明。

关于矢量。中学物理中需要用矢量解决的问题,一般比
较简单,只要有一维矢量运算以及矢量的合成和分解的知识,
也就够用了,不需要系统地讲解矢量运算的知识,因此,物理
量和公式可以不用矢量符号和矢量式来表示,关于矢量的教
学要求可以归结为三点:第一,懂得什么是矢量,什么是标量,
知道它们之间有什么不同;第二,知道矢量加法服从平行四边
形法则,并且会用它对矢量进行合成和分解;第三,知道一维
矢量的运算可以化成代数运算,并且学会这种运算。

关于连接体。一般的连接体问题本来不是很难,对于培
养学生综合运用知识和分析问题的能力也有好处。但是讲了
连接体,会派生出大量的难题,大大增加了学生负担。因此,
从整体来看,目前,在中学以不讲连接体为宜。对于个别学
校,如果学生理解力较强,教师可以自己补充少量不太难的连
接体问题。这样师生都比较主动,也符合因材施教的原则。

不讲连接体问题,并不是不讲隔离法。把所研究的对象
从周围物体中孤立出来或者隔离出来,单独对它进行受力分
析,并运用动力学或静力学的规律来解决问题,这种分析问题
和解决问题的方法,学生应当学会。

(3)教学要符合学生的认识水平和认识规律,努力做到
循序渐进。

物理的概念和规律的特点之一是有严格的含义,但是在
中学阶段,对物理概念的讲解,不宜过分追求严谨。例如即时
速度这个概念,严格的讲,用极限和微分的知识才能讲解清楚。
由于学生没有学过极限和微分,初学者要透彻理解这个概念
比较困难。因此,讲解这个概念,不宜采用数学意义较浓的讨
论,而要着重从物理意义上予以说明。又如加速度的概念,-
开始就区分平均加速度和即时加速度,这固然严谨,学生接受
却有困难。就匀变速运动来讲解加速度,不再区分平均加速
度和即时加速度,可以克服上述困难,便于教学。

学生掌握知识要有个过程。一个概念,只知道它的定义,
远不能说掌握了它。要在具体问题中不断运用它,逐步体会
它的含义,要在逐步揭示与其他概念的联系中,逐步加深对它
的理解。掌握物理规律和某种分析问题的方法,情况也类
似。例如力这个概念,学生对它的认识是逐步丰富起来的。一
开始只明确力是物体间的相互作用;讲过力的合成和分解之
后,又明确了力是矢量,服从矢量运算规则;讲到动力学揭示
出力和加速度的关系,学生才认识到力是产生加速度的原因,
懂得了牛顿第二定律,在这个基础上,再提出力的独立作用
原理加以讨论,对于物体受力分析的教学,也要逐步深入。一
开始就分析很多复杂的事例,企图一下子就掌握好,一劳永逸
地解决问题,实际上是办不到的。

(4)教学中,在讲解知识的同时,应该注意渗透物理学的
研究方法,以提高学生分析和处理物理问题的能力。

研究和处理问题,首先要明确研究对象,这一点看起来简
单,在具体运用中学生却往往不能把握,造成分析上的混乱。
这一点要在教学中通过具体事例强调说明,使学生能够掌握。

研究问题要从简单情况入手,这不仅是为了方便,而且是
一种科学的研究方法。在简单情况下考虑的因素少,容易把
问题搞清楚。然在此基础上逐步把以前未考虑的因素考虑
进去,情况逐渐复杂,研究逐步展开。研究直线运动从匀速运
动开始,研究振动从简谐振动开始,都属于这种情形,希望学
生对此有所体会。

从简单情况入手研究问题,需要理想化的方法,需要科学
的抽象。教材在讲述质点这一概念时,第一次介绍了这种方
法,此后,凡遇到这种情况,也都予以说明,理想实验也属于
理想化方法。伽利略的斜面对接实验是一个理想实验,介绍
这个实验,可以使学生了解用理想实验这种推理形式能够深
入地把握现象的本质。

从简单情况入手以及理想化的方法,需要分清主次,即抓
住主要因素,暂时舍去次要因素,把问题予以简化,这在研究
和处理问题时十分重要,例如在分析物体受力情况时,如果
物体在光滑平面上运动,可以忽略滑动摩擦力;如果物体的截
面积较小而且运动速度不大,可以不考虑空气阻力,都是采用
了这种简化方法。

还有一些研究处理问题的方法,不再一一列举,希望在教
学中予以注意。

(5)高中物理中抽象思维的作用虽然有所加强,但实验
的重要性不能削弱。高中物理仍然是以实验为基础,要重视
实验教学。

对于学生的实验要求有这样的设想。懂得实验原理,能
根据实验课文的叙述自己确定实验步骤,能正确地使用仪器,
会读取数据并设计表格记录数据,知道怎样分析数据得出结
论,会写简明的实验报告-这些是对所有实验的共同要求。
至于分析误差以及对实验结果的进一步讨论等,则只对部分
实验作要求。

习题中的实验题,要求学生课外自己做,课外实验,不作
要求,但应该鼓励学生去做。有的实验活动,需要教师给予必
要的指导,有的器材学生没有,学校可以借给学生使用。

(6)课本中的练习题分为两种。一是练习,设在每节或
每单元之后,是基本练习题,一部分可随堂做,一部分可留为
作业,一是习题,设在每章之后,其中有综合题和较难的题,
一部分可在习题课中解决,一部分可留为作业。习题中个别
有代表性的题或较难的题,给出了解。这类带解的习题可让
学生自己看,也可作为例题讲,由教师酌定。

习题的安排,要求是逐步提高的。头两章内容是基础性
的,学生又是刚到高中学习,因此安排的都是基本题目。从第
三章开始增加了综合题。第七、八章的要求更高些,注意培养
学生灵活运用知识的能力。经验证明,对学生解题能力的培养,
必须循序渐近。一开始就布置过多过难的练习题,往往是师
生负担重,教学效果不够好的重要原因。这一经验教训,应充
分吸取。

每一章后都有复习题。复习题大都很简单,目的是让学
生通过解答复习题自己写出全章的复习提纲。有的复习题要
求高些,让学生理一下全章的基本思路,总结一下学习经验。

(7)高中物理课本(甲种本)第一册的教学内容可按每周
4课时,全学年共128课时讲授完.各章所用的课时数是:引
言6(3)课时(括号内的数字是学生实验的课时数,下同),第
一章力11(2)课时,第二章直线运动14(3)课时,第三章运动
定律12(2)课时,第四章曲线运动12(2)课时,第五章万有引
力定律6课时,第六章物体的平衡7(1)课时,第七章机械能
15(2)课时,第八章动量13(3)课时,第九章机械振动和机械
波18(1)课时,平时复习和机动时间14课时。

\chapter{引言——怎样学好物理知识}
\minitoc[n]
\section{教学要求}
学生在开始学习高中物理时,往往会不适应,感到困难。
这是因为跟初中物理比较起来,高中物理在广度上有所扩大,
在程度上也有明显的提高。抽象思维和推理论证的作用增大
了,数学的应用增多了,学生实验的要求也提高了。课本安排
了《引言》这一部分内容,先集中讲一讲怎样学好物理知识,就
是要在学习方法上给学生一些引导,使他们能较快地适应高
中物理的学习。

学生实验要测量读取数据和分析处理数据。在做第一次
学生实验前,有必要让学生了解误差和有效数字的初步知识。
因此,在教过《引言》之后,要进行学生实验部分的《误差和有
效数字》的教学。

《引言》及《误差和有效数字》的教学要求是:
\begin{enumerate}
    \item 了解进一步学习物理知识的重要性以及怎样才能学
好高中物理。
\item 了解误差的概念和有效数字的意义,知道在实验测量
中要按有效数字规则读数。
\end{enumerate}

在《引言》的教学中,对于教材中提出的学好高中物理应
该注意的几个问题,学生只要有个初步的了解就可以了,不能
要求他们一下子领会得很好,以后还可以结合各章的教学,使
他们逐步加深认识,不断改进学习方法,适应高中阶段的学习
要求。

关于误差的教学,要求学生了解误差的概念,即知道什么
是误差,什么是系统误差和偶然误差,知道误差是不可避免
的,初步知道怎样分析产生误差的原因,但不要求定量地讨论
和计算误差。

由于有效数字的运算规则比较复杂,教材未作介绍,也不
要求教师讲解。只要求学生懂得有效数字的意义,在实验测
量中能按有效数字规则读数,在处理实验数据和解题时,运算
结果一般取两位或三位数字就可以了。

\section{教学建议}
《引言》从回顾初中学过的物理知识讲起,进一步明确物
理知识的重要,指出了高中物理的特点。重点是对怎样学好
高中物理提出了几个应当注意的问题,即如何做好物理实验、
如何学好物理概念和规律以及如何做好练习。

《引言》的教学不必拘泥于课文的叙述,教师可以根据《引
言》的教学要求、自己的教学经验和学生的实际情况,灵活地
组织教学。

\begin{enumerate}
    \item \textbf{消除物理难学的惧怕心理}\quad  在说明高中物理的特点,
指出跟初中物理相比较有明显提高时,还要注意消除一般学
生思想上存在的物理难学的惧怕心理,树立学好高中物理的
信心,这可以从两个方面来说明:一是已有两年初中物理的
学习基础;二是只要注意不断改进学习方法,就会逐步适应高
中物理的学习。
\item \textbf{激发学生学习物理的兴趣}\quad  学生只有对学习物理发
生了兴趣,才能积极主动、心情愉快地学习物理,收到良好的
学习效果,因而在引言的教学中,如何激发学生学习物理的
兴趣,显得十分重要。学生的学习兴趣在很大程度上产生于
求知的渴望,教学中应该多举一些学生熟悉和感兴趣的事例,
来说明物理知识的重要。例如,人造卫星为什么能够围绕地
球运行?电冰箱为什么能够致冷?收音机和电视机为什么能接
收声音和图像信号,把它们再现出来?照相机和电影放映机的
原理又是什么?什么是原子能?等等。要了解这些问题,都需
要学习物理知识。还可以通过能为学生接受的,生动具体的
事例,说明物理学对促进科学技术的发展和人类社会的进步
所起的巨大作用。让学生了解,目前世界已进入了原子能、电
子计算机、自动化、半导体、激光、空间科学等新技术的时代,
现代科学技术正经历一场伟大的革命,使学生把学习物
面向现代化、面向世界、面向未来联系起来,增强他们的学习
责任感,激发他们的求知欲望。
\item \textbf{密切联系实际,指导学习方法}\quad  讲述如何做好物理实
验、如何学好物理概念和规律以及如何做好练习时,切忌照本
宣科,空洞地说教。要通过具体的事例来说明问题。例如,讲
述做实验的要求,可结合学生在初中实验中存在的一般问题,
来说明怎样做是对的,怎样做是不对的;讲述要注意观察演示
实验时,可实际做一、两个简单的实验,具体说明这个实验研
究的是什么问题,应当观察什么,怎样观察,应当记录什么现
象和数据,怎样记录,怎样分析看到的现象和处理实验数据,
最后得到怎样的结论,等等;讲述如何解题时,也应结合具体
例题说明解题的思考方法和一般步骤,给学生做一个良好的
示范。
\item \textbf{误差和有效数字的教学}\quad  这部分内容按照课本的要
求进行教学就可以了,不宜作更多的补充。参考资料中介绍
的有关内容,只供教师参考,不宜向学生讲述。但可以告诉学
生,误差和有效数字之间的关系;有效数字的最后一位就是误
差所在的一位;由误差决定有效数字,这是处理一切有效数字
问题的依据。因此在中学物理实验中,-一般来说,要求估读到
测量仪器最小分度的十分之几。由于误差和有效数字问题比
较复杂,在处理实验数据和解题时,运算结果一般取两位或三
位数字就可以了。还应告诉学生,对于原来有效数字位数不
多的数字,运算结果再取更多的位数,也是毫无意义的。
\end{enumerate}


\section{实验指导}
\subsection{学生实验}
\subsubsection{练习分析实验数据}
这个实验并不要求学生进行实际测量,而是根据课
本给出的某一次实验中所得的数据,来练习分析和处理实验
数据的技能,了解分析实验数据对于得出物理量之间的关系
和研究物理量变化规律所具有的重要意义。

教材要求通过给出的实验数据,用图象方法找出一
定量的水,从容器底部不同直径的排水孔流出时,排尽水的时
间$t$跟排水孔直径$d$的定量关系。在讲解这种方法时,要使学
生了解“自变量”和“因变量”是根据所要研究的两个物理量的
变化关系来确定的。对于容器中一定量的水来说,排尽水的
时间$t$由水孔直径的大小$d$所决定,所以排水孔直径$d$是
自变量,排尽水的时间$t$是因变量。但是,应该让学生注意,
由于研究的问题不同,同一个物理量,有时是自变量,有时又
是因变量。譬如,在研究做匀速直线运动的物体经过的路程
对时间的关系时,时间就不再是因变量,而是自变量了。

在得出$t$-$d$图象是一条曲线之后,可以看出$t$随$d$
的增大而减小,但这只是一个定性的关系,从生活经验或单
从数据表也就足以判断了。所以必须作进一步的分析与猜想,
以便得出定量的关系。由于水是从整个圆孔排出的,排尽水
的时间$t$是否跟圆孔的横截面积$S$存在简单的反比关系呢?
因为圆面积$S$是跟直径$d$的平方成正比的,如果这一猜想成
立,那么,排尽水的时间$t$应该和$d$成反比,即$t$与$1/d^2$成正
比.因此要计算出$1/d^2$的数值,描绘出$t$-$1/d^2$图象,看看是
不是一条直线。要使学生明白建立在实验基础上的合理的猜
想,是研究物理问题的一种重要的思想方法。

在分别画出水深为30厘米和水深为10厘米的
$t$-$1/d^2$图象后,可以让学生自己得出排尽水的时间$t$跟排水
孔横截面积$S$存在着怎样的关系,还可以让他们考虑,这一关
系是否会因排水时的水深不同而有所改变?

\subsubsection{游标卡尺的使用}

这个实验要求学生通过实际操作,了解游标卡尺的
构造和它的读数原理,正确掌握读数方法,学会正确使用游标
卡尺。

实验前可利用游标卡尺的放大模型(也可以按课本
图10.2所示的游标尺部分自制)先使学生理解主尺最小分度
是1毫米、游标尺上有10个小的等分刻度、总长度等于9毫
米、准确度为0.1毫米的游标卡尺的读数原理.使用这种游
标卡尺读数时在0.1毫米后还可估读一位,但不要求估读(详
见参考资料)。

根据学生实际使用的游标卡尺制成游标尺部分的放
大模型,让学生结合所用的游标卡尺,观察了解这种卡尺的主
尺最小分度是多少毫米?游标尺上有多少个小的等分刻度t总
长度等于多少毫米?读数时的准确度为多少毫米?
如果使用准确度为0.05毫米或0.02毫米的游标卡尺,由
于读数误差发生在毫米读数的百分位,因此不应再估读一位
(详见参考资料)。

在使用游标卡尺练习测量金属管的长度、内径和外
径时,课本要求在测长度时每次测量后让金属管绕轴转过
$45^{\circ}$再测量下一次;测内径和外径时,要在管子的两端分别量
出两个互相垂直的内,外径,然后分别求出它们的平均值。应
该使学生了解为什么要这样测量的原因。采用多次测量求平
均值的方法可以减小测量时的偶然误差,但是在这个实验中
多测量是为了弥补金属管可能不是一个理想圆柱体而造成的
误差,这也是一种常规的测量方法。

得出金属管的内径和外径的平均值求得金属管的体
积后,还可以启发学生思考,利用这些数据如何来计算金属管
的管壁厚度?

\subsubsection{螺旋测微器的使用}
这个实验要求学生通过操作了解螺旋测微器的构造
和它的读数原理,正确掌握读数方法,学会正确使用螺旋测
微器。

实验前可利用螺旋测微器的放大模型,让学生结合
所用的螺旋测微器进行观察,了解其结构,并让学生将旋钮$K$
向逆时针方向慢慢旋动,使测微螺杆$P$慢慢后退,同时观察当
$P$后退0.5毫米时,可动刻度$H$恰好转过50格,即转过一周.
因此,可动刻度每转过1格,就相当于沿着螺旋的轴线方向移
动$0.5/50$毫米$=0.01$毫米(不要误认为可动刻度上每1格的
长度-两条刻度线间的距离等于0.01毫米),所以用螺旋
测微器测量长度时可以准确到0.01毫米.

关于读数问题.螺旋测微器的零误差在使用前应先
调整好,使得基本上没有零误差,这样,在读数时只需先读出
固定刻度尺上的毫米读数,再加上可动刻度的读数,而不必考
虑零误差的存在。读数时要特别注意观察固定刻度尺上表示
半毫米的刻线是否已经露出,由于固定刻度尺上的刻线较粗,
有时很难判断究竟半毫米的刻线已经露出还是即将露出,这
时就应看可动刻度$H$上零刻线的位置是在固定刻度尺的准线
之上,还是在准线之下,如果在准线之下,就表示半毫米的刻
线已经露出(图1),则在读数时应加上0.5毫米.此外,在
读数时还应估计一位读数(不可靠的),如果认为可动刻度$H$
上的基条刻线正好跟固定刻度尺$S$上的准线重合,则读数时
最后一位的估计读数应为0, 在记录读数时,这个0也应写
出,虽然这个0并不是可靠的.
如图1所示的读数应为6.540
毫米。

\begin{figure}[htp]
    \centering
\includegraphics[scale=.8]{fig/1.png}
    \caption{}
\end{figure}


关于正确使用螺旋测
微器的方法。应指导学生阅读
课本340页上有关的叙述,并通过演示示范,要求学生严格
遵守。

在完成课本要求的测量后,还可让学生用螺旋测微
器测量一下自己的头发的直径,如果时间允许,还可以事先准
备一些包装香烟用的薄铝箔(去除后面的衬纸,铝箔厚度约为
0.01毫米左右),让学生分别用游标卡尺和螺旋测微器测量它
的厚度,从而体会螺旋测微器是比游标卡尺更精密的湖量长
度的工具。

\section{参考资料}
\subsection{测量误差的估计}
测量的结果不可能绝对精确,总会产生误差,对于测量
结果,可以信任到何种程度,需要知道测量误差,由于误差是
测量值与真实值之差,而真实值本身是不能确切知道的,因此
对于测量误差只能是估计。

\subsubsection{多次测量结果偶然误差的估计}
为了减少偶然误差,在可能的情况下,总是采用多次测
量,以多次测量的算术平均值作为测量结果,即
\[N=\overline{N}=\frac{1}{k}(N_1+N_2+\cdots+N_k)\]
根据误差的统计理论,算术平均值$N$最接近于真实值。

在这种情况下,通常简单的估计方法,是用算术平均偏差
$\delta$来表示多次测量结果的偶然误差。

设第$i$次测量值$N_i$, 与平均值$N$的偏差为$\delta_i$, $i=1,2,\ldots,k$,即
\[\delta_1=N_1-N,\; \delta_2=N_2-N,\ldots,\delta_k=N_k-N\]
则算术平均偏差
\[\delta=\frac{1}{k}(|\delta_1|+|\delta_2|+\cdots+|\delta_k|)\]

\subsubsection{一次测量结果误差估计}
有些情况下,测量不能重复或者不需要精确测量,则需要
估计一次测量结果的误差,这时可以根据测量仪表所注明的
误差来估计.例如2.5级电表,它的一次测量误差可估计为
满刻度的2.5\%, 如果没有注明,可取仪表最小分度值的一半
作为测量误差,例如最小分度为毫米的刻度尺,它的一次测
量误差可估计为0.5毫米.

需要说明的是,测量误差应当包括系统误差和偶然误差
两个方面,有些情况下,主要是系统误差;有些情况下,主要
是偶然误差。估计误差时,要作具体的分析,实际上,往往情
况比较复杂,要对测量结果的误差作出估计,不是一件容易的
事,在中学,并不要求学生估计误差。

\subsection{绝对误差和相对误差}
实际上,常把测量结果写成$N\pm\Delta N$的形式。其中$N$是
测量值,它可以是一次测量值,也可以是多次测量的平均值
$\overline N$; $\Delta N$是测量误差值,叫做绝对误差。绝对误差给出了测量值
的误差范围,但这并不排除多次测量中有的测量值在$N\pm\Delta N$
以外。

用绝对误差不能对测量结果的好坏给出一个十分清楚的
概念。于是引入了相对误差的概念,相对误差用$\Delta N/N$表示,
也叫做百分误差。

相对误差与绝对误差的关系是
\[\Delta N=N\x \frac{\Delta N}{N}\]

误差与有效数字密切相关,由于误差本身是一个估计数,
所以,一般情况下误差的有效数字只取一位,在特殊情况下也
不超过两位,多了是没有意义的。

根据有效数字的含义,有效数字的最后一位是有误差的。
因此,有效数字的最后一位一定要同误差所在的一位取齐。这
就是说,有效数字的位数取决于绝对误差。

相对误差与有效数字之间的关系,大体上讲,有效数字的
位数越多,相对误差就越小;有效数字的位数越少,相对误差
就越大,例如,$1.320\pm 0.001$厘米,有效数字是4位,相对误
差$\frac{0.001}{1.320}\approx 0.08\%$; $1.3\pm 0.1$厘米,有效数字是2位,相对误差$\frac{0.1}{1.3}\approx 8\%$, 一般说来,两位有效数字的相对误差为十分之
几至百分之几,三位有效数字的相对误差为百分之几至千分
之几,依此类推。

\subsection{测量仪器的读数规则}
原教育部颁布的《高中物理教学纲要(草案)》,要求学生
实验测量中能按有效数字规则读数。那么,有效数字的读数
规则是什么,如何要求学生按有效数字规则读数呢?

有效数字的最后一位一定要同误差所在的一位取齐,这
是考虑有效数字的依据。因此,测量仪器的读数规则应当是:
测量误差出现在哪一位,读数时就应读到哪一位,这样,就要
首先估计测量误差,然后再确定读到哪一位。
这里讨论一下电表的读数问题,中学生使用的电表的准
确度是2.5级,即在规定的使用条件下,最大误差不超过满刻
度的2.5\%.

安培表有3安培和0.6安培两个量程,使用3安培量程
时,误差是3安$\x2.5\%=0.075$安,即误差出现在安培的百分
位.这时表的最小分度是0.1安培,即可以准确读到安培的
十分位,因此,使用3安培量程时,应估读一位到安培的百分
位.使用0.6安培量程时,误差0.6安$\x2.5\%=0.015$安,即
误差也是出现在安培的百分位,这时表的最小分度是0.02安
培,即可以准确读到0.02安培,因此,使用0.6安培量程时,
只能读到安培的百分位,可以估读半小格,如果估读到安培
的千分位,则是无意义的了。

伏特表有15伏特和3伏特两个量程使用15伏特量程
时,误差是0.37伏特,即误差出现在伏特的十分位.这时表
的最小分度是0.5伏特,即可以唯确读到0.5伏特。因此,使
用15伏特量程时,应读到伏特的十分位,即可以估该五分之
一小格,使用3伏特量程时,跟安培表3安培量程的情况一
样,应估读一位到伏特的百分位。

可以看出,实验测量中,究竟读取几位有效数字,要作具
体分桥、然而,中学关于误差和有效数字的教学要求不高,并
不要求估计测量误差,在处理实验数据和解题时,运算结果一
般取两位或三位数字就可以了。因此,在中学的实验测量中,
一般来说,可以要求学生估读到测量仪器最小分量的十分
之几。

\subsection{游标卡尺的读数问题}
课本里讲的十分度游标卡尺,游标有10个等分刻度,总长
等于9毫米(图2)。这种卡尺的游标读数值(即主尺与游标每
个分度的差值)是$1.0{\rm mm}-\frac{9.0}{10}{\rm mm}=0.1{\rm mm}$,还有一种十
分度游标卡尺的游标总长等于19毫米(图3)。这种情况下,
主尺上的两个分度(2毫米)与游标上的一个分度相当,因此它
的游标读数值是$2.0{\rm mm}-\frac{19}{10}{\rm mm}=0.1{\rm mm}$,即也是0.1毫米.

\begin{figure}[htp]\centering
    \begin{minipage}[t]{0.48\textwidth}
    \centering
\begin{tikzpicture}[>=latex, scale=.8]
\foreach \x in {1,2,...,14}
{
    \draw(\x/3, 0)--(\x/3,.5);
}
\foreach \x in {0,5,10}
{
    \draw(\x/3, 0)--(\x/3,.8)node[above]{$\x$};
    \draw(\x/3*.9, 0)--(\x/3*.9,-.8)node[below]{$\x$};
}
\foreach \x in {1,2,...,9}
{
    \draw(\x/3*.9, 0)--(\x/3*.9,-.4);
}
\draw(-.5,0)--(5.25,0);
\node at (2.5,1.8){主尺};
\node at (2.25,-.8){游标};
\node at (5,.8){mm};
    \end{tikzpicture}
    \caption{}
    \end{minipage}
    \begin{minipage}[t]{0.48\textwidth}
    \centering
    \begin{tikzpicture}[>=latex, scale=.8]
\foreach \x in {1,2,...,9}
{
    \draw(\x/2*.95, 0)--(\x/2*.95,-.4);
}
\foreach \x in {0,5,10}
{
    \draw(\x/2*.95, 0)--(\x/2*.95,-.8)node[below]{$\x$};
}
\foreach \x in {1,2,...,19,21,22,23}
{
    \draw(\x/4, 0)--(\x/4,.4);
}
\foreach \x in {0,5,...,20}
{
    \draw(\x/4, 0)--(\x/4,.8)node[above]{$\x$};
}
\draw(-.5,0)--(6.25,0);
\node at (2.8,1.8){主尺};
\node at (3,-.8){游标};
\node at (6,.8){mm};
    \end{tikzpicture}
    \caption{}
    \end{minipage}
    \end{figure}



不难看出,游标卡尺的测量误差为游标读数值的一半。因
为当我们采用游标上的某一刻度读数时,这一刻度与主尺上
相当的刻度的距离就不会超过游标读数值的一半。否则,跟
游标上这一刻度左右相邻的两个刻度中,必有一个跟主尺上
的刻度更为接近。

因此,游标读数值是0.1毫米的游标卡尺,误差是0.05毫
米,即误差出现在毫米的百分位。于是,这种游标卡尺,毫米
的百分位可以估读为“0”,表示误差出现在毫米的百分位,如
果无法判断游标上相邻的两条刻度哪一条跟主尺上的刻度重
合或更接近,则毫米的百分位可估读为“5”.如图4所示,读
作0.55毫米.

\begin{figure}[htp]\centering
    \begin{minipage}[t]{0.48\textwidth}
    \centering
\begin{tikzpicture}[>=latex, scale=.8]
\foreach \x in {1,2,...,14}
{
    \draw(\x/3, 0)--(\x/3,.5);
}
\foreach \x in {0,5,10}
{
    \draw(\x/3, 0)--(\x/3,.8)node[above]{$\x$};
    \draw(\x/3*.9+.1667, 0)--(\x/3*.9+.1667,-.8)node[below]{$\x$};
}
\foreach \x in {1,2,...,9}
{
    \draw(\x/3*.9+.1667, 0)--(\x/3*.9+.1667,-.4);
}
\draw(-.5,0)--(5.25,0);
\node at (2.5,1.8){主尺};
\node at (2.35,-1.8){游标};
\node at (5,.8){mm};
    \end{tikzpicture}
    \caption{}
    \end{minipage}
    \begin{minipage}[t]{0.48\textwidth}
    \centering
    \begin{tikzpicture}[>=latex, scale=.8]
\foreach \x in {1,2,...,25}
{
    \draw(\x/5, 0)--(\x/5,.4);
}     
\foreach \x in {0,1,2}
{
    \draw(\x*2, 0)--(\x*2,.6)node[above]{\x};
}   
\draw(-.5,0)--(5.5,0);
\foreach \x in {1,2,...,19}
{
    \draw(\x*.19,0)--(\x*.19,-.4);
}
\foreach \x in {0,25,...,100}
{
    \draw(\x*.19/5,0)--(\x*.19/5,-.6)node[below]{\x};
}
\node at (5.6,.8){cm};
\node at (2.7,1.8){主尺};
\node at (2.7,-1.8){游标};
    \end{tikzpicture}
    \caption{}
    \end{minipage}
    \end{figure}

二十分度(即游标有20个等分刻度)游标卡尺的游标总
长等于19毫米(图5)或等于39毫米(图6)。它们游标读
数值是$1.0{\rm mm}-\frac{19}{20}{\rm mm}=0.05{\rm mm}$或$2.0{\rm mm}-\frac{39}{20}{\rm mm}=
0.05{\rm mm}$,即都是0.05毫米,还有一种十分度游标卡尺,主尺
的最小分度是0.5毫米,游标总长等于4.5毫米(图7)。它
的游标读数值也是0.05毫米(注意:这种情况下,主尺上的一
个分度是0.5毫米,跟游标上的一个分度0.45毫米相当).

\begin{figure}[htp]\centering
    \begin{minipage}[t]{0.48\textwidth}
    \centering
\begin{tikzpicture}[>=latex, scale=1]
\draw(-.5,0)--(5,0);
\node at (5,.5){cm};
\foreach \x in {1,2,...,44}
{
    \draw(\x*.1,0)--(\x*.1,.4);
}
\foreach \x in {0,1,...,4}
{
    \draw(\x,0)--(\x,.7)node[above]{\x};
}
\foreach \x in {5,15,...,35}
{
    \draw(\x*.1,0)--(\x*.1,.6);
}
\foreach \x in {1,2,...,19}
{
    \draw(\x*.195,0)--(\x*.195,-.4);
}
\foreach \x in {0,25,...,100}
{
    \draw(\x*.195/5,0)--(\x*.195/5,-.6)node[below]{\x};
}
\node at (2.3,1.5){主尺};
\node at (2.3,-1.5){游标};

    \end{tikzpicture}
    \caption{}
    \end{minipage}
    \begin{minipage}[t]{0.48\textwidth}
    \centering
    \begin{tikzpicture}[>=latex, scale=1]
        \draw(-.5,0)--(3.5,0)node[above]{mm};
\foreach \x in {1,2,...,6}
{
    \draw(\x*.5,0)--(\x*.5,.6)node[above]{\x};
    \draw(\x*.5-.25,0)--(\x*.5-.25,.4);
}
\foreach \x in {10,20,...,50}
{
    \draw(\x*.045,0)--(\x*.045,-.6)node[below]{\x};
    \draw(\x*.045-.225,0)--(\x*.045-.225,-.4);
}
\draw(0,.6)node[above]{0}--(0,-.6)node[below]{0};
\node at (1.3,1.5){主尺};
\node at (1.3,-1.5){游标};
    \end{tikzpicture}
    \caption{}
    \end{minipage}
    \end{figure}

对于游标读数值是0.05毫米的游标卡尺,误差是0.025
毫米,即误差出现在毫米的百分位,而用这种游标卡尺测量
长度时,可以直接读到毫米的百分位,因此无需再估读了。

还有两种游标读数值是0.02毫米的游标卡尺,如图8
和图9所示,它们的误差是0.01毫米,即误差也是出现在
毫米的百分位,用这种游标卡尺测量长度时,也是可以直接
读到毫米的百分位,因此也无需再估读了。

\begin{figure}[htp]\centering
    \begin{minipage}[t]{0.48\textwidth}
    \centering
\begin{tikzpicture}[>=latex, scale=.8]
\draw (-.5,0)--(6,0);
\foreach \x in {.1,.2,...,5.4}
{
    \draw(\x,0)--(\x,.4);
    
}
\foreach \x in {.1,.2,...,5}
{
    \draw(\x*4.9/5,0 )--(\x*4.9/5,-0.4);
}
\foreach \x in {0,1,...,5}
{
    \draw(\x*1,0)--(\x*1,.7)node[above]{$\x$};
}
\foreach \x in {0,1,...,10}
{
    \draw(\x*4.9/10,0)--(\x*4.9/10,-.7)node[below]{$\x$};
}
\node at (2.5,1.6){主尺};
\node at (2.5,-1.6){游标};
\node at (5.8,.6){cm};
    \end{tikzpicture}
    \caption{}
    \end{minipage}
    \begin{minipage}[t]{0.48\textwidth}
    \centering
    \begin{tikzpicture}[>=latex, scale=.8]
\draw (-.5,0)--(6.5,0);
\foreach \x in {1,3,...,29}
{
    \draw(\x*.2,0)--(\x*.2,.3);
    \draw(\x*.2+.2,0)--(\x*.2+.2,.4);
}
\foreach \x in {0,5,...,15}
{
    \draw(\x*.4,0)--(\x*.4,.6)node[above]{$\x$};
}
\foreach \x in {1,2,...,24}
{
    \draw (\x*4.8/25,0)--(\x*4.8/25,-0.3);
}
\foreach \x in {0,10,...,50}
{
    \draw (\x*4.8/50,0)--(\x*4.8/50,-0.5)node[below]{$\x$};
}



\node at (3.5,1.6){主尺};
\node at (3.5,-1.6){游标};
\node at (6.8,.6){mm};
    \end{tikzpicture}
    \caption{}
    \end{minipage}
    \end{figure}


综上所述,用各种常用的游标卡尺测量长度时,误差都出
现在毫米的百分位。由于游标卡尺的误差问题比较复杂,游
标读数值为0.02毫米和0.05毫米的游标卡尺也不需要估读,
因此中学课本里没有讲解游标卡尺的误差问题,也不要求中
学生使用游标卡尺时进行估读。














