\chapter{动量}

\section{教学要求}

动量是力学中的重要概念,动量守恒定律是自然界最重
要的普遍规律之一。因此在甲种本中单设一章来对有关知识
作较深入的讨论。

这一章的教学要求是:
\begin{enumerate}
\item 理解冲量和动量两个概念,掌握动量定理并会用它解
释一些物理现象。
\item 掌握动量守恒定律并会用它分析、计算有关的问题.
\item 理解弹性碰撞和非弹性碰撞,会计算一维弹性碰撞的
有关问题。
\item 了解反冲运动的原理及其在火箭技术中的应用。
\end{enumerate}

下面对这一章的教学内容作些具体说明。

冲量和动量是两个不容易理解的概念。教材在分析具体
事例的基础上,引用公式进行讨论,得出这两个概念,这样做,
对初学者来说,概念的物理意义可能清楚些,容易理解些。动
量的矢量性在研究动量定理和动量守恒定律时都很重要。但
是学生初学时往往对此认识不够,需要一开始讲解动量时,就
强调它的矢量性。

第二节讲解动量定理。要使学生明确:动量定理$Ft=
mv'-mv$表示的是力在一段时间内连续作用的累积效果与物
体动量变化间的关系;在变力的情况下,公式中的$F$表示变
力在时间$t$内的平均值。

动量守恒定律也可以用牛顿第二定律和第三定律推导出
来。这种讲法的好处是比较简便。然而,动量守恒定律是一
个独立的实验定律,而且它的适用范围比牛顿运动定律更广
泛。因此,教材以实验为基础总结出动量守恒定律,然后说明
这一定律与牛顿运动定律的一致性。

第三节讲述相互作用的物体的动量变化的实验,是为第
四节讲动量守恒定律作准备的。在归纳出动量守恒定律时应
向学生说明,这个定律是在分析研究大量实验事实的基础上
建立起来的,而不是仅靠几个实验得出来的。还应使学生清
楚系统动量守恒的条件,防止解题时不问条件,乱套公式的
倾向。

第五节讲解动量守恒定律和牛顿运动定律的关系。应当
使学生清楚地了解这两个定律在牛顿运动定律运用的范围内
是一致的,这两个定律的区别在于适用范围的不同。牛顿运
动定律只适用于宏观物体的低速运动情况,而动量守恒定律
的适用范围更普遍,不论是宏观物体还是微观粒子,低速运动
还是高速运动,相互作用是什么性质的,都遵守动量守恒
定律。

第六节讲述如何运用动量守恒和动能守恒来研究碰撞问
题。应该使学生明确,对于弹性碰撞需要同时运用动量守恒
和动能守恒来解决。讲解弹性碰撞,要注意培养学生运用两
个守恒定律分析解决问题的能力,进一步认识守恒定律的重
要性,例题中得出的两个钢球发生弹性正碰后的速度表达式,
应要求学生会自行推导,而不是记住它们。同时,要求学生能
够利用表达式讨论一些具体情况.例如,$m_1>m_2$, $m_1=m_2$,
$m_1<m_2$时,两钢球碰撞后的运动情况;$m_1\ll m_2$, $m_1=m_2$, 
$m_1\gg m_2$时,两钢球碰撞中能量的传递情况.并能根据讨论的
结果,解释一些弹性碰撞中发生的现象。

第七节反冲运动中讲述的内容都属于常识性的介绍,目
的是使学生了解物理知识在现代科学技术中的应用,介绍一
些有关的科学技术常识,开阔学生的眼界。

\section{教学建议}
全章教学可分三个单元。第一单元(第一、二节)讲授冲
量和动量的概念以及动量定理。第二单元(第三至五节)讲授
动量守恒定律,这是本章的重点。第三单元(第六、七节)讲
授碰撞和反冲运动,这是动量守恒定律的应用。

\subsection{第一单元}
\subsubsection{冲量和动量概念的引入}

冲量和动量这两个概念也
同功和能一样,不是一引入就能体会它的物理意义的。对这
两个概念的理解,要通过整章的教学过程逐步加深。所以第
一节的教学既要使学生初步了解冲量和动量的意义,特别是
为什么要引入这两个概念以及两个物理量的定义、计算式、单
位和矢量性等,又不能操之过急,要求过高。

课本在引入冲量和动量的概念时,通过日常生活中常见
的事实,以开动汽车为例,说明汽车获得一定的速度不仅同它
的牵引力有关,而且同力的作用时间有关。然后应用牛顿运
动定律和运动学的知识,得出了速度的变化跟力和作用时间
的关系$Ft=mv$. 再通过对这个式子的讨论给出冲量和动量
的定义。这样引入,除了物理意义比较明显、较易理解外,还
可以使学生从一开始就认识这两个概念之间的密切联系,了
解冲量(它表现力的作用)的效果是使物体获得动量。

教学中还应该使学生了解速度和动量的联系和区别。速
度和动量都可以作为运动的量变。但是速度只告诉我们物体
运动的慢快和方向,没有告诉我们使物体运动或者停止运动
需要多大的冲量。动量没有告诉我们物体运动速率的大小,
却能告诉我们使物体运动或停止运动所需的冲量。所以速度
是一个运动学的量,只能用来描述运动,而动量则是一个动力
学的量,它跟冲量即物体运动变化的原因相联系。这里,可以
用课本上以相同的速度运动的铅球和乒乓球的例子来说明。

\subsubsection{动量的变化}

学生初学动量时,往往忽略动量的矢量
性,只注意它的大小,不注意它的方向,以为只要物体的速度
大小不变,动量也就不改变.教学时可利用课本252页的例
题纠正这种错误认识。要想学生考虑钢球从坚硬的障碍物上
弹回,动量的大小并没有变,动量的变化为什么不等于零?原
因是动量是矢量,它的方向改变了,所以动量发生了变化。

教材明确指出,所谓动量的变化就是变化后的动量减去
变化前的动量,因此这里所说的动量的变化实际上就是上一
章所说“增加”或“增量”。具体计算时,由于我们只研究一
直线上的动量变化,所以只要先选定一个方向为正,就可以
把矢量运算简化为代数运算。

\subsubsection{关于打击时的平均作用力}
动量定理可以看作是牛顿第二定律的变形,在这里没有
必要作过多的讲解,可只限于用来解释一些这个定理便于解
释的现象,如为什么茶杯掉在地上要碎,而掉在软的东西上就
不易摔碎等。在计算方面,只限于计算打击、碰撞等问题中的
平均作用力,如课本255页中的例题那样。

课本255页例题后面有一段讨论.讨论中说“如果把铁
锤的重量也考虑在内,那么,这时道钉所受的打击是上面算
出的打击力加上铁锤的重量。”如果学生对此提出怀疑,教学
中可加以说明。应该指出,动量定理$Ft=mv'-mv$中的$F$是
物体所受的合力。所以在这个例题中应该列出如下方程:
\[F=\frac{mv'-mv}{t}=2.5\x10^3{\rm N}\]
而$F=N-G$, 所以
\[N=F+G=2.5\x10^3+49=2549{\rm N}\]

至于什么时候可以忽略铁锤的重量$G$, 这要看$G$与$F$的
差别有多大.象书上例题中$G$为$F$的2\%, 就可以忽略了。

\subsection{第二单元}
本单元是全章的重点。整个单元就是围绕一个中心——
动量守恒定律展开的。

\subsubsection{相互作用物体的动量变化}

第三节教学的关键是要
做好研究相互作用物体的动量变化的演示实验。实验可分为
两部分.第一部分为定性研究,如课本图8.2那样,目的是说
明两个相互作用的物体动量会发生变化,在课堂上可用两个
压紧弹簧的小车代替图8.2中的两个人,也可以用玩具小车
在可动的平板上的运动来演示。

第二部分为定量研究,要求分析课本插页的气垫导轨上
相互作用滑块的闪光照片,得出动量变化总是大小相等方向
相反的结论,教学中要注意引导学生学会如何根据闪光照片
计算滑块速度,并测量动量变化的方法,培养分析原始实验资
料的能力。根据学校的实验设备条件,也可以在课堂上当场
演示,得出同样结论。例如用光电计时器对气垫导轨上滑块
的速度测定来研究共动量的变化;用打点计时器研究相互作
用小车的动量变化;利用平抛原理测定相互作用小球的动量
变化等,有条件的学校还可以让学生自己动手来得出这个结
论。这一节的实验做得越好,对下一节动量守恒定律的学习
就更有利,教学中要十分重视。

\subsubsection{动量守恒的语言表达}

从“相互作用的物体的动量变
化总是大小相等、方向相反”的结论到“系统的动量守恒”的推
理过程,对语言表达的要求较高,教学中教师除应注意自己
授课语言的准确外,还应注意培养学生学习物理的语言表达
能力。推理过程有三个层次:第一是实验结论,“相互作用的
物体的动量变化总是大小相等,方向相反的”;第二是引入系
统总动量的概念,得出“系统的总动量的变化为零”;第三,再
得到系统的总动量守恒,以上三个层次,用公式表达,可与语
言表达对照如下(表中的$p_1$、$p'_1$、$p_2$、$p_2'$都是矢量。如果两个
物体作用前后都在一直线上运动,则上述四个量均代表带正
负号的动量数值。其中最后一式常常写成$m_1v_1+m_2v_2=m_1v'_1
+m_2v_2'$)。

\begin{center}
\begin{tabular}{p{.35\textwidth}c}
    \hline
    用语言表达&用公式表达\\
    \hline
    相互作用物体的动量变化总是大小相等,方向相反的&$p'_1-p_1=-(p'_2-p_2)$\\
    总动量的变化为零&$(p'_1+p'_2)-(p_1+p_2)=0$\\
    系统的总动量守恒&$p'_1+p'_2=p_1+p_2$\\
    \hline
\end{tabular}
\end{center}

\subsubsection{动量守恒定律的条件}
课本明确指出,系统动量守恒
的条件是“系统不受外力或所受外力的合力为零”。教学中要
重视培养学生在应用动量守恒定律时先检验是否符合守恒定
律条件的习惯,防止随意乱套公式,但是有一点应该向学
说明,在有些问题中,系统虽然受到外力作用,而且合力不为
零,但合外力同内力相比非常小,可以忽略不计时,动量守恒
定律仍然可以适用.例如课本260页例题,在列车间相互作
用时,重力和轨道支持力虽然合力为零,但列车和轨道间总有
摩擦力存在。由于这个摩擦力同列车间的内力相比很小,所
以仍可用动量守恒定律。又如手榴弹爆炸时,虽然整个系统
受到重力作用,但比起炸药的爆炸力来重力很小,可以忽略,
因此还是可以用动量守恒定律来计算爆炸后碎块的速度。

\subsubsection{动量守恒定律是普遍适用的物理规律}

动量守恒定
律是自然界最重要的普遍规律之一。教材在第三节末说了
相互作用物体的动量变化总是大小相等方向相反的结论不仅
适用于正碰,而且适用于斜碰。在第四节第一段,教材又说到
这个结论“在任何情况下”都成立,在第四节中,又特别提出三
点说明,对上述“任何情况”作了具体的解释。为了使学生对动
量守恒定律的普遍性有深切的了解,教学中可以多举一些实
例,特别是对说明中第1点可多举例说明,例如,子弹射入木
块、火车车厢连接在一起,课本插页图8.6的碰撞,属于粘合
在一起的例子,炮弹和手榴弹的爆炸属于分裂成碎块的例子。
此外,枪炮发射弹丸,人在船上行走等相互作用的例子,也可
用动量守恒定律来研究.对第2、3点,由于学生知识水平
有限,不可能理解得很具体,只要有一个印象就可以了。

\subsubsection{动量守恒定律中物体的速度}

在两个物体相互作用
对应用动量守恒定律可用表达式
\[m_1v_1+m_2v_2=m_1v'_1+m_2v'_2\]
式中涉及四个速度,要向学生指出,这四个速度必须是相对于
同一参照系的,一般都以地面为参照系。教师在引导学生解决
如人在船上行走之类的问题时,也要注意不要涉及相对速度,
而应把问题局限在相对于同一参照系研究其动量守恒。

\subsubsection{寻找“守恒量”的一个例子}

教材在第四节后安排了
一段阅读材料,要引导学生认真阅读,使学生了解到十六、七
世纪的哲学家如何从观察宇宙间各种物质的不断运动得出宇
宙运动的总量不会减少的看法,笛卡儿和牛顿又怎样去努力
寻找一个物理量来量度这种永恒的运动。把这一段阅读材料
同上一章第十节所说的寻求“守恒量”的重要意义联系起来
(课本244页),可以使学生体会到守恒定律在物理学里的重
要地位。这一段阅读材料中关于笛卡儿寻找量度运动的合适
物理量时的失误和对他的功绩的评述对学生也是有启发的。

\subsubsection{动量守恒定律和牛顿运动定律的关系}

教材第五节
主要说明两个问题,
\begin{enumerate}
 \item 动量守恒定律和牛顿运动定律是一致
的.但现在已认识到,动量守恒定律具有更大的普遍性;
\item 由于动量守恒定律不涉及相互作用的中间过程,所以在处理
某些问题时会更简便。
\end{enumerate}

教学中为了说明这两点,除了按教材讲述外,还可以选择
一道不要求研究中间过程的例题(例如课本260页的例2, 用
动量守恒定律和牛顿运动定律两种方法来解,说明两者的一
致性,同时又可以比较用动量守恒定律解题得更简便一些。

\subsection{第三单元}
\subsubsection{寻找“守恒量”的又一例子}

教材第六节从两个质量
相等的小球作弹性碰撞的演示开始,讨论只有动量守恒定律
还不能解释为什么现象是唯一的。最后得出在这种碰撞中还
应满足动能守恒。这是寻找“守恒量”的又一个例子。通过学
习,学生可以体会到,在一个物理过程中多一个“守恒量”,就
多一个制约因素,只有存在足够的制约因素,现象才能是唯
一的。在哪些现象中有什么守恒量,这就是自然界的规律,学
习物理学就要掌握这些规律。因此寻找“守恒量”是有重要意
义的。这一段教学内容很有启发性,对培养学生分析问题的
能力很有帮助.教学中首先要做好课本图8.7所示的演示实
验。在分析这一实验时,要使学生弄清初末状态的情况,所谓
碰撞的初末状态,指的是两球接触的短暂时间的前后,两
个状态的小球位置都是在最低点,而不是在弹起的最高点。
虽然初末状态的小球位置一样,但它们的速度不一样。在定
义了弹性碰撞以后,教材接着介绍了非弹性碰撞和完全非弹
性碰撞。教师也可在堂上利用课本图8.7的装置演示一下完
全非弹性碰撞.用课本图8.7演示弹性碰撞和完全非弹性碰
撞只能是定性的。有条件的学校可以用气垫导轨和光电计时
器演示,这样就可以作定量的测量,来验证弹性碰撞中动能守
恒,而非弹性碰撞中动能有损失。

\subsubsection{弹性碰撞末速度公式的推导}

课本第六节例题是根
据动量守恒和动能守恒列出方程组解弹性碰撞问题的一个例
子。例题中推导了碰撞后两钢球的速度,教学中要对推导方
法作一示范,並要求学生能自行推导,而不要简单地记忆末
速度公式。对例题后面的讨论,教学中也要加以重视。在讨
论中,可以将练习四的第4题一起进行讨论.从方程解中讨
论几种情况的物理意义,是一种重要的能力。对于弹性碰撞
的讨论是培养这方面能力的好时机。

对于弹性碰撞问题,课本局限在讨论正碰,并且两球中有
一球原来是静止的。对于学有余力的同学,两球初速均不为
零的弹性正碰可以作为练习题让他们自己推导和讨论,只要
他们掌握了推导方法,这样做是不太困难的。至于斜碰则不
要让学生去做了。

\subsubsection{反冲运动的方程}

反冲运动问题可以用动量守恒定
律来解决,如果反冲运动发生前物体是静止的,则动量守恒定
律可写成$MV+mv=0$的形式,其中$M$、$m$分别为向两个方向
运动的物体的质量,$V$、$v$是相应的速度,其正负号由假设的正
方向决定,可以让同学根据动量守恒定律自己写出这一方程。
但同学在写这一方程时,常常有两个容易出错的地方。第一,
如果$M$表示反冲运动发生前的总质量,则方程应改写为
$(M-m)V+mv=0$. 第二,有的同学根据反冲的两部分动量
大小相等写出$MV=mv$, 算出答案往往也是对的。但此式中
$V$和$v$均为绝对值。这种写法与本章前面的做法不一致,要
特别小心。如果反冲系统原来的动量不为零,则不能用这个
办法了。

\subsubsection{关于火箭的教学}

第七节的主要篇幅是介绍反冲运
动在火箭技术中的应用,主要是为了扩大知识面。教材不要
求作定量计算,所以关于火箭的最终速度与喷气速度、质量比
的关系不必介绍计算公式。至于火箭技术的一些原理、应用
及我国火箭技术发展情况,可鼓励学生自己查阅科普小册子
和杂志中的有关资料。这既可以激发学生的学习兴趣,又可
以培养学生自己查阅资料的能力。

\section{实验指导}
\subsection{演示实验}
\subsubsection{冲量的引入}

\begin{figure}[htp]
    \centering
    \includegraphics[scale=.6]{fig/8-1.png}
    \caption{}
\end{figure}

可利用图8.1的装置来演示冲量的作用效果,使小球自
静止开始从斜槽上的某一位置$O$滚下(在$O$处做一记号),斜
槽的底端部分是水平的。观察小球的落地点,在落地点$P$处
可放一塑料小桶,重复实验,使小球恰能落到桶中。使斜槽的
倾角变小,即使小球所受的合力变小,如果仍在$O$点让小球
自静止滚下,则小球将不会落在桶内.若将小球移到$O$点上
方的$O'$点处开始释放,使力的作用时间增长,则小球仍能落
在桶内。这说明了可以用较大的力作用较短的时间,也可以
用较小的力作用较长的时间,使原来静止的物体获得相同的
速度。

实验所用的斜槽,可用两
条平行的粗铁丝焊制而成,或
利用铝材商店出售的用铝合
铁片
金制成的U形铝材来制作
(图8.2)。
\begin{figure}[htp]
    \centering
    \includegraphics[scale=.6]{fig/8-2.png}
    \caption{}
\end{figure}

\subsubsection{动量}
如图8.3所示,使一个乒乓球从一光滑斜槽的顶端自静
止滚下,在光滑桌面上运动一段距离后,被一个固定着的装有
橡皮膜(从气球上剪下,装在架子上不要绷紧)的架子$R$阻挡
后,停止了运动,再用一个大小相同的金属球,把它从斜槽的
同一位置上释放,当它到达桌面时具有相同的速度,但被架子
$R$阻挡时,将会观察到橡皮膜被拉得很长后(阻力$F$和力作用
的时间$t$都比较大)才停止运动。这说明乒乓球和金属球虽
然具有相同的速度,但由于金属球的质量大,动量也大,因此
要使它停止下来,需要更大的冲量。
\begin{figure}[htp]
    \centering
    \includegraphics[scale=.6]{fig/8-3.png}
    \caption{}
\end{figure}


\subsubsection{动量传递和动量守恒}
用气垫导轨演示课本插页图8.3, 图8.4的实验时,可将
一根轻质弹簧先固定在一个滑块上,然后用棉线扎紧使弹簧
处于压缩状态,再把另一个滑块紧靠着这个滑块装有弹簧的
一侧。在两个滑块都静止的情况下,点燃火柴将棉线烧断,即
可看到两个滑块同时分离向相反方向运动的现象。如果两滑
块的质量相等,则它们分离的速度大小相等,若两个滑块的质
量不相等,则质量小的滑块速度大,质量大的滑块速度小。

演示课本插页图8.5, 图8.6的实验时,可在静止滑块靠
近运动滑块的一端,事先粘上一块较软的橡皮泥(如橡皮泥比
较干、硬,可把它切成小块后加些缝纫机油调软),这样,当运
动滑块跟它碰撞时能较好地粘合在一起。

\subsubsection{弹性碰撞}
课本图8.7的演示实验,
也可以用瓷球代替钢球。为了
保证使两个小球在同一竖直平
面内摆动,可采用双线摆的结
构(图8.4).为了增加可见度,
也可以用注满水的乒乓球来代
替钢球进行演示。
\begin{figure}[htp]
    \centering
    \includegraphics[scale=.6]{fig/8-4.png}
    \caption{}
\end{figure}


乒乓球中注水的方法是这样的:用注射器针头先在乒乓
球上戳一小孔,再在小孔的近处插入针头,接上针筒注满水,
拔去针头后可在两个针孔间用弯曲的细铜丝穿入细线,再用
橡皮胶布将小孔封住。

\subsubsection{反冲运动}
反冲小车:
如图8.5所示,在小车上用铝皮做一个支架,上面固定一
根试管,试管略倾斜,管内盛一些水,试管口用橡皮塞(质量可
比软木塞大)塞紧,试管底部安装一个盛有酒精棉花的小盘。
点燃酒精棉花,待试管中的水沸腾后,产生大量蒸汽将橡皮塞
冲出的同时,小车就发生后退现象。

\begin{figure}[htp]\centering
    \begin{minipage}[t]{0.48\textwidth}
    \centering
\includegraphics[scale=.6]{fig/8-5.png}
    \caption{}
    \end{minipage}
    \begin{minipage}[t]{0.48\textwidth}
    \centering
\includegraphics[scale=.6]{fig/8-6.png}
    \caption{}
    \end{minipage}
    \end{figure}

将一长形气球打足气后,用手指捏住打气口,放开后
可观察到气球由于放出气体而发生的反冲现象(图8.6)。

\subsection{学生实验}
\subsubsection{研究弹性碰撞}
这个实验安排两课时完成.第一课时可以结合仪器
的调节与使用来熟悉处理数据的方法,从而进一步理解实验
的设计原理。第二课时进行实际的测量与研究。

这个实验的设计原理比较复杂,应要求学生弄明白
后再行操作。

在调节仪器和实验操作时,要注意以下几点:
\begin{enumerate}
    \item 调节仪器的水平和支放被碰小球的小柱高度和位置。
使两球能在同一高度上发生正碰。
\item 初步调节好支放被碰小球的支柱的高度和位置后,要
进行试测,观察时入射小球的落点位置$P$, 以及发生碰撞时,
入射小球的落点$M$和被碰小球的落点$N$, 看看$P$、$M$、$N$三
个点是否大致在一条直线上,如果偏离很大,则应进一步调节
支放被碰小球的小柱在水平面上的位置,直到碰撞后,$P$、$M$、
$N$三个点看起来基本上在同一直线上,才可以正式做实验。
\item 确定斜槽底部水平部分槽口中心在水平面上的投影,
即入射小球的抛出点在水平面上的投影$O$点时,重垂线的长
度要恰当,应控制在即将接触到纸面的高度上。在这个地方
不要覆盖复写纸,实验时要使斜槽固定使重垂线始终指在
$O$点.
\item 要使学生理解课本图10.19中$O$点的位置,即槽口
重垂线所指的位置,而$O'$点的所在位置,应在原始记录纸上
沿着$OP$直线量度$2r$($r$是小球半径)的距离来确定,如果入
射小球和被碰小球的半径不相等,则距离$OO'$应等于两个小
球的半径之和,小球半径可以用游标卡尺来测量。
\end{enumerate}

利用等式$m_1(OP)=m_1(OM)+m_2(O'N)$研究动量
守恒时,对于式中相同的量取相同的单位。譬如质量的单位
都用千克(或克),距离单位都用米(或厘米)就可以了,不一定
都要用动量的单位进行计算。因为在这个实验中,是用距离
来表示速度的,实际上是
\[v_1=\frac{OP}{\sqrt{2h/g}},\qquad v'_1=\frac{OM}{\sqrt{2h/g}},\qquad v'_2=\frac{O'N}{\sqrt{2h/g}}\]
式中的$h$为小球下落的高度。只要用
$t=\sqrt{2h/g}$
来除上面的等式,式中的各项仍具有动量的单位。

在利用等式$m_1(OP)^2=m_1(OM)^2+m2_(O'N)^2$研究能量
守恒时,同样是相同的量取相同的单位就可以了。在这里要
使学生了解,通常我们总是用国际单位制,但有时为了研究问
题简便,是可以更为方便的单位的。不要把单位问题搞得
那么死板。

实验后还可以让学生思考以下问题:
\begin{enumerate}
\item 在这个实验中为什么入射小球每次都必须从斜槽的
同一高度滚下?
\item 如果入射小球的质量小于被碰小球,将会发生什么现
象?是否同样可以进行研究?
\end{enumerate}

\subsubsection{用冲击摆测弹丸的速度}

对于这个实验的原理,要使学生理解为什么在弹丸
射入摆锤过程中动量守恒而动能不守恒;在摆锤(连同弹丸一
起)向上摆动的过程中机械能守恒而动量不守恒。这是由于
弹丸进入摆锤的过程中,弹丸受到的冲力和摆锤受到的冲力,
这一对相互作用力是属于系统(弹丸和摆锤所组成)的内力,
而它们的重力由悬线的拉力所平衡,因此在弹丸和摆锤相互
作用时,它们所受到的合外力为零,所以可用动量守恒定律来
计算。在这过程中,弹丸由于克服摩擦阻力做功,一部分动能
将转化为内能,所以不能用动能守恒来计算碰撞后的共同
速度。

当摆锤获得速度和弹丸一起运动后,可以把弹丸和摆锤
看成是一个整体,在它们高度上升的过程中,悬绳拉力不做
功,只有重力做功,所以机械能守恒,因此,它们在获得速度、
开始运动时的动能,在到达最高位置时将完全转化成重力
势能。

在进行实验时,要注意以下几点:
\begin{enumerate}
\item 实验前应先将冲击摆装置调水平,要注意在调节悬挂
摆锤的四根细线时,必须使它们的长度相等,这样才能使摆锤
在上升时保持平动。在调节时并且要使摆锤的上沿与刻度盘
上画出的水平虚线对齐,右侧边与偏转角度的零刻度线对齐,
以保证弹丸能够射入摆锤孔内。
\item 应提请学生注意,利用公式$h=\ell(1-\cos\theta)$计算摆锤
的上升高度时,式中的摆长$\ell$是摆线的长度,即悬点与摆锤上
沿之间的距离,而不是悬点与摆锤中心之间的距离。对于为
什么要这样计算摆长的道理,可让学生自己考虑。
\item 为减少机械能损失,调节好摆锤的起始位置后,可使
用弹簧枪的第一档发射速度先试射一二次,观察指针偏转的
最大角度,实验时,使指针预先停留在较小角度上(譬如最大
偏角为$40^{\circ}$, 可以把指针先放在$35^{\circ}$的位置上),然后再进行
发射,读出指针的最大偏角$\theta$. 在使用弹簧枪另外两档发射
速度时也应先进行试射。
\end{enumerate}

对有兴趣的学生可以启发思考下面两个问题
\begin{enumerate}
\item 使用弹簧枪的不同档来发射弹丸时,为什么弹丸的速
度会不相同?
\item 测出弹丸的速度后,如何来计算由弹丸和摆锤组成的
系统的机械能损失(用百分数表示),用测得的数据具体计算
一下,把所算出的结果跟比值$\dfrac{M}{M+m}$
比较一下,看看它们之间
有什么联系?
\end{enumerate}

\subsection{课外实验活动}
\subsubsection{观察反冲现象}
把包装香烟用的铝箔浸湿后,将它反面的一层薄纸用干
布揩去,剪成宽约5厘米、长约20厘米的一块,卷在圆珠笔的
笔芯上做成一个空心铝管。在
封口处浆糊粘牢,用二根细
线将它水平地悬挂起来,在铝
管两端分别插入两根火柴(有
火药的一端向里),使它们刚刚
接触(图8.7),然后点燃一根火柴,在铝管中部加热,当火柴即将燃尽时,由于铝管温度升高,
管内的火柴已达燃点,燃烧产生的气体推动两根火柴向相反
方向从管子两端飞出,若只有一根火柴飞出(另一根与铝管
烧结在一起),则铝管将向反方向摆动。应当注意,做实验时,
人要站在面对空心铝管侧面的位置。

\begin{figure}[htp]
    \centering
    \includegraphics[scale=.6]{fig/8-7.png}
    \caption{}
\end{figure}



\section{习题解答}

\subsection{练习一}
\begin{enumerate}
    \item 用4牛的力推动一个物体,力的作用时间是0.5秒,力的冲量是多少?

    \begin{solution}
        冲量$Ft=4\x0.5=2{\rm N\cdot s}$
    \end{solution}
    \item 使质量为4吨的汽车,从静止达到20$\kmh$的速度,需要多大的冲量?

    \begin{solution}
        汽车动量的变化
\[ p'-p=mv-0=4\x10^3\x\frac{20\x10^3}{3600}
        =2.22\x10^4{\rm kg\cdot m/s}\]       
        所以根据$Ft=mv$, 需要的冲量为$2.22\x10^4{\rm kg\cdot m/s}$.
    \end{solution}
    \item 质量是25千克以0.5$\ms$的速度步行的小孩和质量是0.02千克以800$\ms$的速度飞行的子弹,哪个动量大?

    \begin{solution}
小孩的动量
\[p_1=m_1v_1=25\x0.5=12.5{\rm kg\cdot m/s}\]
子弹的动量
\[p_2=m_2v_2=0.02\x800=16{\rm kg\cdot m/s}\]
所以飞行的子弹动量较大。
    \end{solution}
    \item 质量为8克的玻璃弹球以3$\ms$的速度向左运动,碰到一个物体后弹回,以2$\ms$的速度沿同一直线向右运动,弹球的动量改变了多少?

    \begin{solution}
        若以向右运动的方向为正,则动量的改变:
       \[ p'-p=mv'-mv=8\x10^{-3}\x2-8\x10^{-3}\x(-3)=4\x10^{-2}{\rm kg\cdot m/s}\]
       $ p'-p$的值为正,说明动量的改变方向向右。
    \end{solution}
    \item 以相同的速度分别向竖直和水平方向抛出两个质量相等的物体,抛出时两个物体的动能是否相等?动量是否相等?

    \begin{solution}
        动能相等,动量不等。因为动能是标量,与方向无关,而动量是矢量,方向不同,动量就不等。
    \end{solution}
\end{enumerate}


\subsection{练习二}
\begin{enumerate}
    \item 10千克的物体以10$\ms$的速度作直线运动,在受到一个恒力作用4.0秒钟后,速度变为反向2.0$\ms$.求:
     \begin{enumerate}
        \item 物体在受力前和受力后的动量;
        \item 物体受到的冲量;
        \item 力的大小和方向.
    \end{enumerate}

    \begin{solution}
物体原来速度$v_1=10\ms$,则受到力$F$作用后,速
度变为$v_2=-2.0\ms$.
\begin{enumerate}
\item 受力前物体的动量
\[p_1=mv_1=10\x10=100{\rm kg\cdot m/s}\]
受力后的动量
\[p_2=mv_2=10\x(-2.0)=-20{\rm kg\cdot m/s}\]
\item 根据动量定理,物体受到的冲量
\[Ft=p_2-p_1=-20-100=-120{\rm kg\cdot m/s}-120{\rm N\cdot s}\]
负号表示冲量的方向与原来的速度方向相反。
\item $F=\dfrac{Ft}{t}=\dfrac{-120}{4.0}=-30{\rm N}$
负号表示力的方向与原来的速度方向相反。
\end{enumerate}

    
    \end{solution}
    \item 列车的质量是$2.5\times 10^6$千克,受到的牵引力是$4.0\times 
    10^5$牛,它的速度由10$\ms$增加到24$\ms$需要用多少时间?

    \begin{solution}
        根据动量定理$Ft=mv'-mv$,
\[t=\frac{mv'-mv}{F}=\frac{ 2.5\x10^6\x24-2.5\x10^6\x10}{4.0\x 10^5}= 87.5{\rm s}\]      
列车的速度从10$\ms$增加到24$\ms$需要87.5秒。
    \end{solution}
    \item 一个质量是65千克的人从墙上跳下,以7$\ms$的速度着地,与地面接触后0.01秒停了下来,地面对他的作用力是多大?如果他着地时弯曲双腿,用了1秒钟才停下来,地面对他的作用力又是多大?

    \begin{solution}
若取向上为正方向,则应用动量定理,可得出
\[Ft=0-mv\]
式中$v=-7\ms$.如果落地时力的作用时间为$t=0.01$秒,
地面对人的作用力为
\[F=\frac{-mv}{t}=\frac{-65\x (-7)}{0.01}=4.55\x 10^4{\rm N}\]
说明:上述解答忽略了人所受的重力,这是由于人所受的重力
为$mg=65\x9.8=637{\rm N}$,与上面算出的作用力相比,还不
到2\%, 所以是可以忽略的。但在下面的情况下,人所受的重
力不能忽略。

当人着地时双腿弯曲,力的作用时间为$t'=1$秒,此时
必须考虑到是合力的冲量使动量发生变化,所以若以$F'$表示
地面作用力,则
\[(F'-mg)t'=0-mv\]
\[F'=\frac{-mv}{t'}+mg=\frac{-65\x (-7)}{1}+65\x 9.8=1.1\x 10^{3}{\rm N}\]
    \end{solution}
    \item 跳远时,为什么跳在砂坑里比跳在混凝土路面上安全?钉钉子时,为什么要用铁锤而不用橡皮锤?


    \begin{solution}
    跳远时,从着地到停止下来所经过的时间,跳在沙坑
里比跳在混凝土路面上要长,根据动量定理可知,在动量变化
一定的情况下,跳在沙坑里的平均作用力就较小,比较安全。

钉钉子时,铁锤的质量较大,可以有较大的动量。又因铁
锤比较坚硬,与钉子接触的时间短.根据动量定理,
\[Ft=0-mv,\qquad F=\frac{-mv}{t}\]
因为$mv$较大,$t$较小,所以$F$就很大,容易把
钉子钉入木块等物中去。相反,若用橡皮锤,作用力就较小。
    \end{solution}
    \item 质量为4千克的铅球和质量为0.1千克的皮球以相同的速度运动着,要使它们在相同的时间内停下来,作用在铅球上的力和作用在皮球上的力哪个大?为什么?


    \begin{solution}
    铅球与皮球的速度相同,因为铅球的质量大,所以它
的动量大。要使它停下来,动量的变化也大。根据动量定理,冲
量也必须大。又由于作用时间相同,所以对铅球的作用力应该
比对皮球的作用力大。
    \end{solution}
\end{enumerate}


\subsection{练习三}
\begin{enumerate}
    \item 两个原来静止的在水平面上挨在一起的小车,质量分别是0.5千克和0.2千克,在弹力作用下分开,较重的小车以0.8$\ms$的速度向右运动,求较轻的小车的速度.

    \begin{solution}
因为系统的合外力为零,所以动量守恒。
\[m_1v_1+m_2v_2=0\]
如果以向右的方向为正方向.则根据题意,$m_1=0.5$千克,
$m_2=0.2$千克,$v_1=0.8$米/秒.代入方程,可解得,
\[v_2=\frac{-m_1v_1}{m_2}=\frac{-0.5\x 0.8}{0.2}=-2\ms\]
负号表示较轻的小车的速度与假设的正方向相反,即向左。
    \end{solution}
    \item 在气垫导轨上,一个质量为600克的滑块以15${\rm cm}/{\rm s}$的速度赶上另一个质量为400克速度为10${\rm cm}/{\rm s}$的滑块而发生碰撞,碰撞后两个滑块并在一起,求两个滑块碰撞后的速度.

    \begin{solution}
根据题意,滑块质量分别为$m_1=600{\rm g}=0.6{\rm kg}$,
$m_2=400{\rm g}=0.4{\rm kg}$,$v_1=15{\rm cm/s}=0.15\ms$,$v_2=10{\rm cm/s}=0.10\ms$.设碰撞后的共同速度为$v$. 则根据动量
守恒定律,
\[m_1v_1+m_2v_2=(m_1+m_2)v\]
\[v=\frac{m_1v_1+m_2v_2}{m_1+m_2}=\frac{0.6\x 0.15+0.4\x 0.10}{0.6+0.4}=0.13\ms\]
方向跟原来的方向一致。
    \end{solution}
    \item 一个小孩从静止的小船上水平抛出一个球,球的质量是2.0千克,抛出的速度是20$\ms$.如果小孩和船的总质量为100千克,球抛出时船得到的速度是多大?

    \begin{solution}
    设小孩和船的总质量为$M$, 小球的质量为$m$, 小球
抛出的速度$v$即小球抛出时相对于地面的速度。则小孩和船
的速度$V$可由动量守恒定律求出。设$v$的方向为正。
\[mv+MV=0\]
\[V=-\frac{mv}{M}=\frac{-2.0\x 20}{100}=-0.40\ms\]
负号表示$V$的方向与小球抛出方向相反。
    \end{solution}
    \item 质量为10克速度为300$\ms$的子弹,打进质量为
    24克静止在光滑水平面上的木块中,并留在木块里,子弹进入木块后,木块运动的速度多大?如果子弹把水块打穿,穿过木块后子弹的速度为100$\ms$,这时木块的速度多大?

    \begin{solution}
    设子弹质量为$m$, 速度为$v_1$, 木块质量为$M$. 当子
弹打入木块并留在木块内时,子弹和木块有共同速度$V$,则根
据动量守恒定律,有
\[mv=(m+M)V\]
\[V=\frac{mv}{m+M}=\frac{0.01\x 300}{0.01+0.024}=88.2\ms\]
如果子弹穿过木块后有速度$v'=100\ms$,则木块速度
$V'$可由下式求得
\[mv= mv'+ MV'\]
\[V'=\frac{mv-mv'}{M}=\frac{0.01\x300-0.01\x100}{0.024}=83.3\ms\]
    \end{solution}
    \item 光滑的水平面上停着一辆平车,有两个人在车上相向而行,在什么情况下平车保持静止?在什么情况下平车要运动,运动的方向由什么决定?

    \begin{solution}
        两人动量的大小相等时,平车不动;两人动量的大小
        不等时,平车就要运动。平车运动的方向跟动量小的人的运动
        方向相同。这是因为,两人动量与平车的动量之和应该守恒,
        即为零,因此,平车的动量应与两个人的合动量的大小相等方
        向相反,而两人合动量的方向决定于哪个人的动量大,所以平
        车的动量方向应与动量小的人的动量方向相同。
    \end{solution}
\end{enumerate}


\subsection{练习四}
\begin{enumerate}
\item 两个质量都是3千克的球,各以6$\ms$的速率相向运动,发生正碰后每个球都以原来的速率向相反方向运动.它们的碰撞是弹性碰撞吗?为什么?


\begin{solution}
    是弹性碰撞.两个小球虽然碰撞前后运动方向都发
    生变化,但速度大小不变,所以动能不变,由于动能守恒,所以
    是弹性碰撞。
\end{solution}
\item 一个1.5千克的物体原来静止,另一个0.5千克的以0.2$\ms$的速度运动的物体与它发生弹性正碰,求碰撞后两个物体的速度.

\begin{solution}
根据题意,$m_1=0.5$千克,$v_1=0.2$米/秒,$m_2=1.5$千
克,$v_2=0$. 因为两球发生弹性碰撞,因此满足动量守恒和动能
守恒:
\[\begin{cases}
    m_1v_1=m_1v_1'+m_2v_2'\\
    \frac{1}{2}m_1v_1^2=\frac{1}{2}m_1{v'_1}^2+\frac{1}{2}m_2{v'_2}^2\\
\end{cases}\]
解得,
\[\begin{split}
v'_1&=\frac{m_1-m_2}{m_1+m_2}v_1=\frac{0.5-1.5}{0.5+1.5}\x 0.2=-0.1\ms\\
v'_2&=\frac{2m_1}{m_1+m_2}v_1=\frac{2\x 0.5}{0.5+1.5}\x 0.2=0.1\ms\\
\end{split}\]
$v'_1$为负值,说明质量较小的物体碰撞后速度的方向与原来
相反。
\end{solution}
\item 甲乙两物体在同一直线上同向运动,甲物体在前,乙物体在后,甲物体质量为2千克,速度是1$\ms$;乙物体质量为4千克,速度是3$\ms$.乙物体追上甲物体发生正碰后,两物体仍沿着原来的方向运动,而甲物体的速度变为3$\ms$,乙物体的速度变为2$\ms$,这两个物体的碰撞是弹性碰撞吗?为什么?

\begin{solution}
    设甲物体的质量为$m_1$, 它在碰撞前后的速度为$v_1$、$v_1'$
    乙物体的质量为$m_2$, 它在碰撞前后的速度为$v_2$、$v'_2$. 要
    判断是否弹性碰撞,可检验其碰撞前后的总动能是否守恒。
    
    碰撞前的总动能:
\[\frac{1}{2}m_1v_1^2+\frac{1}{2}m_2v_2^2=\frac{1}{2}\x 2\x 1^2+\frac{1}{2}\x 4\x 3^2=19{\rm J}\]
碰撞后的总动能:
\[\frac{1}{2}m_1{v'_1}^2+\frac{1}{2}m_2{v'_2}^2=\frac{1}{2}\x 2\x 3^2+\frac{1}{2}\x 4\x 2^2=17{\rm J}\]
可见,总动能不守恒,不是弹性碰撞。
\end{solution}
\item 在课文第六节的(8.6)式中,如果$m_2\gg m_1$,就得到$v'_1\approx -v_1,\; v'_2\approx 0$.这组解的物理意义是什么?


\begin{solution}
    在弹性碰撞方程组中解得的末速度公式[(8.6)式]是:
    \[\begin{cases}
        v'_1=\dfrac{m_1-m_2}{m_1+m_2}v_1\\
        v'_2=\dfrac{2m_1}{m_1+m_2}v_1\\
        \end{cases}\]
当$m_2\gg m_1$时,可得$v'_1\approx -v_1$, $v'_2\approx 0$. 这说明:当质量很小的物
体去与质量很大的静止物体发生正碰时,小物体将原速弹回,而大物体几乎不动。
\end{solution}
\end{enumerate}



\subsection{习题}
\begin{enumerate}
    \item 质量为1千克的手榴弹以60$^\circ$角斜抛出去,抛出的速度为10$\ms$,手榴弹到达最高点时炸成两块,一块的质量是0.6千克,以15$\ms$的速度沿原方向运动,求另一块的速度大小和方向.

\begin{figure}[htp]
    \centering
    \includegraphics[scale=.4]{fig/8-8.png}
    \caption{}
\end{figure}

    \begin{solution}
手榴弹以60$^\circ$角斜抛出去,达最高点时的速度
\[v=v_0\cos\theta=10\x\cos60^{\circ}=5\ms\]
(图8.8)。此时手榴弹炸
成两块,爆炸前后的动量应该守恒。(它们所受的重力与爆炸
力相比可忽略不计)。设爆炸后沿原方向运动的一块质量为
$m_1$, 速度为$v_1$, 另一块的质量为$m_2$, 速度为$v_2$. 以原运动方向
为正方向,则
\[mv=m_1v_1+m_2v_2\]
\[v_2=\frac{mv-m_1v_1}{m_2}=\frac{1\x 5-0.6\x 15}{0.4}=-10\ms\]
负号表示方向与手榴弹在最高点的速度方向相反。
    \end{solution}
    \item 对于在一直线上运动的两个物体组成的系统,动量守恒定律的一般表达式为:
\[m_1v_1+m_2v_2=m_1v'_1+m_2v'_2 \]
    在不同情况下,这个表达式往往可以简化为不同形式,试写出下列各种情况下得出的简化的表达式:
\begin{enumerate}
    \item 两个物体原来静止,发生相互作用后分开;
    \item 一个物体原来静止,另一个物体跟它碰撞后粘合在一起并共同沿原来的方向运动;
    \item 一个物体原来静止,另一个运动物体与它正碰后,两物体以不同的速度在原来的直线上运动;
    \item 两个相向运动的物体,相碰后都静止下来.
\end{enumerate}

\begin{solution}
\begin{enumerate}
    \item $0=m_1v_1'+m_2v_2'$
\item 设$m_2$原来静止,则
\[m_1v_1=(m_1+m_2)v\]
式中$v$为粘合后的共同速度。
\item $m_1v_1=m_1v_1'+m_2v'_2$
\item $m_1v_1+m_2v_2=0$
以上几式中的$v_1,v_2,v_1',v_2',v'$等速度的正负要根据与假定正
方向的一致或相反来确定。

\end{enumerate}
\end{solution}
\item 试证明:两个物体碰撞后,它们的速度变化$\Delta v_1=v'_1-v_1$和$\Delta v_2=v'_2-v_2$跟它们的质量成反比,即
\[\frac{\Delta v_1}{\Delta v_2}=-\frac{m_2}{m_1}\]
并利用所得结果来讨论:很轻的物体(如乒兵球)跟一个很重的物体(如课桌)碰撞后,它们的速度变化有什么特征.

\begin{proof}
    两物体碰撞,动量守恒。据两个物体组成系统
    的动量守恒定律一般表达式
    \[m_1v_1+m_2v_2=m_1v'_1+m_2v'_2\]
    移项得
   \[\begin{split}
       m_1(v'_1-v_1)&=-m_2(v'_2-v_2)\\
       \frac{v'_1-v_1}{v'_2-v_2}&=-\frac{m_2}{m_1}
   \end{split} \]
    即
\[\frac{\Delta v_2}{\Delta v_1}=-\frac{m_2}{m_1}\]
\end{proof}
\item 质子的质量是$1.67\times 10^{-27}$千克,速度为$1.0\times 10^7\ms$,与一个静止的氦核碰撞后,质子以$6.0\times 10^6\ms$的速度反弹回来,氦核以$4.0\times 10^6\ms$的速度向前运动.
   \begin{enumerate}
       \item 你能否求出氦核的质量?如果能,是多少?
       \item 你能否求出碰撞时的相互作用力?为什么?
   \end{enumerate}

   \begin{solution}
\begin{enumerate}
    \item 能。可以用动量守恒定律求出。设质子质量为
    $m_1=1.67\x10^{-27}$千克,速度$v_1=1.0\x10^7$米/秒.碰撞后质
    子反弹,速度为$v'_1=-6.0\x10^6$米/秒.氦核的质量为$m_2$, 碰
    撞后速度$v'_2=4.0\x10^6$米/秒.则
\[m_1v_1=m_1v_1'+m_2v_2'\]
\[\begin{split}
    m_2&=\frac{m_1v_1-m_1v_1'}{v_2'}\\
    &=\frac{ 1.67\x10^{-27}\x1.0\x10^7-1.67\x10^{-27} \x(-6.0\x10^6)}{4.0\x10^6}\\
    &=6.68\x10^{-27}{\rm kg}
\end{split}
\]
    \item 不能。因为根据质子的动量变化可以求得质子受到
    的冲量,但由于作用时间未知,所以无法求得作用力。
\end{enumerate}
   \end{solution}
   \item 两个球以相同的速度相向运动,其中一个球的质量是另一个的三倍,相碰后重球停止不动,轻球以二倍的速率弹回,试证明它们发生的是弹性碰撞.

\begin{solution}
    设轻球质量为$m$, 则重球质量为$3m$. 碰撞前速率
    都是$v$, 碰撞后轻球速率是$2v$, 重球静止,则碰撞前总动能为
   \[\frac{1}{2}mv^2+\frac{1}{2}\x 3mv^2=2mv^2\]
    碰撞后总动能为
\[\frac{1}{2}m(2v)^2=2mv^2\]
    可见,碰撞前后动能守恒,为弹性碰撞。
\end{solution}
   \item 在光滑水平面上一个质量为0.2千克的小球以5$\ms$的速度向前运动,途中与另一个质量为0.3千克静止的小球发生正碰.假设碰撞后第二个小球的速度为4.2$\ms$,你算出的第一个小球的速度是多大?想一想,这种情况真的可能发生吗?这道题的毛病出在哪里?

   \begin{solution}
    如果这种情况真的发生,则碰撞前后一定满足动量
    守恒。因此可用动量守恒定律求得第一个小球碰撞后的速度
    $v'_1$. 设第一个小球的质量为$m_1$, 碰撞前速度为$v_1$, 第二个小球
    的质量为$m_2$, 碰撞后的速度为$v'_2$. 则
\[m_1v_1=m_1v_1'+m_2v_2'\]
\[v'_1=\frac{m_1v_1-m_2v_2'}{m_1}=\frac{0.2\x 5-0.3\x 4.2}{0.2}=-1.3\ms\]
但实际上这种情况是不可能发生的。因为碰撞前的总
动能
\[E=\frac{1}{2}m_1v_1^2=\frac{1}{2}\x 0.2\x 5^2=2.5{\rm J}\]
而碰撞后的总动能为
\[\begin{split}
    E'&=\frac{1}{2}m_1{v'_1}^2+\frac{1}{2}m_2{v'_2}^2\\
    &=\frac{1}{2}\x 0.2\x (-1.3)^2+\frac{1}{2}\x 0.3\x 4.2^2\\
    &=2.8{\rm J}
\end{split}\]
碰撞后的总动能大于碰撞前的总动能是不可能的。这道题的毛病在所给的数据不符合实际情况。
   \end{solution}
   \item 一个质量$M=0.2$千克的小球放在高度$h=5$米的直杆顶端(图8.11),一颗质量$m=0.01$千克的子弹以$v_0=500\ms$的速度沿水平方向击中小球,并穿过球心,小球落地处离杆的距离$s=20$米.求子弹落地处离杆的距离.子弹的动能有多少转化成了热能?
   \begin{figure}[htp]\centering
    \begin{tikzpicture}[>=latex, scale=.8]
    \draw (-1,0)-- (6,0);
    \draw (-0.1,0) rectangle (.1, 5);
    \node at (0, 5.6){$M$};
    \draw [<->](-.75,5.2)--node[fill=white]{$h$}(-.75,0);
    
    \draw[fill=gray] (-2, 5.3) --(-1.7, 5.3) to [bend left=15] (-1.45, 5.2) to [bend left=15] (-1.7,5.1)--(-2,5.1)--(-2,5.3); 
    \node at (-1.7, 5.5){$m$};
    \draw[->](-1.1, 5.2)--node[above]{$v_0$}(-.3, 5.2);
    \draw [|<->|](0,-.3)--node[fill=white]{$s$}(2,-.3);
    \draw [|<->|](0,-.8)--node[fill=white]{$s'$}(5.2,-.8);
    \draw (0,0)--(0,-1);
    \draw (2,0)--(2,-.5);
    \draw (5.2,0)--(5.2,-1);
    
    \draw [dashed] (5.2,0) arc (0:87:5.2);
    
    \draw [dashed] plot[domain=0:2, samples=100] function{-1.3*x*x+5.2} ;
    \draw [fill=gray] (0,5.2) circle (.2);
    \draw [dashed, fill=white] (2,.2) circle (.2);
    \draw [dashed, fill=white] (4.9+.2,.55)--(5.1+.2,.55)--(5.1+.2,.25) to [bend left=15](5+.2,0) to [bend left=15](4.9+.2,.25)--(4.9+.2,.55);
    \end{tikzpicture}
    \caption{}
    \end{figure}

    \begin{solution}    
    根据小球落地点离杆
的距离$s$, 利用平抛运动规律,
可求出小球在碰撞后的速度$V$.
\[V=\frac{s}{t}=\frac{s}{\sqrt{2h/g}}=\frac{20}{\sqrt{\dfrac{2\x 5}{9.8}}}=20\x\sqrt{0.98}=19.8\ms\]
再根据动量守恒定律求得子弹在穿过小球后的速度$v'$.
\[mv_0=mv'+MV\]
\[v'=\frac{mv_0-MV}{m}=\frac{0.01\x500-0.2\x19.8}{0.01}=104\ms\]
再根据平抛运动规律求出子弹落地点离杆的距离$s'$, 
\[s'=v't=v'\x\sqrt{\frac{2h}{g}}=104\x \sqrt{\frac{2\x 5}{9.8}}=105{\rm m}\]
设转化为热能的能量为$E$, 则根据能量守恒:
    \[\begin{split}
E&=\frac{1}{2}mv_0^2-\frac{1}{2}m{v'}^2-\frac{1}{2}MV^2\\
&=\frac{1}{2}\x 0.01\x 500^2-\frac{1}{2}\x 0.01\x 104^2-\frac{1}{2}\x 0.2\x 19.8^2\\
&=1.16\x 10^3{\rm J}        
    \end{split}
        \]


    \end{solution}

   \item 略(课本已作解答)。

\item 在上题中,如果宇航员想以最短的时间返回飞船,他开始最多能释放出多少氧气?这时他返回飞船所用的时间是多少?

\begin{solution}
要使返回时间最短,就要使开始释放的氧气最多。这
样反冲速度大,返回时间短,但释放氧气后的剩余氧气又必
须足够字航员在途中呼吸所用。其极端的情况就是所剩的氧
气正好够宇航员途中呼吸,即$m+m_{\text{吸}}=m_{\text{总}}$, 式中$m$就是开始
喷出的氧气质量。根据上题分析,宇航员的反冲速度为$V$,
而$V=-mv/M$。
返回时间
\[t=\frac{d}{V}=-\frac{Md}{mv}\]
在这段时间内宇航员吸氧气
\[m_{\text{吸}}=m_{\text{总}}-m=Rt\]
所以,
\[m_{\text{总}}-m=Rt=-R\frac{Md}{mv}\]
整理得
\[m^2-m_{\text{总}}m-\frac{RMd}{v}=0\]
所以,
\[m=\frac{1}{2}\left(m_{\text{总}}+\sqrt{m^2_{\text{总}}+\frac{4RMd}{v}}\right)\]
(因$m$应取较大值,所以舍去根号前的负号解)。代入数据$m_{\text{总}}=0.5{\rm kg}$
\[\frac{RMd}{v}=\frac{2.5\x 10^{-4}\x 100\x 45}{-50}=-0.0225{\rm kg^2}\]
解得:$m=0.45{\rm kg}$

根据题意,要使宇航员返回时间最短,开始时应释放氧气0.45
千克。宇航员返回时间为
\[t=\frac{d}{V}=-\frac{Md}{mv}=-\frac{100\x 45}{0.45\x (-50)}=200{\rm s}\]

说明:上述返回时间的答案可以验证。看看在这段时间
里宇航员呼吸氧气有没有问题。如果从吸完氧气所需的时间
来计算,则
\[t=\frac{m_{\text{总}}-m}{R}=\frac{0.5-0.45}{2.5\x 10^{-4}}=200{\rm s}\]
与上述答案是一致的。
\end{solution}
\item 速度为$10^5{\rm cm}/{\rm s}$的氦核与静止的质子发生正碰,氦核的质量是质子的4倍,碰撞是弹性的,求碰撞后两个粒子的速度.

\begin{solution}
设氦核质量为$m_1$, 速度为$v_1$, 质子质量为$m_2$. 已知
$m_1=4m_2$. 根据弹性碰撞的动量守恒和动能守恒,列出方程:
\[\begin{cases}
  m_1v_1=m_1v_1’+m_2v_2'\\
\frac{1}{2}m_1v_1^2=\frac{1}{2}m_1{v_1'}^2+\frac{1}{2}m_2{v_2'}^2 
\end{cases}\]
解得碰撞后氦核速度
\[v'_1=\frac{m_1-m_2}{m_1+m_2}v_1=\frac{3m_2}{5m_2}v_1=\frac{3}{5}v_1=\frac{3}{5}\x 10^5=6\x10^4{\rm cm/s}\]
质子速度 
\[v'_2=\frac{2m_1}{m_1+m_2}v_1=\frac{8m_2}{5m_2}v_1=\frac{8}{5}v_1=\frac{8}{5}\x 10^5=1.6\x 10^5{\rm cm/s}\]
$v_1'$、$v_2'$都与$v_1$的方向相同.
\end{solution}
\item 一个质量是$m_1$,动能是$E_K$的物体与一个质量是$m_2$的不动的物体正碰,假定发生的是弹性碰撞,在$m_1=0.01m_2$,$m_1=m_2$,$m_1=100m_2$的情况下,$m_1$传递给$m_2$的动能各是多少?

(有兴趣的同学还可以进一步讨论$m_1$传递给$m_2$的动能最大或最小的条件).

\begin{solution}
    设$m_1$的原速度为$v_1$, 碰撞后两物体的速度分别为
    $v_1'$、$v_2'$。则根据弹性正碰的特点列出方程:
\[\begin{cases}
  m_1v_1=m_1v_1’+m_2v_2'\\
\frac{1}{2}m_1v_1^2=\frac{1}{2}m_1{v_1'}^2+\frac{1}{2}m_2{v_2'}^2 
\end{cases}\]
解得:
\[v'_1=\frac{m_1-m_2}{m_1+m_2}v_1,\qquad v'_2=\frac{2m_1}{m_1+m_2}v_1\]
$m_1$传递给$m_2$的动能
\[\begin{split}
    E'_{k_2}-0=E'_{k_2}&=\frac{1}{2}m_2{v_2'}^2=\frac{1}{2}m_2\left(\frac{2m_1}{m_1+m_2}\right)^2 v_1^2\\
    &=\frac{1}{2}m_1v_1^2\cdot \frac{4m_1m_2}{(m_1+m_2)^2}\\
    &=\frac{4m_1m_2}{(m_1+m_2)^2}E_k
\end{split}\]

\begin{itemize}
\item 当$m_1=0.01m_2$时,
\[E'_{k_2}=\frac{4\x 0.01m_2^2}{1.01^2m_2^2}E_k=0.039E_k\]
说明传递给$m_2$的动能只占$m_1$原动能的3.9\%.
\item 当$m_1=m_2$时,$E'_{k_2}=E_k$. 说明$m_1$的动能全部传递给$m_2$.
\item 
当$m_1=100m_2$时,
$$E'_{k_2}=\frac{400m^2_2}{101^2\cdot m^2_2}E_k=0.039E_k$$
说明传递给$m_2$的动能也只占$m_1$原动能的3.9\%.
\end{itemize}

$m_1$传递给$m_2$的动能为最大的情况,就是将自己的动能
全部传给$m_2$的情况,即上面所说的$m_1=m_2$的情况。

从式子$E'_{k_2}=\dfrac{4m_1m_2}{(m_1+m_2)^2}E_k$ 可见,
\[E'_{k_2}=\frac{4m_1m_2}{m_1^2+2m_1m_2+m_2^2}E_k=\frac{4}{\dfrac{m_1}{m_2}+2+\dfrac{m_2}{m_1}}E_k\]

\begin{itemize}
    \item 当$m_1\gg m_2$时,$\dfrac{m_2}{m_1}\to 0$, 而$\dfrac{m_1}{m_2}\to \infty$, 所以$E'_{k_2}\to 0$. 
    \item 当
$m_1\ll m_2$时,$\dfrac{m_1}{m_2}\to 0$, 而$\dfrac{m_2}{m_1}\to \infty$, 所以$E'_{k_2}\to 0$
\end{itemize}

所以,当$m_1\gg m_2$或$m_1\ll m_2$时,$m_1$传递给$m_2$的动能最小(等于零)。
\end{solution}
\item 在有些原子反应堆里,要让中子与原子核碰撞,以便把中子的速率迅速降低下来.为此,是选用较重的还是较轻的原子核效果较好?为什么?


\begin{solution}
    要使中子速率迅速降低下来,就要使中子与原子核
    碰撞的过程中将动能传递给原子核。根据上题的讨论,被撞
    原子核的质量越接近中子质量,传递动能越多,中子的速率就
    降低得越快。所以选用较轻的原子核效果较好。
\end{solution}
\end{enumerate}


\section{参考资料}
\subsection{机械运动中动量及动能的区别}

课本里的阅读材料:《笛卡儿和动量守恒定律》中已经提
到动量这个概念是笛卡尔、牛顿先后提出的,并且笛卡儿明确
地把物体的质量和速度的乘积作为物体“运动量”的量度。在
历史上由于他们的影响,在十七世纪四十年代至八十年代,科
学界普遍承认$mv$是机械运动唯一的量度。

在这同一时期内,由于惠更斯对完全弹性碰撞的研究,得
出了“各个质量和各个速度的平方乘积之和,在碰撞前后不
变”的结论。

1686年德国数学家莱布尼兹通过对落体运动的分析,认
为物体的质量和速度平方的乘积——活力——才是机械运动
的真正量度,从而与笛卡尔的主张展开了争论。

关于两种运动量度的争论,持续了近二百年,许多著名的
数理学家参加到争论中。后来随着力学本身的发展,人们对
这两种量度取得了清楚的认识。

牛顿第二定律
\[\frac{\dd(mv)}{\dd t}=F\]
所表现的只是运动的原因(力)
和结果(动量变化)之间的瞬时关系。如果考察力在一段时间
内的累积效应,可由上式得出:
\[\dd(mv)=F\cdot \dd t\]
\[\therefore\quad mv_2-mv_1=\int^{t_2}_{t_1}F\cdot \dd t\]
这就是动量定理:在一段时间内物体动量的变化,等于物体在
同一时间内所受外力的冲量,如果要根据物体在力的作用下
所通过的距离来考察力的作用效果,即力的空间积累效应,则
可得出:
\[\begin{split}
    F\cdot \dd s&=\frac{\dd(mv)}{\dd t}\cdot \dd s\\
    \int^{t_2}_{t_1}F\cdot \dd s&=\int^{t_2}_{t_1}v\cdot \dd (mv)
\end{split}\]
\[\therefore\quad \frac{1}{2}mv^2_2-\frac{1}{2}mv^2_1=\int^{t_2}_{t_1}F\cdot \dd s\]
这就是动能定理:物体动能的增加,等于外力对物体所做
的功。

所以,在力学中动量的变化表现着力的时间累积效应,动
量的变化与外力的冲量相等;动能的变化表现着力的空间累
积效应,动能的变化与外力的功相等。动量是与冲量密切联
系着的,动量决定物体反抗阻力能够移动多么久;动能是与
功密切联系着的,动能决定物体反抗阻力能够移动多么远。


\subsection{动量守恒定律的适用范围比牛顿运动定律广}
动量守恒定律比牛顿运动定律的适用范围要广。近代的
科学实验和理论分析都表明:在自然界中,大到天体的相互作
用,小到质子、中子等基本粒子间的相互作用都遵守动量守恒
定律。

在天文学中发现过这样一种现象:在太空的某个地方有
时会突然发出非常明亮的光,这就是超新星,可是它很快就
暗淡下来,经过几十个昼夜亮度就会减弱一半,光要从这样
一颗超新星出发到达地球需要几百万年,而相比之下超新星
从发光到熄灭的时间就显得太短了,在光到达我们这里以前,
超新星早已烧光了。

当光从超新星到达地球时,它给地球一个轻微的推动,而
与此同时地球却无法给超新星一个轻微的推动,因为它已消
失了,因此,如果我们想象一下超新星与地球之间的相互作
用力,在同一瞬间也就不是什么大小相等,方向相反了。此时,
牛顿第三定律显然已不适用了。

虽然如此,动量守恒定律还是正确的。不过,我们必须把
光也考虑在内。当超新星发射光时,星体反冲,得到动量,同
时光也带走了大小相等、方向相反的动量。经过几百万年光
到达地球时,光把它的动量给了地球。这里要注意的是:动量
不仅可以为实物所携带,而且可以以辐射的方式传递动量,当
我们考虑到这点时,动量守恒定律还是正确的。

\subsection{相对论的动量}

在牛顿力学里,动量定义为$mv$, 质量$m$是个不变的量。根
据牛顿第二定律,一个恒定的力,持续作用于一个物体,可以
使该物体有任意大的高速度。但是在现实中,真空中的光速是
极限速度,并且在任何条件下物体的速度都不可能超过真空
中的光速。因此,在高速运动时,认为质量,以及动量,是与速
度无关的,是不正确的。

相对论告诉我们,在高速运动时质量不再是一个不变的
量,而是随着运动的速度接近光的速度$c$而增大,如果用$m_0$
表示静止物体的质量,则以速度$v$运动的物体的质量$m$可以
用下式表示:
\[m=\frac{m_0}{\sqrt{1-\dfrac{v^2}{c^2}}}\]
相对论的动量仍定义为,
\[p=mv=\frac{m_0v}{\sqrt{1-\dfrac{v^2}{c^2}}}\]
在采用这样定义的情况下,牛顿本人所用的第二定律的表
达式
\[\frac{\dd p}{\dd t}=F,\qquad F=\frac{\dd}{\dd t}\cdot\frac{m_0v}{\sqrt{1-\dfrac{v^2}{c^2}}}\]
在接近光速的情况下也同样适用了,因为随着运动速度的增
大,决定物体惯性大小的质量也增大。当$v\to c$时,$m\to \infty$, 所以加速度趋于零,不论力作用多长时间,速度也不会超过
光速。

对于静止质量$m_0=0$, 而速度为$c$的光子来说,它的动量$p=E/c$
可以这样推得:

因为$E=mc^2$,所以
\[\begin{split}
    E^2=m^2c^4&=m^2c^4-m^2v^2c^2+m^2v^2c^2\\
    &=m^2c^4\left(1-\frac{v^2}{c^2}\right)+p^2c^2=m^2_0c^4+p^2c^2
\end{split}\]
在$v\to c$, $m_0\to 0$时,$E^2=p^2c^2$, 所以
\[p=\frac{E}{c}\]


