\chapter{直线运动}\minitoc[n]
\section{教学要求}
这一章讲授的运动学知识,跟第一章一样,都是基础性
的,是后面学习动力学所必需的预备知识.

为了减少学生学习的困难,适应学生的知识水平和接受
能力,本章只讲直线运动,而把运动的合成和分解以及平抛和
斜抛的知识移到第四章的曲线运动中去讲.

通过这一章的教学,应该使学生了解一些描述物体运动
的基本概念和方法,掌握匀速直线运动和匀变速直线运动的
规律,会用这些规律来分析解决一些比较简单的实际问题.

这一章的教学要求是:
\begin{enumerate}
\item 了解参照物的概念,知道研究物体的运动要选择合适
的参照物,了解质点的概念,知道在什么情况可以把物体看
成质点.了解位移的概念,知道位移和路程的区别.
\item 明确什么是匀速直线运动,掌握匀速直线运动的公
,理解匀速直线运动的图象的物理意义.
\item 明确什么是匀变速直线运动,理解平均速度、即时速
,加速度等概念,明确知道速度和加速度的区别,掌握匀变
直线运动的公式,理解匀变速直线运动的图象的物理意义.
\item 认识自由落体运动和竖直上抛运动的特点和规律.
\end{enumerate}

下面对这一章的教学内容作些具体说明.

第一节开始先复习初中学过的机械运动和参照物的
概念,以加强与初中知识的联系,同时强调参照物的重要性,
使学生初步了解怎样选择参照物.第一节还简单介绍了平动
和转动,目的是使学生对物体运动的这两种基本形式有所认
识,后面用到时方便.教材没有给平动和转动下严格的定义,
也不要求补充讲解,只要求学生知道平动和转动的特点和
区别.

质点是力学中的一个重要概念,它是通过科学抽象得出
的理想化模型,运用理想化模型来研究问题,是物理学经常
用的方法.学生在这里初次接触这个问题,需要引起他们注
意,因此把质点单独作为一节来讲述.关于质点的定义,教材
采用了有质量的点这种说法,目的在于强调质点是物理学上
的点,不同于几何学中所说的点,在“质点”这节的最后,给出
了研究质点运动的基本线索——确定质点在任一时刻的位置
和速度,是为了使学生明确讲解本章知识的思路.

描述物体的运动,首先要懂得如何描述物体的位置和位
置变化,为要讲解位置的坐标表示和位移的概念.对位移
的坐标表示,只要求学生知道位移的数值可以用初末位置的
坐标来表示;在后面计算位移的公式中,除了平抛和斜抛外,
都不要求写出位移的坐标表示,而只写出位移本身.

匀速直线运动的知识,学生在初中已学过.这里要在复
习的基础上予以扩展和提高,为讲授匀变速直线运动作好准
备.用比值来定义物理量是物理学中常用的方法,这里用位
移和时间的比值重新定义了速度.在用比值给出速度的定义
之后,说明速度在数值上等于单位时间内位移的大小,使学生
既懂得可以用比值来定义速度,又能跟初中学过的知识联系
起来,理解其意义.

从速度的定义式$v=s/t$,
可以直接写出匀速运动的位移
公式$s=vt$. 根据这个公式就可以确定做匀速运动的物体在
任意时间内的位移,从而确定物体在任意时刻的位置.指出
这一点,可以使学生和前面提出的研究物体运动的总线索联
系起来,认识这一公式表示出了匀速运动的规律.

关于匀速直线运动的定义,教材没有明确“相等的时间”
是“任意”的.这样做是为了把问题叙述得简明一些,减少一
些过细的分析和冗长的论述.

第五节讲解匀速直线运动的图象,主要要求学生会认识
图象,知道图象的物理意义,会画简单的图象.学生刚开始学
习图象,不要求他们用图象去解决比较复杂的问题,以免增加
教学上的难度和学生的负担.教学中,应注意引导学生把数
学中学过的函数及其图象的知识运用到物理中来.这一节讲
述了用速度图象求位移,为后面用面积法推导匀变速运动的
位移公式做准备.

关于即时速度的概念,着重讲述它的物理意义.教材中
写了用数学语言可以精确地表达即时速度,只是让学生知道
个意思,不要求细讲.

为了减少学生学习加速度的概念时的困难,本章只讲匀
变速运动的加速度,不讨论一般变加速运动的情况,不引入
平均加速度和即时加速度的概念,以后各章中再逐步扩大对
加速度的认识,例如讲牛顿第二定律时,提到加速度可以改
变,即外力随时间改变,加速度就随时间而改变,讲圆周运动
时进一步认识加速度的方向的改变.

关于匀变速直线运动位移公式,教材采用了求面积的方
法推导,比较直观形象.这种通过计算面积来求物理量的方
法在物理学中经常用到,需要学生熟悉.但是,对于什么是无
限分割,限于学生的数学知识水平,不能细讲,只要求他们能
体会其意思就行了.

教材没有把匀加速运动和匀减速运动分开来讲,而是把
它统一看作匀变速运动,用统一的位移公式和速度公式来处
理.这样便于后面用统一的方法来处理竖直上抛等问题,有
助于培养学生的概括能力.

匀变速直线运动的两个基本公式皮映了匀变速直线运动
的规律并包括了前面讲过的匀速直线运动.要使学生清楚这
一点,提高他们对两个基本公式的认识.这里提到初速不为
零的匀变速直线运动,可以看作是速度为0.的匀速直线运动
和初速为零的匀变速直线运动的合运动,为了使学生理解这
几种运动之间的关系,加深对匀变速直线运动公式的认识.关
于运动合成的知识这里不宜多讲,到第四章再具体讲解.
教材通过对自由落体闪光照片的分析,得出自由落体运
动是匀变速直线运动,利用闪光照片来研究物体的运动情况,
这种方法以后还要用到.

对竖直上抛运动,教材是把它作为统一的匀变速运动来
处理的,以提高学生的理解能力和运用知识的能力.如果这
样处理有困难,也可以先把上升过程和下降过程分开来计算,
再用统一的运动来处理.讨论竖直上抛物体的上升时间、上
升的最大高度、下落时间、落地速度等,是为了使学生练习运
用匀变速运动的一般规律来分析具体问题,而不是单纯地记
住几个计算公式.

\section{教学建议}
这一章的内容可以分为四个单元:

第一单元(第1节——第3节)讲授描述物体运动的一些预
备知识.

第二单元(第4节——第5节)讲述匀速直线运动.

第三单元(第6节——第10节)讲述匀变速直线运动.

第四单元(第11节——第12节)讲述自由落体运动和竖直
上抛运动.

这一章的重点是第三单元,并通过第四单元两种常见的
实例来巩固和加深对于匀变速直线运动的认识.

\subsection{第一单元}
这一单元是在复习初中学过的机械运动有关知识的基础
上讲述参照物、质点、平动与转动、位置和位移等基本概念,内
容比较简单,但对以后的学习是重要的.

\subsubsection{运动的相对性}

使学生从理性上接受“在自然界中没
有不运动的初体”并且跟“总有许多物体停在原地不动”这种
日常经验统一起来,逻辑上自然要求提出参照物和运动的相
对性问题.为了避免引伸出去讲得很多,可以不提出“运动
的相对性”这个概念,要通过实例使学生清楚地知道:
\begin{enumerate}
    \item 一个物体对某个参照物来说是运动还是静止,要看这个物体
    对参照物来说位置是否变化;
    \item 对于相同的运动,由于选取
    的参照物不同,观测得出的结果可以是不同的;
    \item 虽然参照
    物的选取是人为的,但是在实际选参照物时,总是要使观测方
    便和使运动的描述尽可能简单,例如,在研究地面上物体的运
    动时,一般总是选取地面作参照物.
\end{enumerate}

质点是力学中的一个重要概念.在讲这个概念时,首先
要抓住“物体都具有大小和形状,在运动中物体中各点的位置
变化一般说来是各不相同的,所以要详细描述物体的运动,并
不是一件简单的事情”这一段叙述作为出发点,结合实例(例
如,改变桌子的方位)先讨论在运动中物体中各点的位置变化
各不相同的问题,然后再结合另外一些实例(远途行驶的汽
车、公转轨道上的地球,等等)讨论物体的大小和形状可以忽
略的问题,使学生知道在什么情况下可以只考虑问题的主要
方面、忽略次要因素的影响,使本来并不简单的事情得到简
化,从而便于找出它的规律来.这种使事物或问题简化、理想
化、抽象化、模型化的思想方法和研究方法,在“自由落体运
动”中以及后面还有很多内容都要接触到.

在讲质点的概念时,另一种情况,即平动的物体可以看成
质点,也要讲到,不可忽略.它也是科学的抽象.

\subsubsection{位置和位移}
 这一节一开始就提出,研究物体(质点)
的运动,首先要确定物体(质点)的位置,并可梁取建立坐标系
的方法来确定.在教学中要向学生指出,这里可以把所选定
的参照物,作为坐标系的原点以确定运动物体(质点)的位置,
并按上一章一维矢量的运算方法,来确定质点的位移.但是,
为了计算的方便,更常见的是取质点的初位置作为坐标的
原点.

这一节,要重点讲清楚“位移”这个新概念.主要是要讲
清位移是表示质点在运动过程中位置变化的物理量,它只与
运动的起点(初位置)和终点(末位置)有关,而与物体运动的
路径无关.要通过实际例子讲清楚位移不但有大小,而且有
方向,是个矢量,并把它跟初中学过的路程的概念进行对比和
区别.在讲位移和路程的区别时,要着重提醒学生,只有质点
始终向着同一个方向作直线运动时,位移的大小才等于路程.
在直线运动中,对位移的坐标表示,学生只要知道用初、末位
置的坐标来得出位移的数值,其方向则可按规定好的坐标轴
的正方向依一维矢量的运算方法得知.

\subsection{第二单元}
这一单元就它的内容-匀速直线运动来说,是学生所
熟知的,但是其中速度的概念和匀速直线运动的图象,比初中
的要求高得多,而且在方法上又要为以后学习其他内容作准
备,教学时要予以重视.

\subsubsection{匀速运动的速度}

这一节的教学在讲了什么是匀速
直线运动之后,重点应放在速度的定义、意义和矢量性上.对
中学生来说,用比值的形式来定义一个物理量,也是他们所不
熟悉的.关键在于使同学们认识位移跟时间的比值表示的就
是他们在日常生活中很熟悉的物体运动的快慢,应该从匀速
直线运动的定义出发,即物体在相等时间内的位移相等,推出
匀速运动的位移和时间的比值是一个不随时间而改变的恒
量,这个比值越大,表示物体在相同时间里的位移越大,即运
动得越快.

用位移跟时间的比值来定义速度,突出了速度概念的矢
量性.速度矢量的方向,就是物体位移的方向,根据一维矢量
的运算法则,由于匀速直线运动是沿着同一方向运动的,只要
取位移的方向为正方向,它的位移和速度就都是正值.

了解时间和时刻的区别,对于初学者来说也是很必要的.
这里,不需要在形式上给时间和时刻下什么定义,可以通过具
体的例子说明两者的区别.如通常所说的“几秒内”、“第几秒
内”都是表示一段时间,而“第几秒末”表示的就是一个时刻.
物体在运动过程中,每一时刻都有一定的对应位置,而一定的
时间,则对应于一段位移.

\subsubsection{匀速运动的图象}

物体运动的规律不仅可以用数学
公式去描述它,还可用图象的方法去描述它,而且用图象的方
法,有时更形象直观一些.这一节教学就是要引导学生把函
数图象的知识运用到物理中来.

讲解匀速直线运动的图象,主要是让学生学会认识图象,
理解图象的物理意义,会画简单的图象,学习怎样用图象来求
位移和速度.在图象的绘制上,一开始就要培养学生严格的
科学态度,一定要用直角或三角尺作图,标明横轴和纵轴所代
表的物理量及其单位,并选择适宜的标度.要让学生理解,在
匀速直线运动的速度图象中,可以用长方形“面积”的数值来
表示位移的大小,是因为长方形的一条边长在数值上等于物
体运动时间$t$的长短,另一条边长在数值上等于物体速度$v$
的大小,因此长方形的面积在数上恰好等于物体在时间内
的位移$s=vt$的大小;它的单位是“${\rm m/s\x s=m}$”,而不是
“${\rm m^2}$”.弄清楚这些知识,一方面有利于理解图象所表示的物
理意义,另一方面也有利于培养学生灵活运用数学知识解决
物理问题的能力.学生刚开始学习图象,要从数与形的关系
以及函数图象与物理量的关系,着重引导学生理解和领会图
象的物理意义及其应用,而不要补充利用图象去解二个物体
相向运动或同向运动、速度不同,何时相遇,或要求解释图
线相交点及负斜率的意义等比较复杂的问题.

\subsection{第三单元}

第一单元是本章的重点.教好这一章的关键,在于讲好即
时速度和加速度这两个重要的物理概念.从平均速度引入即
时速度,从即时速度的变化引入加速度,从加速度的定义式引
入匀变速直线运动的速度公式并进一步得出速度图象,从匀
变速直线运动的速度图象,引入匀变速运动的位移公式,再从
匀变速运动的速度公式和位移公式,得出两个有用的推论,是
这一单元教学的主要线索.

\subsubsection{即时速度}

讲述即时速度的概念,可以先让学生粗略
地了解运动物体在某一时刻(或某一位置)都有一定的速度,
这个速度就是物体的即时速度.然后再进一步讲述它的物理
意义.

运动物体在每一时刻(或每一位置)都有一定的速度这一
点,结合日常生活经验,学生是不难理解的.例如自行车和汽
车,在你面前驶过的瞬间,它们的快慢是不同的;百米赛跑,运
动员们在到达终点时的冲刺速度也各不相同,等等.这些事
例都可以说明运动物体在每一时刻(或每一位置)都有一定的
速度.学生难以理解的是即时速度的物理意义,这里主要应
该让学生理解即时速度也就是在足够短的时间里(或位移上)
运动物体的平均速度,因为在足够短的时间里,物体速度的变
化很小,已不能为测量仪器所分辨.在这样的条件下,物体的
运动在测量误差允许的范围内,可以认为是匀速的.

\subsubsection{加速度}

匀变速运动的加速度是本单元另一个重要一
的物理概念,教材在讲清楚什么是匀变速直线运动的基础上,
仿照匀速运动中定义速度的方法,用速度的变化和所用的时
间的比值来定义加速度,并说明了它的矢量性.教学中要强
调:
\begin{enumerate}
    \item 匀变速直线运动的速度在不断改变,而加速度是保持
不变的;
\item 加速度既是矢量,就要遵循一维矢量的运算方法,
根据事先选定的运动的正方向,用带有正负号的数值来表示
它,一般取初速度$v_0$的方向为正方向.
\end{enumerate}
因此,当$v_t>v_0$时$a=\frac{v_t-v_0}{t}$一
定是正值,表示$a$与$v_0$的方向相同;当$v_t<v_0$时,$a$
一定是负值,表示$a$与$v_0$的方向相反.

学生初学时对速度、速度的变化和加速度这几个物理量
在概念上往往混淆不清,教学中应该注意澄清.《速度和加速
度的区别》这段阅读材料,有助于弄清这些问题,应该引导学
生认真阅读,并提出问题来了解、检查学生阅读后理解的情
况.例如可以提出:在平直路面上行驶的汽车,在离开车站
和即将靠站时,汽车的加速度的方向是相同的吗?速度的方向
是相同的吗?汽车在匀加速行驶时,速度对时间的变化率以及
位移对时间的变化率都是不变的吗?等等.教师还应该引导
学生通过一些具体问题的讨论,认识加速度是表示速度变化
快慢的物理量,跟速度的大小没有直接关系.速度大的物体,
加速度不一定大;速度小的物体加速度不一定小.另外,速度
变化的大小,不仅与加速度的大小有关,还跟加速的时间长短
有关.速度变化大的物体,加速度也不一定大,速度变化小的
物体,加速度也不一定小.

\subsubsection{速度公式和速度图象}
在导出匀变速运动的速度公
式时,可先从加速度$a=\frac{v_t-v_0}{t}$
,过渡到速度的变化$v_t-v_0=at$, 然后得出速度公式$v_t=v_0+at$, 这样既可分步分别阐
明和强调公式的物理意义,并可引导学生在理解概念区别的
基础上,进一步认识匀变速运动的速度和加速度的联系.课本
中不把匀加速运动跟匀减速运动分开来处理,而统一为匀变
速运动,只是做匀减速运动时,加速度为负值,这样可以避免
把公式搞得太多,把问题看得太绝对,造成机械的记忆和硬套
公式,加速度的方向也似乎可以不考虑了.

根据速度公式$v_t=v_0+at$, 可以画出匀变速运动的速度
图象,要引导学生弄清楚:图线通过原点和不通过原点的物
理意义,纵轴上的截距和图线的斜率的物理意义,以及斜率的
大小和正负的物理意义等.把这些基本内容掌握了,便有利
于学生的认图、用图与作图.概括地说,这一节教学中,对公式
的导出,不要简单地处理为数学公式的变换,对图象教学不要
单纯地处理成数学上数与形的关系,仅仅作出公式的函数图
象.要强调公式,图象的特点及其变化所表示的物理意义.

\subsubsection{位移公式}

在引用匀速运动的速度图线和横轴之间
的面积表示位移这种方法来求匀变速运动的位移时,要讲一
讲为什么将时间轴尽量加以分割,使折线下的面积,尽量逼近
速度图线和横轴之间的面积,从而可以用它来表示匀变速运
动的位移.让学生熟悉和体会这种方法,对于培养学生的科
学思维能力是有好处的.但是限于学生的数学水平,也不宜
做过细的分析.得出位移公式以后,应通过例题说明加速度
矢量的方向如何根据一维矢的规定来表示,这也是要求学
生熟悉的.

为了使学生学会用不同的方法来解题,并在此基础上选
择简便的方法来求解,可举一些已知条件不同的题目,由同学
讨论、分析、判断、比,让他们自己去体会,不宜由教师归纳
为几条,以免代替和抑制学生的思维.但教师还应该做必要
的指导,例如为了培养良好的习惯,解题时要先弄清楚题目中
所描述的整个运动过程;对复杂的问题,要分步考虑,一步步
地分析出所求的未知量和已知量之间的关系,而这个“关系”
从数学上说是公式,从物理上说就是运动规律.在运用时要
思考它是否符合这个规律,在熟悉分步运算的同时,要引导
学生逐步学会列出方程,利用文字运算来解题,有的题目,要
注意一题多解,以提高学生掌握物理规律和分析物理问题的
能力.

\subsection{第四单元}
这一单元是用匀变速运动的知识来研究自由落体和竖直
上抛这两种常见的运动,认识这两种运动的特点和规律.这
也是对匀变速直线运动基本规律的应用和巩固.

\subsubsection{自由落体运动}

这一节教学应依次掌握好下列各个
环节:
\begin{enumerate}
\item 做好课本图2.20毛钱管演示实验,以表明在管中
空气被抽出后,重量不同的物体下落的快慢相同.
\item 把管中
空气抽出后,如果忽略余下的稀薄气体的作用,就可以近似地
看成是没有空气的空间;在没有空气的空间里,物体下落时才
是“只受重力的作用”;因此,自由落体运动是理想化的运动模
型.
\item 让同学实际测量课本图2.21频闪照片中小球在各个
相等时间里的位移,以鉴别小球自由落下时是作什么运动.有
频闪设备的学校,可以根据自己实际拍摄的频闪照片分组测
量数据;没有频闪设备的学校,也可用其他实验来代替.在得
出:“自由落体运动是初速度为零的匀变速运动”的结论以后,
紧接着要做出的第二个结论便是:不同的自由落体,它们的
运动情况相同,也就是在同一地点,一切物体在自由落体运动
中的加速度(重力加速度$g$)都相同,
\item 既然一切物体在自由
落体运动中的加速度都相同,它必然是一定值,我们可通过实
验来测定它,接着就让学生根据课本88页给出的数据表去计
算$\Delta s$和$\Delta s$的平均值,逐步引导学生重视数据处理,培养这
方面的能力.   
 \item 引导学生认真阅读《伽利略对自由落体运动
的研究》,以培养自学阅读能力和逻辑思维能力,从中学习用
外推法研究物理现象和规律的思路和方法,并且还可以使学
生获得一些物理学史的知识.
\end{enumerate}


\subsubsection{竖直上抛运动}

竖直上抛运动一直有这样两种处理
方法:一是把运动分为匀减速上升和自由下落两个过程,分
开来进行计算;另一种是把它看成是向上的匀速运动和向下
的自由落体运动这两个分运动的合成运动,前一种方法比较
直观,但是在运算时比较繁也容易错;后一种方法比较品象,
如果不进一步具体分析,就只能从运动公式的形式上来说明
两项分别表示两个分运动,学生也不容易体会.

课本在这一章不过分强调运动的合成,也避免学生不容
易理解的运动的独立性,把整个上抛运动看成是一个统一的
匀变速直线运动,这样既体现了课本中不把匀加速运动和匀
减速运动分开来的处理方法,在理解上也并不困难.而另一种
方法,把上升运动和下降运动分为两步来计算,留给同学自己
去尝试,也是可取的,在这方面的要求,对程度不同的学生可
以因材施教.

上抛运动的位移和速度的方向,仍应强调按一维矢量的
统一规定:对速度矢量来说,是以上抛运动初速度$v_0$的方向
为正方向;位移矢量是以初位置(抛出点)为原点,以初速度的
方向为正方向,物体位于抛出点下方时,位移方向向下,位移
是负值;加速度矢量也是以初速度的方向为正方向,而由于重
力加速度的方向跟初速度的方向相反,因此重力加速度$g$总
是取负值.当$g$取绝对值时,上抛运动的公式即可写成:
\[v_t=v_0-gt,\qquad s=v_0t-\frac{1}{2}gt^2\]
这样把各矢量的方向规定一并
弄清楚,就不致于混淆.

在应用上抛运动的公式讨论几个具体问题时,要引导学
生掌握它们的特征,例如物体上升到最大高度时,特征是到
达最高点时即时速度为零,据此可以很容易算出物体上升的
时间和上升的最大高度;物体落回到初位置时的特征是位移
为零,据此可以很容易得出落回原地的时间和物体着地时的
速度.引导同学理解和掌握这些特征,并不是要同学记忆几
条,而是要通过分析和练习让同学去领会.

在解题计算的过程中,还要引导同学理解方程同时有两
个解的物理意义.

\section{实验指导}
\subsection{演示实验}
\subsubsection{观察匀速直线运动和匀变速直线运动}

利用节拍器、斜面(长约1.50米)和小车观察匀速直线运
动和匀变速直线运动的实验装置如图2.1所示.事先调节好
斜面的倾斜程度,使得小车恰能沿斜面匀速下滑,做好垫木位
置的记号,然后再将垫木移右些,使小车下滑时做加速运动.
\begin{figure}[htp]
    \centering
    \includegraphics[scale=.8]{fig/2-1.png}
    \caption{}
\end{figure}

打开节拍器,当听到节拍器发出一个信号时,立即释放小
车,使它自某一固定位置$O$下滑.用一木块阻挡小车,调整阻
挡木块的位置,重复几次实验,使得小车撞击木块时发出的
声音恰巧和节拍器发出的第二个信号(以小车开始释放时的
信号作为第一个信号)重合.在阻挡木块的这一位置($A$)上,
用事先准备好的箭头标出(用胶纸把箭头贴在斜面的侧边).
用同样的方法来确定小车和木块的撞击声恰和节拍器发出的
第三个信号、第四个信号重合时的木块位置$B$和$C$, 并分别用
箭头标出.用米尺量度$OA$、$AB$和$BC$的长度,发现$OA<AB
<BC$, 然后将垫木移到事先准备好的位置,重做实验,直到测
得$OA=AB=BC$. 这表明小车在相等时间里的位移都相等,
所以小车的运动是匀速运动.

在演示匀变速直线运动时,则可以将垫木放在事先调整
好的另一位置上,用上述方法观察小车在相等时间内经过了
不相等的距离.通过调节节拍器的频率,使得小车从静止开
始释放在各相等时间里发生的位移之比$OA:AB:BC=1:3:5$
(譬如可调节到使$OA=16{\rm cm}$, $AB=48{\rm cm}$, $BC=90{\rm cm}$),在
这基础上还可进一步得出$AB-OA=BC-AB$, 即匀变速直线
运动中,在连续相等时间内的位移差是一常数.

\subsubsection{测量匀变速直线运动的即时速度}

\begin{figure}[htp]
    \centering
    \includegraphics[scale=.8]{fig/2-2.png}
    \caption{}
\end{figure}


可利用斜面、小车和节拍器采用上述的实验方法,测
出小车在各连续相等时间内的位移$OA=s_1$, $AB=s_2$, $BC=s_3$
(图2.2).节拍器发出信号的时间间隔为$T$, 根据匀变速直
线运动公式可知:
\begin{align}
    s_1&=\frac{1}{2}aT^2\\
    s_2&=v_AT+\frac{1}{2}aT^2
\end{align}

将(2.1)、(2.2)式相加,$s_1+s_2=v_A T+aT^2$.

$\because\quad v_A=aT$

$\therefore\quad s_1+s_2=v_AT+v_AT=2v_AT,\qquad v_A=\dfrac{s_1+s_2}{2T}$

也$\dfrac{s_1+s_2}{2T}$就等于小车开始运动$2T$时间内的平均速
度,所以匀变速直线运动中某一段时间的中间时刻的即时速
度就等于在这一段时间内的平均速度.

同理,可以测出当小车经过位置$B$时的即时速度$v_B=\dfrac{s_2+s_3}{2T}$

利用打点计时器来测量.
\begin{figure}[htp]
    \centering
\includegraphics[scale=.8]{fig/2-3.png}
    \caption{}
\end{figure}
如图2.3所示,在一端装有定滑轮的长木板上,放一条有
细绳的小车,通过定滑轮在细绳的另一端挂有几个钩码,固定
在小车后面的纸带和打点计时器连在一起.接通电源待打点
计时器正常工作后,释放小车.取下纸带后请一位学生选定
连续的几个计数点(可用每打五次点的时间作为时间的单
位),并要求学生毫米刻度尺测量出各相邻计数点间的距离
$s_1,s_2,s_3,s_4,\ldots$, 教师可将纸带以及选定的计数点放大后画
在黑板上,并将学生实际测得的数据标出,用跟前述相同的方
法来处理数据,求出打点计时器打下各计数点时小车的即时
速度.

\subsubsection{测量匀变速直线运动的加速度}
可采用测量匀变速直线运动的即时速度时相同的实
验装置,取得数据,然后根据匀变速直线运动$\Delta s=aT^2$来求
加速度.$a=\dfrac{\Delta s}{T^2}$

用上述装置取得数据,算出打点计时器在打下各计
数点时小车的即时速度后,然后用画$v$-$t$图象的方法求出图
线的斜率,从而得出加速度$a=\dfrac{\Delta v}{\Delta t}$

算出打点计时器在打下各计数点时小车的即时速度
后,可以任取几段不同的时间及其相应的初速度和末速度的
数据,根据加速度的定义式$a=\dfrac{v_t-v_0}{t}$
,来分别求出这几段时
间内的加速度$a_1,a_2,a_3,\ldots$, 然后再求这些加速度的平均值.

\subsubsection{空气阻力对落体运动的影响}
准备一架调节好的托盘天平,先将一个乒乓球和一个小
铁球放在托盘天平上比较它们的重量,可看到小铁球比较重.
把这两个球效在同一高度上同时下落,则铁球先落地.又把乒
乓球和一块较大的泡沫塑料平板放在天平上比较它们的重
量,可看到泡沫塑料板较重,把它们放在同一高度上同时下
落,则乒乓球先落地.再把一张纸裁成两半,把其中的一半揉
成纸团和另一半放在天平上称,它们是等重的,使它们从同一
高度同时下落,结果揉成纸团的那一半先落地.

这个演示说明了,比较重的物体可以先落地也可以后落
地,即使等重的物体落地的时间也有先后.因此使得物体落
地的时间有先后的原因不是由于重力的大小而是由于空气阻
力大小的影响.受到空气阻力大的物体总是后落地.

\subsubsection{在空气阻力很小时,不同物体同时落下}
这可以用课本图2.20所示牛顿管(又称毛钱管)的传统
实验来进行.演示时可以先不抽空气,当把管子迅速倒转来
时,金属片很快下落,羽毛则下落较慢.然后抽气(抽气要用
管壁很厚的橡皮管),抽气后再演示,发现羽毛和金属片同时
落到管子的底部,最后再将空气放入管中,则羽毛又比金属片
下落得慢.这证明了:在空气阻力很小时,一切物体在同一高
度上的落地时间都是相等的.

\subsubsection{研究自由落体的闪光照片}
课本图2.21自由落体的闪光照片表明了自由落体运动
是初速度为零的匀变速直线运动.关于这幅照片,要求学生
理解以下几点:
\begin{enumerate}
\item 这不是许多个小球,而是表明一个自由下落的小球
在经过各个相等时间
($1/30$秒)时的位置.
\item 从每隔相等时间来看,小球下落的距离越来越大,说
明小球是作变速运动.
\item 照片上小球最初几个位置比较密集,因此可选择某
一个间距较大的位置作为位置1开始测量.小球的位置都取
小球球心(也可以取小球的上缘或下缘),这样来量度相邻两
个位置间的距离$s_1,s_2,s_3,\ldots$, 再算出相邻的相等时间内的
距离之差$\Delta s_1=s_2-s_1,\Delta s_2=s_3-s_2,\Delta s_3=s_4-s_3,\ldots$指
导学生阅读课本88页的数据表,发现$\Delta s$基本上都是接近
的,因此可以证明自由落体运动是初速度为零的匀变速直线
运动.
\item 要注意课本数据表中的数据是根据照片中的刻度尺
读取的,而不是在照片上用毫米刻度尺测量的.
\item 从数据表所列的数据可以计算出自由落体运动的加
速度(即重力加速度$g$)的数值.
\end{enumerate}

\subsubsection{利用打点计时器来研究自由落体运动}
可以按照课本图10.16,利用铁架台把打点计时
器固定起来,用手提住夹有重物的纸带,接通电源,当打点计
时器正常工作后,松开纸带,让重物拖着纸带自由下落.对纸
带上记录的点的分布情况进行分析,可以证明自由落体运动
是初速度为零的匀变速直线运动,而且可以求出重力加速度
$g$的数值.

这个实验也可以让全体学生自己做,这样可以增加练习
使用打点计时器的次数,并再一次练习对实验数据的分析和
处理.


\subsection{学生实验}
\subsubsection{练习使用打点计时器}
实验前要首先弄清楚所使用的打点计时器需要多大
的工作电压,打点的时间间隔是多少.

用手拉动纸带时,速度不要过小,要水平,直到全部
把纸带拉出,这样,即可观察到纸带上被打下的一系列点.

从纸带上能看得清的某个点数起,数一数纸带上共
有多少个点,计算一下在这段距离内纸带运动的时间$t$是多少
秒?要注意如果共有几个点,已知每两个点间经过的时间是
0.02秒,则运动的总时间$t=(n-1)\x0.02$秒.


\subsubsection{研究匀变速直线运动}
这个实验对于数据处理的要求较高,内容较多,要用
两课时完成.实验的具体要求是:
\begin{enumerate}
\item 从分析纸带上的点的分布来判断小车是否做匀变速
直线运动.
\item 在确认小车是做匀变速直线运动的前提下,利用纸带
上的数据来计算出小车在各个时刻的即时速度.
\item 通过画出速度-时间图象,来计算小车做匀变速直线
运动的加速度.
\end{enumerate}

按课本图10.9的装置把实验器材装好,先不要接通
交流电源,用手挡住小车,在细绳的一端挂上三个50克的钩
码.释放后,观察小车运动时拖着的纸带通过打点计时器限
位孔的位置是否恰当.适当调整并重新固定打点计时器的位
置使得限位孔正对着小车的运动方向,然后把纸带穿好,接通
电源,待打点计时器正常工作后释放小车.

取下纸带,观察纸带上的点,会发现开始时的几个点
很密集,为了减小测量误差,可从间距较大的点(譬如相距
几个毫米)开始进行测量,选定连续的几个计数点(不少于五
个),并要求学生参照课本图10.10, 将所选定的计数点标出
$A,B,C,D,E,\ldots$, 量出各计数点间的间距$s_1,s_2,s_3,s_4,\ldots$也
标在纸带上(图2.4),以此作为原始数据记录.
\begin{figure}[htp]
    \centering
    \includegraphics{fig/2-4.png}
    \caption{}
\end{figure}

为了测量方便,可以用每打五次点的时间作为时间的单
位,这样,在两个计数点间的时间间隔$T=5\x0.02=0.1$秒.

根据课本练习九第6题,可得
\[\begin{split}
    \Delta s_1&=s_2-s_1=aT^2\\
    \Delta s_2&=s_3-s_2=aT^2\\
    \Delta s_3&=s_4-s_3=aT^2
\end{split}\]

在匀变速直线运动中,加速度$a$是恒量,因此通过对纸带
上各个相邻计数点间距离的测量,算出各相邻计数点间的距
离之差$\Delta s$均相等,则可证明小车的运动是匀变速直线
运动.

怎样求出打点计时器在打下各计数点时小车的即时
速度?

在确认小车是做匀变速直线运动的前提下,可以利用速
度图象来求小车的加速度,这就首先需要求出小车从开始计
时(即打点计时器打下$A$点时)起,经过$T,2T,3T,\ldots$也就是
打点计时器打下$B,C,D,\ldots$各点时的即时速度.
\[v_1=\frac{s_1+s_2}{2T},\quad v_2=\frac{s_2+s_3}{2T},\quad v_3=\frac{s_3+s_4}{2T}\]

要注意:从课本图10.10所示纸带上所选定的这几个计
数点,应用上述方法只能测得即时速度$v_1$、$v_2$和$v_3$, 若要测出
打下$E$点时小车的即时速度,则必须在纸带上再确定经过时
间为$4T$时的计数点$F$, 测出$EF$间的距离$s_5$, 则$v_4=\dfrac{s_4+s_5}{2T}$.
同理,若要测出打下$A$点时小车的即时速度,则必须在纸带
上的$A$点之前再确定一个计数点$O$, 然后测量$O$和$A$点间的
距离$s_0$, 则$v_A=\dfrac{s_0+s_1}{2T}$.
要注意这个$v_0$并不等于零,它表示
在实验中开始计时时刻的初速度.

求出打点计时器在打下各计数点时的即时速度后,
就可设计一个能表示时间和其对应的即时速度数值的数据表
格.在坐标纸上建立一个平面直角坐标系,用横坐标表示时
间,用纵坐标表示速度,然后在坐标平面上标出$(T,v_1),(2T,
v_2),(3T,v_3),\ldots$各数据点,数据点不得少于五个.把这些点
连结起来可以画出一条直线,画直线时应尽量使多数的点落
在这条直线上,不在直线上的各点,应使它们比较均匀地分布
在直线的两旁,这就是在这条直线两侧的点数以及这些点到
直线的平均距离应大致相等,这就得出小车的速度图线.

\begin{figure}[htp]
    \centering
    \includegraphics[scale=.8]{fig/2-5.png}
    \caption{}
\end{figure}

求出速度图线的斜率就可以得出小车的加速度.如
果画出的$v$-$t$图象如图2.5所示,应该怎样求出这条直线的
斜率呢?要从图线上选取相隔较远的两个点,如$P$和$Q$,分
别从图象上读出它们的坐标$(t_P,v_P)$和$(t_Q,v_Q)$, 即可求得加
速度
\[a=\frac{\Delta v}{\Delta t}=\frac{v_Q-v_P}{t_Q-t_P}\]

可启发学生思考以下问题:
\begin{enumerate}
    \item 如果不用速度图象来求小车的加速度,是否还有其他
的方法?
\item 实验中为什么不直接用$a=\dfrac{v_5-v_1}{4T}$
来求加速度,而要
在图线上另找$P$、$Q$两点通过求图线的斜率来得出加速度,这
样做有什么好处?
\end{enumerate}

这个实验的课时安排,可以在第一课时内完成实验
的准备、使用打点计时器打出纸带以及分析纸带上点的分布、
确定计数点、测量数据判断小车是否做匀变速直线运动等内
容.求出打点计时器在打下各计数点时小车的速度、画出$v$-$t$
图象,求得小车的加速度等内容可安排在第二课时完成.

\subsection{课外实验活动}
\subsubsection{滴水法测重力加速度}
这个实验的原理是根据自由落体运动是初速度为零
的匀变速直线运动,水滴下落的距离$h$跟运动时间$t$的平
方成正比$h=\frac{1}{2}gt^2$, 则 
\[g=\frac{2h}{t^2}\]

因此只要测出水滴下落的距离和下落的时间,便可测得
重力加速度.

这个实验中的水滴下落距离是易于测量的,比较困
难的是时间的测定.对于课本介绍的测时间的方法,要多次
耐心地调整阀门(自来水笼头)的大小,才能使水滴从阀门落
到盘子经过的时间正好等于阀门滴下水滴的时间间隔.为了
便子调整,盘子可以倒过来放在水槽里(或者用大口瓶上的金
属盖代替盘子,使盖子的顶部朝上).由于盘子下面跟水槽间
有一空腔,使得水滴在盘子上的响声比较清脆,阀门离盘
子的距离不要太近,太近了不容易区别两次滴水的时间间隔.
(譬如水滴下落的距离约为0.5米左右时,半分钟里约有
90—100个水滴从阀门滴下,这样,水滴下落的时间就正好等
于相继滴下的两个水滴之间的时间间隔).

用这一方法测定的重力加速度的数值是近似的,但
实验方法比较巧妙而且简单.

\subsubsection{用秒表测量玩具手枪子弹射出的速度}
这个实验的原理是,以初速$v_0$、竖直上抛的物体,从
开始抛出直到落回抛出点所经过的总时间$t=2v_0/g$,
只要测出
玩具手枪的子弹从发射到落回发射点的时间$t$, 当地的$g$值
可以由教师给出,即可测出玩具手枪子弹射出时的初速度
$v_0=\frac{1}{2}gt$.

由于实验条件的限制,所测得的子弹初速度是近似
的.为了能使子弹基本上做竖直上抛运动,可以设法使玩具
手枪固定起来(譬如可将手枪缚在一张方木凳的边上,使枪口
和凳面相平,并使枪管竖直向上),试着先发射一发子弹调节
枪管的位置,使子弹不做明显的斜抛运动就可以了.实验时
可以在手枪的另一侧再放一个相同高度的木凳,从扳动手枪
扳机发射子弹的同时开始计时,当子弹落到木凳时再按下秒
表,测出子弹做竖直上抛运动的总时间$t$. 如果没有秒表,也
可以用手表近似地计时.

\section{习题解答}
\subsection{练习一}
\begin{enumerate}
    \item 两辆在公路上直线行驶的汽车,它们的距离保持不
变,试说明用什么样的物体做参照物,两辆汽车都是静止的,
用什么样的物体做参照物,两辆汽车都是运动的.能否找到
这样一个参照物,一辆汽车对它是静止的,另一辆汽车对它是
运动的?为什么?


\begin{solution}
用其中任意一辆汽车里的座椅做参照物,两辆汽车
都是静止的;用车外公路旁的树木、房屋做参照物,两辆汽车
都是运动的.因为两辆车的距离不变,它们保持相对静止,所
以不可能找到一个参照物,一辆车对它是静止的,而另一辆车
对它却是运动的.
\end{solution}

\item 小孩从滑梯上滑下,钢球沿斜槽滚下,石块从手中落
下,这些物体中哪些是做平动的?

\begin{solution}
根据运动过程中物体各部分的运动是否完全相同来
判断:小孩从滑梯上滑下是平动,钢球沿斜槽滚下不是平动;
石块从手中落下时如果没有翻转则也是平动.
\end{solution}

\item 研究自行车轮的转动,能不能把自行车当作质点?研
究在马路上行驶的自行车的度,能不能把自行车当作质点?

\begin{solution}
    研究自行车轮的转动时,不能把自行车当作质点;研
    究自行车的行驶速度时,可以把它当作质点.
\end{solution}
\end{enumerate}


\subsection{练习二}
\begin{enumerate}
    \item 质点做什么运动,位移的大小才等于路程?
    
\begin{solution}
    质点始终向同一方向做直线运动时,位移的大小等
于路程.
\end{solution}
    \item 课本图2.6表示做直线运动的质点从初位置
$A$经过$B$运动到$C$, 然后从$C$返回,运动到末位置$B$, 设$AB$
长7米,$BC$长5米.求质点的位移的大小和路程.

\begin{solution}
    如图所示,初位置为$A$, 末位置为$B$, 所以位移的大
小为7米($AB$长).
\[\text{路程}=AB+BC+CB=7+5+5=17{\rm m}\]
\end{solution}
\item 在课本图2.4中汽车初位置的坐标是$-2$千
米,末位置的坐标是1千米.求汽车的位移的大小和方向.

\begin{solution}
    汽车的位移$s=1-(-2)=3$千米,由于
位移为正值,方向跟坐标轴正方向一致,即由西向东.
\end{solution}
\end{enumerate}

\subsection{练习三}

\begin{enumerate}
    \item 光在真空中沿直线传播的速度为$3.0\times 10^8$$\ms$.
\begin{enumerate}
    \item 一光年(光在一年中传播的距离)有多少千米?
    \item 最靠近我们的恒星(半人马座$\alpha$星)离我们$4.0\times 10^{13}$千米,它发出的光要多长时间才到达地球?
\end{enumerate}    

\begin{solution}
\begin{enumerate}
    \item $1\text{光年}=365\x24\x60\x60{\rm s}\x3.0\x10^5{\rm km/s}=9.5\x10^{12}{\rm km}$
    \item $t=\dfrac{4.0\x 10^{13}}{9.5\x10^{12}}=4.2\text{年}$
    
    最靠近我们的恒星发出的光要4.2年才能到达地球.
\end{enumerate}
\end{solution}
\item  在技术上常用$\kmh$作速度的单位.试求1$\ms$合多少$\kmh$.

\begin{solution}
    \[\frac{1{\rm m}}{1{\rm s}}=\frac{1/1000{\rm km}}{1/3600{\rm h}}=\frac{3600}{1000}{\kmh}=3.6\kmh\]
\end{solution}
\item 光在空气中的速度可以认为等于光在真空中的速
度.声音在空气中的速度是340$\ms$.一个人看到闪电后5
秒听到雷声,打雷的地方离他大约多远?

\begin{solution}
    设打雷的地方跟观侧者的距离为$s$, 声速$v=
340{\rm m/s}$,由于光速很快,可以认为闪电发出后,即刻被看到,于
是时间$t=5$秒即是雷声传播的时间,所以$s=vt=340
\x5=1700{\rm m}$.即打雷的地方离他1700米远.
\end{solution}

\end{enumerate}

\subsection{练习四}
\begin{figure}[htp]
    \centering\begin{minipage}[t]{0.48\textwidth}
\centering
    \begin{tikzpicture}[>=stealth,  thick, scale=.9]
    \draw [<->](0,5)node[right]{$s$(千米)}--(0,0)--(4,0)node[right]{$t$(小时)};
   
    \foreach \x in {1,2,3,...,12}
    {
        \draw(\x/4, 0) --(\x/4, .2);
    }
    
    \foreach \y in {1,2,3,...,8}
    {
        \draw(0,\y/2)--(.2, \y/2);
    }
    \node at (-.2,-.2){$0$};
    \node at (1.5,-.2){$1$};
    \node at (3,-.2){$2$};
    
    \node at (-.5,1){$200$};
    \node at (-.5,2){$400$};
    \node at (-.5,3){$600$};
    
    \draw [dashed] (1.5/2,0)--(1.5/2,1.5)--(0,1.5);
    \draw [dashed] (1.5,0)--(1.5,3)--(0,3);
    \draw [dashed] (1.5+1.5/6,0)--(1.5+1.5/6,3.5)--(0,3.5);
    \draw[ultra thick](0,0)--(1.5*1.5,4.5);
    
    \end{tikzpicture}
    
    \caption{}
\end{minipage}
\begin{minipage}[t]{0.48\textwidth}
\centering
\begin{tikzpicture}[>=stealth,  thick, scale=.9]
    \draw [<->](0,5)node[right]{$v$(千米/小时)}--(0,0)--(4,0)node[right]{$t$(小时)};
   
    \foreach \x in {1,2,3,...,12}
    {
        \draw(\x/4, 0) --(\x/4, .2);
    }
    
    \foreach \y in {1,2,3,...,8}
    {
        \draw(0,\y/2)--(.2, \y/2);
    }
    \node at (-.2,-.2){$0$};
    \node at (1.5,-.2){$1$};
    \node at (3,-.2){$2$};
    
    \node at (-.5,1){$200$};
    \node at (-.5,2){$400$};
    \node at (-.5,3){$600$};
    \node at (-.5,4){$800$};
    \draw[ultra thick](0,3)--(1.5*1.5,3);
    
    \end{tikzpicture}
    \caption{}
\end{minipage}
    \end{figure}

\begin{enumerate}
    \item 图2.6是一架民航飞机的位移图象.从这个图象求
    出
    \begin{enumerate}
        \item 飞机在30分钟内的位移;
        \item 飞行700千米所用的时间;
        \item 飞行速度并画出速度图象.
    \end{enumerate}

\begin{solution}
\begin{enumerate}
    \item 从位移图象上看出,
    时间为30分钟的点所对应的位
    移为300千米.
    \item 从位移图象上看出,位
    移为700千米的点所对应的时
    间为70分钟.
    \item 从位移图象上时间为1小时、位移为600千米的点,即
    可求得速度为600千米/小时.
    速度图象如图2.7所示.
\end{enumerate}
\end{solution}

\item  图2.8是一辆火车运动的位移图象.线段$OA$和$BC$
所表示的运动,哪个速度大?各等于多大?线段$AB$与横轴平
行,表示火车做什么运动?速度是多大?火车在3小时内的位移
是多少?通过80千米用多长时间?画出火车的速度图象.

\begin{solution}
    由于位移图象线段$OA$的斜率大于线段$BC$的斜率,
    所以$OA$线段所表示的运动速度大;从位移图象上$A$点的坐标可以求得
\[v_{OA}=\frac{90}{1.5}=60{\rm kmh}\]
从$B$、$C$两点的坐标可以求得
\[v_{BC}=\frac{140-90}{3-2}=50{\rm kmh}\]

    线段$AB$表示火车静止,速度为零.从$C$的坐标可知火
    车在3小时内的位移是140千米.在图线上取位移为80千
    米的点,可以求得火车通过这段位移需用80分钟.根据前述
    各段作出火车的速度图象如图2.9所示.
\end{solution}

\begin{figure}[htp]\centering
    \begin{minipage}[t]{0.48\textwidth}
    \centering
\begin{tikzpicture}[>=stealth,  thick, xscale=.8]
    \draw [<->](0,4.5)node[right]{$s$(千米)}--(0,0)--(4,0)node[right]{$t$(小时)};
   
    \foreach \x in {1,2,3,...,9}
    {
        \draw(\x/3, 0) --(\x/3, .2);
    }
    
    \foreach \y in {1,2,3,...,8}
    {
        \draw(0,\y/2)--(.2, \y/2);
    }
    \node at (-.2,-.2){$0$};
    \node at (1,-.2){$1$};
    \node at (2,-.2){$2$};
    \node at (3,-.2){$3$};
    
    \node at (-.5,.5){$20$};
    \node at (-.5,1.5){$60$};
    \node at (-.5,2.5){$100$};
    \node at (-.5,3.5){$140$};
    
    \draw [dashed] (1,0)--(1,1.5)--(0,1.5);
    \draw [dashed] (1+1/3,0)--(1+1/3,2)--(0,2);
    \draw [dashed] (3,0)--(3,3.5)--(0,3.5);
    \draw [dashed] (2,0)--(2,2.25);
    
    \draw[ultra thick](0,0)--(1.5 ,2.25);
    \draw[ultra thick](2,2.25)node[right]{$B$}--(1.5 ,2.25)node[left]{$A$};
    \draw[ultra thick](2,2.25)--(3,3.5)node[right]{$C$}--(4,4.75);
       
    \end{tikzpicture}
    \caption{}
    \end{minipage}
    \begin{minipage}[t]{0.48\textwidth}
    \centering
    \begin{tikzpicture}[>=latex, scale=1]
        \draw [<->](0,4.5)node[right]{$v$(千米/小时)}--(0,0)--(3.5,0)node[right]{$t$(小时)};
       
        \foreach \x in {1,2,...,6}
        {
            \draw(\x/2, 0) --(\x/2, .1);
        }
        
        \foreach \y in {1,2,3,...,8}
        {
            \draw(0,\y/2)--(.1, \y/2);
        }
        \node at (-.2,-.2){$0$};
\foreach \x in {1,2,3}
{
    \node at (\x,-.2){$\x$};
}
        \node at (-.25,1){$20$};
        \node at (-.25,2){$40$};
        \node at (-.25,3){$60$};
        \draw[ultra thick](0,3)--(1.5,3)node[above]{$A$};
        \draw[ultra thick](2,2.5)node[above]{$B$}--(3,2.5)node[above]{$C$};
        \draw[ultra thick](1.5,0)--(2,0);
\draw[dashed](1.5,3)--(1.5,0);
\draw[dashed](2,2.5)--(2,0);
\draw[dashed](3,2.5)--(3,0);
    \end{tikzpicture}
    \caption{}
    \end{minipage}
    \end{figure}

    \item  有两个物体,从同一点开始向相同方向做匀速运动,
    速度分别是3$\ms$和5$\ms$,在同一个坐标平面上画出它们
    的位移图象和速度图象,并根据这两种图象分别求出它们在5
    秒内的位移.
    
\begin{solution}
    题中所述的两个物体的位移图象如图2.10所示,速度图象如图2.11所示,从位移图象上可以看出,时间为5秒
    时,图线$A$上$P_A$点的位移为15米,图线$B$上$P_B$点的位移为
    25米.从速度图象上求得图线$A$5秒内的“面积”为$3\x5=15$米,图线$B$5秒内的“面积”为$5\x5=
    25$米,于是,从位移图象和速度图象得到的结果相同,即两个物体5秒内的位移分别是15米和25米.
\end{solution}

\begin{figure}[htp]\centering
    \begin{minipage}[t]{0.48\textwidth}
    \centering
\begin{tikzpicture}[>=latex, xscale=.7]
    \draw [<->](0,4)node[right]{$S$(米)}--(0,0)--(5.5,0)node[right]{$t$(秒)};
\foreach \x in {1,2,...,5}
{
    \draw(\x,0)node[below]{\x}--(\x,.1);
}
\foreach \x in {1,2,3}
{
    \draw(0,\x)node[left]{\x0}--(.1,\x);
    \draw(0,\x-.5)--(.1,\x-.5);
}
\draw[dashed](0,2.5)--(5,2.5)--(5,0);
\draw[dashed](0,1.5)--(5,1.5);
\draw[very thick](0,0)--(5,2.5)node[right]{$B$};
\draw[very thick](0,0)--(5,1.5)node[right]{$A$};
\node at (5,2.5)[above]{$P_B$};
\node at (5,1.5)[left]{$P_A$};
\node at (-.25,-.25){$O$};
    \end{tikzpicture}
    \caption{}
    \end{minipage}
    \begin{minipage}[t]{0.48\textwidth}
    \centering
    \begin{tikzpicture}[>=latex, scale=.7]
        \fill[pattern=north east lines](0,0) rectangle (5,5);
\fill[pattern=north west lines](0,0) rectangle (5,3);
  \draw [<->](0,6)node[right]{$v$(米/秒)}--(0,0)--(6,0)node[right]{$t$(秒)};     
\foreach \x in {1,2,...,5}
{
    \draw(\x, 0)node[below]{\x}--(\x,.1);
    \draw(0,\x)node[left]{\x}--(.1,\x);
}
\node at (-.25,-.25){$O$};
\draw[very thick](0,3)--(6,3)node[right]{$A$};
\draw[very thick](0,5)--(6,5)node[right]{$B$};
\draw[dashed](5,0)--(5,5);


    \end{tikzpicture}
    \caption{}
    \end{minipage}
    \end{figure}
\end{enumerate}


\subsection{练习五}

\begin{enumerate}
    \item 一辆汽车,起初以30$\kmh$的速度匀速行驶了30千米,然后又以60$\kmh$的速度匀速行驶了30千米.一位
    同学认为这辆汽车在这60千米中的平均速度是$1/2$(30千米/
    时+60$\kmh$)=45$\kmh$.这个结果对不对?

\begin{solution}
    按平均速度的定义应是位移与时间的比值,所以这
辆汽车的平均速度
\[\bar v=\frac{s_1+s_2}{t_1+t_2}\]
由于$s_1=30$千米,$s_2=30$千米,$t_1=1$小时,
$t_2=0.5$小时,所以
\[\bar v=\frac{60}{1.5}=40\kmh\]
因此,那位同学的看法是不对的.
\end{solution}
    \item 骑自行车的人沿着坡路下行,在第1秒内通过1米,
    第2秒内通过3米,在第3秒内通过5米,在第4秒内通过7米.求
    最初两秒内、最后两秒内以及全部运动时间内的平均速度.

    \begin{solution}
最初两秒:
\[\bar v=\frac{1+3}{2}=2\ms\]
最后两秒:
\[\bar v=\frac{5+7}{2}=6\ms\]
全部时间:
\[\bar v=\frac{1+3+5+7}{4}=4\ms\]
    \end{solution}
    \item 在一个速度是$v$的匀速直线运动中,各段时间内的
    平均速度以及整个运动的平均速度各是多大?每一时刻的即
    时速度是多大?

    \begin{solution}
        匀速直线运动是速度不变的运动,所以各段和全
部时间的平均速度以及每一时刻的即时速度都是$v$. 
    \end{solution}
    \item 火车以70$\kmh$的速度经过某一路标,子弹以
    600$\ms$的速度从枪筒射出.这里指的是什么速度?
    
\begin{solution}
    这里指的都是即时速度.
\end{solution}
\end{enumerate}

\subsection{练习六}
\begin{enumerate}
\item  加速度为零的运动是什么运动?
   

\begin{solution}
    是匀速直线运动.
\end{solution}
\item  有人说:速度越大表示加速度也越大.这话对吗?为什么?   

\begin{solution}
    不对,加速度是指速度变化的快慢,其大小取决于
    单位时间内速度变化的大小.高速运动的物体,速度虽然大,
    单位时间内速度的变化却不一定大,因而加速度也并不一
    定大.
\end{solution}
\item  汽车的加速性能是反映汽车质量的重要标志.汽车
从一定的初速度$v_0$加速列一定的末速度$v_t$,用的时间越少,表
明它的加速性能越好.下表是三种型号汽车的加速性能的实
验数据,求它们的加速度.

\begin{center}
\begin{tabular}{ccccc}
\hline
汽车型号 & 初速度$v_0$ & 末速度$v_t$ & 时间$t$ & 加速度$a$\\
& (km/h)& (km/h)& (s)& (m/s$^2$)\\
\hline
某型号高级轿车& 20& 50& 7 \\
某型号4吨载重汽车& 20& 50& 38\\
某型号8吨载重汽车& 20& 50& 50\\
\hline
\end{tabular}
\end{center}
   
\begin{solution}
    由公式$a=\dfrac{v_t-v_0}{t}$可得
\begin{enumerate}
    \item 高级轿车的加速度
    \[a_1=\frac{(50-20)\kmh}{7{\rm s}}=\frac{(50-20)\x 10^3{\rm m}}{7\x 3600{\rm s}}=1.19\msq\]
    \item 4吨载重汽车的加速度
    \[a_2=\frac{(50-20)\kmh}{38{\rm s}}=\frac{(50-20)\x 10^3{\rm m}}{38\x 3600{\rm s}}=0.22\msq\]
    \item 8吨载重汽车的加速度
    \[a_3=\frac{(50-20)\kmh}{50{\rm s}}=\frac{(50-20)\x 10^3{\rm m}}{50\x 3600{\rm s}}=0.17\msq\]
\end{enumerate}
\end{solution}
\item   以18$\ms$的速度行驶的火车,制动后经15秒停止,
求火车的加速度.
   

\begin{solution}
   \[a=\dfrac{v_t-v_0}{t}=\frac{0-18}{15}=-1.2\msq\]
\end{solution}
\end{enumerate}


\subsection{练习七}
\begin{enumerate}
\item 机车原来的速度是36$\kmh$,在一段下坡路上加
速度为$0.20\msq$,机车行驶到下坡末端,速度增加到54$\kmh$.求机车通过这段下坡路所用的时间.   

\begin{solution}
   \[a=\frac{v_t-v_0}{t},\qquad t=\frac{v_t-v_0}{a}\]
   由于$v_t=54\kmh=15\ms$,$v_0=36\kmh=10\ms$,$a=0.20\msq$,所以
\[t=\frac{15-10}{0.20}=25{\rm s}\]
\end{solution}
\item 一辆做匀变速运动的汽车,初速度是34$\kmh$,
4.0秒末速度变为42$\kmh$.如果保持加速度不变,6.0秒
末、7.0秒末的速度是多大?   

\begin{solution}
    用$v_0$和$v_4, v_6,v_7$分别表示汽车的初速度和4秒末、
    6秒末、7秒末的速度,则有
\[a=\frac{v_4-v_0}{4}=\frac{(42-34){\kmh}}{4{\rm s}}=\frac{8\kmh}{4{\rm s}}\]
所以
\[\begin{split}
    v_6&=v_0+a\x 6{\rm s}=34\kmh+\frac{8\kmh}{4{\rm s}}\x 6{\rm s}=46\kmh\\
    v_7&=v_0+a\x 7{\rm s}=34\kmh+\frac{8\kmh}{4{\rm s}}\x 7{\rm s}=48\kmh
\end{split}\]
\end{solution}
\item 匀变速运动的加速度是$-4.0\msq$.在某一时刻,
速度为$20\msq$.试求这一时刻后 4.0秒末和5.0秒末的速度.   

\begin{solution}
    由$v_t=v_0+at$,得
    4秒末的速度
    \[v_4=20{\rm m/s}+(-4.0\msq)\x4{\rm s}=4\ms\]
    5秒末的速度
    \[v_5=20{\rm m/s}+(-4.0\msq)\x5{\rm s}=0\ms\]
\end{solution}
\end{enumerate}


\subsection{练习八}
\begin{enumerate}
\item      钢球在斜槽上做初速度为零的匀变速运动,开始运动
后0.2秒内通过的路程是3.0厘米,1秒内通过的路程是多少?如
果斜面长1.5米,钢球由斜面顶端滚到底端需要多长时间?
   
\begin{solution}
由公式$s=\dfrac{1}{2}at^2$得$a=\dfrac{2s}{t^2}$.
把$t=0.2{\rm s}$,$s=3.0{\rm cm}=0.03{\rm m}$代入得
\[a=\frac{2\x 0.03}{(0.2)^2}=1.5\msq\]
由此得1秒内通过的路程
\[s_1=\frac{1}{2}at^2=\frac{1}{2}\x1.5\msq\x(1{\rm s})^2=0.75{\rm m}\]
又由公式$s=\dfrac{1}{2}at^2$得
\[t=\sqrt{\frac{2s}{a}}\]
把$a=1.5\msq$和$s=1.5{\rm m}$
代入可得钢球滚到斜面底端所需的时间
\[t= \sqrt{\frac{2\x 1.5{\rm m}}{1.5\msq}}=1.4{\rm s} \] 
\end{solution}

\item     飞机着陆后做匀变速运动,速度逐渐减小,已知初
速度是60$\ms$,加速度的大小是6.0$\msq$,求飞机着陆后5.0
秒内通过的路程.
   
\begin{solution}
\[\begin{split}
    s&=v_0t+\frac{1}{2}at^2\\
    &=60\x 5+\frac{1}{2}\x (-6.0)\x 5^2\\
    &=225{\rm m}
\end{split}\]
\end{solution}

\item     一辆汽车原来匀速行驶,然后1.0$\msq$的加速度
加快行驶,经12秒行驶了180米.汽车开始加速时的速度是多
大?
   
\begin{solution}
由公式$s=v_0 t+\dfrac{1}{2}at^2$, 可得$v_0=\dfrac{s}{t}-\dfrac{1}{2}at$.
把$s=180{\rm m}$,$t=12$s,$a=1.0\msq$代入上式得
\[v_0=\frac{180}{12}-\frac{1}{2}\x 1.0\x 12=15-6=9\ms\]   
\end{solution}

\item     骑自行车的人以5.0$\ms$的初速度登上斜坡,得到
$-40{\rm cm}/{\rm s}^2$的加速度,经过10秒钟,在斜坡上通过多长的
距离?
   

\begin{solution}
    由公式$s=v_0 t+\dfrac{1}{2}at^2$, 得:
    \[s=5\x 10+\frac{1}{2}\x(-0.4)\x (10)^2=30{\rm m}\]
\end{solution}
\item    汽车以36$\kmh$的速度行驶.刹车后得到的加
速度的大小为4$\msq$.从刹车开始,经过3秒钟,汽车通过的
距离是多少?
   
\begin{solution}
    解此题要注意两点:一是这里的加速度为负值;二是求出
    从刹车到车停止运动的时间$t$, 如果小于3秒,则求距离时用
    时间$t$; 如果大于3秒,则求距离时用的时间为3秒.
   
    由$a=\dfrac{v_t-v_0}{t}$
    ,得从刹车开始到车停止的时间$t=\dfrac{v_t-v_0}{a}$
    
    把$v_t=0$, $v_0=36\kmh=10\ms$代入上式得
\[a=\frac{0-10}{-4}=2.5{\rm s}\]
    所求的距离
    \[s=10\ms\x2.5{\rm s}+\frac{1}{2}\x (-4)\msq\x (2.5{\rm s})^2=12.5{\rm m}\]
\end{solution}

\end{enumerate}




\subsection{练习九}
\begin{enumerate}
\item 一个做匀变速运动的物体,初速度为3.0$\ms$,经过10秒钟,速度变为9.0$\ms$,它在这10秒钟内的平均速度是多大?   

\begin{solution}
由$\bar v=\dfrac{1}{2}(v_0+v_t)$,得:
\[\bar v=\frac{3+9}{2}=6\ms\]
\end{solution}
\item 从长3.0米的斜面顶端由静止滚下来的小球,末速度是2.5$\ms$,求小球滚动所用的时间.   

\begin{solution}
   由公式$s=\bar vt$和$\bar v=\dfrac{1}{2}(v_0+v_t)$,得
   \[s=\frac{v_0+v_t}{2}t\]
   所以
   \[t=\frac{2s}{v_0+v_t}=\frac{2\x 3.0}{0+2.5}=2.4{\rm s}\]
\end{solution}
\item 一辆汽车以12$\ms$的速度行驶,走到一个下坡,得到0.40$\msq$的加速度,汽车通过下坡末端的速度是16$\ms$,这个下坡的长度是多长?   

\begin{solution}
   由公式$v^2_t-v^2_0=2as$,得
   $$s=\dfrac{v^2_t-v^2_0}{2a}=\frac{16^2-12^2}{2\x 0.4}=140{\rm m}$$
\end{solution}
\item 子弹射中墙壁前的速度是400$\ms$,射到墙壁后穿进墙壁20厘米,子弹在墙内的运动可以看作匀变速运动,求子弹在墙壁内的加速度和运动时间.   

\begin{solution}
    由公式$v^2_t-v^2_0=2as$,得子弹在墙壁内的加速度
    $$a=\dfrac{v^2_t-v^2_0}{2s}=\frac{0-400^2}{2\x 0.20}=-4\x 10^5{\rm m/s^2}$$
由$s=\bar v t$,得子弹在墙壁内的运动时间
\[t=\frac{s}{\bar v}=\frac{2s}{v_0+v_t}=\frac{2\x 0.20}{400+0}=0.001{\rm s}\]
\end{solution}
\item 试证明做匀变速运动的物体在一段时间内的平均速度等于这段时间的中间时刻的即时速度.

\begin{proof}
方法一:由$s=\bar v t$和$s=v_0t+\dfrac{1}{2}at^2$, 可得
\[\bar v=\frac{s}{t}=v_0+\frac{1}{2}at =v_0+a\left(\frac{t}{2}\right)\]
即$\bar v$等于求平均速度这段时间的中间时刻的即时速度.

方法二:由$v_t=v_0+at$, 可得中间时刻的即时速度$v'$
\[v'=v_0+a\left(\frac{t}{2}\right)=v_0+\frac{1}{2}at =\frac{1}{2}(v_0+v_0+at)=\frac{1}{2}(v_0+v_t)=\bar v\]
\end{proof}
\item 做匀变速运动的物体,在各个连续相等时间$t$内的位移分别是$s_1, s_2, s_3,\ldots,s_n$.如果加速度是$a$,试证明:
\[\Delta s=s_2-s_1=s_3-s_2=\cdots=s_n-s_{n-1}=at^2 \]

\begin{solution}
    设初速度为$v_0$, 由$s=v_0t+\dfrac{1}{2}at^2$和$v_t=v_0+at$,得:
\[\begin{split}
    s_1&=v_0t+\frac{1}{2}at^2\\
    s_2&=(v_0+at)t+\frac{1}{2}at^2=v_0t+\frac{1}{2}at^2+at^2\\
    s_3&=(v_0+at+at)t+\frac{1}{2}at^2=(v_0+2at)t+\frac{1}{2}at^2=v_0t+\frac{1}{2}at^2+2at^2\\
s_{n-1}&=[v_0+(n-2)at]t+\frac{1}{2}at^2=v_0 t+\frac{1}{2}at^2+(n-2)at^2\\
s_{n}&=[v_0+(n-1)at]t+\frac{1}{2}at^2=v_0 t+\frac{1}{2}at^2+(n-1)at^2\\
\end{split}\]
$\therefore\quad \Delta s=s_2-s_1=s_3-s_2=\cdots=s_n-s_{n-1}=at^2$
\end{solution}
\end{enumerate}


\subsection{练习十}
\begin{enumerate}
	\item 为了测出井口到井里水面的深度,让一个小石块从井口落下,经过2.0秒后听到石块落到水面的声音,求井口到水面的大约深度(不考虑声音传播所用的时间).

    \begin{solution}
 由公式$s=\dfrac{1}{2}gt^2$,得: 
 \[s=\frac{1}{2}\x 9.8\x 2^2=19.6{\rm m}\]      
    \end{solution}
\item 一个自由下落的物体,到达地面的速度是39.2$\ms$,这个物体是从多高落下的?落到地面用了多长时间?

\begin{solution}
由$v^2_t=2gs$,得物体下落时的高度
\[s=\frac{v^2_t}{2g}=\frac{39.2^2}{2\x 9.8}=78.4{\rm m}\]
由$v_t=gt$,得物体落到地面所用的时间
\[t=\frac{v_t}{g}=\frac{39.2}{9.8}=4{\rm s}\]
\end{solution}
\item 一个物体从22.5米高的地方下落,到达地面时的速度是多大?下落最后1秒内的位移是多大?

\begin{solution}
    由$v^2_t=2gs$,得物体到达地面时的速度
\[v_t=\sqrt{2gs}=\sqrt{2\x 9.8\x 22.5}=21{\rm m/s}\]
由$v_t=gt$,得物体下落的时间 
\[t=\frac{v_t}{g}=\frac{21}{9.8}=2.14{\rm s}\]
前1.14秒内位移:
\[s'=\frac{1}{2}gt^2=\frac{1}{2}\x 9.8\x 1.14^2=6.4{\rm m}\]
最后1秒内位移:
\[s=22.5-6.4=16.1{\rm m}\]
\end{solution}
\end{enumerate}




\subsection{练习十一}

\begin{enumerate}
	\item 在竖直上抛运动中,$v_t$与$v_0$何时方向相同,何时相反?$v_t$与$a$何时方向相同,何时相反?

    \begin{solution}
        在上升运动中,$v_t$和$v_0$方向相同;$v_t$和$a$方向相反.
        在下降运动中,$v_t$和$v_0$方向相反;$v_t$和$a$方向相同. 
    \end{solution}
\item 竖直向上射出的箭,初速度是35$\ms$,上升的最大高度是多大?从射出到落回原地一共用多长时间?落回原地的速度是多大?

\begin{solution}
箭上升到最大高度$H$时,$v_t=0$, 由此得$v^2_0=2gH$,
所以
\[H=\frac{v^2_0}{2g}=\frac{35^2}{2\x 10}=61{\rm m}\]
由于$v_t=v_0+gt=0$, 箭的上升时间
\[t=\frac{v_0}{g}=\frac{35}{10}=3.5{\rm s}\]
由射出到落回原地共用时间$T=2t=2\x3.5=70{\rm s}$.
落回原地速度跟抛出的初速度大小相等即$35\ms$.

说明:为了计算的方便,解题时取$g=10\msq$, 以下两
题同.
\end{solution}
\item 竖直上抛的物体,初速度是30$\ms$,经过2.0秒、3.0秒、4.0秒,物体的位移分别是多大?通过的路程分别是多长?各秒末的速度分别是多大?

\begin{solution}
上升的最大高度
\[H=\frac{v^2_0}{g}=\frac{30^2}{2\x 10}=45{\rm m}\]
由$s=v_0t-\dfrac{1}{2}gt$得:
\begin{enumerate}
    \item 当$t=2.0$s时,位移
    \[s=30\ms \x2.0{\rm s}-\frac{1}{2}\x 10\msq\x(2.0{\rm s})^2=40{\rm m}<H\]
    $\therefore\quad $路程$s'=40{\rm m}$.$v_t=v_0-gt=30-10\x2.0=10\ms$.
\item 当$t=3.0$s时,位移$$s=30\ms\x3.0{\rm s}-\frac{1}{2}
\x10\msq\x(3.0{\rm s})^2=45{\rm m}=H$$
$\therefore\quad $路程$s'=45{\rm m}$.$v_t=v_0-gt=30-10\x3.0=0$.
\item 当$t=4.0$s时,位移$$s=30\ms\x4.0{\rm s}-\frac{1}{2}
\x10\msq\x(4.0{\rm s})^2=40{\rm m}<H$$
$\therefore\quad $路程$s'=45+(45-40)=50{\rm m}$.
$v_t=v_0-gt=30-10\x4.0=-10\ms$.
\end{enumerate}

\end{solution}
\item 在课文的例题中,求经过1秒后石子离地面的高度以及石子这时的速度.先分上升运动和下降运动两步来计算,再用统一的公式来计算,并加以比较.

\begin{solution}
\begin{enumerate}
    \item 石子上升时间
    \[t_1=\frac{v_0}{g}=\frac{4}{10}=0.4{\rm s}\]
    1秒内石子的下降时间$$t_2=1-0.4=0.6{\rm s}$$

    石子上升的最大高度
\[h_1=\frac{v_0^2}{2g}=\frac{4^2}{2\x 10}=0.8{\rm m}\]
0.6
秒内石子下降的高度
\[h_2=\frac{1}{2}gt^2=\frac{1}{2}\x 10\x0.6^2=1.8{\rm m}\]
石子这时的速度$v_t=gt=10\x0.6=6\ms$,方向
向下.
\item 由$s=v_0t-\dfrac{1}{2}gt^2$,
得石子在抛出1秒后的高度
\[s=4\x1-\frac{1}{2}\x10\x1^2=-1{\rm m}\]
即在抛出点下方1米处,离地面高度为$15-1=14{\rm m}$.

石子这时的速度$v_t=v_0-gt=4-10\x1=-6\ms$.
\end{enumerate}

比较:解(b)较解(a)简单,但求出的石子的速度和相对于抛
出点的位移,都是负值,必须搞清它们的物理意义,关键在于
掌握各矢量的方向.
\end{solution}
\end{enumerate}




\subsection{习题}
\begin{enumerate}
	\item 物体的加速度为零时,它的速度是否一定为零?物体的速度为零时,它的加速度是否一定为零?各举一个例子.

    \begin{solution}
        物体的加速度为零时,速度不一定为零,例如火车在
        平直轨道上匀速行驶,物体的速度为零时,它的加速度不一
        定为零,例如竖直上抛运动中当物体到达最高点时速度为零,
        加速度为$g=9.8\msq$.   
    \end{solution}
	\item 汽车以26$\kmh$的速度行驶了2小时,跟目的地还有一半路程,要想在40分钟内到达目的地,在后一半路程中汽车应该以多大速度行驶?

    \begin{solution}
        由$s_1=v_1t_1$, 得$s_1=25\x2=50{\rm km}$.
        因此,汽车在40分钟内走完另一半路程所需的速度为
    \[v_2=\frac{s_2}{t_2}=\frac{50{\rm km}}{40{\rm min}}=75\kmh\]    
    \end{solution}
	\item 矿井里的升降机,从静止开始加速上升,经过3秒速度达到3$\ms$,然后以这个速度匀速上升25秒,最后减速上升,经过2秒到达井口时,正好停下来,求矿井深度.

    \begin{solution}
        矿井的深度等于升降机各段上升高度之和,即
\[\begin{split}
     s&=v_1t_1+v_2t_2+v_3t_3\\
     &=\frac{0+3}{2}\x 3+3\x 25+\frac{3+0}{2}\x 2\\
     &=4.5+75+3=82.5{\rm m}
\end{split}\]
    \end{solution}
	\item 一架飞机以7.0$\msq$的加速度做匀加速飞行,计算它的速度由240$\kmh$增加到600$\kmh$所发生的位移和所用的时间.

    \begin{solution}
        初速$v_0=240\kmh=\dfrac{200}{3}\ms$,末速$v_t=600\kmh=\dfrac{500}{3}\ms$

由$a=\dfrac{v_t-v_0}{t}$,得所用的时间
\[t=\frac{v_t-v_0}{a}=\frac{\frac{500}{3}+\frac{200}{3}}{7.0}=14.3{\rm s}\]
这段时间内飞机的位移
\[s=vt=\frac{1}{2}\left(\frac{200}{3}+\frac{500}{3}\right)\x 14.3=1.67\x 10^3{\rm m}\]
    \end{solution}
	\item 火车制动后经过20秒停下来,在这段时间内前进120米.求火车开始制动时的速度和火车的加速度.

    \begin{solution}
由公式$s=\bar v t$和$\bar v=\dfrac{v_0+v_t}{2}$
解得开始制动时的速度
\[v_0=\frac{2s}{t}-v_t\]
由于$v_2=0$, 所以
\[v_0=\frac{2s}{t}=\frac{2\x 120}{20}=12\ms\]
火车的加速度
\[a=\frac{v_t-v_0}{t}=\frac{0-12}{20}=-0.6\msq\]        
    \end{solution}
	\item 汽车从静止开始做匀变速运动,通过一段距离,速度达到14$\ms$,汽车通过这段距离的一半时,速度是多大?

    \begin{solution}
        设汽车的加速度为$a$, 通过距离$s$获得的末速度为$v$, 通过的距离为$s$之半时,获得的速度为$v'$. 由于$v_0=0$, 所
        以有
\[\begin{split}
    v^2&=2as\\
    {v'}^2&=2a\left(\frac{s}{2}\right)=as
\end{split}\]
$\therefore\quad \dfrac{ {v'}^2}{v^2}=\dfrac{as}{2as}=\dfrac{1}{2}$,
由此得:
\[v'=\frac{\sqrt{2}}{2}v\]
已知$v=14\ms$,$\therefore\quad v'=\dfrac{\sqrt{2}}{2}\x 14=9.9\ms$
\end{solution}
	\item 一个物体从塔顶上下落,在到达地面前最后一秒内通过的位移是整个位移的9/25.求塔高.

    \begin{solution}
设塔高为$s$米,下落时间为$t$秒,因此$s=\dfrac{1}{2}gt^2$,
同理可知物体在$(t-1)$秒内落下的距离为
\[s'=\frac{1}{2}g(t-1)^2\]
由题设知
\[\frac{s'}{s}=\frac{25-9}{25}=\frac{16}{25}\]
所以
\[\frac{(t-1)^2}{t^2}=\frac{16}{25}\]
即\[\frac{t-1}{t}=\frac{4}{5}\]
解得$t=5$s.

因此塔高
\[s=\frac{1}{2}gt^2=\frac{1}{2}\x10\x25=125{\rm m}\]
    \end{solution}
	\item 自由落下的物体在某一点速度是19.6$\ms$,在另一点的速度是39.2$\ms$.求这两点间的距离和经过这段距离所用的时间.

    \begin{solution}
由公式$v^2=2gs$, 得物体速度为$19.6\ms$时下降的
高度为
\[s_1=\frac{v^2_1}{2g}=\frac{19.6^2}{2\x 9.8}=19.6{\rm m}\]
物体速度为39.2米时下降的高度为
\[s_2=\frac{v^2_2}{2g}=\frac{39.2^2}{2\x 9.8}=78.4{\rm m}\]
两点之间的距离$\Delta s=s_2-s_1=78.4-19.6=58.8{\rm m}$

由$\Delta s=\dfrac{v_1+v_2}{2}t$, 可得经过这段距离所用的时间
\[t=\frac{2\Delta s}{v_1+v_2}=\frac{2\x 58.8}{19.6+39.2}=2{\rm s}\]
    \end{solution}
	\item 一个竖直上抛的物体,经过4.0秒落回原地,经过1.0秒,2.0秒,3.0秒,物体的速度分别是多大?物体的位移分别是多大?通过的路程分别是多长?

    \begin{solution}
已知物体上抛,落回原地的时间$T=4$s,而上升时
间$t'$等于下落时间$t''$,所以$t'=t''=T/2=2$s.

由于物体上抛到最高点的速度$v_{t'}=0$, 则由公式$v_t=v0
-gt$, 得上抛的初速度$v_0=gt'=10\x2=20\ms$.

由公式$v_t=v_0-gt$和$s=v_0t-\dfrac{1}{2}gt^2$得:
\begin{enumerate}
    \item 经过1秒物体的速度
    $$v_1=20-10\x1=10\ms$$
    物体的位移
    $$s_1=20\x1-\frac{1}{2}\x 10\x1^2=15{\rm m}$$
    经过的路程也是15m.
\item 经过2秒物体的速度
\[v_2=20-10\x2=0\]
物体的位移
\[s_2=20\x 2-\frac{1}{2}\x 10\x 2^2=20{\rm m}\]
经过的路程也是20m.
\item 经过3秒物体的速度
\[v_3=20-10\x3=-10\ms\]
物体的位移
\[s_3=20\x3-\frac{1}{2}\x 10\x3^2=15{\rm m}\]
通过的路程
\[20+(20-15)=25{\rm m}\]
\end{enumerate}

    \end{solution}
	\item 气球以10$\ms$的速度匀速竖直上升,从气球上掉下一个物体,经17秒到达地面.求物体刚脱离气球时气球的高度.

    \begin{solution}
        物体从气球上掉下后到达地面时的位移为
\[\begin{split}
    s&=v_0t-\frac{1}{2}gt^2\\
    &=10\x 17-\frac{1}{2}\x 10\x 17^2\\
    &=-1275{\rm m}
\end{split}\]
所以,物体刚脱离气球时气球的高度为1275米.
    \end{solution}
	\item 初速度为零的匀变速运动,在第1秒内、第2秒内、第
	3秒内……的位移分别是$s_I,s_{II},s_{III},\ldots$.试证明:$s_I,s_{II},s_{III},\ldots$之比等于从1开始的连续奇数之比,即:
$$s_I:s_{II}:s_{III}\cdots=1:3:5\cdots$$

提示:设物体在1秒内、2秒内、3秒内……发生的位移是
$s_1,s_2,s_3,\ldots$,那么
$$s_I=s_1, s_{II}=s_2-s_1, s_{III}=s_3-s_2,\ldots$$

\begin{solution}
\[\begin{split}
    \text{1秒内位移:} & s_1=\frac{1}{2}at^2_1, \qquad t_1=1\\
    \text{2秒内位移:} & s_2=\frac{1}{2}at^2_2=\frac{1}{2}a(2t_1)^2\\
    \text{3秒内位移:} & s_3=\frac{1}{2}at^2_3=\frac{1}{2}a(3t_1)^2\\
    \vdots& \qquad \vdots
\end{split}\]
\[\begin{split}
    \text{第1秒内位移:} & s_I=s_1=\frac{1}{2}at^2_1=1\left(\frac{1}{2}at^2_1\right)\\
    \text{第2秒内位移:} & s_{II}=s_2-s_1=\frac{1}{2}a\left[(2t_1)^2-t^2_1\right]=3\left(\frac{1}{2}at^2_1\right)\\
    \text{第3秒内位移:} & s_{III}=s_3-s_2=\frac{1}{2}a\left[(3t_1)^2-(2t_1)^2\right]=5\left(\frac{1}{2}at^2_1\right)\\
    \vdots& \qquad \vdots
\end{split}\]
$\therefore\quad s_I:s_{II}:s_{III}\cdots=1:3:5\cdots$
\end{solution}
	\item 从楼顶上落下一个铅球,通过1米高的窗子用了0.1秒的时间.楼顶比窗台高多少米?

    \begin{solution}
设从楼顶到窗台的距离为$H$. 铅球从下落到通过窗
子上沿和窗台所用的时间为$t_1$和$t_2$, 即时速度分别为$v_1$和
$v_2$.

从自由落体速度公式可得$v_1=gt_1$, $v_2=gt_2$, 
设窗高为$h$($=1$米),铅球通过$h$的平均速度
\[\bar v=\frac{v_1+v_2}{2}=\frac{g(t_1+t_2)}{2}\]
从题设条件可知
\[\bar v=\frac{1}{0.1}=10\ms\]
所以
\[\frac{g}{2}(t_1+t_2)=10\ms\]
\[t_1+t_2=\frac{10\x 2}{g}=\frac{20}{10}=2{\rm s}\]
由题设条件还可知道$t_2-t_1=0.1$s,所以$t_2=1.05{\rm s}$.

因此,从楼顶到窗台的距离
\[H=\frac{1}{2}gt^2=\frac{1}{2}\x 10\x 1.05^2=5.5{\rm m}\]   \end{solution}
\end{enumerate}

\section{参考资料}
\subsection{伽利略的比萨斜塔实验}
许多著作记述了伽利略曾经做过比萨斜塔实验.

比萨是位于意大利半岛北部地区的一座古城.在流过这
座古城的阿诺河畔矗立着高56米的比萨斜塔.始建于1174
年,14世纪竣工.由于塔基问题,塔身发生了倾斜.据说,年
青的伽利略为了证明自己的论断,邀请了许多人到斜塔旁观
看他的实验.伽利略在塔上拿着两个质量相差很大而体积相
同的硬木球和铁球,让它们同时从手中自由下落,结果两个球
同时触地.于是,两千年来人们一直信奉的亚里士多德的观
点:重的物体落得快,轻的物体落得慢,终于被事实否定了,比
萨斜塔实验广为流传,比萨斜塔也随着伽利略在科学上的成
就而闻名于世.

但是,科学史家对伽利略是否在比萨斜塔上做过落体实
验持有不同的看法.从上世纪后期一直在争论着,至今仍是
悬而未决的疑案.

认为伽利略做过比萨斜塔实验的根据是维维安尼所写的
《伽利略传》(1654年出版).维维安尼是和伽利略晚年一起
生活的学生,他手中有伽利略的许多笔记和书信.另外,在伽
利略的著作中几处都提到由高塔上坠落重物的事.因此,有
人推测伽利略有可能做过比萨斜塔实验.

认为伽利略没有做过比萨斜塔实验的理由是伽利略无须
通过落体实验,只要采用逻辑推理的方法就可以否定亚里士
多德的观点.而且在伽利略的著作中找不到这个实验的记
载.在跟维维安尼同时代的其他历史资料中也没有这个实验
材料、因此,有人怀疑维维安尼有可能把别人的实验误记到伽
利略的名下,如说伽利略实验用的两个球,其中一个比另一个
重10倍,这跟比利时物理学家斯台文所做的落体实验情况
相同.

这两种看法都还缺乏充分的证据.在没有发现新的历史
资料的情况下,是难于统一认识的.

\subsection{用逐差法求加速度值}
课本练习九第6题已经证明:$$\Delta s=s_2-s_1=s_3-
s_2=\cdots=s_n-s_{n-1}=at^2$$ 

这样,在做学生实验七研究匀变速
直线运动的时候,是否可以根据相邻的距离之差$\Delta s_1,\Delta s_2,\Delta s_3,\ldots,\Delta s_{n-1}$, 分别除以$T^2$, 再取其平均值,从而得出加速
度$a$的值呢?下面看一下这个求解过程.
\[\begin{split}
    a&=\frac{\Delta s_1+\Delta s_2+\Delta s_3+\cdots+\Delta s_{n-1}}{T^2(n-1)}\\
    &=\frac{(s_2-s_1)+(s_3-s_2)+(s_4-s_3)+\cdots+(s_n-s_{n-1})}{T^2(n-1)}\\
    &=\frac{s_n-s_1}{T^2(n-1)}
\end{split}\]
可以看出,中间的各数值$s_2,s_3,\ldots, s_{n-1}$在平均过程中都已消
去,不起作用,只有首尾两个数值$s_1$和$s_n$才起作用.这样,也
就不能起到利用多个数据来减少偶然误差的作用.如果$s_1$和
$s_n$的误差很大,则求出的$a$误差也就很大了.

实际处理数据是用逐差法,把连续的数据前后对半分成
两组,将后一半的第一个数据与前一半的第一个数据相减,后
一半的第二个数据与前一半的第二个数据相减……下面我
们看一下这个求解过程.

把$s_1,s_2,s_3,\ldots,s_n$对半分为两组,每组有$m=n/2$
个数据,
前一半为$s_1,s_2,\ldots,s_m$, 后一半为$s_{m+1},s_{m+2},\ldots,s_n$, 相应的差值
是$\Delta s_1=s_{m+1}-s_1,\Delta s_2=s_{m+2}-s_2,\ldots, \Delta s_m=s_{n}-s_m$.由这些差值
求得的加速度值分别是:
\[a_1=\frac{\Delta s_1}{mT^2},\quad a_2=\frac{\Delta s_2}{mT^2},\ldots, a_m=\frac{\Delta s_m}{mT^2} \]

因此,取其平均值求得的加速度
\[\begin{split}
   a&=\frac{a_1+a_2+\cdots+a_m}{m}=\frac{\Delta s_1+\Delta s_2+\cdots+\Delta s_m}{m^2T^2}\\
   &=\frac{(s_{m+1}-s_1)+(s_{m+2}-s_2)+\cdots+(s_{n}-s_m)}{m^2T^2}\\
   &=\frac{(s_{m+1}+s_{m+2}+\cdots+s_n)-(s_1+s_2+\cdots+s_m)}{m^2T^2} 
\end{split}\]
可以看出,所有数据$s_1,s_2,\ldots,s_n$都得到了利用,因而减少了偶
然误差.

\subsection{物理学中的理想化方法、理想化模型和理想实验}

物理学研究对象受许多因素影响,但在一定条件下可以
抓住其主要因素和本质,将其他因素撇开,在此基础上进行抽
象概括,把错综复杂的问题归结为比较简单的问题进行研究,
这就是物理学研究中的理想化方法,用这种方法把研究对象
简化成的抽象模型就是物理学中的理想模型.用这种方法进
行的假想实验就是理想实验.

理想化模型在物理学研究中被广泛应用,常见的理想化
模型有:质点、刚体、弹性体、塑性体、理想气体、理想流体、弹
簧振子、单摆、点电荷、试探电荷、无限长直导线、无限大平板、
点磁荷、纯电阻(纯电感、纯电容)、光线、薄透镜、点光源等,在
各类物理书籍里,有时清楚地说明所讨论的是哪一种理想模
型(如点电荷、光线等)或者对实物附加某些说明(如“两球作
完全弹性碰撞”等).但在大多数场合,都不加说明,要我们
自己判定.不过在大多数情况下还是有一定规律可循的,如
质点力学中列举的粒子、小球、子弹、汽车、火箭、地球以至太
阳,应看作质点.在刚体力学中提到的杠杆、飞轮、圆板、皮带
轮,都看作是刚体.在流体力学中研究的空气和水,看作是理
想流体.

在用理想化方法处理问题时,考虑什么因素,舍去什么因
素不是固定不变的,随之研究对象的实际情况、研究范围和条
件的变动而变动.因此同一个物体(研究对象)可以看成不同
的理想模型,从而也会得出不同的结论.例如一个物体在不
同情况下可分别视为质点、刚体或弹性体.我们求一根搁在
墙角上的均匀铁棒里的应力时,必须先把铁棒看成是一个质
量集中于重心$O$的质点,求出重力,然后把铁棒看成刚体,利
用力的平衡条件求出各个未知力,再把铁棒看成弹性体,用材
料力学的方法求出应力.假如不这样逐步选取不同的理想模
型,这个问题是解不出的.

理想实验在物理学发展中具有极其重要的地位和意义.这是因为日常的具体事物虽然直观,但各种现象、因素、过程
交织在一起,往往掩盖事物本质的一面.运用理想实验进行
研究所揭示的特性和规律,只能以抽象的形式出现,虽然它与
具体现象远了,但离真理近了.普朗克曾说过“物理世界观之
愈益远离感性世界,无非就是与现实世界愈益接近”.例如经
典物理的鼻祖伽利略就是通过理想斜面实验揭示了惯性定律
的物理本质(课本图3.1).而爱因斯坦提出狭义相对论的基
础,同时性的相对性就是通过理想实验形象地加以说明的,此
外海森堡提出的“测不准原理”也运用了理想化实验.





