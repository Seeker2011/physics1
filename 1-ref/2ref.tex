\chapter{直线运动}
\section{教学要求}
这一章讲授的运动学知识,跟第一章一样,都是基础性
的,是后面学习动力学所必需的预备知识。

为了减少学生学习的困难,适应学生的知识水平和接受
能力,本章只讲直线运动,而把运动的合成和分解以及平抛和
斜抛的知识移到第四章的曲线运动中去讲。

通过这一章的教学,应该使学生了解一些描述物体运动
的基本概念和方法,掌握匀速直线运动和匀变速直线运动的
规律,会用这些规律来分析解决一些比较简单的实际问题。

这一章的教学要求是:
\begin{enumerate}
\item 了解参照物的概念,知道研究物体的运动要选择合适
的参照物,了解质点的概念,知道在什么情况可以把物体看
成质点。了解位移的概念,知道位移和路程的区别。
\item 明确什么是匀速直线运动,掌握匀速直线运动的公
,理解匀速直线运动的图象的物理意义。
\item 明确什么是匀变速直线运动,理解平均速度、即时速
,加速度等概念,明确知道速度和加速度的区别,掌握匀变
直线运动的公式,理解匀变速直线运动的图象的物理意义。
\item 认识自由落体运动和竖直上抛运动的特点和规律.
\end{enumerate}

下面对这一章的教学内容作些具体说明。

第一节开始先复习初中学过的机械运动和参照物的
概念,以加强与初中知识的联系,同时强调参照物的重要性,
使学生初步了解怎样选择参照物。第一节还简单介绍了平动
和转动,目的是使学生对物体运动的这两种基本形式有所认
识,后面用到时方便。教材没有给平动和转动下严格的定义,
也不要求补充讲解,只要求学生知道平动和转动的特点和
区别。

质点是力学中的一个重要概念,它是通过科学抽象得出
的理想化模型,运用理想化模型来研究问题,是物理学经常
用的方法。学生在这里初次接触这个问题,需要引起他们注
意,因此把质点单独作为一节来讲述。关于质点的定义,教材
采用了有质量的点这种说法,目的在于强调质点是物理学上
的点,不同于几何学中所说的点,在“质点”这节的最后,给出
了研究质点运动的基本线索——确定质点在任一时刻的位置
和速度,是为了使学生明确讲解本章知识的思路。

描述物体的运动,首先要懂得如何描述物体的位置和位
置变化,为要讲解位置的坐标表示和位移的概念。对位移
的坐标表示,只要求学生知道位移的数值可以用初末位置的
坐标来表示;在后面计算位移的公式中,除了平抛和斜抛外,
都不要求写出位移的坐标表示,而只写出位移本身。

匀速直线运动的知识,学生在初中已学过。这里要在复
习的基础上予以扩展和提高,为讲授匀变速直线运动作好准
备。用比值来定义物理量是物理学中常用的方法,这里用位
移和时间的比值重新定义了速度。在用比值给出速度的定义
之后,说明速度在数值上等于单位时间内位移的大小,使学生
既懂得可以用比值来定义速度,又能跟初中学过的知识联系
起来,理解其意义。

从速度的定义式$v=s/t$,
可以直接写出匀速运动的位移
公式$s=vt$. 根据这个公式就可以确定做匀速运动的物体在
任意时间内的位移,从而确定物体在任意时刻的位置。指出
这一点,可以使学生和前面提出的研究物体运动的总线索联
系起来,认识这一公式表示出了匀速运动的规律。

关于匀速直线运动的定义,教材没有明确“相等的时间”
是“任意”的。这样做是为了把问题叙述得简明一些,减少一
些过细的分析和冗长的论述。

第五节讲解匀速直线运动的图象,主要要求学生会认识
图象,知道图象的物理意义,会画简单的图象。学生刚开始学
习图象,不要求他们用图象去解决比较复杂的问题,以免增加
教学上的难度和学生的负担。教学中,应注意引导学生把数
学中学过的函数及其图象的知识运用到物理中来。这一节讲
述了用速度图象求位移,为后面用面积法推导匀变速运动的
位移公式做准备。

关于即时速度的概念,着重讲述它的物理意义。教材中
写了用数学语言可以精确地表达即时速度,只是让学生知道
个意思,不要求细讲。

为了减少学生学习加速度的概念时的困难,本章只讲匀
变速运动的加速度,不讨论一般变加速运动的情况,不引入
平均加速度和即时加速度的概念,以后各章中再逐步扩大对
加速度的认识,例如讲牛顿第二定律时,提到加速度可以改
变,即外力随时间改变,加速度就随时间而改变,讲圆周运动
时进一步认识加速度的方向的改变。

关于匀变速直线运动位移公式,教材采用了求面积的方
法推导,比较直观形象。这种通过计算面积来求物理量的方
法在物理学中经常用到,需要学生熟悉。但是,对于什么是无
限分割,限于学生的数学知识水平,不能细讲,只要求他们能
体会其意思就行了。

教材没有把匀加速运动和匀减速运动分开来讲,而是把
它统一看作匀变速运动,用统一的位移公式和速度公式来处
理。这样便于后面用统一的方法来处理竖直上抛等问题,有
助于培养学生的概括能力。

匀变速直线运动的两个基本公式皮映了匀变速直线运动
的规律并包括了前面讲过的匀速直线运动。要使学生清楚这
一点,提高他们对两个基本公式的认识。这里提到初速不为
零的匀变速直线运动,可以看作是速度为0.的匀速直线运动
和初速为零的匀变速直线运动的合运动,为了使学生理解这
几种运动之间的关系,加深对匀变速直线运动公式的认识。关
于运动合成的知识这里不宜多讲,到第四章再具体讲解。
教材通过对自由落体闪光照片的分析,得出自由落体运
动是匀变速直线运动,利用闪光照片来研究物体的运动情况,
这种方法以后还要用到。

对竖直上抛运动,教材是把它作为统一的匀变速运动来
处理的,以提高学生的理解能力和运用知识的能力。如果这
样处理有困难,也可以先把上升过程和下降过程分开来计算,
再用统一的运动来处理。讨论竖直上抛物体的上升时间、上
升的最大高度、下落时间、落地速度等,是为了使学生练习运
用匀变速运动的一般规律来分析具体问题,而不是单纯地记
住几个计算公式。

\section{教学建议}
这一章的内容可以分为四个单元:

第一单元(第1节——第3节)讲授描述物体运动的一些预
备知识。

第二单元(第4节——第5节)讲述匀速直线运动.

第三单元(第6节——第10节)讲述匀变速直线运动。

第四单元(第11节——第12节)讲述自由落体运动和竖直
上抛运动。

这一章的重点是第三单元,并通过第四单元两种常见的
实例来巩固和加深对于匀变速直线运动的认识。

\subsection{第一单元}
这一单元是在复习初中学过的机械运动有关知识的基础
上讲述参照物、质点、平动与转动、位置和位移等基本概念,内
容比较简单,但对以后的学习是重要的。

\subsubsection{运动的相对性}

使学生从理性上接受“在自然界中没
有不运动的初体”并且跟“总有许多物体停在原地不动”这种
日常经验统一起来,逻辑上自然要求提出参照物和运动的相
对性问题。为了避免引伸出去讲得很多,可以不提出“运动
的相对性”这个概念,要通过实例使学生清楚地知道:
\begin{enumerate}
    \item 一个物体对某个参照物来说是运动还是静止,要看这个物体
    对参照物来说位置是否变化;
    \item 对于相同的运动,由于选取
    的参照物不同,观测得出的结果可以是不同的;
    \item 虽然参照
    物的选取是人为的,但是在实际选参照物时,总是要使观测方
    便和使运动的描述尽可能简单,例如,在研究地面上物体的运
    动时,一般总是选取地面作参照物。
\end{enumerate}

质点是力学中的一个重要概念。在讲这个概念时,首先
要抓住“物体都具有大小和形状,在运动中物体中各点的位置
变化一般说来是各不相同的,所以要详细描述物体的运动,并
不是一件简单的事情”这一段叙述作为出发点,结合实例(例
如,改变桌子的方位)先讨论在运动中物体中各点的位置变化
各不相同的问题,然后再结合另外一些实例(远途行驶的汽
车、公转轨道上的地球,等等)讨论物体的大小和形状可以忽
略的问题,使学生知道在什么情况下可以只考虑问题的主要
方面、忽略次要因素的影响,使本来并不简单的事情得到简
化,从而便于找出它的规律来。这种使事物或问题简化、理想
化、抽象化、模型化的思想方法和研究方法,在“自由落体运
动”中以及后面还有很多内容都要接触到。

在讲质点的概念时,另一种情况,即平动的物体可以看成
质点,也要讲到,不可忽略。它也是科学的抽象。

\subsubsection{位置和位移}
 这一节一开始就提出,研究物体(质点)
的运动,首先要确定物体(质点)的位置,并可梁取建立坐标系
的方法来确定。在教学中要向学生指出,这里可以把所选定
的参照物,作为坐标系的原点以确定运动物体(质点)的位置,
并按上一章一维矢量的运算方法,来确定质点的位移。但是,
为了计算的方便,更常见的是取质点的初位置作为坐标的
原点。

这一节,要重点讲清楚“位移”这个新概念。主要是要讲
清位移是表示质点在运动过程中位置变化的物理量,它只与
运动的起点(初位置)和终点(末位置)有关,而与物体运动的
路径无关。要通过实际例子讲清楚位移不但有大小,而且有
方向,是个矢量,并把它跟初中学过的路程的概念进行对比和
区别。在讲位移和路程的区别时,要着重提醒学生,只有质点
始终向着同一个方向作直线运动时,位移的大小才等于路程。
在直线运动中,对位移的坐标表示,学生只要知道用初、末位
置的坐标来得出位移的数值,其方向则可按规定好的坐标轴
的正方向依一维矢量的运算方法得知。

\subsection{第二单元}
这一单元就它的内容-匀速直线运动来说,是学生所
熟知的,但是其中速度的概念和匀速直线运动的图象,比初中
的要求高得多,而且在方法上又要为以后学习其他内容作准
备,教学时要予以重视。

\subsubsection{匀速运动的速度}

这一节的教学在讲了什么是匀速
直线运动之后,重点应放在速度的定义、意义和矢量性上。对
中学生来说,用比值的形式来定义一个物理量,也是他们所不
熟悉的。关键在于使同学们认识位移跟时间的比值表示的就
是他们在日常生活中很熟悉的物体运动的快慢,应该从匀速
直线运动的定义出发,即物体在相等时间内的位移相等,推出
匀速运动的位移和时间的比值是一个不随时间而改变的恒
量,这个比值越大,表示物体在相同时间里的位移越大,即运
动得越快。

用位移跟时间的比值来定义速度,突出了速度概念的矢
量性。速度矢量的方向,就是物体位移的方向,根据一维矢量
的运算法则,由于匀速直线运动是沿着同一方向运动的,只要
取位移的方向为正方向,它的位移和速度就都是正值。

了解时间和时刻的区别,对于初学者来说也是很必要的。
这里,不需要在形式上给时间和时刻下什么定义,可以通过具
体的例子说明两者的区别。如通常所说的“几秒内”、“第几秒
内”都是表示一段时间,而“第几秒末”表示的就是一个时刻。
物体在运动过程中,每一时刻都有一定的对应位置,而一定的
时间,则对应于一段位移。

\subsubsection{匀速运动的图象}

物体运动的规律不仅可以用数学
公式去描述它,还可用图象的方法去描述它,而且用图象的方
法,有时更形象直观一些。这一节教学就是要引导学生把函
数图象的知识运用到物理中来。

讲解匀速直线运动的图象,主要是让学生学会认识图象,
理解图象的物理意义,会画简单的图象,学习怎样用图象来求
位移和速度。在图象的绘制上,一开始就要培养学生严格的
科学态度,一定要用直角或三角尺作图,标明横轴和纵轴所代
表的物理量及其单位,并选择适宜的标度。要让学生理解,在
匀速直线运动的速度图象中,可以用长方形“面积”的数值来
表示位移的大小,是因为长方形的一条边长在数值上等于物
体运动时间$t$的长短,另一条边长在数值上等于物体速度$v$
的大小,因此长方形的面积在数上恰好等于物体在时间内
的位移$s=vt$的大小;它的单位是“${\rm m/s\x s=m}$”,而不是
“${\rm m^2}$”.弄清楚这些知识,一方面有利于理解图象所表示的物
理意义,另一方面也有利于培养学生灵活运用数学知识解决
物理问题的能力。学生刚开始学习图象,要从数与形的关系
以及函数图象与物理量的关系,着重引导学生理解和领会图
象的物理意义及其应用,而不要补充利用图象去解二个物体
相向运动或同向运动、速度不同,何时相遇,或要求解释图
线相交点及负斜率的意义等比较复杂的问题。

\subsection{第三单元}

第一单元是本章的重点。教好这一章的关键,在于讲好即
时速度和加速度这两个重要的物理概念。从平均速度引入即
时速度,从即时速度的变化引入加速度,从加速度的定义式引
入匀变速直线运动的速度公式并进一步得出速度图象,从匀
变速直线运动的速度图象,引入匀变速运动的位移公式,再从
匀变速运动的速度公式和位移公式,得出两个有用的推论,是
这一单元教学的主要线索。

\subsubsection{即时速度}

讲述即时速度的概念,可以先让学生粗略
地了解运动物体在某一时刻(或某一位置)都有一定的速度,
这个速度就是物体的即时速度。然后再进一步讲述它的物理
意义。

运动物体在每一时刻(或每一位置)都有一定的速度这一
点,结合日常生活经验,学生是不难理解的。例如自行车和汽
车,在你面前驶过的瞬间,它们的快慢是不同的;百米赛跑,运
动员们在到达终点时的冲刺速度也各不相同,等等。这些事
例都可以说明运动物体在每一时刻(或每一位置)都有一定的
速度。学生难以理解的是即时速度的物理意义,这里主要应
该让学生理解即时速度也就是在足够短的时间里(或位移上)
运动物体的平均速度,因为在足够短的时间里,物体速度的变
化很小,已不能为测量仪器所分辨。在这样的条件下,物体的
运动在测量误差允许的范围内,可以认为是匀速的。

\subsubsection{加速度}

匀变速运动的加速度是本单元另一个重要一
的物理概念,教材在讲清楚什么是匀变速直线运动的基础上,
仿照匀速运动中定义速度的方法,用速度的变化和所用的时
间的比值来定义加速度,并说明了它的矢量性。教学中要强
调:
\begin{enumerate}
    \item 匀变速直线运动的速度在不断改变,而加速度是保持
不变的;
\item 加速度既是矢量,就要遵循一维矢量的运算方法,
根据事先选定的运动的正方向,用带有正负号的数值来表示
它,一般取初速度$v_0$的方向为正方向。
\end{enumerate}
因此,当$v_t>v_0$时$a=\frac{v_t-v_0}{t}$一
定是正值,表示$a$与$v_0$的方向相同;当$v_t<v_0$时,$a$
一定是负值,表示$a$与$v_0$的方向相反。

学生初学时对速度、速度的变化和加速度这几个物理量
在概念上往往混淆不清,教学中应该注意澄清。《速度和加速
度的区别》这段阅读材料,有助于弄清这些问题,应该引导学
生认真阅读,并提出问题来了解、检查学生阅读后理解的情
况。例如可以提出:在平直路面上行驶的汽车,在离开车站
和即将靠站时,汽车的加速度的方向是相同的吗?速度的方向
是相同的吗?汽车在匀加速行驶时,速度对时间的变化率以及
位移对时间的变化率都是不变的吗?等等。教师还应该引导
学生通过一些具体问题的讨论,认识加速度是表示速度变化
快慢的物理量,跟速度的大小没有直接关系。速度大的物体,
加速度不一定大;速度小的物体加速度不一定小。另外,速度
变化的大小,不仅与加速度的大小有关,还跟加速的时间长短
有关。速度变化大的物体,加速度也不一定大,速度变化小的
物体,加速度也不一定小。

\subsubsection{速度公式和速度图象}
在导出匀变速运动的速度公
式时,可先从加速度$a=\frac{v_t-v_0}{t}$
,过渡到速度的变化$v_t-v_0=at$, 然后得出速度公式$v_t=v_0+at$, 这样既可分步分别阐
明和强调公式的物理意义,并可引导学生在理解概念区别的
基础上,进一步认识匀变速运动的速度和加速度的联系。课本
中不把匀加速运动跟匀减速运动分开来处理,而统一为匀变
速运动,只是做匀减速运动时,加速度为负值,这样可以避免
把公式搞得太多,把问题看得太绝对,造成机械的记忆和硬套
公式,加速度的方向也似乎可以不考虑了。

根据速度公式$v_t=v_0+at$, 可以画出匀变速运动的速度
图象,要引导学生弄清楚:图线通过原点和不通过原点的物
理意义,纵轴上的截距和图线的斜率的物理意义,以及斜率的
大小和正负的物理意义等。把这些基本内容掌握了,便有利
于学生的认图、用图与作图。概括地说,这一节教学中,对公式
的导出,不要简单地处理为数学公式的变换,对图象教学不要
单纯地处理成数学上数与形的关系,仅仅作出公式的函数图
象。要强调公式,图象的特点及其变化所表示的物理意义。

\subsubsection{位移公式}

在引用匀速运动的速度图线和横轴之间
的面积表示位移这种方法来求匀变速运动的位移时,要讲一
讲为什么将时间轴尽量加以分割,使折线下的面积,尽量逼近
速度图线和横轴之间的面积,从而可以用它来表示匀变速运
动的位移。让学生熟悉和体会这种方法,对于培养学生的科
学思维能力是有好处的。但是限于学生的数学水平,也不宜
做过细的分析。得出位移公式以后,应通过例题说明加速度
矢量的方向如何根据一维矢的规定来表示,这也是要求学
生熟悉的。

为了使学生学会用不同的方法来解题,并在此基础上选
择简便的方法来求解,可举一些已知条件不同的题目,由同学
讨论、分析、判断、比,让他们自己去体会,不宜由教师归纳
为几条,以免代替和抑制学生的思维。但教师还应该做必要
的指导,例如为了培养良好的习惯,解题时要先弄清楚题目中
所描述的整个运动过程;对复杂的问题,要分步考虑,一步步
地分析出所求的未知量和已知量之间的关系,而这个“关系”
从数学上说是公式,从物理上说就是运动规律。在运用时要
思考它是否符合这个规律,在熟悉分步运算的同时,要引导
学生逐步学会列出方程,利用文字运算来解题,有的题目,要
注意一题多解,以提高学生掌握物理规律和分析物理问题的
能力。

\subsection{第四单元}
这一单元是用匀变速运动的知识来研究自由落体和竖直
上抛这两种常见的运动,认识这两种运动的特点和规律。这
也是对匀变速直线运动基本规律的应用和巩固。

\subsubsection{自由落体运动}

这一节教学应依次掌握好下列各个
环节:
\begin{enumerate}
\item 做好课本图2.20毛钱管演示实验,以表明在管中
空气被抽出后,重量不同的物体下落的快慢相同.
\item 把管中
空气抽出后,如果忽略余下的稀薄气体的作用,就可以近似地
看成是没有空气的空间;在没有空气的空间里,物体下落时才
是“只受重力的作用”;因此,自由落体运动是理想化的运动模
型.
\item 让同学实际测量课本图2.21频闪照片中小球在各个
相等时间里的位移,以鉴别小球自由落下时是作什么运动。有
频闪设备的学校,可以根据自己实际拍摄的频闪照片分组测
量数据;没有频闪设备的学校,也可用其他实验来代替。在得
出:“自由落体运动是初速度为零的匀变速运动”的结论以后,
紧接着要做出的第二个结论便是:不同的自由落体,它们的
运动情况相同,也就是在同一地点,一切物体在自由落体运动
中的加速度(重力加速度$g$)都相同,
\item 既然一切物体在自由
落体运动中的加速度都相同,它必然是一定值,我们可通过实
验来测定它,接着就让学生根据课本88页给出的数据表去计
算$\Delta s$和$\Delta s$的平均值,逐步引导学生重视数据处理,培养这
方面的能力.   
 \item 引导学生认真阅读《伽利略对自由落体运动
的研究》,以培养自学阅读能力和逻辑思维能力,从中学习用
外推法研究物理现象和规律的思路和方法,并且还可以使学
生获得一些物理学史的知识。
\end{enumerate}


\subsubsection{竖直上抛运动}

竖直上抛运动一直有这样两种处理
方法:一是把运动分为匀减速上升和自由下落两个过程,分
开来进行计算;另一种是把它看成是向上的匀速运动和向下
的自由落体运动这两个分运动的合成运动,前一种方法比较
直观,但是在运算时比较繁也容易错;后一种方法比较品象,
如果不进一步具体分析,就只能从运动公式的形式上来说明
两项分别表示两个分运动,学生也不容易体会。

课本在这一章不过分强调运动的合成,也避免学生不容
易理解的运动的独立性,把整个上抛运动看成是一个统一的
匀变速直线运动,这样既体现了课本中不把匀加速运动和匀
减速运动分开来的处理方法,在理解上也并不困难。而另一种
方法,把上升运动和下降运动分为两步来计算,留给同学自己
去尝试,也是可取的,在这方面的要求,对程度不同的学生可
以因材施教。

上抛运动的位移和速度的方向,仍应强调按一维矢量的
统一规定:对速度矢量来说,是以上抛运动初速度$v_0$的方向
为正方向;位移矢量是以初位置(抛出点)为原点,以初速度的
方向为正方向,物体位于抛出点下方时,位移方向向下,位移
是负值;加速度矢量也是以初速度的方向为正方向,而由于重
力加速度的方向跟初速度的方向相反,因此重力加速度$g$总
是取负值。当$g$取绝对值时,上抛运动的公式即可写成:
\[v_t=v_0-gt,\qquad s=v_0t-\frac{1}{2}gt^2\]
这样把各矢量的方向规定一并
弄清楚,就不致于混淆。

在应用上抛运动的公式讨论几个具体问题时,要引导学
生掌握它们的特征,例如物体上升到最大高度时,特征是到
达最高点时即时速度为零,据此可以很容易算出物体上升的
时间和上升的最大高度;物体落回到初位置时的特征是位移
为零,据此可以很容易得出落回原地的时间和物体着地时的
速度。引导同学理解和掌握这些特征,并不是要同学记忆几
条,而是要通过分析和练习让同学去领会。

在解题计算的过程中,还要引导同学理解方程同时有两
个解的物理意义。

\section{实验指导}
\subsection{演示实验}
\subsubsection{观察匀速直线运动和匀变速直线运动}

利用节拍器、斜面(长约1.50米)和小车观察匀速直线运
动和匀变速直线运动的实验装置如图2.1所示.事先调节好
斜面的倾斜程度,使得小车恰能沿斜面匀速下滑,做好垫木位
置的记号,然后再将垫木移右些,使小车下滑时做加速运动。
\begin{figure}[htp]
    \centering
    \includegraphics[scale=.8]{fig/2-1.png}
    \caption{}
\end{figure}

打开节拍器,当听到节拍器发出一个信号时,立即释放小
车,使它自某一固定位置$O$下滑.用一木块阻挡小车,调整阻
挡木块的位置,重复几次实验,使得小车撞击木块时发出的
声音恰巧和节拍器发出的第二个信号(以小车开始释放时的
信号作为第一个信号)重合。在阻挡木块的这一位置($A$)上,
用事先准备好的箭头标出(用胶纸把箭头贴在斜面的侧边)。
用同样的方法来确定小车和木块的撞击声恰和节拍器发出的
第三个信号、第四个信号重合时的木块位置$B$和$C$, 并分别用
箭头标出。用米尺量度$OA$、$AB$和$BC$的长度,发现$OA<AB
<BC$, 然后将垫木移到事先准备好的位置,重做实验,直到测
得$OA=AB=BC$. 这表明小车在相等时间里的位移都相等,
所以小车的运动是匀速运动。

在演示匀变速直线运动时,则可以将垫木放在事先调整
好的另一位置上,用上述方法观察小车在相等时间内经过了
不相等的距离。通过调节节拍器的频率,使得小车从静止开
始释放在各相等时间里发生的位移之比$OA:AB:BC=1:3:5$
(譬如可调节到使$OA=16{\rm cm}$, $AB=48{\rm cm}$, $BC=90{\rm cm}$),在
这基础上还可进一步得出$AB-OA=BC-AB$, 即匀变速直线
运动中,在连续相等时间内的位移差是一常数。

\subsubsection{测量匀变速直线运动的即时速度}

\begin{figure}[htp]
    \centering
    \includegraphics[scale=.8]{fig/2-2.png}
    \caption{}
\end{figure}


可利用斜面、小车和节拍器采用上述的实验方法,测
出小车在各连续相等时间内的位移$OA=s_1$, $AB=s_2$, $BC=s_3$
(图2.2)。节拍器发出信号的时间间隔为$T$, 根据匀变速直
线运动公式可知:
\begin{align}
    s_1&=\frac{1}{2}aT^2\\
    s_2&=v_AT+\frac{1}{2}aT^2
\end{align}

将(2.1)、(2.2)式相加,$s_1+s_2=v_A T+aT^2$.

$\because\quad v_A=aT$

$\therefore\quad s_1+s_2=v_AT+v_AT=2v_AT,\qquad v_A=\dfrac{s_1+s_2}{2T}$

也$\dfrac{s_1+s_2}{2T}$就等于小车开始运动$2T$时间内的平均速
度,所以匀变速直线运动中某一段时间的中间时刻的即时速
度就等于在这一段时间内的平均速度。

同理,可以测出当小车经过位置$B$时的即时速度$v_B=\dfrac{s_2+s_3}{2T}$

利用打点计时器来测量。
\begin{figure}[htp]
    \centering
\includegraphics[scale=.8]{fig/2-3.png}
    \caption{}
\end{figure}
如图2.3所示,在一端装有定滑轮的长木板上,放一条有
细绳的小车,通过定滑轮在细绳的另一端挂有几个钩码,固定
在小车后面的纸带和打点计时器连在一起。接通电源待打点
计时器正常工作后,释放小车。取下纸带后请一位学生选定
连续的几个计数点(可用每打五次点的时间作为时间的单
位),并要求学生毫米刻度尺测量出各相邻计数点间的距离
$s_1,s_2,s_3,s_4,\ldots$, 教师可将纸带以及选定的计数点放大后画
在黑板上,并将学生实际测得的数据标出,用跟前述相同的方
法来处理数据,求出打点计时器打下各计数点时小车的即时
速度。

\subsubsection{测量匀变速直线运动的加速度}
可采用测量匀变速直线运动的即时速度时相同的实
验装置,取得数据,然后根据匀变速直线运动$\Delta s=aT^2$来求
加速度。$a=\dfrac{\Delta s}{T^2}$

用上述装置取得数据,算出打点计时器在打下各计
数点时小车的即时速度后,然后用画$v$-$t$图象的方法求出图
线的斜率,从而得出加速度$a=\dfrac{\Delta v}{\Delta t}$

算出打点计时器在打下各计数点时小车的即时速度
后,可以任取几段不同的时间及其相应的初速度和末速度的
数据,根据加速度的定义式$a=\dfrac{v_t-v_0}{t}$
,来分别求出这几段时
间内的加速度$a_1,a_2,a_3,\ldots$, 然后再求这些加速度的平均值。

\subsubsection{空气阻力对落体运动的影响}
准备一架调节好的托盘天平,先将一个乒乓球和一个小
铁球放在托盘天平上比较它们的重量,可看到小铁球比较重。
把这两个球效在同一高度上同时下落,则铁球先落地。又把乒
乓球和一块较大的泡沫塑料平板放在天平上比较它们的重
量,可看到泡沫塑料板较重,把它们放在同一高度上同时下
落,则乒乓球先落地。再把一张纸裁成两半,把其中的一半揉
成纸团和另一半放在天平上称,它们是等重的,使它们从同一
高度同时下落,结果揉成纸团的那一半先落地。

这个演示说明了,比较重的物体可以先落地也可以后落
地,即使等重的物体落地的时间也有先后。因此使得物体落
地的时间有先后的原因不是由于重力的大小而是由于空气阻
力大小的影响。受到空气阻力大的物体总是后落地。

\subsubsection{在空气阻力很小时,不同物体同时落下}
这可以用课本图2.20所示牛顿管(又称毛钱管)的传统
实验来进行。演示时可以先不抽空气,当把管子迅速倒转来
时,金属片很快下落,羽毛则下落较慢。然后抽气(抽气要用
管壁很厚的橡皮管),抽气后再演示,发现羽毛和金属片同时
落到管子的底部,最后再将空气放入管中,则羽毛又比金属片
下落得慢。这证明了:在空气阻力很小时,一切物体在同一高
度上的落地时间都是相等的。

\subsubsection{研究自由落体的闪光照片}
课本图2.21自由落体的闪光照片表明了自由落体运动
是初速度为零的匀变速直线运动。关于这幅照片,要求学生
理解以下几点:
\begin{enumerate}
\item 这不是许多个小球,而是表明一个自由下落的小球
在经过各个相等时间
($1/30$秒)时的位置.
\item 从每隔相等时间来看,小球下落的距离越来越大,说
明小球是作变速运动。
\item 照片上小球最初几个位置比较密集,因此可选择某
一个间距较大的位置作为位置1开始测量.小球的位置都取
小球球心(也可以取小球的上缘或下缘),这样来量度相邻两
个位置间的距离$s_1,s_2,s_3,\ldots$, 再算出相邻的相等时间内的
距离之差$\Delta s_1=s_2-s_1,\Delta s_2=s_3-s_2,\Delta s_3=s_4-s_3,\ldots$指
导学生阅读课本88页的数据表,发现$\Delta s$基本上都是接近
的,因此可以证明自由落体运动是初速度为零的匀变速直线
运动。
\item 要注意课本数据表中的数据是根据照片中的刻度尺
读取的,而不是在照片上用毫米刻度尺测量的。
\item 从数据表所列的数据可以计算出自由落体运动的加
速度(即重力加速度$g$)的数值。
\end{enumerate}

\subsubsection{利用打点计时器来研究自由落体运动}
可以按照课本图10.16,利用铁架台把打点计时
器固定起来,用手提住夹有重物的纸带,接通电源,当打点计
时器正常工作后,松开纸带,让重物拖着纸带自由下落。对纸
带上记录的点的分布情况进行分析,可以证明自由落体运动
是初速度为零的匀变速直线运动,而且可以求出重力加速度
$g$的数值。

这个实验也可以让全体学生自己做,这样可以增加练习
使用打点计时器的次数,并再一次练习对实验数据的分析和
处理。


\subsection{学生实验}
\subsubsection{练习使用打点计时器}
实验前要首先弄清楚所使用的打点计时器需要多大
的工作电压,打点的时间间隔是多少。

用手拉动纸带时,速度不要过小,要水平,直到全部
把纸带拉出,这样,即可观察到纸带上被打下的一系列点。

从纸带上能看得清的某个点数起,数一数纸带上共
有多少个点,计算一下在这段距离内纸带运动的时间$t$是多少
秒?要注意如果共有几个点,已知每两个点间经过的时间是
0.02秒,则运动的总时间$t=(n-1)\x0.02$秒.


\subsubsection{研究匀变速直线运动}
这个实验对于数据处理的要求较高,内容较多,要用
两课时完成。实验的具体要求是:
\begin{enumerate}
\item 从分析纸带上的点的分布来判断小车是否做匀变速
直线运动。
\item 在确认小车是做匀变速直线运动的前提下,利用纸带
上的数据来计算出小车在各个时刻的即时速度。
\item 通过画出速度-时间图象,来计算小车做匀变速直线
运动的加速度。
\end{enumerate}

按课本图10.9的装置把实验器材装好,先不要接通
交流电源,用手挡住小车,在细绳的一端挂上三个50克的钩
码。释放后,观察小车运动时拖着的纸带通过打点计时器限
位孔的位置是否恰当。适当调整并重新固定打点计时器的位
置使得限位孔正对着小车的运动方向,然后把纸带穿好,接通
电源,待打点计时器正常工作后释放小车。

取下纸带,观察纸带上的点,会发现开始时的几个点
很密集,为了减小测量误差,可从间距较大的点(譬如相距
几个毫米)开始进行测量,选定连续的几个计数点(不少于五
个),并要求学生参照课本图10.10, 将所选定的计数点标出
$A,B,C,D,E,\ldots$, 量出各计数点间的间距$s_1,s_2,s_3,s_4,\ldots$也
标在纸带上(图2.4),以此作为原始数据记录.
\begin{figure}[htp]
    \centering
    \includegraphics{fig/2-4.png}
    \caption{}
\end{figure}

为了测量方便,可以用每打五次点的时间作为时间的单
位,这样,在两个计数点间的时间间隔$T=5\x0.02=0.1$秒。

根据课本练习九第6题,可得
\[\begin{split}
    \Delta s_1&=s_2-s_1=aT^2\\
    \Delta s_2&=s_3-s_2=aT^2\\
    \Delta s_3&=s_4-s_3=aT^2
\end{split}\]

在匀变速直线运动中,加速度$a$是恒量,因此通过对纸带
上各个相邻计数点间距离的测量,算出各相邻计数点间的距
离之差$\Delta s$均相等,则可证明小车的运动是匀变速直线
运动。

怎样求出打点计时器在打下各计数点时小车的即时
速度?

在确认小车是做匀变速直线运动的前提下,可以利用速
度图象来求小车的加速度,这就首先需要求出小车从开始计
时(即打点计时器打下$A$点时)起,经过$T,2T,3T,\ldots$也就是
打点计时器打下$B,C,D,\ldots$各点时的即时速度。
\[v_1=\frac{s_1+s_2}{2T},\quad v_2=\frac{s_2+s_3}{2T},\quad v_3=\frac{s_3+s_4}{2T}\]

要注意:从课本图10.10所示纸带上所选定的这几个计
数点,应用上述方法只能测得即时速度$v_1$、$v_2$和$v_3$, 若要测出
打下$E$点时小车的即时速度,则必须在纸带上再确定经过时
间为$4T$时的计数点$F$, 测出$EF$间的距离$s_5$, 则$v_4=\dfrac{s_4+s_5}{2T}$。
同理,若要测出打下$A$点时小车的即时速度,则必须在纸带
上的$A$点之前再确定一个计数点$O$, 然后测量$O$和$A$点间的
距离$s_0$, 则$v_A=\dfrac{s_0+s_1}{2T}$.
要注意这个$v_0$并不等于零,它表示
在实验中开始计时时刻的初速度。

求出打点计时器在打下各计数点时的即时速度后,
就可设计一个能表示时间和其对应的即时速度数值的数据表
格。在坐标纸上建立一个平面直角坐标系,用横坐标表示时
间,用纵坐标表示速度,然后在坐标平面上标出$(T,v_1),(2T,
v_2),(3T,v_3),\ldots$各数据点,数据点不得少于五个。把这些点
连结起来可以画出一条直线,画直线时应尽量使多数的点落
在这条直线上,不在直线上的各点,应使它们比较均匀地分布
在直线的两旁,这就是在这条直线两侧的点数以及这些点到
直线的平均距离应大致相等,这就得出小车的速度图线。

\begin{figure}[htp]
    \centering
    \includegraphics[scale=.8]{fig/2-5.png}
    \caption{}
\end{figure}

求出速度图线的斜率就可以得出小车的加速度.如
果画出的$v$-$t$图象如图2.5所示,应该怎样求出这条直线的
斜率呢?要从图线上选取相隔较远的两个点,如$P$和$Q$,分
别从图象上读出它们的坐标$(t_P,v_P)$和$(t_Q,v_Q)$, 即可求得加
速度
\[a=\frac{\Delta v}{\Delta t}=\frac{v_Q-v_P}{t_Q-t_P}\]

可启发学生思考以下问题:
\begin{enumerate}
    \item 如果不用速度图象来求小车的加速度,是否还有其他
的方法?
\item 实验中为什么不直接用$a=\dfrac{v_5-v_1}{4T}$
来求加速度,而要
在图线上另找$P$、$Q$两点通过求图线的斜率来得出加速度,这
样做有什么好处?
\end{enumerate}

这个实验的课时安排,可以在第一课时内完成实验
的准备、使用打点计时器打出纸带以及分析纸带上点的分布、
确定计数点、测量数据判断小车是否做匀变速直线运动等内
容。求出打点计时器在打下各计数点时小车的速度、画出$v$-$t$
图象,求得小车的加速度等内容可安排在第二课时完成。

\subsection{课外实验活动}
\subsubsection{滴水法测重力加速度}
这个实验的原理是根据自由落体运动是初速度为零
的匀变速直线运动,水滴下落的距离$h$跟运动时间$t$的平
方成正比$h=\frac{1}{2}gt^2$, 则 
\[g=\frac{2h}{t^2}\]

因此只要测出水滴下落的距离和下落的时间,便可测得
重力加速度。

这个实验中的水滴下落距离是易于测量的,比较困
难的是时间的测定。对于课本介绍的测时间的方法,要多次
耐心地调整阀门(自来水笼头)的大小,才能使水滴从阀门落
到盘子经过的时间正好等于阀门滴下水滴的时间间隔。为了
便子调整,盘子可以倒过来放在水槽里(或者用大口瓶上的金
属盖代替盘子,使盖子的顶部朝上)。由于盘子下面跟水槽间
有一空腔,使得水滴在盘子上的响声比较清脆,阀门离盘
子的距离不要太近,太近了不容易区别两次滴水的时间间隔。
(譬如水滴下落的距离约为0.5米左右时,半分钟里约有
90—100个水滴从阀门滴下,这样,水滴下落的时间就正好等
于相继滴下的两个水滴之间的时间间隔)。

用这一方法测定的重力加速度的数值是近似的,但
实验方法比较巧妙而且简单。

\subsubsection{用秒表测量玩具手枪子弹射出的速度}
这个实验的原理是,以初速$v_0$、竖直上抛的物体,从
开始抛出直到落回抛出点所经过的总时间$t=2v_0/g$,
只要测出
玩具手枪的子弹从发射到落回发射点的时间$t$, 当地的$g$值
可以由教师给出,即可测出玩具手枪子弹射出时的初速度
$v_0=\frac{1}{2}gt$。

由于实验条件的限制,所测得的子弹初速度是近似
的。为了能使子弹基本上做竖直上抛运动,可以设法使玩具
手枪固定起来(譬如可将手枪缚在一张方木凳的边上,使枪口
和凳面相平,并使枪管竖直向上),试着先发射一发子弹调节
枪管的位置,使子弹不做明显的斜抛运动就可以了。实验时
可以在手枪的另一侧再放一个相同高度的木凳,从扳动手枪
扳机发射子弹的同时开始计时,当子弹落到木凳时再按下秒
表,测出子弹做竖直上抛运动的总时间$t$. 如果没有秒表,也
可以用手表近似地计时。

\section{习题解答}
\subsection{练习一}
\begin{enumerate}
    \item 两辆在公路上直线行驶的汽车,它们的距离保持不
变,试说明用什么样的物体做参照物,两辆汽车都是静止的,
用什么样的物体做参照物,两辆汽车都是运动的。能否找到
这样一个参照物,一辆汽车对它是静止的,另一辆汽车对它是
运动的?为什么?


\begin{solution}
用其中任意一辆汽车里的座椅做参照物,两辆汽车
都是静止的;用车外公路旁的树木、房屋做参照物,两辆汽车
都是运动的。因为两辆车的距离不变,它们保持相对静止,所
以不可能找到一个参照物,一辆车对它是静止的,而另一辆车
对它却是运动的。
\end{solution}

\item 小孩从滑梯上滑下,钢球沿斜槽滚下,石块从手中落
下,这些物体中哪些是做平动的?

\begin{solution}
根据运动过程中物体各部分的运动是否完全相同来
判断:小孩从滑梯上滑下是平动,钢球沿斜槽滚下不是平动;
石块从手中落下时如果没有翻转则也是平动。
\end{solution}

\item 研究自行车轮的转动,能不能把自行车当作质点?研
究在马路上行驶的自行车的度,能不能把自行车当作质点?

\begin{solution}
    研究自行车轮的转动时,不能把自行车当作质点;研
    究自行车的行驶速度时,可以把它当作质点。
\end{solution}
\end{enumerate}


\subsection{练习二}
\begin{enumerate}
    \item 质点做什么运动,位移的大小才等于路程?
    
\begin{solution}
    质点始终向同一方向做直线运动时,位移的大小等
于路程。
\end{solution}
    \item 课本图2.6表示做直线运动的质点从初位置
$A$经过$B$运动到$C$, 然后从$C$返回,运动到末位置$B$, 设$AB$
长7米,$BC$长5米.求质点的位移的大小和路程。

\begin{solution}
    如图所示,初位置为$A$, 末位置为$B$, 所以位移的大
小为7米($AB$长).
\[\text{路程}=AB+BC+CB=7+5+5=17{\rm m}\]
\end{solution}
\item 在课本图2.4中汽车初位置的坐标是$-2$千
米,末位置的坐标是1千米.求汽车的位移的大小和方向。

\begin{solution}
    汽车的位移$s=1-(-2)=3$千米,由于
位移为正值,方向跟坐标轴正方向一致,即由西向东。
\end{solution}
\end{enumerate}


