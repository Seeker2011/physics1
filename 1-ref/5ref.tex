\chapter{万有引力定律}\minitoc[n]
\section{教学要求}
万有引力定律的发现,是人类在认识自然规律方面取得
的一个重大成果,对人类文化历史的发展有重要意义.万有
引力定律在研究天体的运动和人造地球卫星等方面有着重要
的应用.鉴于这一规律的重要,把它单独列为一章,使内容集
中,中心突出.

这一章的教学要求是:
\begin{enumerate}
\item 了解开普勒三定律,掌握万有引力定律.
\item 了解万有引力定律在天文学上的应用,了解地球上物
体的重量变化的原因.
\item 了解有关人造卫星的知识,会推导第一宇宙速度.
\end{enumerate}

下面对这一章的教学内容作些具体说明.

为讲解万有引力定律的建立作准备,第一节先介绍行星
的运动,关于人类对行星运动规律的认识过程,只要求学生
了解个梗概,知道开普勒三定律是在前人长期观察研究的基
础上总结出来的,这一节的重点是讲解开普勒三定律,使学
生对定律的内容有所了解.学生在学习本章时,还不具备椭
圆的知识,教学中需要对椭圆的焦点、半长轴作简单的介绍.

万有引力定律的教学,主要是让学生知道牛顿如何在开
普勒三定律的基础上推导出万有引力定律的思路.在介绍牛
顿建立万有引力定律之前,提到了胡克等人猜想到引力与距
离的平方成反比,是为了说明万有引力定律的建立经历了一
个过程,不是只靠个别天才人物的灵感创造的.引力与太阳
质量的正比关系,可以直接给出,不要求作进一步的讨论.地
球对月球的吸引力与地面物体所受的重力是同一种性质的
力,让学生自己通过练习计算得出,以获得较深刻的印象.如
果学生对这个题目的推理过程不很理解,也可以作些必要的
引导和说明.

万有引力定律揭示了支配天体运动的规律,把地上的运
动和天上的运动统一起来,打破了以往人们对天体运动的神
秘感,增强了人们认识自然的信心,讲述万有引力定律,应该
使学生对此有所认识.

卡文迪许实验是历史上的著名实验,它测定了万有引力
恒量的值.鉴于这个实验的重要,单独作为一节来讲述.这
个实验,不要求演示,通过介绍这个实验,使学生认识这个实
验的重要作用,领会前人是怎样进行巧妙的设计来测出万有
引力恒量值的,启发他们进一步认识培养和训练灵活运用知
识能力的重要性.

列举几个物体间引力大小的例子,是为了说明一般物体
间的引力非常小,而天体之间的引力非常大.正是这个巨大
的引力支配着天体的运动.因而万有引力定律主要用于研究
天体的运动.

天体质量的计算,说明应用万有引力定律和圆周运动的
知识,可以确定无法直接测定的天体质量.在天文学上,太
阳、地球等天体的质量就是根据行星或卫星的轨道半径和周
期来求得的.海王星、冥王星的发现,说明万有引力定律不仅
能对观察到的天体运动作出解释,而且能预言尚未观察到的
天体的存在.这是理论指导实践的典型事例.

关于人造地球卫星的教学,重点是讲述发射人造卫星的
原理,得出第一宇宙速度后,要指出卫星进入轨道的水平速
度大于7.9${\rm km/s}$,小于11.2${\rm km/s}$时,卫星绕地球运动的
轨道将不是圆,而是椭圆,进而说明速度增大到11.2${\rm km/s}$
后,不再绕地球运行,而成为围绕太阳运动的一颗行星.至于
轨道为什么会成为椭圆,限于学生的知识水平,中学阶段不能
讲解.对于第二宇宙速度和第三宇宙速度,只要求简单介绍,
使学生知道它们的意义就行了.

这一章的习题大多是综合性的,灵活性也有所提高,要
注意向学生讲明解决这类问题的思路,以培养他们灵活运用
知识,逐步提高解题能力.

\section{教学建议}
本章内容是按照人类对万有引力定律的认识过程,围绕
运动和力的关系而逐步展开的.

学习本章内容是对前几章知识的综合和提高,在教学中
应着重培养学生综合运用新旧知识对问题进行推理和分析的
能力.此外,引导学生体会建立万有引力定律程中所体现
的科学方法,以及激发学生对未知世界的探索精神和科学的
想象力,也是本章教学中不可忽视的方面.

本章分为两个单元,第一单元包括第一至第三节,概括
地介绍了万有引力定律建立的历史进程,其中包括对引力恒
量的测定.第二单元包括第四至第六节,介绍了万有引力定
律的某些应用.

\subsection{第一单元}

万有引力定律的建立过程,对于已确立的定律、新的假
说、理论推导和实验观测之间如何相互影响和补充,提供了一
个很好的范例.因此本单元的教学可以按产生这一定律的历
史背景、定律的建立和定律的实验验证这三个层次来展开,这
样可以使学生对这一理论获得一个整体的认识,从而体会到
具有突破性的重大物理理论的建立,并不是偶然的,它反映了
人类对自然界的认识不断深化和完善的过程.

\subsubsection{开普勒定律}

第一节的重点是介绍开普勒三定律,讲
述时可指出,开普勒定律是一种描述性的经验定律.开普勒
定律描述了行星运动的规律,但没有提出和解决行星为什么
这样运动的问题,这个重要的问题是牛顿在他的运动定律的
基础上解决的.

鉴于学生在学习本节时还没学过椭圆知识,因此可结合
课本图5.2作简单解释.教学中应当指出该图表示的行星椭
圆轨道是一个十分夸张的示意图,事实上大部分行星的椭圆
轨道都十分接近于圆形,因此可对开普勒第一、第二定律作近
似处理,即认为行星以太阳为圆心作匀速圆周运动,而第三定
律中椭圆的半长轴可以当作圆形轨道的半径$R$. 应使学生明
确本章所有对天体运动的分析计算都是在上述近似处理的基
础上用匀速圆周运动的动力学方法进行的.

在说明开普勒第三定律中$k$值是一个与行星无关的恒量
时,可指出$k$值只与行星所环绕的那个天体有关,至于为什么
会这样,以后将会作进一步深入讨论.最后可将练习一的第
1题作为课堂练习,让不同小组的同学分别算出各行星的$k$
值加以比较.通过这一练习学生对太阳系中的$k$值与各行星
无关便有了具体认识.对计算结果中$k$值的差异可简单指出
这是由于表中原始数据不太精确(仅三位有效数字),而且开
普勒第三定律本身也是近似的定律.(见参考资料2)

\subsubsection{万有引力定律的建立}

第二节是全章的重点也是教
学中的一个难点.通过本节学习要使学生认识牛顿所建立的
万有引力定律不仅解决了行星运动的起因,而且揭示了自然
界物体间普遍存在的一种基本相互作用.为此,教学中可围
绕地和天统一这个中心突出点:第一,牛顿如何将天体运
动规律(开普勒第三定律)和在地球上得出的力学规律联系起
来,进行演绎,从而导出平方反比定律的.分析中只需指出引
力还与太阳质量$M$成正比这一结论,不必对常数$k$和太阳质
量$M$的关系作进一步讨论.第二,牛顿如何推广平方反比定
律,将天体间的引力和地面上的重力统一起来,使之成为一条
字宙万物间的普适物理定律.教材中对后一点的陈述较简
练,并在练习二中设计了一道题,引导学生通过推导和计算来
理解这段陈述.教学中可将此题(练习二4)作为课堂练习
导学生边练习边分析.练习时,可根据保持月球在其轨道
上运动的力也就是把地面上的物体放在那个位置所受到的重
力这一思路,画出示意图来帮助学生分析.

在平方反比定律的推广中,要将$g$和$a_R$加以比较,把两
种不同运动形式的加速度联系在一起,认为它们出自同一性
质的力,学生往往感到不易理解,这主要还是由于学生仍习
惯于从运动表现形式上来比较物体的受力情况,错误地认为
物体运动形式不相同,它所受的力也一定不相同,而对物体运
动方式是由受力和初始运动状态所共同决定的这一点,缺乏
足够的认识.要让学生认识地面上的苹果和天空中的月亮虽
然受到同一性质的力——地球引力的作用,但并不因此决定
它们有相同的运动形式.苹果的初速度为零,它便自由下落;
如果给它一个水平方向的初速度,它就作平抛运动;如果这个
水平速度越来越大,苹果也有可能绕着地表作匀速圆周运动.
在练习二4中适当点明这一点,不仅能帮助学生明确牛顿推
广平方反比定律的合理性,而且也为后面“人造卫星”一节的
教学作了一定的准备.

为了培养学生演绎推理的思维能力,进行上述课堂练习
时,教师要将以下三个层次交代清楚,即:
\begin{enumerate}
    \item 提出假设:牛
顿设想使月球围绕地球运行的力和地面上的重力属于同一性
质的力,都来自地球引力;
\item 根据假设进行演绎推导:练习
二的1、2、3;
\item 用已知的观察数据验证推导的结论,从
而证实假设是否成立:练习二的4、5.
\end{enumerate}

在归纳时,教师
可指出这种研究方法与第二章第十节阅读材料中所介绍的伽
利略研究匀变速运动的方法是一致的,从而引导学生对这一
物理学基本研究方法有更深一步的体会.

\subsubsection{对万有引力定律公式的理解}

引入万有引力定律公
式后要引导学生认识以下两点:

第一,平方反比定律公式的
形式是学生在学习物理中第一次遇到,在以后的学习中还要
接触,可引导学生注意这一公式在数学形式上的特点,并点明
这种与距离平方成反比的数学形式反映了自然界物质相互作
用所遵循的一种重要方式.此外必须明确对两个相距不太远
的非球形物体,不可简单地把两物中心间距作为R代入公式
来计算,这样只能作出粗略的估算.

第二,在说明引力与两个物体质量的乘积成正比时,要指
出两个不接触物体间的相互引力作用也是服从牛顿第三定律
的,即使很大质量和很小质量之间的相互吸引力也是大小相
等的.这一点似乎与学生的普通常识相矛盾,由于学生一般
遇到的都是卫星绕行星、行星绕恒星、地球表面物体自由下落
之类的问题,所以往往容易产生似乎只是质量大的物体吸引
质量小的物体,或者质量大的物体对质量小的物体的引力大
的错误观念,为了帮助学生理解这一点,可将练习二中的1、2作为课堂讨论题进行分析,并且举潮汐为例说明不仅地球
吸引月球,而且月球也吸引地球,潮汐就是质量小的物体也吸
引质量大的物体的具体例证.

\subsubsection{万有引力恒量} 
$G$是学生接触到的不多几个具有重
要地位的物理学普适恒量之一,要向学生指出,$G$作为万有
引力定律中的比例常数,不能单纯从数学角度去理解,要充分
认识它所表征的物理意义.要使学生理解比例常数是在描述
某种物理规律时经常出现的,各个常数有其特定的物理意义.
让学生回忆一下过去有比例常数的公式,如第一章中的胡克
定律$f=kx$中的 $k$, 表示某种材料在弹性限度内的力学性质,
因材料而异,不带普适性.而$G$表征质点间引力作用的
性质,它的数值等于两个质量各为1千克的质点相距1米的
相互吸引力,是适用于任何物体的普适恒量.以上这些比例
常数都有单位,单位由相关物理量决定.可以让学生自己确
定一下$G$的单位.其次要注意让学生对$G$的数值非常小有
个感性认识,防止学生产生一种错误观念:诸如认为固体之
所以成形,主要是由于物质颗粒间的万有引力使它们结合在
一起等,本章中许多数据用指数表示,并出现了不少指数运
算,为此可简单向学生介绍一下什么是数量级和怎样进行数
量级的估算.可结合本节教材中最后一段让学生自己估算一
下几种不同情况下引力的数量级,从而对一般物体之间和
天体之间引力大小的巨大差异有一个鲜明的认识.并建议用
以下的板图(或投影片)形象化地表示,也可
要求学生自己画在笔记本上,这比只用文字表示更易留下较
深的印象.

\subsubsection{卡文迪许实验}

在介绍卡文迪许实验装置时,可绘制
扭秤装置俯视图,也可做一个扭秤模型,说明扭秤装置中的
$T$形架增大了引力$F$的力臂,从而使石英细丝在$m$、$m'$两球
间微小的引力作用下产生一定的扭转形变,而$T$形架上的小
镜又利用光的反射定律把这一微弱的形变效应放大,加大标
尺与小镜间距离又能增大标尺上光点的偏转距离.在此可提
醒学生回忆一下第14页阅读材料:“显示微小形变的装置”.
正是这种“三次放大”的作用,扭秤才能较准确的测定微小的
作用力,例如,现代形式的卡文迪许装置能测出的引力约为
$6\x10^{-10}$牛,一根人发的重量是它的一万倍.这类利用“光杠
杆”作用的扭秤装置,是所有机械装置中最灵敏的装置之一.

\begin{center}
\begin{tabular}{p{.35\textwidth}p{.1\textwidth}p{.4\textwidth}}
    \hline
    两个物体 &引力的数量级 &相当于\\
    \hline
相距1米的两个1千克物体 & $10^{-10}$N  &  一粒砂子重量的1千万分之一 或 一根头发重量的十万分之一\\
相距10厘米的两个100克苹果 & $10^{-8}$N  &  一粒砂子重量的10万分之一\\
相距1米的两个成人 & $10^{-7}$N  &  一粒砂子重量的万分之一\\
相距100米的两艘万吨轮 & $10^0$N  &  两只鸡蛋的重量\\
相距$4\x 10^8$米的地球和月球 & $10^{20}$N  & 拉断钢索\\
相距$10^{11}$米的太阳和地球 & $10^{22}$N  &  可将直径为几千米的钢柱拉断\\
    \hline
\end{tabular}
\end{center}

\subsection{第二单元}
本单元运用的公式及相应的物理量较多,问题的综合程
度和灵活性又较前一章有所提高,学生往往不注意作有条理
的分析而惯于套用现成公式,单纯作公式代换,并易犯单位
和运算的错误,因此在教学过程中需注意帮助学生掌握综合
运用万有引力定律和匀速圆周运动的动力学方法分析具体问
题的基本思路.有关单位统一、指数运算等也要注意作出示范.

\subsubsection{万有引力定律在天文学上的应用} 

在引入“天体质量
算”这一课题时,可先提出能不能用简单的实验方法直接测
定地球或太阳的质量的问题,启发学生思考,并引导他们自己
用万有引力定律和圆周运动的知识,一步步导出计算天体质
量的公式.然后指出,计算某天体质量时只需知道围绕该天
体运行的行星(或卫星)的轨道半径$R$和周期$T$, 因这两个量
是可以测定的.

\subsubsection{地球上物体重量的变化}

通过本节学习应使学生了
解影响地球上物体重量变化的三个因素:纬度、离地高度和地
质结构,其中纬度(即地理位置)的变化是主要因素,关键是
使学生明确由于地球的自转,重力仅是引力的一个分力,而且
引力本身又从两极到赤道逐渐变小,此外应向学生指出课本
图5.4仅仅是一个示意图,地球的椭球状以及向心力相对于
引力的大小都是夸大了的,引力和重力之间的夹角也是极
小的.

本节教学中还有必要向学生指出处理某一物理量的变与
不变是相对的,必须根据所研究问题的要求来决定,由于地
球上的$g$随纬度、高度变化的相对数值很小,在一般计算中并
不考虑$g$的变化,而将它作为常数处理.


\subsubsection{人造地球卫星}

由于人造卫星问题的综合性较强,所
涉及的概念较多,学生往往搞不清其中的关系,常犯的错误
是把卫星绕地球运行的速率和第一宇宙速度(环绕速度)相混
淆.应该使学生明确:卫星绕地运行速率的表示式$v=\sqrt{\dfrac{GM}{r}}$,
对所有在圆形轨道上的地球卫星普遍适用,$v$的大小随$r$而
改变.而环绕速度表示式$v=\sqrt{gR_{\text{地}}}$仅适用于在近地圆形轨
道上运行的卫星,式中$g=9.8\msq$, 环绕速度的值为$7.9{\rm km/s}$,是个定值.

根据计算式$v=\sqrt{\dfrac{GM}{r}}$
,离地越远的卫星,$r$越大,$v$越
小.学生往往感到这一结论与课本图5.6中卫星进入轨道的
水平速度越大,轨道偏离地球越远的情况相矛盾,应该使学
生认识,卫星在椭圆轨道上运行时,它在各点的运动速度是不
同的.根据开普勒第二定律,卫星在近地点速度大,在远地点
速度小,在轨道上的平均速度也比在近地点的速度小,卫星进
入轨道的水平速度,只是卫星在近地点的速度,并不能反映出
它在椭圆轨道上各点的实际速度.根据公式$v=\sqrt{\dfrac{GM}{r}}$,
可
以用练习三1的方法,以卫星在近地点和远地点到地心距离
的平均值作为平均轨道半径,近似地求出卫星在轨道上的平
均运行速度.

教学中也可将练习四3作为课堂讨论练习题,通过对这
一问题的具体分析,进一步引导学生明确以上几点,在解答
该题时学生往往会将$2\pi R$, 除以周期80分钟,得出运行速率
$v\approx 8.4{\rm km/s}$,又根据这一速率大于$7.9{\rm km/s}$,便断定可
发射这样一颗卫星,学生之所以会得出这错误结论,是由
于不明确卫星的$T$、$r$和$v$之间有着确定的关系,因而错误地
将$R_{\text{地}}$作为轨道半径$r$来求运行速率$v$. 其次又将运行速率
错误地与进入轨道的最低水平速率混同起来.遇到这类错误,
可将它写在黑板上让学生共同来分析产生错误的原因.

关于第二、第三宇宙速度,只需指出只有当卫星获得足够
大的速度时它才能摆脱地球,甚至太阳的引力羁绊,而不必作
其他补充.

卫星中的超重、失重问题,主要抓住加速度向上还是向下
这个关键进行分析.对卫星在轨道上的失重情况,应使学生
理解此处所谓向下的加速度就是指向地心的向心加速度.

\section{实验指导}
\subsection{演示实验}

\subsubsection{天文挂图}

太阳系八大行星围绕太阳运动的示意
图,以及八大行星大小比较示意图,使学生对太阳系的结构有
一定性的形象了解.

\subsubsection{卡文迪许扭秤实验模型}
可根据课本图5.3的卡文迪许扭秤实验示意图,自制模
型,使学生了解扭秤装置的构造原理,以及如何利用光杠杆的
放大作用,读出石英丝的微小扭转形变.同时可以画出俯视
示意图(图5.1).模型中射到平面镜$M$上的光是从平行光管
$S$射出的,反射光点投到圆弧形刻度盘上.

\begin{figure}[htp]
    \centering
    \begin{tikzpicture}[>=latex]
\draw[dashed](-.05,-3) rectangle (.05,3);
\draw[rotate=-15](-.05,-3) rectangle (.05,3);
   \draw(45:3.5) circle(5pt)node[right=3pt]{$m'$};
\draw(45+180:3.5) circle(5pt)node[left=3pt]{$m'$};
\draw(75:3)[fill=white] circle(3.5pt)node[above=3pt]{$m$};
\draw(75+180:3)[fill=white] circle(3.5pt)node[below=3pt]{$m$};
\draw(90:3)[fill=white] circle(3.5pt);
\draw(90+180:3)[fill=white] circle(3.5pt);
\node at (0,0)[left]{$M$};
\draw[very thick] (40:5) arc (40:-40:5);
\draw[thick](5,0)node[right=.5]{$S$}--(0,0)--(-30:5)node[right]{$P$};
\draw[->,thick](5,0)--(2.5,0);
\draw[->,thick](0,0)--(-30:2.5);
\draw[dashed](0,0)--(-15:2);
\draw(-15:1) arc (-15:0:1);\node at (-7.5:1.3){$\theta$};
\draw (-15:.8) arc (-15:-30:.8);\node at (-22.5:1.2) {$\theta$};
\draw(5,-.1)--(5.5,-.1);
\draw(5,.1)--(5.5,.1);
\tkzDefPoint(75:3){A}  \tkzDefPoint(75+180:3){B}
\tkzDefPoint(45:3.5){A1}  \tkzDefPoint(45+180:3.5){B1}
\tkzDrawLines[->, add = 0 and -.7](A,A1 A1,A B,B1 B1,B)
\draw(-90:1.2) arc (-90:-105:1.2);
\node at (-97.5:1.5){$\theta$};
\draw[decorate,decoration={brace,raise=8pt}] (A) --node[above=9pt]{$r$}node {$F$} (A1);
\draw[decorate,decoration={brace,raise=8pt}] (B) --node[below=9pt]{$r$} node {$F$}  (B1);

    \end{tikzpicture}
    \caption{}
\end{figure}



\subsubsection{挂图——人造地球卫星、宇宙飞船}

目的在于使学生了解人造地球卫星和宇宙飞船在轨道上
运行的原理,以及通讯卫星为什么可以实现全球电视转播的
原理.

有条件的可以播放人造地球卫星的发射、空间站、宇
宙飞船的运行(包括字航员的失重状态)、航天飞机的发射与
返航等资料影片或录象,以增加感性认识,提高学习兴趣.




\section{习题解答}
	
\subsection{练习一}
\begin{enumerate}
	\item 下表给出了太阳系九大行星平均轨道半径和周期的数值.从表中任选三个行星验证开普勒第三定律,并计算恒
	量$k=R^3/T^2$的值.
\begin{center}
	\begin{tabular}{ccc}
\hline
行星    &  平均轨道半径(m)  & 周期(s)\\
\hline
水星      &  $5.79\times 10^{10}$    & $7.60\times 10^8$ \\ 
金星    &  $1.08\times 10^{11}$    &  $1.94\times 10^7$ \\ 
地球    &  $1.49\times 10^{11}$    &  $3.16\times 10^7$ \\ 
火星    &  $2.28\times 10^{11}$    &  $5.94\times 10^7$ \\ 
木星    &  $7.78\times 10^{11}$    &  $3.74\times 10^8$ \\ 
土星    &  $1.43\times 10^{12}$    &  $9.30\times 10^8$ \\ 
天王星    & $2.87\times 10^{12}$     &  $2.66\times 10^9$ \\ 
海王星    &  $4.50\times 10^{12}$    &  $5.20\times 10^9$ \\ 
冥王星    & $5.9\times 10^{12}$     &  $7.82\times 10^9$ \\ 
\hline
	\end{tabular}
\end{center}


\begin{solution}
	取地球、火星、木星为例来验证开普勒第三定律.
地球:
\[k=\frac{R^3}{T^2}=\frac{(1.49\x 10^{11})^3}{(3.16\x 10^{7})^2} =3.31\x 10^{18}  {\rm m^3/s^2}\]

火星:\[k=\frac{R^3}{T^2}=\frac{(2.28\x 10^{11})^3}{(5.97\x 10^{7})^2} =3.36\x 10^{18}   {\rm m^3/s^2}\]
木星:\[k=\frac{R^3}{T^2}=\frac{(7.78\x 10^{11})^3}{(3.74\x 10^{8})^2} = 3.37\x 10^{18}  {\rm m^3/s^2}\]
故:\[\bar k=\frac{(3.31+3.36+3.37)\x 10^{18} }{3}=3.35\x 10^{18}  {\rm m^3/s^2}\]
\end{solution}

\item 有一个名叫谷神的小行星(质量$1.00\times 10^{21}$kg),它的轨道半径是地球的2.77倍,求出它绕太阳一周需要多少年.

\begin{solution}
	由开普勒第三定律$k=\dfrac{R^3}{T^2}$,取$k=3.35\x 10^{18}  {\rm m^3/s^2}$,则
\[T=\sqrt{\frac{R^3}{k}}=\sqrt{\frac{(1.49\x 10^{11}\x 2.77)^3}{3.35\x 10^{18}}}=1.45\x 10^8{\rm s}\]
\end{solution}
说明:题中给出的谷神小行星的质量与解题无关,目的是
使学生明确行星运行周期仅取决于轨道半径,同时培养学生
合理利用已知数据的能力.
\end{enumerate}





\subsection{练习二}
\begin{enumerate}
	\item 你能说出你对地球的引力是多少吗?

	\begin{solution}
		人对地球的引力大小等于地球对人的引力大小,也
		就是大的体重.
	\end{solution}
	
\item “我们说苹果落向地球,而不说地球向上运动碰到苹果,是因为地球的质量比苹果大得多,地球对苹果的引力比苹果对地球的引力大得多.”这种说法对吗?为什么?

\begin{solution}
	这种说法不对.地球对苹果的引力大小与苹果对地
	球的引力大小相等,是一对作用力和反作用力,因为地球质量
	比苹果大得多,所以它产生的加速度就比苹果产生的加速度
	(即重力加速度)小得多,几乎等于零,所以我们说苹果落向地
	球,面不说地球向上运动碰到苹果.
\end{solution}

\item 两个质量都是4千克的铅球,相距0.1米远,它们之间的引力是多少?

\begin{solution}
	据万有引力公式
\[F=G\frac{Mn}{R^2}=6.67\x 10^{-11}\x \frac{4^2}{0.1^2}=1.07\x 10^{-7}{\rm N}\]
\end{solution}

\item 用$M$表示地球的质量,$R$表示地球的半径,$T$表示月球到地球的距离.试证明,在地球引力作用下,
\begin{enumerate}
	\item 地面上物体的重力加速度$g=\dfrac{GM}{R^2}$;
	\item 月球的加速度$a_{\text{月}}=\dfrac{GM}{r^2_{\text{月地}}}$;
	\item 已知$r_{\text{月地}}=60R$,利用(a)(b)求$a_{\text{月}}/g$;
	\item 已知$r_{\text{月地}}=3.8\times 10^8$米,月球绕地球运行的周期$T=27.3$天,计算月球绕地球运行的向心加速度$a_{\text{月}}$.
	\item 已知重力加速度$g=9.8\msq$.用(d)中算出的$a_{\text{月}}$,求$a_{\text{月}}/g$.
	
	比较(c)(e)中求出的$a_{\text{月}}/g$是否相等.如果相等,则表明地球
	对月球的引力和对地面物体的引力都遵守平方反比定律,因
	而是同一种性质的力,牛顿就是根据这一结果证明地球对月球的引力和地面上物体所受的重力是同一种力的.
\end{enumerate}

	\begin{proof}
\begin{enumerate}
	\item 地球表面物体的重力等于地球对物体的引力,有
\[mg=G\frac{Mm}{R^2}\quad \Rightarrow\quad g=G\frac{M}{R^2}\]
\item 假设月球绕地球运行所需的向心力就是地球对月球
的万有引力.$a_{\text{月}}$是月亮绕地球运行的向心加速度,方向指向地
球.则有
\[ma_{\text{月}}=G\frac{Mm}{r^2_{\text{月地}}}\quad \Rightarrow\quad a_{\text{月}}=\frac{GM}{r^2_{\text{月地}}}\]
\item \[\frac{a_{\text{月}}}{g}=\frac{GM\cdot R^2}{r^2_{\text{月地}}\cdot GM}=\frac{R^2}{r^2_{\text{月地}}}\]
将$r^2_{\text{月地}}=60R$代入上式,得
\[\frac{a_{\text{月}}}{g}=\frac{R^2}{(60R)^2}=\frac{1}{3600}=2.8\x 10^{-4}\]
\item 由匀速圆周运动向心力公式得
\[\begin{split}
	a_{\text{月}}=\omega^2r_{\text{月地}}=\left(\frac{2\pi}{T}\right)^2\cdot r_{\text{月地}}
&=\frac{4\pi^2}{T^2}\cdot r_{\text{月地}}\\
&=\frac{2\x (3.14)^2}{(27.3\x 86400)^2}\x 3.8\x 10^{8}\\
&=2.69\x 10^{-3}\msq
\end{split}\]
\item \[\frac{a_{\text{月}}}{g}=\frac{2.69\x 10^{-3}}{9.8}=2.8\x 10^{-4}\]
\end{enumerate}
比较(c)、(e)的结果相等,说明牛顿的假说是正确的.
	\end{proof}
	
\end{enumerate}




\subsection{练习三}
\begin{enumerate}
	\item 应用人造地球卫星可以测定地球的质量.我国1970年4月24日发射的第一颗人造地球卫星,其周期是114分,它的近地点是439千米,远地点是2384千米,以卫星在近地点和远地点时到地心距离的平均值作为卫星轨道的平均半径,试计算地球的质量.

	\begin{solution}
		取地球半径为6370千米,则卫星轨道的平均半径为		
	\[R=\frac{439+2384}{2}+6370=7782{\rm km}\]
		卫星绕地球运动的向心力等于地球对卫星的引力,即有
\[m\omega^2 R=G\frac{Mm}{R^2},\qquad \left(\frac{2\pi}{T}\right)^2 R=G\frac{Mm}{R^2}\]
则:
\[M=\frac{4\pi^2\cdot R^3}{GT^2}\]
代入数据得
\[M=\frac{4\x 3.14^2\x (7.782\x 10^6)^3}{6.67\x 10^{-11}\x (114\x 60)^2}=5.96\x 10^{24}{\rm kg}\]
	\end{solution}
	
	\item 登月密封舱在离月球表面112千米的空中沿圆形轨道运行,周期是120.5分钟,月球的半径是1740千米,根据这些数据计算月球的质量和平均密度.

	\begin{solution}
密封舱绕月球所需的向心力就是月球对密封舱的引
力.

由$m\cdot \dfrac{4\pi^2}{T^2}\cdot R=G\dfrac{Mn}{R^2}$,得$M=\dfrac{4\pi^2 R^3}{GT^2}$

密封舱轨道半径
\[R=112+1740=1852{\rm km}\]
代入数据
\[M=\frac{4\x3.14^2\x(1.852\x10^8)^3}{6.67\x10^{-11}\x(120.5\x60)^2}
=7.19\x10^{22}{\rm kg}\]
平均密度
\[\rho=\frac{M}{V}=\frac{M}{\frac{4}{3}\pi R^3}=\frac{7.19\x 10^{22}\x 3}{4\x 3.14\x (1.740\x 10^{6})^3}=3.26\x 10^{3}{\rm kg/m^3}\]
	\end{solution}
	
\end{enumerate}


\subsection{练习四}

在下列各题中,地球质量取$M=6.0\times 10^{24}{\rm kg}$.
\begin{enumerate}
	\item 图5.8 $A$、$B$、$C$是在地球大气层外圆形轨道上运行的三颗人造卫星,$A$、$B$的质量相同,它们的轨道速率是否也相同?$B$、$C$的质量不同,它们的轨道速率是否也不同?

\begin{figure}[htp]
\centering\begin{tikzpicture}[>=latex]
\draw (0,0) circle (15pt);
\node at (0,0){地球};
\draw (0,1.5) arc (90:15:1.5);
\draw (0,3) arc (90:15:3);
\draw [<-](0,2.2) arc (90:70:2.2);
\node at (0,2.2)[left]{卫星运行方向};
\draw[fill=white] (60:1.5) circle (2pt) node[above]{$A$};
\draw[fill=white] (75:3) circle (2pt) node[above]{$B$};
\draw[fill=white] (40:3) circle (5pt);
\node at  (40:3.2) [right]{$C$};
\end{tikzpicture}
\caption{}
\end{figure}


\begin{solution}
	由卫星绕地球运行所需向心力即为地球对卫星的引力,即
\[\frac{mv^2}{r}=G\frac{M_{\text{地}}m}{r^2}\]
得卫星的轨道速率
\[v=\sqrt{\frac{GM_{\text{地}}}{r}}\]
由于
式中$G$和$M_{\text{地}}$为常量,所以,卫
星的轨道速率只与其轨道半径的平方根成反比.由于卫星$A$
方向
和$B$不在同一轨道上,即$r_A\ne r_B$,即速率不等.

卫星$B$和卫星$C$同在一条
轨道上运行.即$r_B=r_C$, 则$v_B=v_C$,即速率相等,跟卫星$B$、$C$的质量无关.
\end{solution}


\item  假定一颗人造地球卫星正在离地面700千米高空的圆周轨道上运转,计算它的速率和周期.

\begin{solution}
	同上题由$\dfrac{mv^2}{r}=G\dfrac{Mm}{r^2}$,得速率
$v=\sqrt{\dfrac{GM}{r}}$,代入数据:
\[v=\sqrt{\frac{6.67\x 10^{-11}\x 6.0\x 10^{24}}{(6400+700)\x 10^3}}=7.5{\rm km/s}\]
周期
\[T=\frac{2\pi R}{v}=\frac{2\x 3.14\x (6400+700)\x 10^3}{7.5\x 10^3}=5.95\x 10^3{\rm s}\approx 99{\rm min}\]
\end{solution}

\item 能否发射一颗周期是80分钟的人造地球卫星?说明你的理由.

\begin{solution}
	解法1:
	若卫星的周期为80分钟,则
\[\frac{m\left(\frac{2\pi}{T}\right)^2}{R}=G\frac{Mm}{R^2},\qquad \frac{4\pi^2}{T^2}=\frac{GM}{R^3}\]
因此:
\[R^3=\frac{GMT^2}{4\pi^2}=\frac{6.67\x 10^{-11}\x 6.0\x 10^{24}\x (80\x 60)^2}{4\x 3.14^2}=2.34\x 10^{20}{\rm m^3}\]
\[R\approx 6.2\x 10^{6}{\rm m}\]
取地球半径为$6.4\x10^6$m,则$R<R_{\text{地}}$.

由于卫星飞行的圆周半径不可能小于地球半径,故不可
能发射这样一颗卫星.

解法2:根据$T^2=\dfrac{4\pi^2R^3}{GM}$,卫星轨道半径$R$越大,周期$T$
越长.

靠近地球表面以第一宇宙速度运行卫星的周期最短,为
\[T=\frac{2\pi R}{v}=\frac{2\x 3.14\x 6.4\x 10^{6}}{7.9\x 10^3}=5.1\x 10^3{\rm s}=85{\rm min}\]

而题中卫星周期为80分钟,所以不可能发射一颗运行周
期比以第一宇宙速度运行的卫星还短的地球卫星.
\end{solution}

\end{enumerate}





\subsection{习题}
\begin{enumerate}
	\item 在一次测定引力恒量的实验里,已知一个质量是0.80千克的球,以$1.0\times 10^{-10}$牛的力吸引另一个质量是$4.0\times 10^{-3}$千克的球.这两个球相距$4.0\times 10^{-2}$米.地球表面的重力加速度是9.8$\msq$,地球的半径是6400千米.根据这些数据计算地球的质量.

	\begin{solution}
		因为地球对物体的引力就是物体所受的重力,所以$mg=G\dfrac{Mm}{R^2}$,由此 得:
		\[M=\frac{gR^2_{\text{地}}}{G}\]
		先根据$F=G\dfrac{m_1m_2}{r^2}$,求出$G=\dfrac{Fr^2}{m_1m_2}$,代入上式得:
	\[M=\frac{gR^2_{\text{地}}m_1m_2}{Fr^2}=\frac{9.8\x (6.4\x 10^6)^2\x 0.80\x 4.0\x 10^{-3}}{1.3\x 10^{-10}\x (4.0\x 10^{-2})^2}=6.2\x 10^{24}{\rm kg}\]
	\end{solution}
	
	\item 行星的质量为$M$,一个围绕它作匀速图周运动的卫星的轨道半径是$R$,周期是$T$.试用两种方法求出卫星轨道上的向心加速度.

	\begin{solution}
		解法1:根据向心加速度的公式有
	\[a_n=\omega^2 R=\left(\frac{2\pi}{T}\right)^2\cdot R=\frac{4\pi^2 R}{T^2}\]
		解法2:根据卫星所需向心力等于行星对它的引力有
	\[ma_n=G\frac{Mm}{R^2}\]
	所以:$a_n=\dfrac{GM}{R^2}$
	\end{solution}
		
	\item 应用通讯卫星可以实现全地球的电视转播,这种卫星位于赤道的上方,相对于地面静止不动,犹如悬在空中一样,叫做同步卫星.同步卫星的周期是多大?计算它的高度和速率.

	\begin{solution}
		同步卫星的周期与地球自转的周期相同.
		\[T=24\x3600=86400{\rm s}\]
		根据
$m\dfrac{4\pi^2}{T^2}\cdot r=G\dfrac{Mm}{r^2}$,得:
\[r^3=\frac{GM}{4\pi^2}\cdot T^2\]
故卫星轨道半径为
\[r=\left(\frac{GMT^2}{4\pi^2} \right)^{1/3}=\left(\frac{6.67\x 10^{-11}\x 6.0\x 10^{24}\x 86400^2}{4\x 3.14^2}\right)^{1/3}=4.23\x 10^7{\rm m}\]
上式也可由下式得出:
\[v=\sqrt{\frac{GM}{r}},\qquad \frac{2\pi r}{T}=\sqrt{\frac{GM}{r}}\]

卫星高度
\[h=r-R_{\text{地}}=4.23\x 10^{7} - 6.4\x 10^6=3.59\x 10^7{\rm m}\]
卫星运行速率
\[v=\frac{2\pi r}{T}=\frac{2\x3.14\x4.23\x10^7}{86400}=3.07\x10^3{\rm m/s}\]
或者:
\[v=\sqrt{\frac{GM}{r}}=\sqrt{\frac{6.67\x 10^{-11}\x 6.0\x 10^{24}}{4.23\x 10^7}}=3.07\x 10^3\ms\]

		说明:此题说明卫星的轨道半径$r$、运行周期$T$和速率
$v$之间存在着确定的关系.对同步地球卫星讲,$T$是确定的,
因此所有的同步卫星均在赤道平面的同一轨道上,以相同速
率运行.
	\end{solution}
	
	\item 试用万有引力定律证明:对于某个行星的所有卫星来说,$R^3/T^2$是一个恒量.其中$R$是卫星的轨道半径,$T$是卫星的运行周期.

	\begin{proof}
	卫星围绕行星运行所需的向心力就是它们之间的
万有引力.即有
\[G\frac{Mm}{R^2}=m\omega^2 R,\qquad G\frac{M}{R^2}=\frac{4\pi^2}{T^2}\cdot R,\qquad \frac{GM}{R^3}=\frac{4\pi^2}{T^2}\]
由此得
\[\frac{R^3}{T^2}=\frac{GM}{4\pi^2}\]

对某个行星的所有卫星来说,$M$恒定,是该行星的质量,$\dfrac{GM}{4\pi^2}$
是恒量,所以$R^3/T^2$也是一个恒量.
	\end{proof}
	
	\item 行星的密度是$\rho$,靠近行星表面的卫星运行周期是$T$.试证明$\rho T^2$是一个普遍适用的恒量,即它对任何行星都相同.

	\begin{proof}
球体积
$V=\frac{4}{3}\pi R^3$, 所以行星的密度
\[\rho=\frac{M}{\frac{4}{3}\pi R^3}\]
由$m\omega^2 R=G\dfrac{Mm}{R^2}$
得
\[\frac{M}{R^3}=\frac{4\pi^2}{GT^2}\]
将上式变形得
\[\frac{M}{\frac{4}{3}\pi R^3}=\frac{3\pi}{GT^2}\]
即
\[\rho T^2=\frac{3\pi}{GT^2}\]
由此得
\[\rho T^2=\frac{3\pi}{G}\]
由于$G$是普适恒量,所以$\rho T^2$对于任何一个行星都相同.
	\end{proof}
	
	\item 一艘宇宙飞船飞近某一个不知名的行星,并进入靠近该行星表面的圆形轨道,宇航员着手进行预定的考察工作.宇航员能不能仅仅用一只表通过测定时间来测定该行星的密度?说明理由.

	\begin{solution}
		根据题意,字航员可用表测定该飞船在行星表面附
		近绕行星运行一周所需要的时间$T$, 利用第5题的公式
		\[\rho=\frac{3\pi}{GT^2}\]
		将$T$值代入,即可算出密度.
	\end{solution}
	
	\item 不考虑地球的自转,求出用地球半径$R$、地面重力加速度$g$和引力恒量$G$表示的地球密度的公式.

	\begin{solution}
不考虑地球的自转,可认为物体重量就是地球对它
的引力,即
\[mg=G\frac{Mm}{R^2}\]
将$M=\frac{4}{3}\pi R^3\rho$代入上式得
\[g=\frac{4}{3}\pi\rho GR\]
则地球的密度
\[\rho=\frac{3g}{4\pi GR}\]
	\end{solution}
	
	\item 用火箭把宇航员送到月球上,如果他已知月球的半径,那么他用一个弹簧秤和一个已知质量的砝码,能否测出月球的质量?应该怎样测定?	

	\begin{solution}
只要已知月球半径,便可测出月球质量.具体步骤
如下:
\begin{enumerate}
\item 可以先将质量已知的砝码挂在弹簧秤上,测出读数
$W$, 由$W=mg_{\text{月}}$, 可求出$g_{\text{月}}=W/m$;
\item 然后由砝码重量等于月球对它的引力得
\[mg_{\text{月}}=G\frac{M_{\text{月}}m}{R^2_{\text{月}}},\qquad g_{\text{月}}=\frac{GM_{\text{月}}}{R^2_{\text{月}}}\]
故
\[M_{\text{月}}=\frac{g_{\text{月}}R^2_{\text{月}}}{G}=\frac{WR^2_{\text{月}}}{G_m}\]
月球半径$R_{\text{月}}$已知,$W$由弹簧测
出,$G$是普适恒量,$m$已知,因此可以计算出月球质量$M$.
\end{enumerate}


	\end{solution}
	

\end{enumerate}
	
	
\section{参考资料}
\subsection{太阳系中的最大和最小}

太的体积约是地球体积
的130万倍,地球的体积约是月球体积的50倍.太阳的质量
约是地球质量的33万倍,地球质量约是月球质量的81倍.下
表列出了太阳系八大行星及冥王星距太阳的平均距离、体积、质量、表
面平均重力加速度、平均密度、自转周期最大和最小的星球.

\begin{center}
\begin{tabular}{p{.45\textwidth}p{.25\textwidth}p{.2\textwidth}}
\hline
&  最小  &最大\\
\hline
距太阳平均距离(以地球与太阳平均距离为单位) & 水星(约为0.4)&冥王星(约39.5)\\
体积(以地球体积为单位) & 水星(约为0.056)&木星(约1316)\\
质量(以地球质量为单位) & 水星(约为0.055)&木星(约318)\\
表面平均重力加速度($\msq$)&  水星(3.6)&木星(约26)\\
平均密度(克/厘米)& 土星(0.7)&地球(5.5)\\
自转周期&木星(9时50分)& 金星(244.3日)\\
\hline
\end{tabular}
\end{center}

\subsection{开普勒定律}
开普勒研究所根据的资料都是凭肉眼
观测的.随着望远镜等精密仪器的出现,发现开普勒定律只
是近似的,行星实际运行的情况与开普勒定律有少许偏离.造
成这种情况的有以下两个原因:由于太阳也受到行星的吸引,
它也有加速度,并不是静止不动的,实际上太阳和行星都绕它
们的质心各自沿椭圆轨道运动,此时行星椭圆轨道半长轴(平
均半径)立方与运行周期平方之比已不再是常数,而应修正为
\[\frac{R_1^3}{T^2_1}\cdot \frac{R^3_2}{T^2_2}=\frac{M+m_1}{M+m_2}=\frac{1+\dfrac{m_1}{M}}{1+\dfrac{m_2}{M}}\]
式中的$R_1$和$R_2$分别是质量为
$m_1$
和$m_2$的行星轨道半长轴,$T_1$和$T_2$分别是它们的运行周
期,$M$是太阳的质量,实际上太阳系中质量最大的行星是木
星,它的质量是太阳质量的$1/1047$, 上式之比与1相差极微.
所以开普勒第三定律虽然只是近似的,但近似程度是相当高
的.以上结论只考虑了行星与太阳间的相互吸引,在理论力
学中称为二体问题,如果要考虑任一行星还受到其他行星的
吸引,则成为多体问题,此时只能用微扰法来近似求解.

\subsection{万有引力定律建立的历史进程}

在古代和中世纪,引
力被认为是位置的一种性质.亚里士多德认为“宇宙中的万
物都有它的指定位置,一旦脱离原位,就要回复回去”以此来
解释石头落地的问题.哥白尼设想太阳、月球和各个行星都有
自己的引力体系,地球上空的石头会落向最近的引力体系,即
落向地面.伽利略提出惯性概念时虽已意识到约束行星沿闭
合轨道需要力的作用,但没指出这力的性质.开普勒在探索
行星运动的规律时,也产生了寻求行星运动原因的思路,他认
为是太阳发出的磁力推动着行星的公转.

英国物理学家胡克提出了一切天体都具有倾向于其中心
的吸引力,它不但吸引其本身的各个部分,还吸引其作用范围
的其他天体.这就是行星绕太阳作椭圆运动的原因,他还提
出了这个引力反比于距离的平方,但他一直未能从理论上证
明这一点.

牛顿在1665—1666年间想到“把推动月球在轨道上运行
的力和地面上的重力加以比较”,可是由于在计算上遇到的困
难,他的研究迟迟没有进展.

1685年,牛顿从理论上解决了把太阳、月球、地球都当成
一个个质点的问题,采用了地球半径的新数据,证明地面上物
体坠落和月球沿闭合轨道运行是出于同一原因,并把这一结
论推广到所有的行星运动中去,从而提出了著名的万有引力
定律.

\subsection{卡文迪许}

亨利·卡文迪许(1731—1810)是近代著名
英国科学家,他一生从事大量的化学、电学实验,不疲倦地埋
头于实验研究工作达50年之久.大约在库仑确定著名的静
电学基本定律的同时,他独自发现并测得电荷间的作用力跟
距离平方成反比的规律,还独立提出了电势的概念,1798年,
已近垂暮之年的卡文迪许运用构思巧妙的精密“扭秤”实验技
巧测定出地球的平均密度为$5.481{\rm g/cm^3}$(现代公认值为
$5.517{\rm g/cm^3}$),由此可推算出万有引力常数是$6.754\x10^{11}{\rm N\cdot m^2/kg^3}$
(现代公认值为$6.668\pm 0.005\x10^{11}{\rm N\cdot m^2/kg^3}$).他被公认为是最伟大的实验科学家之一.英国科学家
坡印廷盛赞他“开创了测量弱力的新时代”.

\subsection{三种宇宙速度}


\subsubsection{第一宇宙速度} 课本中已经讲过了第一宇宙速度的
推导过程和数值,即$v_1=\sqrt{Rg}=7.9{\rm km/s}$.

\subsubsection{第二宇宙速度} 即物体能够脱离地球引力而不再回
到地球所需的最小发射速度,通常用$v_2$表示.

如果知道从地球上射出的物体至少需要有多大的动
能,才能够克服地球的引力逃到无限远处(即脱离地球的引力
范围),就可以求出$v_2$.要知道这个动能就必须求出物体从地
面移到无限远处反抗地球引力所做的功.

由于把物体移到无限远处的过程中,地球对物体的引力
是变化的,所以不能照恒力做功那样简单地用$W=Fs\cos\theta$
来求,而要用到积分,用$W$表示反抗地球引力从地面到无限
远处所做的功,则
\[W=\int^{\infty}_R \frac{GMm}{r^2}\dd r=\frac{GMm}{R} \]
由于在地面上
\[G\frac{Mm}{R^2}=mg,\qquad G\frac{M}{R^2}=g\]
所以
\[W=mgR\]

根据动能定理知道,物体需要具有的动能应该等于这个
功,即
\[\frac{1}{2}mv^2_2=mgR\]
所以
\[v_2=\sqrt{2Rg}\]
由于$\sqrt{Rg}=v_1=7.9{\rm km/s}$,所以
\[v_2=\sqrt{2}v_1=11.2{\rm km/s}\]

\subsubsection{第三宇宙速度}

使物体不但挣脱地球的引力,而目.
挣脱太阳的引力,逃到太阳系以外去,物体所需要的速度叫第
三宇宙速度,通常用$v_3$表示.

我们知道,地球以约$30{\rm km/s}$的速度绕太阳运动,地球
上的物体也随着地球以这个速度绕太阳运动.正象物体挣脱
地球引力所需的速度等于它绕地球运动的速度的2倍那
样,地球上的物体挣脱太阳引力所需的速度为$30{\rm km/s}$的
2倍,所以地球上的物体挣脱太阳引力所需要的速度为
$30\x\sqrt{2}=42.3{\rm km/s}$.

由于地球上的物体已经具有绕太阳运动的$30{\rm km/s}$的
速度,要使它相对于太阳的速度达到$42.3{\rm km/s}$,只要使它
在沿着地球运行方向增加$12.3{\rm km/s}$的速度就行了,但是,-
要使物体脱离太阳,首先要使它脱离地球,因此,除了给予物
体$\frac{1}{2}mv^2$($v$代表$12.3{\rm km/s}$的速度)的动能以外,还必须给
予物体
$\frac{1}{2}mv^2_2$($v_2$代表第二宇宙速度)的动能.也就是说,必
须给予物体的动能为
$\frac{1}{2}mv^2+\frac{1}{2}mv^2_2$.如果用$v_3$代表第三宇
宙速度,这个动能就等于$\frac{1}{2}mv^2_3$.所以
\[\frac{1}{2}mv^2_3=\frac{1}{2}mv^2+\frac{1}{2}mv^2_2\]
\[v_3=\sqrt{v^2+v^2_3}=\sqrt{12.3^2+11.2^2}=16.7{\rm km/s}\]

\subsection{不同高度上卫星的环绕速度}

\begin{center}
\begin{tabular}{ccccccccc}
\hline
高度& 0&300&500&1000&3000&5000&35900& 380000\\
(km) &&&&&&&(同步轨道)  &(月球轨道)\\
\hline
环绕速度 &7.91&7.73&7.62&7.36&6.53&5.29&2.77&0.97\\
(km/s)\\
周期&84.4&90.5&94.5&105&150&201&23h 56min&28d\\
(min)\\
\hline
\end{tabular}
\end{center}

\subsection{五个国家第一颗卫星比较}
\begin{center}
\begin{tabular}{cccccc}
\hline
&中国&苏联&美国&法国&日本\\
\hline
发射日期&1970.4.24&1957.10.4&1958.2.1&1965.11.26&1970.2.11\\
质量(kg)&173&83.6&13.97*&40&38*\\
\hline
\end{tabular}

* 包括最后一级运载火箭壳体.
\end{center}

\subsection{我国发射的十七颗卫星简况}

\begin{center}
	\begin{tabular}{ccp{.3\textwidth}p{.25\textwidth}}
		\hline
		名称	&	发射时间	&	工作情况	&	其他\\
		\hline
		人造地球卫星&	1970.4.24&	播送《东方红》乐曲\\
		科学实验卫星&		1971.3.3&		向地面播送科学实验数据\\
		人造地球卫星&		1975.7.26&		星上各种仪器工作正常\\
		人造地球卫星&1975.11.26&卫星各种系统工作正常&三天后,按计划返回地面\\
人造地球卫星&1975.12.16&卫星工作正常\\
人造地球卫星&1976.8.30&卫星工作正常\\
人造地球卫星&1976.12.7&卫星工作正常&按预定计划准确返回地面	\\
人造地球卫星&1978.1.26&卫星运行良好,完成了科学实验任务&按预定计划成功地返回地面\\
空间物理探测卫星&1981.9.20&各系统工作正常,不断向地面发送各种科学探测和试验数据&用一枚火箭发射三颗卫星\\
科学试验卫星&1982.9.20&卫星运行良好,仪器工作正常&运行五天后,按预定计划返回地面\\
科学试验卫星&1983.8.19&卫星运行良好,各系统工作正常&按预定计划准返回地面\\
试验卫星&
1984.1.29&取得了重要成果\\
试验通信卫星&1984.4.8&进入预定轨道,设备工作正常\\
科学实验卫星&1984.9.12&卫星运行和工作正常&按预定计划准确返回地面\\
实用通信广播卫星&1986.2.1&进入预定轨道,设备工作正常\\
		\hline
	\end{tabular}
\end{center}









