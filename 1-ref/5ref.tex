\chapter{万有引力定律}
\section{教学要求}
万有引力定律的发现,是人类在认识自然规律方面取得
的一个重大成果,对人类文化历史的发展有重要意义。万有
引力定律在研究天体的运动和人造地球卫星等方面有着重要
的应用。鉴于这一规律的重要,把它单独列为一章,使内容集
中,中心突出。

这一章的教学要求是:
\begin{enumerate}
\item 了解开普勒三定律,掌握万有引力定律。
\item 了解万有引力定律在天文学上的应用,了解地球上物
体的重量变化的原因。
\item 了解有关人造卫星的知识,会推导第一宇宙速度.
\end{enumerate}

下面对这一章的教学内容作些具体说明。

为讲解万有引力定律的建立作准备,第一节先介绍行星
的运动,关于人类对行星运动规律的认识过程,只要求学生
了解个梗概,知道开普勒三定律是在前人长期观察研究的基
础上总结出来的,这一节的重点是讲解开普勒三定律,使学
生对定律的内容有所了解。学生在学习本章时,还不具备椭
圆的知识,教学中需要对椭圆的焦点、半长轴作简单的介绍。

万有引力定律的教学,主要是让学生知道牛顿如何在开
普勒三定律的基础上推导出万有引力定律的思路。在介绍牛
顿建立万有引力定律之前,提到了胡克等人猜想到引力与距
离的平方成反比,是为了说明万有引力定律的建立经历了一
个过程,不是只靠个别天才人物的灵感创造的。引力与太阳
质量的正比关系,可以直接给出,不要求作进一步的讨论。地
球对月球的吸引力与地面物体所受的重力是同一种性质的
力,让学生自己通过练习计算得出,以获得较深刻的印象。如
果学生对这个题目的推理过程不很理解,也可以作些必要的
引导和说明。

万有引力定律揭示了支配天体运动的规律,把地上的运
动和天上的运动统一起来,打破了以往人们对天体运动的神
秘感,增强了人们认识自然的信心,讲述万有引力定律,应该
使学生对此有所认识。

卡文迪许实验是历史上的著名实验,它测定了万有引力
恒量的值。鉴于这个实验的重要,单独作为一节来讲述。这
个实验,不要求演示,通过介绍这个实验,使学生认识这个实
验的重要作用,领会前人是怎样进行巧妙的设计来测出万有
引力恒量值的,启发他们进一步认识培养和训练灵活运用知
识能力的重要性。

列举几个物体间引力大小的例子,是为了说明一般物体
间的引力非常小,而天体之间的引力非常大。正是这个巨大
的引力支配着天体的运动。因而万有引力定律主要用于研究
天体的运动。

天体质量的计算,说明应用万有引力定律和圆周运动的
知识,可以确定无法直接测定的天体质量。在天文学上,太
阳、地球等天体的质量就是根据行星或卫星的轨道半径和周
期来求得的。海王星、冥王星的发现,说明万有引力定律不仅
能对观察到的天体运动作出解释,而且能预言尚未观察到的
天体的存在。这是理论指导实践的典型事例。

关于人造地球卫星的教学,重点是讲述发射人造卫星的
原理,得出第一宇宙速度后,要指出卫星进入轨道的水平速
度大于7.9${\rm km/s}$,小于11.2${\rm km/s}$时,卫星绕地球运动的
轨道将不是圆,而是椭圆,进而说明速度增大到11.2${\rm km/s}$
后,不再绕地球运行,而成为围绕太阳运动的一颗行星。至于
轨道为什么会成为椭圆,限于学生的知识水平,中学阶段不能
讲解。对于第二宇宙速度和第三宇宙速度,只要求简单介绍,
使学生知道它们的意义就行了。

这一章的习题大多是综合性的,灵活性也有所提高,要
注意向学生讲明解决这类问题的思路,以培养他们灵活运用
知识,逐步提高解题能力。

\section{教学建议}
本章内容是按照人类对万有引力定律的认识过程,围绕
运动和力的关系而逐步展开的。

学习本章内容是对前几章知识的综合和提高,在教学中
应着重培养学生综合运用新旧知识对问题进行推理和分析的
能力。此外,引导学生体会建立万有引力定律程中所体现
的科学方法,以及激发学生对未知世界的探索精神和科学的
想象力,也是本章教学中不可忽视的方面。

本章分为两个单元,第一单元包括第一至第三节,概括
地介绍了万有引力定律建立的历史进程,其中包括对引力恒
量的测定。第二单元包括第四至第六节,介绍了万有引力定
律的某些应用。

\subsection{第一单元}

万有引力定律的建立过程,对于已确立的定律、新的假
说、理论推导和实验观测之间如何相互影响和补充,提供了一
个很好的范例。因此本单元的教学可以按产生这一定律的历
史背景、定律的建立和定律的实验验证这三个层次来展开,这
样可以使学生对这一理论获得一个整体的认识,从而体会到
具有突破性的重大物理理论的建立,並不是偶然的,它反映了
人类对自然界的认识不断深化和完善的过程。

\subsubsection{开普勒定律}

第一节的重点是介绍开普勒三定律,讲
述时可指出,开普勒定律是一种描述性的经验定律。开普勒
定律描述了行星运动的规律,但没有提出和解决行星为什么
这样运动的问题,这个重要的问题是牛顿在他的运动定律的
基础上解决的。

鉴于学生在学习本节时还没学过椭圆知识,因此可结合
课本图5.2作简单解释。教学中应当指出该图表示的行星椭
圆轨道是一个十分夸张的示意图,事实上大部分行星的椭圆
轨道都十分接近于圆形,因此可对开普勒第一、第二定律作近
似处理,即认为行星以太阳为圆心作匀速圆周运动,而第三定
律中椭圆的半长轴可以当作圆形轨道的半径$R$. 应使学生明
确本章所有对天体运动的分析计算都是在上述近似处理的基
础上用匀速圆周运动的动力学方法进行的。

在说明开普勒第三定律中$k$值是一个与行星无关的恒量
时,可指出$k$值只与行星所环绕的那个天体有关,至于为什么
会这样,以后将会作进一步深入讨论。最后可将练习一的第
1题作为课堂练习,让不同小组的同学分别算出各行星的$k$
值加以比较。通过这一练习学生对太阳系中的$k$值与各行星
无关便有了具体认识。对计算结果中$k$值的差异可简单指出
这是由于表中原始数据不太精确(仅三位有效数字),而且开
普勒第三定律本身也是近似的定律.(见参考资料2)

\subsubsection{万有引力定律的建立}

第二节是全章的重点也是教
学中的一个难点。通过本节学习要使学生认识牛顿所建立的
万有引力定律不仅解决了行星运动的起因,而且揭示了自然
界物体间普遍存在的一种基本相互作用。为此,教学中可围
绕地和天统一这个中心突出点:第一,牛顿如何将天体运
动规律(开普勒第三定律)和在地球上得出的力学规律联系起
来,进行演绎,从而导出平方反比定律的。分析中只需指出引
力还与太阳质量$M$成正比这一结论,不必对常数$k$和太阳质
量$M$的关系作进一步讨论。第二,牛顿如何推广平方反比定
律,将天体间的引力和地面上的重力统一起来,使之成为一条
字宙万物间的普适物理定律。教材中对后一点的陈述较简
练,並在练习二中设计了一道题,引导学生通过推导和计算来
理解这段陈述.教学中可将此题(练习二4)作为课堂练习
导学生边练习边分析。练习时,可根据保持月球在其轨道
上运动的力也就是把地面上的物体放在那个位置所受到的重
力这一思路,画出示意图来帮助学生分析。

在平方反比定律的推广中,要将$g$和$a_R$加以比较,把两
种不同运动形式的加速度联系在一起,认为它们出自同一性
质的力,学生往往感到不易理解,这主要还是由于学生仍习
惯于从运动表现形式上来比较物体的受力情况,错误地认为
物体运动形式不相同,它所受的力也一定不相同,而对物体运
动方式是由受力和初始运动状态所共同决定的这一点,缺乏
足够的认识。要让学生认识地面上的苹果和天空中的月亮虽
然受到同一性质的力——地球引力的作用,但並不因此决定
它们有相同的运动形式。苹果的初速度为零,它便自由下落;
如果给它一个水平方向的初速度,它就作平抛运动;如果这个
水平速度越来越大,苹果也有可能绕着地表作匀速圆周运动。
在练习二4中适当点明这一点,不仅能帮助学生明确牛顿推
广平方反比定律的合理性,而且也为后面“人造卫星”一节的
教学作了一定的准备。

为了培养学生演绎推理的思维能力,进行上述课堂练习
时,教师要将以下三个层次交代清楚,即:
\begin{enumerate}
    \item 提出假设:牛
顿设想使月球围绕地球运行的力和地面上的重力属于同一性
质的力,都来自地球引力;
\item 根据假设进行演绎推导:练习
二的1、2、3;
\item 用已知的观察数据验证推导的结论,从
而证实假设是否成立:练习二的4、5。
\end{enumerate}

在归纳时,教师
可指出这种研究方法与第二章第十节阅读材料中所介绍的伽
利略研究匀变速运动的方法是一致的,从而引导学生对这一
物理学基本研究方法有更深一步的体会。

\subsubsection{对万有引力定律公式的理解}

引入万有引力定律公
式后要引导学生认识以下两点:

第一,平方反比定律公式的
形式是学生在学习物理中第一次遇到,在以后的学习中还要
接触,可引导学生注意这一公式在数学形式上的特点,並点明
这种与距离平方成反比的数学形式反映了自然界物质相互作
用所遵循的一种重要方式。此外必须明确对两个相距不太远
的非球形物体,不可简单地把两物中心间距作为R代入公式
来计算,这样只能作出粗略的估算。

第二,在说明引力与两个物体质量的乘积成正比时,要指
出两个不接触物体间的相互引力作用也是服从牛顿第三定律
的,即使很大质量和很小质量之间的相互吸引力也是大小相
等的。这一点似乎与学生的普通常识相矛盾,由于学生一般
遇到的都是卫星绕行星、行星绕恒星、地球表面物体自由下落
之类的问题,所以往往容易产生似乎只是质量大的物体吸引
质量小的物体,或者质量大的物体对质量小的物体的引力大
的错误观念,为了帮助学生理解这一点,可将练习二中的1、2作为课堂讨论题进行分析,並且举潮汐为例说明不仅地球
吸引月球,而且月球也吸引地球,潮汐就是质量小的物体也吸
引质量大的物体的具体例证。

\subsubsection{万有引力恒量} 
$G$是学生接触到的不多几个具有重
要地位的物理学普适恒量之一,要向学生指出,$G$作为万有
引力定律中的比例常数,不能单纯从数学角度去理解,要充分
认识它所表征的物理意义。要使学生理解比例常数是在描述
某种物理规律时经常出现的,各个常数有其特定的物理意义。
让学生回忆一下过去有比例常数的公式,如第一章中的胡克
定律$f=kx$中的 $k$, 表示某种材料在弹性限度内的力学性质,
因材料而异,不带普适性。而$G$表征质点间引力作用的
性质,它的数值等于两个质量各为1千克的质点相距1米的
相互吸引力,是适用于任何物体的普适恒量。以上这些比例
常数都有单位,单位由相关物理量决定。可以让学生自己确
定一下$G$的单位。其次要注意让学生对$G$的数值非常小有
个感性认识,防止学生产生一种错误观念:诸如认为固体之
所以成形,主要是由于物质颗粒间的万有引力使它们结合在
一起等,本章中许多数据用指数表示,並出现了不少指数运
算,为此可简单向学生介绍一下什么是数量级和怎样进行数
量级的估算。可结合本节教材中最后一段让学生自己估算一
下几种不同情况下引力的数量级,从而对一般物体之间和
天体之间引力大小的巨大差异有一个鲜明的认识。并建议用
以下的板图(或投影片)形象化地表示,也可
要求学生自己画在笔记本上,这比只用文字表示更易留下较
深的印象。

\subsubsection{卡文迪许实验}

在介绍卡文迪许实验装置时,可绘制
扭秤装置俯视图,也可做一个扭秤模型,说明扭秤装置中的
$T$形架增大了引力$F$的力臂,从而使石英细丝在$m$、$m'$两球
间微小的引力作用下产生一定的扭转形变,而$T$形架上的小
镜又利用光的反射定律把这一微弱的形变效应放大,加大标
尺与小镜间距离又能增大标尺上光点的偏转距离。在此可提
醒学生回忆一下第14页阅读材料:“显示微小形变的装置”.
正是这种“三次放大”的作用,扭秤才能较准确的测定微小的
作用力,例如,现代形式的卡文迪许装置能测出的引力约为
$6\x10^{-10}$牛,一根人发的重量是它的一万倍.这类利用“光杠
杆”作用的扭秤装置,是所有机械装置中最灵敏的装置之一。

\begin{center}
\begin{tabular}{p{.35\textwidth}p{.1\textwidth}p{.4\textwidth}}
    \hline
    两个物体 &引力的数量级 &相当于\\
    \hline
相距1米的两个1千克物体 & $10^{-10}$N  &  一粒砂子重量的1千万分之一 或 一根头发重量的十万分之一\\
相距10厘米的两个100克苹果 & $10^{-8}$N  &  一粒砂子重量的10万分之一\\
相距1米的两个成人 & $10^{-7}$N  &  一粒砂子重量的万分之一\\
相距100米的两艘万吨轮 & $10^0$N  &  两只鸡蛋的重量\\
相距$4\x 10^8$米的地球和月球 & $10^{20}$N  & 拉断钢索\\
相距$10^{11}$米的太阳和地球 & $10^{22}$N  &  可将直径为几千米的钢柱拉断\\
    \hline
\end{tabular}
\end{center}

\subsection{第二单元}
本单元运用的公式及相应的物理量较多,问题的综合程
度和灵活性又较前一章有所提高,学生往往不注意作有条理
的分析而惯于套用现成公式,单纯作公式代换,並易犯单位
和运算的错误,因此在教学过程中需注意帮助学生掌握综合
运用万有引力定律和匀速圆周运动的动力学方法分析具体问
题的基本思路。有关单位统一、指数运算等也要注意作出示范。

\subsubsection{万有引力定律在天文学上的应用} 

在引入“天体质量
算”这一课题时,可先提出能不能用简单的实验方法直接测
定地球或太阳的质量的问题,启发学生思考,并引导他们自己
用万有引力定律和圆周运动的知识,一步步导出计算天体质
量的公式。然后指出,计算某天体质量时只需知道围绕该天
体运行的行星(或卫星)的轨道半径$R$和周期$T$, 因这两个量
是可以测定的。

\subsubsection{地球上物体重量的变化}

通过本节学习应使学生了
解影响地球上物体重量变化的三个因素:纬度、离地高度和地
质结构,其中纬度(即地理位置)的变化是主要因素,关键是
使学生明确由于地球的自转,重力仅是引力的一个分力,而且
引力本身又从两极到赤道逐渐变小,此外应向学生指出课本
图5.4仅仅是一个示意图,地球的椭球状以及向心力相对于
引力的大小都是夸大了的,引力和重力之间的夹角也是极
小的。

本节教学中还有必要向学生指出处理某一物理量的变与
不变是相对的,必须根据所研究问题的要求来决定,由于地
球上的$g$随纬度、高度变化的相对数值很小,在一般计算中并
不考虑$g$的变化,而将它作为常数处理。


\subsubsection{人造地球卫星}

由于人造卫星问题的综合性较强,所
涉及的概念较多,学生往往搞不清其中的关系,常犯的错误
是把卫星绕地球运行的速率和第一宇宙速度(环绕速度)相混
淆。应该使学生明确:卫星绕地运行速率的表示式$v=\sqrt{\dfrac{GM}{r}}$,
对所有在圆形轨道上的地球卫星普遍适用,$v$的大小随$r$而
改变。而环绕速度表示式$v=\sqrt{gR_{\text{地}}}$仅适用于在近地圆形轨
道上运行的卫星,式中$g=9.8\msq$, 环绕速度的值为$7.9{\rm km/s}$,是个定值。

根据计算式$v=\sqrt{\dfrac{GM}{r}}$
,离地越远的卫星,$r$越大,$v$越
小.学生往往感到这一结论与课本图5.6中卫星进入轨道的
水平速度越大,轨道偏离地球越远的情况相矛盾,应该使学
生认识,卫星在椭圆轨道上运行时,它在各点的运动速度是不
同的。根据开普勒第二定律,卫星在近地点速度大,在远地点
速度小,在轨道上的平均速度也比在近地点的速度小,卫星进
入轨道的水平速度,只是卫星在近地点的速度,并不能反映出
它在椭圆轨道上各点的实际速度。根据公式$v=\sqrt{\dfrac{GM}{r}}$,
可
以用练习三1的方法,以卫星在近地点和远地点到地心距离
的平均值作为平均轨道半径,近似地求出卫星在轨道上的平
均运行速度。

教学中也可将练习四3作为课堂讨论练习题,通过对这
一问题的具体分析,进一步引导学生明确以上几点,在解答
该题时学生往往会将$2\pi R$, 除以周期80分钟,得出运行速率
$v\approx 8.4{\rm km/s}$,又根据这一速率大于$7.9{\rm km/s}$,便断定可
发射这样一颗卫星,学生之所以会得出这错误结论,是由
于不明确卫星的$T$、$r$和$v$之间有着确定的关系,因而错误地
将$R_{\text{地}}$作为轨道半径$r$来求运行速率$v$. 其次又将运行速率
错误地与进入轨道的最低水平速率混同起来。遇到这类错误,
可将它写在黑板上让学生共同来分析产生错误的原因。

关于第二、第三宇宙速度,只需指出只有当卫星获得足够
大的速度时它才能摆脱地球,甚至太阳的引力羁绊,而不必作
其他补充。

卫星中的超重、失重问题,主要抓住加速度向上还是向下
这个关键进行分析。对卫星在轨道上的失重情况,应使学生
理解此处所谓向下的加速度就是指向地心的向心加速度。

\section{实验指导}
\subsection{演示实验}

\subsubsection{天文挂图}

太阳系八大行星围绕太阳运动的示意
图,以及八大行星大小比较示意图,使学生对太阳系的结构有
一定性的形象了解。

\subsubsection{卡文迪许扭秤实验模型}
可根据课本图5.3的卡文迪许扭秤实验示意图,自制模
型,使学生了解扭秤装置的构造原理,以及如何利用光杠杆的
放大作用,读出石英丝的微小扭转形变。同时可以画出俯视
示意图(图5.1)。模型中射到平面镜$M$上的光是从平行光管
$S$射出的,反射光点投到圆弧形刻度盘上。

\subsubsection{挂图——人造地球卫星、宇宙飞船}

目的在于使学生了解人造地球卫星和宇宙飞船在轨道上
运行的原理,以及通讯卫星为什么可以实现全球电视转播的
原理。

有条件的可以播放人造地球卫星的发射、空间站、字
宙飞船的运行(包括字航员的失重状态)、航天飞机的发射与
返航等资料影片或录象,以增加感性认识,提高学习兴趣。













