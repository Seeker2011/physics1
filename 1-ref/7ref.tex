
\chapter{机械能}\minitoc[n]
\section{教学要求}
通过本章的学习,要使学生认识到:各种形式的能可以
相互转化,而且在转化过程中保持守恒;功是能的转化的
量度。

能的转化和守恒,是贯穿全部物理学的基本规律之一。解
决力学问题,从能量的观点人手分析,往往是很方便的。因此,
在本章的教学中,要特别注意培养学生从能量的观点来分析
问题。

这一章的教学要求是:
\begin{enumerate}
\item 理解功的概念,掌握功的计算公式,了解正负功的
意义。
\item 理解功率的概念,掌握功率的公式$P=Fv$.
\item 理解动能的概念,掌握动能的公式,掌握动能定理,会
运用动能定理解决力学问题。
\item 理解重力势能的概念,掌握重力势能的公式,掌握重
力做功与重力势能变化的关系。了解重力做功与路径无关是
引入重力势能的前提。
\item 了解弹性势能的概念,知道弹力做功与弹性势能
方关。
\item 掌握机械能守恒定律,会运用这个定律解决力学
问题。
\item 理解功和能的关系,知道功是能的转化的量度,学习
用能的观点来分析决力学问题。
\end{enumerate}


下面对这一章的教学内容作些具体说明。

为了给讲解动能和重力势能做好准备,教材安排了“能
量”一节,让学生对什么是能量以及功和能的关系有一个概括
的了解。什么是能量?许多课本中常常引用的“能是物体做功
的本领”,这个定义,严格说来有些不妥。从状态函数的角度
给出能量的普遍定义,中学生又难于接受 因此,教材沿用初
中物理的提法:一个物体能够做功,就说这个物体具有能量。
这并不是给能量下定义,只是使学生初步认识能量。

动能的公式实际上是动能定理的特殊情形,可以把动能
和动能定理合在一起来讲,直接推导出动能定理,同时给出动
能的定量表示。这样的讲法看起来简洁,但对初学者来说知识
太密集了 从便于接受的角度来考虑,还是把动能和动能定
理分作两节来讲为宜。

严格地说,应在重力做功与路径无关的基础上得出重力
势能的概念,这样讲,学生接受起来有一定的困难,因此教
材的讲法是,先通过克服重力做功引入重力势能,然后说明重
力做功与路径无关,正因为这样,才能引入重力势能的概念
这一点不能要求过高,使学生大体上有个印象就可以了,讲
解重力势能的相对性时,要使学生明确知道,参考平面的选择
可视研究问题的方便而定。重力势能的正负,使学生理解它
的意义即可。负的重力势能在力学中基本不用,在电学中要
讲到负的电势能。

关于势能属于系统,也不要求细讲;以后本章提到重力
势能时,仍认为是物体所具有的,因为采取这样的讲法,学生
容易理解,可以避免因过分严格而造成学习上的困难。但初
步了解势能属于系统,对今后学习内能有好处。

机械能守恒是指机械能在转化中守恒,因此在讲守恒定
律之前,应让学生对机械能的相互转化获得深刻的印象。这
里,要求学生对功这个概念有进一步的理解。在本章第一节
给出功的概念及其量度;第三节把功和能这两个概念联系起
来,使学生知道做功的过程总伴随着能量的改变,而且做多
少功,能量就改变多少,但未提及能的转化;这一节讲了机械
能的相互转化,并把做功和能的转化联系起来,使学生对功的
概念的理解进一步加深。

关于机械能守恒定律,在证明中,先得到结论,重力做多
少功,就有多少重力势能转化成等量的动能,然后移项得出机
械能守恒定律的表达式。这样处理,目的是强调等量转化,以
期学生对机械能守恒的理解能够具体些;同时有助于在本章
最后概括出功是能量转化的量度这一结论。教材是就自由落
体这个具体例子来证明机械能守恒定律的,没有作一般性的
证明。但是要求学生明确知道,在只有重力做功的情形下,不
论物体做直线运动还是曲线运动,守恒定律都成立。由于弹性
势能不作定量讨量,因此在守恒定律的表达中没有提到弹力
做功:面是在定律的表达之后提到在有弹性势能参加的相互
转化中,如果只有弹力做功,机械能也保持守恒。

第十节通过例题讲解如何应用机械能守恒定律时,应着
重说明以下几点。第一,说明用守恒定律求得的答案与用动
力学和运动学求得的答案相同,使学生确信守恒定律的有效
性。第二,说明运用守恒定律只涉及起始和终了状态,不涉
及中间过程的细节,因此用它来处理问题相当简便。第三,说
明有的问题只用守恒定律还不能完全解决,还需要用其他知
识,希望学生体会这一点,培养自己综合运用知识的能力。第
四,强调从能量观点分析问题的重要性,本节最后说明寻求
“守恒量”是物理研究工作的一个重要方面,希望学生能对守
恒定律的重要有所了解。

最后一节功和能的教学,主要是在前面的知识基础上,进
一步明确其他力做机械功的过程实际上是机械能与其他形式
的能相互转化的过程,而且做了多少机械功,机械能就改变多
少。最后得出结论:能量在转化中保持守恒,功是能量转化
的量度。

为了简明易懂,在最后一节的讲述中没有涉及弹性势能,
其他力是指重力以外的其他力。用绳子拉物体,绳的拉力属
于弹力,但作为外力来处理。这类问题不要引导学生去研究,
它已超出了本书的要求。

\section{教学建议}
本章是从做功和能量变化的角度,来研究物体在力的作
用下运动状态的改变的。为此,引入功、功率、动能、势能等概
念,介绍了动能定理和机械能守恒定律两条规律。这些概念
和规律在物理学上占有重要地位。

这一章可分两个单元。第一单元(第一、二节)讲述功和
功率这两个概念,第二单元(第三至十一节)讲解能的概念以
及不同情况下功和能的关系。本章教学的重点是使学生掌握
动能定理和机械能守恒定律这两条重要的物理规律。

\subsection{第一单元}
\subsubsection{功的概念的建立}

在物理学里,许多概念一建立起来
就能体会它明确的物理意义,如速度是用来反映运动快慢的
物理量,加速度是用来反映速度变化快慢的物理量等。但功这
个概念不是这样 教材先给功下一个明确的定义,然后要在
研究功和能的关系时才能逐步体会到功是物体能量转化的量
度,这一认识,要逐步渗透并贯穿于整章教学的过程中。因
此,在第一单元的教学中,首先是要让学生准确掌握功的定义
和计算 对功的意义的理解,不要急于求成,要在整章教学中
一步-步地引导,使学生逐步理解。

在介绍功的计算公式,研究力的方向与运动方向成$\alpha$角
的情况时,教材是将力$F$分解成两个分力$F_1$和$F_2$, 然后计算
出分力的功,从而得出$W=Fs\cos\alpha$的。这一分析过程包含
了分力所做的功的总和等于合力所做的功这一重要思路。教
师在教学中可适当地启发学生思考和体会这一点,以便在以
后的学习中应用。

得出公式$W=Fs\cos\alpha$后,应该通过对不同的$\alpha$角度的
讨论,理解什么情况下力$F$做功,什么情况下力$F$不做功,从
而与前面所说的功的两个不可缺少的因素相呼应。要使学生
明白,只要夹角$\alpha$不等于$90^{\circ}$, 力对物体就做了功.

对程度较好的学生,还可以进一步分析一下什么叫“物体
在力的方向上的位移”,如图7.1所示,$s\cos\alpha$就是物体在力
方向上的位移,也就是位移$s$在力的方向上的投影,用它乘
上力$F$, 也可以得到$W=Fs\cos\alpha$.
\begin{figure}[htp]
    \centering
\begin{tikzpicture}[>=latex]
\fill[pattern=north east lines](-1.5,-.2) rectangle (5.5,0);
\draw(-1.5,0)--(5.5,0);
\draw(-.7,0) rectangle (.7,1);
\draw[dashed](-.7+4,0) rectangle (.7+4,1);
\draw[dashed](0,.5)--(4,.5)--+(120:2);
\draw(0,.5)--(0,-1); \draw(0+4,.5)--(0+4,-1);
\draw[<->](0,-.75)--node[fill=white]{$s$}(4,-.75);
\draw[->, thick](0,.5)--+(30:1.5)node[below]{$F$};
\draw(.5,.5) arc (0:30:.5)node[right]{$\alpha$};
\draw[dashed](0,.5)--+(30:5);
\draw[decorate,decoration={brace,raise=5pt}] (0,.5) --node[above=10pt]{$s\cos\alpha$}+ (30:2*1.732);
\end{tikzpicture}
    \caption{}
\end{figure}

\subsubsection{正功和负功}

正确地理解正功和负功对以后的学习
很重要。由于公式$W=Fs\cos\alpha$ 中$F$和$s$均为绝对值,所以$W$
的正负完全决定于$\cos\alpha$ 的正负,要使学生体会到,$\cos\alpha >0$($F$
与$s$的夹角为锐角)意味着力$F$对物体产生位移$s$有一定贡
献,可以理解为力$F$对物体确实做了功.而当$\cos\alpha <0$时,$F
$与$s$的夹角为钝角,力$F$对物体产生位移$s$实际上起了阻碍
作用,所做的是负功。这时,物体要继续产生位移,必须克服
力$F$的阻碍,所以力$F$对物体做负功,又可以表达为物体克服
力$F$做功(取正值)。学了动能定理之后,学生就能进一步从
能量改变的角度体会正功和负功的意义。

\subsubsection{汽车牵引力与速度的关系}

教材推导了公式$P=Fv$。
要提醒学生注意,这是在力的方向与位移方向相同的情况下
推出来的,一般的实际问题都属于这种情况。

利用公式$P=Fv$来讨论牵引力与速度的关系,是以汽车
的输出功率一定为前提的,当然,汽车开动时实际功率不一定
保持不变,驾驶员可以用控制混合气体流量的办法来控制功
率,然后再用换档的办法来调节速度,从而改变牵引力的大
小,因此,公式$P=Fv$是有实际意义的。

为了使学生熟悉公式$P=Fv$的应用,课本安排了一个例
题。在讲解这个例题时,如何分析卡车由开始到匀速行驶的过
程是很重要的,它可以培养学生养成分析物理过程的习惯,避
免简单地套用公式。这一段分析实际上包含了三个内容,一是
在功率一定的条件下,利用公式$P=Fv$讨论$F$与$v$的关系;第
二是在不同$v$的情况下讨论$F$与阻力$f$的合力如何变化;第
三是当$F_{\text{合}}$减小时,加速度减小,但速度继续增大,当$F_{\text{合}}$减小
到零时,加速度为零,速度不再增加。此时的速度就是汽车在
上定功率下的最大速度。教学
时把这三个内容分析清楚了,
学生就比较容易理解了,为了
教学方便起见,在分析这一问
题时,建议教师在黑板上画出示意图,如图7.2所示.
\begin{figure}[htp]
    \centering
    \includegraphics[scale=.8]{fig/7-2.png}
    \caption{}
\end{figure}


\subsubsection{第二单元}
这一单元从第三节一直到第十一节,内容很多、第三节
相当于这一单元的引言。在这一节里所说的人们对功能关系
的基本认识——做功总是伴随着能量的改变,而且做多少功,
能量就改变多少一是整个单元教学的主线。第十一节是单
元的结束语。第三节和第十一节给出了这一章的基本思路,
应予以足够的重视。

\subsubsection{关于能量的概念}

要对能量下一个严格的定义是困
难的,因此,在新课本中不给出能量的定义,只沿用了初中课
本中的一个粗浅的说法:一个物体能够对外界做功,我们
就说这个物体具有能量。为了使学生能够比初中有更进一步
的理解,教材讨论了怎样定量地确定能量的变化问题,从而得
出用做功的多少来确定能量变化的多少这样一个基本认识。

讲解这一节教材的关键是要分析好一些实例:要分析流
动的河水、举高的铁锤等物体怎样对别的物体做功,自己的能
量又怎样变化。这样多次重复,就能使学生对上述基本认识
有深刻的印象。

这里有一个问题教学时应加以注意。本章前两节只讲力
对物体做功,即做功的主体是力。而这里说物体做功,怎么理
解?实际上这就是指物体克服阻力做功(即阻力对物体做负
功),例如流动的河水克服水轮机的阻力而做功;举高的铁锤
克服木桩的阻力而做功,等等,有了这一认识,就能对教材224
页中说的什么时候物体能量增加,什么时候能量减少有正确
的理解。要使学生明白,物体克服阻力做功,能量要减少,外
力对物体做正功,物体的能量要增加。

\subsubsection{动能概念和动能定理的建立}

教材分成三个层次来
建立起动能这个概念,第一步,先说运动的物体能够做功,因
而具有能量,称为动能。这一点是直接根据上一节所讲的能
量的概念来讲的第二步,讨论如何通过做功来定量地
确定动能的方法。这是以上一节中所讲的功和能的关系的基
本认识为依据的。第三步,具体推导动能的定量表达式。前
两步由于直接引用上一节关于能量的结论,所以容易被学生
所接受,第三步的推导应用了牛顿第二定律和运动学的一些
知识,其推导方法同下一节动能定理的推导方法是一样的,
只不过情况较为简单(假设物体原来是静止的)。所以这一节
教材同前后各节的联系是很紧密的。教学中要引导学生注意
这种联系,使学生对功和能的关系的认识能一环扣一环,逐步
加深理解。

动能定理是一条适用范围很广的物理规律,但教材在推
导这条规律时是由特殊到一般逐步扩大的。先把第四节的推
导扩大到初速不为零的情况,得到外力对物体所做的功等于
物体动能的增加的结论。然后将此结论推广到外力与
运动方向相反的情况,最后再推广到物体受到几个力作用的
情况。但是,尽管经过几次推广,教学中还是应该引导学生注
意这条规律的适用范围。在中学范围内,动能定理只应用在
研究对象是一个物体(质点)、作用力是恒力的情况,对作用力
是变力的情况,动能定理也是适用的,但学生没有学过变力作
功的计算,无法应用动能定理来解决实际问题。对于几个物
体组成的物体系,动能定理必须改变形式,否则不能适用(见
参考资料)

\subsubsection{“动能的增量”和“动能的增加”}

课本在表述动能定
理时没有用“动能的增量”这种提法,而说“动能的增加”,学生
比较容易理解,但对程度较好的学生也可介绍“增量”这种提
法。不管使用何种提法,都要使学生理解,变化后的动能与变
化前的动能之差,即
\[\frac{1}{2}mv^2_2-\frac{1}{2}mv^2_1\]
可以是正的,也可以
是负的。

\subsubsection{动能定理的应用}

讲解动能定理的例题要达到两个
目:一是学会应用动能定理解题的一般方法,即首先要分
析物体受力情况,并研究整个过程中外力所做的总功。然后
看初末状态的动能。最后列出方程,解出未知量,二是要让
学生明白同一例题用牛顿第二定律和运动学知识也可以解。
但在不要求研究过程中加速度和时间的情况下,用动能定理
要比用牛顿第二定律解题方便得多。

教师在教学过程中如果感到书上例题不够,可适当补充。
例如可以补充一道物体初速不为零的例题。但例题和习题不
必做得太多,这不仅是因为要减轻学生的负担,而且是对动能
定理这个教学重点的处理要适当。如果仅仅从算题的角度来
理解动能定理的重要,认为用它可以计算本章的各种问题,那
就必然会削弱其他知识的学习和理解。例如有的同学就认为
连势能这样的概念也没有必要引入,这种看法显然对培养学
生从能量角度考虑问题是不利的。


\subsubsection{如何使学生正确理解重力势能的概念}

重力势能是
一个重要的概念,对学生来说,掌握这个概念又是一个难点。为
了使学生掌握好这个难点,课本先从功和能的基本认识出发
引入重力势能概念,然后再讲重力做功的特点,并将重力势能
的一些性质(相对性、正负的意义、属于系统共有等)分散在第
六、七节中讲述。这样编排有利于学生理解重力势能的概念。
教学中应按教材的顺序逐步讲解,使学生的认识步步深入。
为了帮助学生正确理解重力势能的概念,再作如下几点建议:
 
重力势能的引入同第四节中动能的引入思路是一样
的。先讲被举高的物体能够做功,所以具有能量,称为重力势
能;然后从功和能的基本认识出发研究如何定量地确定势能
的大小;最后通过简单的例子(匀速举高一个物体)来引入重
力势能的计算式$E_p=mgh$. 因此讲解这部分内容时,可以与
动能的引入对照起来讲,使学生觉得并不陌生。

要使学生明白,课本231页第二小节里说的“克服重
力做多少功(即重力做负功),重力势能就增加多少;重力对物
体做多少功,重力势能就减少多少”这个结论始终成立,与物
体是否还有其他力做功以及物体的动能变化与否无关。

如果把这个结论同课本第三、四节中讨论功和能的关系
时说的“外力对物体做多少功,物体的动能就增加多少”进行
对照,会发现表述形成上不一致。为什么外力对物体做功能使
物体的能量增加,而重力对物体做功会使物体能量减少呢?这
是学生在学习过程中很容易混淆的一个问题。许多学生就是
由于在这里产生了混淆又不能及时澄清,采取了死记硬背的
学习方法,影响了整章的学习效果,其实,这两句话是不矛盾
的。外力对物体做功,使物体增加的是动能;量力对物体做正
功,所减少的是重力势能。此时物体动能增加与否要看合外
力是否对物体作正功。例如自由落体运动,物体只受到重力
作用,重力作为外力对物体作正功,物体的动能增加。但同时
重力势能减少。从这里也可以看出,重力做功使物体动能增
加,这是以减少重力势能为代价的,因此,重力作功的问题总
是伴随着能量的转化。这就把重力势能的问题同第九节机械
能守恒定律联系起来了,为学习第九节作了准备。

用这个观点再来分析课本图7.9的例子,可以看到,外力
$F$对物体做功,本来应该使物体动能增加,但重力作为外力对
物体作负功,能使物体动能减少,结果物体动能不变。不管动
能变或不变,重力作了负功,重力势能一定增加、这一结论仍
然成立。

关于重力做功的特点,教材通过物体从$A$点到$C$点的
不同途径(折线、直线、曲线)的计算,得出重力对物体所做的
功只跟起点和终点的位置有关,而跟物体运动的路径无关的
结论。这一结论的得来并不困难,但后面一段讨论,正因为
重力做功有这样的特点,才能引人重力势能的概念,就比较难
理解了,教材是用反证法来论证这一点的。教学中能使学生
通过阅读和讲解对这个问题有个印象就可以了。这对今后学
习静电学等知识有好处。教学中不要提保守力、耗散力的概
念,但应把重力做功和摩擦力做功的情况对照起来讲。

第六、七节内容较多,在两节学完以后建议小结一
下,突出两个重点。一个是重力势能的计算式$E_p=mgh$; 另
一个是重力做功与重力势能变化的关系。至于“克服重力做
多少功,重力势能就增加多少;重力对物体做多少功,重力势
能就减少多少”这一结论是否要再抽象为“重力做的功等于
物体重力势能增加的负值”,则要根据学生的接受能力而定。
对程度较好的学生,这样做也未尝不可。


\subsubsection{弹性势能的定性研究}

关于弹性势能,课本没有作定
量讨论,只作定性介绍.这一节包括三个方面的内容:
\begin{enumerate}
\item 什
么是弹性势能?教学中可采取与重力势能相对照来引人的方
法,并通过实例来加以解释,其中特别要注意弹性势能是属于
发生弹性形变的系统的。
\item 弹力作功与弹性势能变化的关系,这是达一节教材的主要内容,要使学生明白,克服弹力
做多少功,弹性势能就增加多少;弹力对其他物体做多少功,
弹性势能就减少多少,这个规律与重力做功跟重力势能变化
的关系是一样的,这个关系始终成立,跟物体(被弹力作用的
物体)是否还受其他外力、动能增加与否无关。
\item 弹簧的弹
性势能大小与形变大小和倔强系数有关,这一点不作定量计
算,但定性掌握它们的关系对于学习振动的知识是很有用的。
\end{enumerate}

\subsubsection{机械能的转化和守恒}

要讲机械能的守恒,先要讲机
械能的转化。没有动能和势能的相互转化,就无所谓机械能
的守恒,因此,首先要通过一些实例的分析,研究物体的动能
和势能的转化,为了使教学更充实些,除了教材所举的自由
落体、光滑斜面、竖直上抛及弹簧、弓箭等例子外,还可以举出
平抛、斜抛、光滑曲面等例子,来说明这些过程中机械能是如
何转化的。

要启发学生注意,势能的变化是由于重力和弹力做功而
引起的。但重力作为外力,又可以改变物体的动能(动能定
理)。如果重力做正功,重力势能将减少,动能将增加,意味着
重力势能转化为动能。反之也一样,这样不仅可以帮助学生理
解为什么课本中说“机械能的相互转化是通过重力或弹力做
功来实现的”,也为推导机械能守恒定律提供了思路。

在得到机械能守恒定律后,一定要强调条件。课本在表达
时只说“只有重力做功”,这是因为弹性势能不作定量讨论,只
限于定量研究重力势能与动能的转化问题。但在这一节的最
后也提到了如只有弹力做功,机械能也是守恒的。这一点,让
学生了解一下就可以了,需要强调的是,“只有重力做功”不
等于“只受重力作用”。物体可以受其他外力作用,只要这些
力不对物体做功,机械能就是守恒的。

应用机械能守恒定律解题时,首先要检验是否符合守恒
的条件,如果符合,则明确初状态和末状态后就可以列方程解
出未知量了,如果除重力和弹力外还有其他外力做功,则机械
能不守恒,但这种情况仍可以应用动能定理来解题。不管用
动能定理还是用机械能守恒定律解题都只涉及起始和终了状
态,不涉及中间过程的细节,因此相对于用牛顿第二定律和
运动学来解题要简便得多。此外,用守恒定律解题还有更深
刻的意义,因为自然界的许多规律就是通过寻找“守恒量”而
找到的。教学中讲一讲这个问题对开拓学生的视野是有好处
的。

\subsubsection{除重力、弹力外还有其他力做功的情况}

教材第十一
节是前几节的自然延伸,也是整章的小结。这一节的教学处理
得当,能把全章知识联系起来,做到融会贯通。建议教师在教
学时注意以下几点:

教材通过一系列实例来分析,除重力和弹力外如果
还有其他外力做功,则物体的机械能要增加,且其他力做多少
功,物体的机械能就增加多少。反之,如果物体克服其他外力
做功,机械能将减少。为了使学生对这一段分析理解得更具
体,可采取两种办法,对一般程度的同学,可以举一个有具体
数字的例子,证明其他力(如汽车牵引力)做的功等于物体机
械能的增加;对程度较好的同学,还可以从动能定理出发进行
推导.推导时不要涉及弹力做功.例如对课本245页上所举
的第二个实例(图7.3)应用动能定理可以写出如下的式子:
\[W_{\text{牵}}+W_G=\frac{1}{2}mv^2_2-\frac{1}{2}mv^2_1\]
由于重力做功等于重力势能增加的负值,即
$W_{\text{牵}}=mgh_1-mgh_2$,代入上式,得
\[W_{\text{牵}}+mgh_1-mgh_2 =\frac{1}{2}mv^2_2-\frac{1}{2}mv^2_1 \]
整理后,可得:
\[W_{\text{牵}}=\left(\frac{1}{2}mv^2_2+mgh_2\right)-\left(\frac{1}{2}mv^2_1+mgh_1\right)=E_2-E_1\]
\begin{figure}[htp]
    \centering
\begin{tikzpicture}[>=latex, scale=1.3]
    \fill[pattern=north east lines](-1,-.2) rectangle (1.5,0);
\draw(-1,0)--(1.5,0);
\draw (0,1.5) rectangle (.5,2);
\draw[dashed] (0,3.5) rectangle (.5,4);
\draw[|<->|](-.2,1.5)--node[fill=white]{$h_1$}(-.2,0);
\draw[|<->|](-.7,3.5)--node[fill=white]{$h_2$}(-.7,0);

\draw[<->](.25,1.75-.7)node[right]{$G$}--(.25,1.75)--(.25,1.75+.7)node[right]{$F$};
\draw[<->](.25,3.75-.7)node[right]{$G$}--(.25,3.75)--(.25,3.75+.7)node[right]{$F$};
\tkzDrawPoint(.25,1.75) \tkzDrawPoint(.25,3.75)
\draw[->](.75,1.5)--node[right]{$v_1$}(.75,2);
\draw[->](.75,3.5)--node[right]{$v_2$}(.75,4);

\end{tikzpicture}
    
    \caption{}
\end{figure}


采用推导的方法比较简捷明
了,而且有利于把其他力做功引起
的机械能增加同动能定理中的外力做的总功使物体动能增加
区别开来。但对基础不太好的同学,这样推导不易真正理解,
因此不宜采用。

得出上述其他力做功使机械能变化的结论后,课本
接着阐述了机械能的变其实是机械能同其他形式的能的转
化。这部分叙述是以这样两句话为依据的:“增加了的机械能
并不是凭空产生的,”“减少的机械能也不能无影无踪地消
失。”这实际上就是能量的转化和守恒的思想,由于前面已
学过机械能守恒定律,又讲过寻找“守恒量”的意义,所以这里
这样说学生是能够接受的。

在这一节的教学中,可以也应该将整个第二单元所
讨论的功和能的关系作一次整理,这一单元讨论功和能的关
系大致有这样几个层次:

\begin{enumerate}
    \item 先定性讨论做功能位能量发生变化。
    \item 定量研究合力做功(或外力做功的代数和)能使物体
    动能增加(动能定理)。
    \item 在这些外力中有两个力是特殊的;重力和弹力,重力
    做功,重力势能减少;克服重力做功,重力势能增加。弹力做
    功,弹性势能减少;克服弹力做功,弹性势能增加。
    \item 如果除重力和弹力外,其他力不做功,则机械能守恒。
    \item 如果除重力和弹力外,还有其他力做功,则机械能与
    其他形式的能发生转化,但能量仍守恒,
\end{enumerate}

总之,做功总是跟能量变化相联系,而且两者在数量上是
相等的,也就是说,功是能量转化的量度,以上几个关系可用
下面的示意图表示出来,由于课本不定量研究弹性势能,所
以示意图中也可以把弹力和弹性势能略去。

\begin{center}
\includegraphics[scale=.8]{fig/1.pdf}
\end{center}

\section{实验指导}
\subsection{演示实验}
\subsubsection{物体的动能跟它的质量和速度有关}
如图7.4所示,使小车沿一光滑斜面下滑.斜面倾角约
$10^{\circ}$—$15^{\circ}$即可,斜面底端接着一个水平木板,在水平木板上
铺一层毛巾布。在斜面和平面的接合处,用毛巾布堵塞以减少
小车下滑到接合处时发生撞击或弹跳。
















































































